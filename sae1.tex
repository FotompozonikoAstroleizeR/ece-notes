\documentclass[11pt,a4paper,notitlepage,fleqn,draft]{article}

\usepackage{amsmath}
\usepackage{amsfonts}
\usepackage{amssymb}
\usepackage{libs/commath2}
\usepackage[table]{xcolor}
\usepackage[hidelinks,draft=false]{hyperref}
\usepackage[skins,theorems]{tcolorbox}
\usepackage{titlesec}
\usepackage{tikz}
\usepackage{libs/circuitikz} % use our own recent version to make sure some bugs are fixed
\usepackage{pgfplots}
\usepackage{mathtools}
\usepackage[makeroom]{cancel}
\usepackage{mathrsfs}
\usepackage{wrapfig}
%\usepackage{subcaption}
%\usepackage{floatrow}
\usepackage{esint}
\usepackage{enumitem}
%\usepackage{bm}
\usepackage{relsize}
\usepackage{xfrac}
\usepackage{comment}
\usepackage{siunitx}
\usepackage{multicol}
%\usepackage{MnSymbol}
\usepackage[obeyDraft,disable]{todonotes}
%\usepackage{morefloats} % oh no!
%\usepackage[linesnumbered,lined]{algorithm2e}
\usepackage{glossaries}
\usepackage{xifthen}


\pgfplotsset{compat=1.13}
\usetikzlibrary{arrows.meta}
\usetikzlibrary{patterns}
\usetikzlibrary{decorations.pathmorphing}
\usetikzlibrary{decorations.markings}
\usetikzlibrary{backgrounds}
\usetikzlibrary{shapes.misc}
\usetikzlibrary{shapes.multipart}
\usetikzlibrary{shadows.blur}
\usetikzlibrary{fadings}
\usetikzlibrary{intersections}
\usetikzlibrary{arrows.meta}
\usetikzlibrary{calc}
\usetikzlibrary{matrix}
\usetikzlibrary{positioning}
\usetikzlibrary{shapes}
\usetikzlibrary{shadings}

\tcbuselibrary{breakable}
\tcbuselibrary{skins}
\tcbuselibrary{xparse}

\tikzset{cross/.style={cross out, draw,
        minimum size=2*(#1-\pgflinewidth),
        inner sep=0pt, outer sep=0pt}}
\tikzset{
    mark position/.style args={#1(#2)}{
        postaction={
            decorate,
            decoration={
            	post length=1mm, % ??? Magic to fix "Dimension
            	pre length=1mm, % ???  too large" errors.
                markings,
                mark=at position #1 with \coordinate (#2);
            }
        }
    }
}
\tikzset{
	arrow at/.style args={#1}{
		postaction={
			decorate,
			decoration={
				post length=1mm, % ??? Magic to fix "Dimension
				pre length=1mm, % ???  too large" errors.
				markings,
				mark=at position #1 with {\arrow{>}};
			}
		}
	}
}
\makeatletter
\tikzset{
  use path for main/.code={%
    \tikz@addmode{%
      \expandafter\pgfsyssoftpath@setcurrentpath\csname tikz@intersect@path@name@#1\endcsname
    }%
  },
  use path for actions/.code={%
    \expandafter\def\expandafter\tikz@preactions\expandafter{\tikz@preactions\expandafter\let\expandafter\tikz@actions@path\csname tikz@intersect@path@name@#1\endcsname}%
  },
  use path/.style={%
    use path for main=#1,
    use path for actions=#1,
  }
}
\makeatother

\pgfmathdeclarefunction{sinc}{1}{%
	\pgfmathparse{abs(#1)<0.01 ? int(1) : int(0)}%
	\ifnum\pgfmathresult>0 \pgfmathparse{1}\else\pgfmathparse{sin(#1 r)/#1}\fi%
}
\pgfmathdeclarefunction{gauss}{2}{%
	\pgfmathparse{1/(#2*sqrt(2*pi))*exp(-((x-#1)^2)/(2*#2^2))}%
}

\usepackage[left=2cm,right=2cm,top=2cm,bottom=2cm]{geometry}

%\usepackage[no-math]{fontspec}
%\usepackage{fontspec}
\usepackage{mathspec}
%\usepackage{newtxtext,newtxmath}
%\usepackage{unicode-math}
%\setmainfont{texgyretermes-regular.otf}
%\setsansfont{texgyreheros-regular.otf}
%\newfontfamily\greekfont[Script=Greek]{Linux Libertine O}
%\newfontfamily\greekfontsf[Script=Greek]{Linux Libertine O}
\usepackage{polyglossia}
%\newfontfamily\greekfont[Script=Greek]{texgyretermes-regular.otf}
\newfontfamily\greekfontsf[Script=Greek]{texgyreheros-regular.otf}
\newfontfamily\greekfonttt[Script=Greek]{Latin Modern Mono}
%\usepackage[greek]{babel}
\setdefaultlanguage{greek}
\setotherlanguage{english}

%\usepackage[utf8]{inputenc}
%\usepackage[greek]{babel}


%\usepackage{tkz-euclide} % loads  TikZ and tkz-base
%\usetkzobj{angles} % important you want to use angles

\newlist{enumparen}{enumerate}{1}
\setlist[enumparen]{label=(\arabic*)}
\newlist{enumpar}{enumerate}{1}
\setlist[enumpar]{label=\arabic*)}

\newlist{enumgreek}{enumerate}{1}
\setlist[enumgreek]{label=\alph*.}
\newlist{enumgreekparen}{enumerate}{1}
\setlist[enumgreekparen]{label=(\alph*)}
\newlist{enumgreekpar}{enumerate}{1}
\setlist[enumgreekpar]{label=\alph*)}


\newlist{enumroman}{enumerate}{1}
\setlist[enumroman]{label=(\roman*)}

\newlist{enumlatin}{enumerate}{1}
\setlist[enumlatin]{label=(\alph*)}

\newlist{invitemize}{itemize}{1}
\setlist[invitemize]{noitemsep,label=}

\input{libs/fiximplies}
\input{libs/sphere}

\makeatletter
\let\anw@true\anw@false

%\newcommand{\attnboxed}[1]{\textcolor{red}{\fbox{\normalcolor\m@th$\displaystyle#1$}}}
\makeatother
\tcbset{highlight math style={enhanced,colframe=red,colback=white,%
        arc=0pt,boxrule=1pt,shrink tight,boxsep=1.5mm,extrude by=0.5mm}}
\newcommand{\attnboxed}[1]{\tcbhighmath[colback=red!5!white,drop fuzzy shadow,arc=0mm]{#1}}
\newcommand{\infoboxed}[1]{%
	\tcbhighmath[colframe=blue!50!white,colback=blue!5!white,arc=0mm]{#1}}
\titleformat{\section}{\bf\Large}{Κεφάλαιο \thesection}{1em}{}
\newtcolorbox{attnbox}[1]{colback=red!5!white,%
    colframe=red!75!black,fonttitle=\bfseries,title=#1}
\newtcbox{quickattnbox}[1]{colback=red!5!white,%
	colframe=red!75!black,fonttitle=\bfseries,title=#1}
\newtcolorbox{infobox}[1]{colback=blue!5!white,%
    colframe=blue!75!black,fonttitle=\bfseries,title=#1}

\tcbset{frogbox/.style={enhanced jigsaw,%
		overlay first={\foreach \x in {0cm} {
				\begin{scope}[shift={([xshift=-0.2cm]title.west)}]
					\draw[very thick,green!65!black!50!white,latex-] (0,0) -- ++(-1.5,0);
\end{scope}}}}}
\tcbset{frogtitle/.style={
attach boxed title to top left=
{xshift=0mm,yshift=-0.50mm},
boxed title style={skin=enhancedfirst jigsaw,
	bottom=0mm,
	interior style={fill=none,
		left color=green!20!black,
		right color=gray}}
}}
\DeclareTColorBox{exercise}{ O{} }{
	enhanced jigsaw,
	breakable,parbox=false,
	%title style={left color=gray!50!white!50!green,opacity=.5,right color=white},
	subtitle style={%boxrule=1pt,
		colback=yellow!50!red!25!white,fontupper=\bfseries},
	coltitle=black,colbacktitle=green!90!black!25!white,colframe=black,
	frame hidden,
	boxrule=0mm,
	%boxrule=1mm,
	leftrule=0.8pt,toprule=0.8pt,rightrule=0pt, %reserve space
	borderline west={0.8pt}{0pt}{white!25!black},%---- draw line
	borderline north={0.8pt}{0pt}{white!25!black},%---- draw line
	interior hidden,
	%frame style={left color=black,right color=white},
	sharp corners=all,
	%frogbox, %TODO: frogbox
	before lower={\tcbsubtitle[before skip=\baselineskip]{Λύση}},lower separated=false,
	before title={\textbf{Άσκηση\ifthenelse{\isempty{#1}}{}{: }}},
	title={\ifthenelse{\isempty{#1}}{\hspace{0pt}}{#1}}%
}

\AtBeginDocument{%
\let\arg\relax
\let\Re\relax
\let\Im\relax
\DeclareMathOperator{\arg}{Arg}
\DeclareMathOperator{\Re}{Re}
\DeclareMathOperator{\Im}{Im}
}
\DeclareMathOperator{\sinc}{sinc}
\DeclareMathOperator{\sgn}{sgn}
\DeclareMathOperator{\erf}{erf}
\DeclareMathOperator{\cov}{cov}
\DeclareMathOperator{\atand}{atan2}

\newenvironment{absolutelynopagebreak}
{\par\nobreak\vfil\penalty0\vfilneg
	\vtop\bgroup}
{\par\xdef\tpd{\the\prevdepth}\egroup
	\prevdepth=\tpd}

\DeclareSIUnit \voltampere { VA } %apparent power 
\DeclareSIUnit \var { VAr } %volt-ampere reactive - idle power 
\DeclareSIUnit \decade { dec } %decade

% Global amount of samples
% Set to a higher value (e.g. 200) for nicer graphs
% Set to a low value (e.g. 10) for performance
% NOTE: Check the sample variables below for further measurements
\newcommand*{\gsamples}{200}

% Equals command as a workaround for CircuiTikZ bug
% not allowing the = sign in labels
\newcommand*{\equals}{=}

\newcommand{\nesearrow}{%
	\,%
	\smash{\raisebox{-1.1ex}
		{$%
			\stackrel{\displaystyle\nearrow}{\displaystyle\searrow}%
			$}}%
}
\newcommand{\degree}{^{\circ}} % not great
\newcommand\numberthis{\addtocounter{equation}{1}\tag{\theequation}} % add an equation number to a number-less math environment

% Provided commands
\providecommand\dif{d}
\providecommand\od[2]{\frac{#1}{#2}}

\newtcbtheorem[number within=section,list inside=thm]{theorem}{Θεώρημα}%
{colback=green!5,colframe=green!35!black,colbacktitle=green!35!black,fonttitle=\bfseries,enhanced,attach boxed title to top left={yshift=-2mm,xshift=-7mm},width=.9\textwidth,arc=.7mm}{th}
\newtcbtheorem[number within=section,list inside=defn]{defn}{Ορισμός}%
{colback=blue!5,colframe=cyan!35!black,colbacktitle=blue!35!black,fonttitle=\bfseries,enhanced,attach boxed title to top left={yshift=-2mm,xshift=-2mm}}{def}

% Locus plot utilities
\tikzset{locus/.style={orange!50!red!70!brown}}
\tikzset{locuspole/.style={draw=red!30!black,cross,inner sep=2.5pt,fill=white,fill opacity=.6,thick,label={[below]-90:#1}}}
\tikzset{locuszero/.style={draw=red!30!black,circle,inner sep=2pt,fill=white,fill opacity=.6,thick,label={[below]-90:#1}}}
\tikzset{locusbreak/.style={rounded corners=1.5pt,inner sep=2pt,draw,top color=brown,bottom color=black,fill opacity=.8,label={[below]-90:#1}}}

% New plotting utilities
\def\lowsamples{18}
\def\hisamples{40}
\def\timecolour{blue!50!cyan}

\tikzstyle{timecolour}=[\timecolour]



\title{ΣΑΕ 1
	\\
	{ 
		\normalsize Συστήματα Αυτομάτου Ελέγχου I
		\\
		\normalsize Σημειώσεις από τις παραδόσεις
	}}
\date{Οκτώβριος-Ιανουάριος 2017
	\\
	{ 
		\small Τελευταία ενημέρωση: \today
	}
}
\author{
	Για τον κώδικα σε \LaTeX, ενημερώσεις και προτάσεις:
	\\
	\url{https://github.com/kongr45gpen/ece-notes}}

\setallmainfonts(Digits,Latin,Greek){Asana Math}
\setmainfont{Noto Serif}
\setsansfont{Ubuntu}
\usepackage{polyglossia}
\newfontfamily\greekfont[Script=Greek,Scale=1.00]{Liberation Serif}

\hypersetup{pdftitle = {ΣΑΕ 1}}


\begin{document}
\maketitle

\hrule
\vspace{50pt}
	
Συστήματα Αυτομάτου Ελέγχου
	
Υπεύθυνη καθηγήτρια: Ζωή Δουλγέρη, ασκήσεις από τον Παπαγεωργίου Δημήτρη - δεν υπάρχει διαχωρισμός ασκήσεων και θεωρίας.

\section{Συστήματα}
Γνωρίζουμε από τα προηγούμενα μαθήματα τι είναι το σύστημα. Σκοπός του μαθήματος είναι να σχεδιάσουμε έναν "ελεγκτή" ώστε ένα σύστημα να έχει μια επιθυμητή έξοδο.

Για παράδειγμα, αν έχουμε έναν κινητήρα που επιθυμούμε να ελέγξουμε, μπορούμε να τον παραστήσουμε με το παρακάτω σχήμα:

\begin{tikzpicture}
\draw (0,0) rectangle (1,1) node[midway] {$\Sigma$};
\draw[->] (1,0.5) -- ++(1,0) node[right] {$y$};
\draw (1.5,0.5) -- (1.5,-1) -- (0,-1);
\draw (0,-0.75) rectangle (-0.5,-1.25) node[midway] {$H$};
\draw[->] (-0.5,-1) -- (-2,-1) -- (-2,0.25);
\draw (-2,0.5) circle (0.25);
\draw[->] (-3,0.5) -- (-2.25,0.5) node[above,midway] {$r$};
\draw (-1.35,0.25) rectangle (-0.65,0.85) node[midway] {$E$};
\draw (-1.75,0.5) -- (-1.35,0.5) ;
\draw (-0.65,0.5) -- (0,0.5) node[above left] {$u$};
\end{tikzpicture}

όπου:
\begin{itemize}
	\item \( \Sigma \) είναι ο κινητήρας
	\item \( u \) είναι η τάση εισόδου (που ρυθμίζουμε εμείς)
	\item \( y \) είναι η έξοδος του συστήματος, εδώ η ταχύτητα του κινητήρα
	\item Η ροπή του φορτίου εκφράζει την είσοδο της διαταραχής
	\item \( H \) είναι ένας μετρητής που μπορούμε να έχουμε για να ελέγχουμε την ταχύτητα
	του κινητήρα
	\item \( E \) είναι ο ελεγκτής που θέλουμε να υλοποιήσουμε, ώστε να ρυθμίζει την
	τάση \( u \) εισόδου του κινητήρα για να πετύχουμε την επιθυμητή ταχύτητα.
\end{itemize}

Έχουμε και μία \textbf{είσοδο αναφοράς} που καθορίζει την επιθυμητή έξοδο του συστήματος.

Στα πλαίσια των ΣΑΕ βρίσκουμε το μαθηματικό μοντέλο του συστήματος, καθώς και το μαθηματικό
μοντέλο του ελεγκτή, και τα υλοποιούμε με φυσικό τρόπο (για παράδειγμα μέσω κυκλωματικών
στοιχείων, μικροελεγκτών, arduino κ.ά).

Παραδείγματα συστημάτων αυτομάτου ελέγχου είναι: Τα αμορτισέρ του αυτοκινήτου, οι κινητήρες
των CD drives, οι κινητήρες των γραμμών παραγωγής (ώστε για παράδειγμα να είμαστε σίγουροι
ότι τα υλικά περνάν από έναν κλίβανο ακριβώς για 30 λεπτά, διατηρώντας σταθερή την
ταχύτητα μεταφοράς τους), κ.ά.

\subsection{Μοντελοποίηση συστημάτων}
\paragraph{}
Για το σύστημα ενός σώματος στο οποίο ασκείται δύναμη, έχουμε πολύ απλά:
\[
	F = m\ddot x
\]

Για μια δύναμη ελατηρίου, ισχύει \( F = \kappa \cdot \delta x \), και για μια δύναμη
απόσβεσης/ιξώδους: \( F = d\dot x \)

\paragraph{Ανάρτηση αυτοκινήτου}
Θεωρούμε ότι η ανάρτηση ενός αυτοκινήτου αποτελείται από ένα ελατήριο και έναν αποσβεστήρα:

\begin{tikzpicture}
\draw (0,0) rectangle (2,1) node[midway] {$m$};
\draw (0.5,0) -- ++(0,-0.4)
-- ++(-30:0.2)
-- ++(-150:0.4)
-- ++(-30:0.4)
-- ++(-150:0.4)
-- ++(-30:0.4)
-- ++(-150:0.4)
-- ++(-30:0.2)
-- (0.5,-2);
;
\draw (1.5,0) -- (1.5,-1);
\draw (1.25,-1) -- (1.75,-1);
\draw (1.1,-0.9) -- (1.1,-1.2) -- (1.9,-1.2) -- (1.9,-0.9);
\draw (1.5,-1.2) -- (1.5,-2);

\draw (-0.5,-2) -- (2.5,-2);
\draw[gray] (2.5,1.2) -- (2.5,-2.7);

\draw (1,-2) -- (1,-2.5);
\filldraw (1,-2.5) circle(1pt) node[below left] {$\rho$};

\draw[dashed] (0.5,0) -- (3,0) node[right] {$x_0$};
\draw[dashed] (0.5,-2) -- (3,-2) node[right] {$x_i$};
\end{tikzpicture}

Και όπως πριν προκύπτει η σχέση:
\[
m\ddot x_0 + b(\dot x_0 - \dot x_i) + \kappa (x_0-x_i) = 0
\]
η οποία μπορεί να μετασχηματιστεί κατά Laplace:
\begin{align*}
	m\ddot x_0 + b\dot x_0 + \kappa x_0 &= b\dot x_i + \kappa \dot x_i \\
	ms^2X_0(s) + bsX_0(s) + \kappa X_0(s) &= X_1(s)bs + \kappa X_1(s) \\
	\Aboxed{\frac{X_0(s)}{X_1(s)} &= \frac{bs+\kappa}{ms^2+bs+\kappa} }
\end{align*}



Αυτή είναι μία απλή μέθοδος μοντελοποίησης συστημάτων, αλλά η μοντελοποίηση δεν είναι
αντικείμενο αυτού του μαθήματος.

\subsection{Ορισμοί}

\begin{defn}{}{}
	\begin{itemize}
	\item
	\textbf{Συνάρτηση μεταφοράς:} \(
	\displaystyle G(s) = \frac{Y(s)\quad\text{\small (έξοδος)}}{U(s) \quad
		\text{\small (είσοδος)} }
	= \frac{N(s)\quad \text{\small (αριθμητής)}}{D(s)\quad \text{\small (παρονομαστής)} }
	\)
	\item \textbf{Χαρακτηριστικό πολυώνυμο:} \( D(s) \)
    \end{itemize}
\end{defn}
Θυμόμαστε ότι στα φυσικά συστήματα δεν γίνεται να έχουμε βαθμό του αριθμητή μεγαλύτερο από
το βαθμό του παρονομαστή.
\begin{defn}{Μορφές έκφρασης συνάρτησης μεταφοράς}{}
	\begin{align*}
		H(s) &= \frac{K(s+z_1)\cdots(s+z_m)}{(s+p_1)\cdots(s+p_n)} \\
		H(s) &= \frac{G(1+s\tau_{n+1})\cdots(1+s\tau_{n+m})}{(1+s\tau_1)\cdots(1+s\tau_n)}
		\quad \text{όπου } G=\frac{kz_1\cdots z_m}{p_1\cdots p_m}
	\end{align*}
\end{defn}
\begin{defn}{}{}
	\begin{itemize}
    \item \textbf{Πόλοι} ονομάζονται οι τιμές \( p \) για τις οποίες ισχύει:
    \[
    \mathlarger{\lim_{s\to p}  \left\lvert H(s) \right\rvert} = \infty
    \]
    \item \textbf{Μηδενικά} ονομάζονται οι τιμές \( z \) για τις οποίες ισχύει:
    \[
    \mathlarger{\lim_{s\to z}  \left\lvert H(s) \right\rvert} = 0
    \]
    \end{itemize}
\end{defn}
\begin{theorem}{Σύνδεση εν σειρά}{}
	Όταν συνδέουμε δύο απομονωμένα συστήματα εν σειρά, για τις συναρτήσεις μεταφοράς τους
	ισχύει:
	\[
	G(s) = G_1(s)G_2(s)
	\]
	
	\begin{center}
	\begin{tikzpicture}[scale=1.2]
	\draw[->] (0,0) -- (1,0) node[above,midway] {$X_1(s)$};
	\draw (1,-0.3) rectangle (2,0.3) node[midway] {$G_1(s)$};
	\draw (2,0) -- (3,0) node[above,midway] {$X_2(s)$};
	\draw (3,-0.3) rectangle (4,0.3) node[midway] {$G_2(s)$};
	\draw[->] (4,0) -- (5,0) node[above,midway] {$X_3(s)$};
	\end{tikzpicture}
    \end{center}
\end{theorem}
\paragraph{Παράδειγμα} \hspace{0pt}


\begin{circuitikz}[american,scale=1.4]
	\ctikzset{bipoles/thickness=3}
	\draw (0,2) to [V=$u(s)$] (0,0);
	\draw[color=green!50!black] (0,2)
	to[R=$R_1$] (2,2)
	to[C=$C_1$] (2,0)
	-- (0,0);
	
	\draw[color=green!50!cyan!50!black] (2,2)
	to[R=$R_2$] (4,2)
	to[C=$C_2$] (4,0)
	-- (2,0);
	
	\draw (4,2) to[short,-*] (5,2);
	\draw (4,0) to[short,-*] (5,0);
	\draw (5,2) to[open, v^=$y$] (5,0);
\end{circuitikz}

Για το παραπάνω κύκλωμα, αν και έχουμε δύο συστήματα ενωμένα σε σειρά, δεν μπορούμε να
εφαρμόσουμε το θεώρημα στο παραπάνω κύκλωμα, αφού τα επιμέρους κυκλώματα δεν είναι απομονωμένα
και παρουσιάζουν σύνθετες αντιστάσεις εισόδου και εξόδου. Πράγματι, αν επιλύσουμε το
κύκλωμα:
\begin{align*}
	G_1(s)G_2(s) &= \frac{1}{(R_1C_1s+1)}\frac{1}{(R_2C_2s+1)} \\
	\frac{y(s)}{u(s)} &= \frac{1}{R_1C_1R_2C_2s^2+(R_1C_1+R_2C_2+\underline{R_1C_2})s+1}
\end{align*}
Παρατηρούμε τον όρο \( R_1C_2s \) που δεν υπάρχει στον απλό πολλαπλασιασμό των δύο
συστημάτων.

\paragraph{Άσκηση}
Ποιές από τις παρακάτω συναρτήσεις μεταφοράς έχουν \textbf{ρυθμούς} (ρίζες του παρονομαστή)
που δεν είναι πόλοι;
\begin{enumroman}
	\item \( \displaystyle \frac{s+8}{(s+3)(s+10)} \)
	\item \( \displaystyle \frac{s+1}{(s+1)^2(s+2)} \)
	\item \( \displaystyle \frac{s+9}{(s+2)^2+9} \)
	\item \( \displaystyle \frac{s+1}{(s+1)(s+2)} \)
\end{enumroman}
\subparagraph{Απάντηση}
\begin{enumroman}
	\item Έχει μηδενικό στο \( -8 \) και πόλλους στα \( -3 \) και \( -10 \).
	\item Έχει μόνο έναν πόλο στο \( -1 \) και στο \( -2 \).
	\item Έχει μηδενικό στο \( -9 \) και πόλους στα \( -2+j3 \) και \( -2-j3 \).
	\item Έχει μόνο πόλο στο \( -2 \).
\end{enumroman}

Η λύση αυτή μπορεί να προκύψει από τους ορισμούς του πόλου και του μηδενικού.

\subsection{Σύστημα κλειστού βρόγχου}

\begin{tikzpicture}[scale=1.3]
\draw[->] (-0.25,0) -- (0.5,0) node[above,midway] {$r(s)$};
\draw (0.75,0) circle (0.25);
\draw (1,0) -- ++(0.5,0) node[above,midway,green!50!black,scale=0.8] {$w(s)$};
\draw (1.5,-0.5) rectangle (3,0.5) node[midway] {$G(s)$};
\draw (3,0) -- (3.75,0);
\draw (4,0) circle (0.25);
\draw[->] (4.25,0) -- ++(1,0) node[above right] {$y$};

\draw[<-] (4,0.25)  node[left] {$+$} --++ (0,0.5)
node[above,rectangle,align=center,scale=0.8]
{είσοδος\\διαταραχής\\$d(s)$};

\draw (4.5,0) -- ++(0,-1.5) -- (3,-1.5);
\draw (1.5,-2) rectangle (3,-1) node[midway] {$H(s)$};
\draw[->] (1.5,-1.5) -- (0.75,-1.5) node[above,midway,green!50!black,scale=0.8] {$f(s)$} 
-- ++(0,1.25) node[right,xshift=1mm,yshift=-1mm] {$-$};
\end{tikzpicture}

Ορίζουμε:
\begin{alignat*}{2}
	\text{συνάρτηση μεταφοράς κλειστού βρόγχου: } && T(s) &= \frac{y(s)}{r(s)} \\
	\text{συνάρτηση μεταφοράς εισόδου διαταραχής: } && T_d(s) &= \frac{y(s)}{d(s)}
\end{alignat*}

Για να υπολογίσουμε την έξοδο του συστήματος, αν δεν λάβουμε υπ' όψιν
την είσοδο διαταραχής:
\begin{align*}
	y(s) &= G(s)w(s)
	\\ &= G(s)\left( r(s)-f(s) \right) \\
	y(s) &= G(s)\left[ r(s)-H(s)y(s) \right] \\
	y(s)\left[1+G(s)H(s)\right] &= G(s)r(s) \\
	y(s) &= \frac{G(s)r(s)}{1+G(s)H(s)} \\
	T(s) &= \frac{G(s)}{1+G(s)H(s)}
\end{align*}

Αν συμπεριλάβουμε και την είσοδο διαταραχής, το ζητούμενο είναι η είσοδος αυτή να μην
επηρεάζει καθόλου (ή όσο το δυνατόν λιγότερο) την έξοδο. Έχουμε:
\begin{align*}
    y(s) &= d(s) + G(s)w(s)
    \\ &= d(s) - G(s)f(s) \\ &= d(s) - G(s)H(s)y(s) \implies \\
    y(s)\left[ 1+G(s)H(s) \right] &= d(s) \implies \\
	T_d(s) &= \frac{y(s)}{d(s)} = \frac{1}{1+G(s)H(s)}
\end{align*}

Συνοπτικά:
\begin{theorem}{Συναρτήσεις μεταφοράς σε σύστημα κλειστού βρόγχου}{}
	Για ένα σύστημα κλειστού βρόγχου με είσοδο \( r(s) \), είσοδο διαταραχής
	\( d(s) \), συνάρτηση \( G(s) \) στην ευθεία διαδρομή και \( H(s) \) στη
	διαδρομή ανάδρασης, οι συναρτήσεις μεταφοράς είναι:
	\begin{align*}
		T(s) &= \left. \frac{y(s)}{r(s)} \right\rvert_{d(s) = 0}
		= \frac{G(s)}{1+G(s)H(s)} \\
		T_{d}(s) &= \left. \frac{y(s)}{d(s)} \right\rvert_{r(s) = 0}
		=
		\frac{1}{1+G(s)H(s)}
	\end{align*}
	και η συνολική έξοδος του συστήματος είναι:
	\[
	y(s) = T(s)r(s) + T_d(s)d(s)
	\]
\end{theorem}

Παρατηρούμε ότι το χαρακτηριστικό πολυώνυμο είναι το ίδιο στις δύο συναρτήσεις μεταφοράς.

\paragraph{Παράδειγμα} \hspace{0pt}

\begin{tikzpicture}
\draw[->] (-0.25,0) -- (0.5,0) node[above,midway] {$r(s)$};
\draw (0.75,0) circle (0.25);
\draw (1,0) -- ++(0.5,0);
\draw (1.5,-0.5) rectangle (3,0.5) node[midway] {$\displaystyle\frac{\kappa}{s+a}$};
\draw (3,0) -- (4,0);
\draw[->] (4,0) -- ++(1,0) node[above right] {$y$};

\draw (4,0) -- ++(0,-1.5) -- (3,-1.5);
\draw[dashed] (3,-1.5) -- (1.5,-1.5);
\draw (1.5,-1.5) -- (0.75,-1.5) -- ++(0,1.25) node[below right] {$-$};
\end{tikzpicture}

Θα υπολογίσουμε την έξοδο του συστήματος \underline{χωρίς ανάδραση} και \underline{με ανάδραση} σε βηματική είσοδο \( r(s) \rightarrow \mathrm u(t) \).
\subparagraph{Χωρίς ανάδραση} \hspace{0pt}

Το σύστημα χωρίς ανάδραση είναι το παραπάνω χωρίς τον κάτω βρόχο:
\begin{tikzpicture}[baseline,scale=0.7]
\draw[->] (-0.25,0) -- (0.5,0) node[above,midway] {$r(s)$};
\draw (0.75,0) circle (0.25);
\draw (1,0) -- ++(0.5,0);
\draw (1.5,-0.5) rectangle (3,0.5) node[midway] {$\frac{\kappa}{s+a}$};
\draw (3,0) -- (4,0);
\draw[->] (4,0) -- ++(1,0) node[above right] {$y$};
\end{tikzpicture}

Και ισχύει:
\begin{align*}
	y(s) &= r(s)\frac{\kappa}{s+a} \\
	y(s) &= \kappa\frac{1}{s}\frac{1}{s+a} \\
	y(t) &= \kappa\left( 1-e^{\sfrac{-t}{\tau}} \right)
\end{align*}
(όπου η σταθερά χρόνου \( \tau = \frac{1}{a} \))

Για \( t\to \infty \) το αποτέλεσμα είναι \( y(t) = \kappa \).

\subparagraph{Με ανάδραση}
\begin{align*}
	y(s) &= \frac{G(s)r(s)}{1+G(s)} \implies \dots \implies
	\tau' = \frac{1}{a+\kappa}
\end{align*}
Παρατηρούμε πως το σύστημα αυτό φτάνει πολύ πιο γρήγορα στην τελική του τιμή. Αυτό φαίνεται
αν συγκρίνουμε τις σταθερές χρόνου μεταξύ τους, σκεπτόμενοι ότι λειτουργούν ως συντελεστές
στην εκθετική συνάρτηση:

\begin{tikzpicture}
\draw (-2,0) -- (2,0);
\draw (0,-2) -- (0,2);

\draw (-1.5,0) node[cross=4pt,thick,blue] {} node[below,yshift=-1mm] {$-a-\kappa$};
\draw (-0.5,0) node[cross=4pt,thick,yellow!50!brown!50!red] {} node[above,yshift=1mm] {$-a$};

\begin{scope}[xshift=4cm]
\draw (-1,0) -- (3,0);
\draw (0,-2) -- (0,2);

\draw [very thick, color=blue, domain=0:3,variable=\t, samples=\gsamples, smooth]
plot (\t,{ 1-exp(-\t) }) node[below] {$-a$};
\draw [very thick, color=yellow!50!brown!50!red, domain=0:3,variable=\t, samples=\gsamples, smooth]
plot (\t,{ 1-exp(-5*\t) }) node[above,xshift=-2cm] {$-a-k$};
\end{scope}
\end{tikzpicture}

\subsection{Ισοδύναμα λειτουργικά διαγράμματα}
Για τη διευκόλυνσή της εύρεσης της συνάρτησης μεταφοράς, μπορούμε αντί να βρούμε την έξοδο
αλγεβρικά χρησιμοποιώντας ενδιάμεσες συναρτήσεις, να εφαρμόσουμε κανόνες όπως τους
παρακάτω:
\begin{infobox}{}
	\begin{itemize}
		\item
		\begin{tikzpicture}[scale=1.3,baseline]
		\begin{scope}[yshift=2mm]
		\draw[->] (0,0) -- (0.75,0);
		\draw (0.75,-0.7/2) rectangle ++(1,0.7) node[midway] {$H$};
		\draw[->] (1.75,0) -- (2.5,0) node[right] {$y_1$};
		\draw[->] (2,0) -- (2,-0.5) -- (2.5,-0.5) node[right] {$y_2$};
		\end{scope}
		
		\draw[->,very thick,gray!20!black] (4,0) to[bend left=10] ++(1.5,0);
		
		\begin{scope}[xshift=6.5cm]
		\draw (0,0) -- (0.5,0);
		\draw (0.5,0) -- (0.5,0.3) -- (1,0.3);
		\draw (0.5,0) -- (0.5,-0.3) -- (1,-0.3);
		\draw (1,-0.5) rectangle (1.7,-0.1) node[midway] {$H$};
		\draw[->] (1.7,-0.3) -- ++(0.5,0) node[right] {$y_2$};
		\draw (1,0.5) rectangle (1.7,0.1) node[midway] {$H$};
		\draw[->] (1.7,0.3) -- ++(0.5,0) node[right] {$y_1$};
		\end{scope}
		\end{tikzpicture}
		\item
		\begin{tikzpicture}[scale=1.3,baseline]
		\begin{scope}[]
		\draw[->] (0,0) -- (0.75,0) node[above,midway] {$u$};
		\draw (0.75,0) -- (0.75,0.5) -- (1.25,0.5);
		\draw (1.25,0.5-0.7/2) rectangle ++(1,0.7) node[midway] {$H$};
		\draw[->] (2.25, 0.5) -- (2.75,0.5) node[right] {$y$};
		\draw[->] (0.75,0) -- (0.75,-0.5) -- (2.75, -0.5) node[right]{$y_1$};
		\end{scope}
		
		\draw[->,very thick,gray!20!black] (4,0) to[bend left=10] ++(1.5,0);
		
		\begin{scope}[xshift=6.5cm,yshift=0.3cm]
		\draw[->] (0,0) -- (0.75,0) node[above,midway] {$u$};
		\draw (0.75,-0.5/2) rectangle ++(0.75,0.5) node[midway] {$H$};
		\draw[->] (1.5,0) -- (2.25,0) node[right] {$y$};
		\draw[->] (1.75,0) -- (1.75,-0.5) -- (2.25,-0.5);
		\draw (2.25,-0.5-0.5/2) rectangle ++(0.75,0.5) node[midway] {$\frac{1}{H}$};
		\draw[->] (3,-0.5) -- ++(0.5,0) node[right] {$y_1$};
		\end{scope}
		\end{tikzpicture}
		\item
		\begin{tikzpicture}[scale=1.3,baseline]
		\begin{scope}[]
		\draw[->] (0,0.5) -- ++(0.75,0);
		\draw (0.75,0.5-0.5/2) rectangle ++(0.75,0.5) node[midway] {$H_1$};
		\draw[->] (1.5,0.5) -- (2.25,0.5) -- (2.25,0.2) node[right,yshift=1mm,xshift=1mm] {$+$};
		\draw[->] (0,-0.5) -- ++(0.75,0);
		\draw (0.75,-0.5-0.5/2) rectangle ++(0.75,0.5) node[midway] {$H_2$};
		\draw[->] (1.5,-0.5) -- (2.25,-0.5) -- (2.25,-0.2);
		\draw (2.25,0) circle (2mm);
		\draw[->] (2.45,0) -- ++(0.75,0) node[above,pos=.9] {$y$};
		\end{scope}
		
		\draw[->,very thick,gray!20!black] (4,0) to[bend left=10] ++(1.5,0);
		
		\begin{scope}[xshift=6.5cm]
		\draw[] (0,0.5) -- ++(0.6,0);
		\draw (0.6,0.5-0.5/2) rectangle ++(1,0.5) node[midway] {$\sfrac{H_1}{H_2}$};
		\draw[->] (1.6,0.5) -- (2.25,0.5) -- (2.25,0.2) node[right,yshift=1mm,xshift=1mm] {$+$};
		\draw[->] (0,-0.5) -- (2.25,-0.5) -- (2.25,-0.2);
		\draw (2.25,0) circle (2mm);
		\draw (3.05,-0.4/2) rectangle ++(0.8,0.4) node[midway] {$H_2$};
		\draw[] (2.45,0) -- ++(0.6,0);
		\draw[->] (3.85,0) -- ++(0.6,0) node[right] {$y$};
		\end{scope}
		\end{tikzpicture}
	\end{itemize}
\end{infobox}

\paragraph{Παράδειγμα} \hspace{0pt}

\begin{tikzpicture}
\def\h{0.6}
\def\l{1}
\def\ll{0.75}
\draw[->] (0,0) node[green!70!black,scale=.8,opacity=.8,left] {$r(s)$} -- (1.5-0.2,0);
\begin{scope}[xshift=1.5cm]
\draw (0,0) circle (2mm);
\draw (0.2,0) -- ++(0.75,0);
\draw (0.2+0.75, -\h/2) rectangle ++(\l, \h) node[midway] {$H_1$};
\draw[->] (\l+0.2+0.75,0) -- (2.8,0) node[green!70!black,scale=.8,opacity=.8,above,midway] {$w(s)$};
\draw (3,0) circle (2mm);
\end{scope}
\begin{scope}[xshift=4.5cm]
\draw (0.2,0) -- ++(0.75,0);
\draw (0.2+0.75, -\h/2) rectangle ++(\l, \h) node[midway] {$H_2$};
\draw[] (\l+0.2+0.75,0) -- (3.5,0) node[right] {$y(s)$};
\end{scope}

\draw (0.5,0) -- ++(0,1.5) -- (4.5,1.5) -- (4.5,1);
\draw (4.5-\ll/2,1) rectangle (4.5+\ll/2,0.6) node[midway] {$\kappa$};
\draw[->] (4.5,0.6) -- (4.5,0.2) node[right,xshift=1mm,yshift=1mm] {$+$};

\draw[->] (4.5+2.5,0) -- ++(0,-1.5) -- (1.5,-1.5) -- (1.5,-0.2) node[right,xshift=1mm,yshift=-1mm] {$-$};
\end{tikzpicture}

Χρησιμοποιώντας τους παραπάνω κανόνες, ή την προηγούμενη μέθοδο, μπορούμε να βρούμε:
\[
T(s) = \frac{\kappa H_2(s)+H_2(s)H_1(s)}{1+H_1(s)H_2(s)}
\]

Ενδεικτικά, με βοηθητικές συναρτήσεις, οι πράξεις γίνονται ως εξής:
\begin{align*}
	y(s) &= H_2(s)\cdot\left( \kappa r(s) + w(s) \right)
	\\ &= H_2 \cdot \left[ \kappa r + H_1 \left( r - y \right) \right]
	\\ &= \kappa H_2 r + H_2H_1r - H_1H_2y \implies
	\\ y \cdot (1+H_1H_2) &= \kappa H_2 r + H_2H_1 r \implies
	\\ y &= \frac{\kappa H_2 r + H_2H_1 r}{1+H_1H_2} \implies
	\\ T(s) &= \frac{\kappa H_2(s) + H_2(s)H_1(s) }{1+H_1(s)H_2(s)}
\end{align*}


\section{Προδιαγραφές}
Ορίζουμε κάποιες προδιαγραφές που επιθυμούμε να πληροί η έξοδος του συστήματος, όπως η
ακρίβεια θέσης, η ταχύτητα της απόκρισης, η ευστάθεια κλπ. Για να μετρήσουμε ποσοτικά αυτά
τα κριτήρια, ορίζουμε νέα μεγέθη και χρησιμοποιούμε διάφορες συναρτήσεις ως
"εισόδους αναφοράς", όπως την κρουστική \( \delta(t) \) (για μελέτη ευστάθειας),
τη βηματική \( \mathrm u(t) \), την ράμπα, την ημιτονοειδή (για μελέτη
απόκρισης συχνότητας και ταχύτητας) κλπ.

\subsection{Ακρίβεια}
Το ζητούμενο της ακρίβειας είναι η τελική έξοδος να είναι κοντά στην επιθυμητή είσοδο.

Για να υπολογίσουμε την τελική έξοδο, δεν χρειάζεται να υπολογίσουμε τον αντίστροφο Μ/Σ
Laplace της συνάρτησης για να πάμε στο πεδίο του χρόνου, αλλά αρκεί να χρησιμοποιηθεί το θεώρημα της τελικής τιμής:
\[
f(\infty) = \lim_{s\to 0} sF(s)
 \]

Για παράδειγμα, για βηματική είσοδο (\( \mathrm u(t) \rightarrow \frac{1}{s}\))
σε ένα σύστημα (ss = steady state):
\begin{align*}
	y_{\mathrm{ss}} &= \lim_{s\to 0} sF(s) = \lim_{s\to 0} sH(s)\frac{1}{s} =
	\lim_{s\to 0} H(s)
\end{align*}

\begin{comment}
\begin{tikzpicture}
\def\h{0.6}
\def\l{1.2}
\def\ll{0.75}
\draw[->] (0,0) node[left] {$r(s)$} -- (1-0.2,0);
\begin{scope}[xshift=1cm]
\draw (0,0) circle (2mm);
\draw (0.2,0) -- ++(0.75,0);
\draw (0.2+0.75, -\h/2) rectangle ++(\l, \h) node[midway] {$G_1(s)$};
\draw (\l+0.2+0.75,0) -- (3,0);
\draw  (3, -\h/2) rectangle ++(\l, \h) node[midway] {$G_2(s)$};
\draw (3+\l,0) -- (5-0.2,0);
\draw[->] (5,1) -- ++(0,-1+0.2) node[midway,right] {$d(s)$};
\draw (5,0) circle (2mm);
\draw[->] (5+0.2,0) -- ++(1,0) node[right] {$y$};
\end{scope}

\draw[->] (4.5+2.2,0) -- ++(0,-1.5) -- (4,-1.5);
\draw (4,-1.5-\h/2) rectangle ++(-\l,\h) node[midway] {$H(s)$};
\draw[->] (4-\l,-1.5) -- (1,-1.5) -- (1,-0.2);
\end{tikzpicture}
\end{comment}

Για να μελετήσουμε την ακρίβεια, ορίζουμε το σφάλμα:
\begin{defn}{Σφάλμα}{}
	\[
	e(s) = r(s) - y(s)
	\]
\end{defn}

Χρησιμοποιούμε διάφορες εισόδους για να βρούμε διάφορα είδη σφαλμάτων του συστήματος:

\begin{alignat*}{5}
\begin{tikzpicture}[scale=0.4,baseline]
\draw (-0.2,0) -- (2,0);
\draw (0,-0.2) -- (0,2);
\draw [very thick, color=blue] (0,1.8) -- (2,1.8);
\end{tikzpicture}\quad
y(t) &= \mathrm u(t) &&\quad y(s) = \frac{1}{s} &\qquad e_{\mathrm ssp} & \quad{\text{σφάλμα θέσης}} \\
\begin{tikzpicture}[scale=0.4,baseline]
\draw (-0.2,0) -- (2,0);
\draw (0,-0.2) -- (0,2);
\draw [very thick, color=blue] (0,0) -- (2,2);
\end{tikzpicture}\quad
y(t) &= t &&\quad y(s) = \frac{1}{s^2} &\qquad e_{\mathrm ssv} & \quad{\text{σφάλμα ταχύτητας}} \\
\begin{tikzpicture}[scale=0.4,baseline]
\draw (-0.2,0) -- (2,0);
\draw (0,-0.2) -- (0,2);
\draw [very thick, color=blue, domain=0:2,variable=\t, samples=\gsamples, smooth]
plot (\t,\t^2/2);
\end{tikzpicture}\quad
y(t) &= \sfrac{t^2}{2} &&\quad y(s) = \frac{1}{s^3} &\qquad e_{\mathrm ssa} & \quad{\text{σφάλμα επιτάχυνσης}}
\end{alignat*}

Για παράδειγμα, για το σύστημα κλειστού βρόγχου, θυμόμαστε ότι:

\begin{tikzpicture}[scale=1]
\draw[->] (-0.25,0) -- (0.5,0) node[above,midway] {$r(s)$};
\draw (0.75,0) circle (0.25);
\draw (1,0) -- ++(0.5,0);
\draw (1.5,-0.5) rectangle (3,0.5) node[midway] {$H(s)$};
\draw (3,0) -- (3.75,0);
\draw[->] (3.25,0) -- ++(1,0) node[above right] {$y$};

\draw (3.6,0) -- ++(0,-1.5) -- (3,-1.5) -- (1.5,-1.5);
\draw[dashed] (1.5,-2) rectangle (3,-1) node[midway,opacity=.5] {$G(s)$};
\draw[->] (1.5,-1.5) -- (0.75,-1.5) -- ++(0,1.25) node[right,xshift=1mm,yshift=-1mm] {$-$};
\end{tikzpicture}

Αν θεωρήσουμε ότι η \( G(s) \) είναι \( 1 \), τότε:
\begin{align*}
	\frac{y(s)}{r(s)} &= \frac{H(s)}{1+H(s)} \\
	\text{όπου } H(s) &=
	\frac{G(τ_{n+1}s+1)\cdots(τ_{n+m}s+1)}{s^N(1+sτ_1)\cdots(1+sτ_n)} \implies
	\\
	e(s) &= r(s)-y(s) = \frac{1}{1+H(s)}r(s) \\
	e_{\mathrm{ss}} &= \lim_{s\to 0}se(s)
\end{align*}
(δηλαδή \( N \) είναι η τάξη του τυχόν πόλου στο 0).

Εφαρμόζοντας τις διάφορες συναρτήσεις ως εισόδους, σύμφωνα με τα παραπάνω, έχουμε:
\begin{alignat*}{3}
	e_{\mathrm{ssp}} &= \frac{1}{\displaystyle 1+\underbrace{\lim_{s\to 0}H(s)}_{K_\mathrm{pos}}}
	= \frac{1}{1+{\color{green!50!black}K_\mathrm{pos}}}
	&&= \begin{cases}
	\frac{1}{1+{\color{green!50!black}G}} \qquad &\text{για } N = 0 \\
	\frac{1}{1+{\color{green!50!black}\infty}} = 0 \qquad &\text{για } N \geq 1
	\end{cases}
	\\
	e_{\mathrm{ssv}} &= \frac{1}{\displaystyle \underbrace{\lim_{s\to 0} sH(s)}_{K_v} }
	= \frac{1}{\color{cyan!50!black}K_v} &&=
	\begin{cases}
	\frac{1}{\color{cyan!50!black}0} = \infty \qquad &\text{για } N = 0
	\\
	\frac{1}{\color{cyan!50!black}G}\qquad &\text{για } N=1
	\\
	\frac{1}{\color{cyan!50!black}\infty} = 0\qquad &\text{για } N\geq 2
	\end{cases}
	\\
	e_{\mathrm{ssa}} &= \frac{1}{\displaystyle \underbrace{\lim_{s\to 0} s^2H(s)}_{K_a}}
	= \frac{1}{\color{orange!50!black}K_a}
	&&= \begin{cases}
	\infty \qquad &\text{για } N \leq 1 \\
	\frac{1}{\color{orange!50!black} G} \qquad &\text{για } N = 2 \\
	0 \qquad &\text{για } N > 2
	\end{cases}
\end{alignat*}

Δε συζητάμε για ακρίβειες πέραν της επιτάχυνσης, επειδή σπάνια τα συστήματα έχουν πάνω από
2 ολοκληρωτές.

\begin{defn}{Ολοκληρωτής}{}
	Ένας \textbf{πόλος στο 0} λειτουργεί σαν \textbf{ολοκληρωτής}, επειδή
	έχει την ιδιότητα να ολοκληρώνει το σήμα εισόδου.
\end{defn}
\begin{defn}{Τύπος συστήματος}{}
	Ο \textbf{τύπος του συστήματος} είναι ο αριθμός των ολοκληρωτών που έχει.
\end{defn}

\paragraph{Παράδειγμα}
Έστω το σύστημα:

\begin{tikzpicture}[scale=1]
\draw[->] (-0.25,0) -- (0.5,0) node[above,midway] {$r(s)$};
\draw (0.75,0) circle (0.25);
\draw (1,0) -- ++(0.5,0);
\draw (1.5,-0.5) rectangle (3,0.5) node[midway] {$H(s)$};
\draw (3,0) -- (3.75,0);
\draw (4,0) circle (0.25);
\draw[->] (4.25,0) -- ++(1,0) node[above right] {$y$};
\draw ({(3+1.5)/2},0.5) node[above] {$\frac{1}{s(s+a)}$};

\draw[<-] (4,0.25)  node[right,xshift=1mm,yshift=1mm] {$+$} --++ (0,0.5)
node[above,rectangle,align=center,scale=0.8]
{$d(s)=\frac{1}{s}$};

\draw[->] (4.5,0) -- ++(0,-1.5) -- (0.75,-1.5) -- ++(0,1.25)
node[right,xshift=1mm,yshift=-1mm] {$-$};
\end{tikzpicture}

Ποιά θα είναι η έξοδος του συστήματος στη μόνιμη κατάσταση αν προσθέσουμε την είσοδο
διαταραχής;

\subparagraph{}

Γνωρίζουμε για το σύστημα κλειστού βρόχου ότι:
\begin{align*}
	\frac{y(s)}{d(s)} &= \frac{1}{1+H(s)} \\
	\lim_{s\to 0}sy(s) &= \lim_{s\to 0} s\frac{d(s)}{1+H(s)}
	= \lim_{s\to 0} s\frac{\frac{1}{s}}{1+H(s)} = \lim_{s\to 0}\frac{1}{1+H(s)} = 0
%	\frac{y(s)}{r(s)} &= \frac{H(s)}{1+H(s)}\\
%	y(s) &= \frac{H(s)}{1+H(s)}r(s) + \frac{1}{1+H(s)}d(s)
\end{align*}
Δηλαδή \( y_{\mathrm{ss}} = 0 \), άρα το σύστημα έχει πάλι τέλεια ακρίβεια, και σφάλμα θέσης
0.

\paragraph{Παράδειγμα} \hspace{0pt}

Το παρακάτω διάγραμμα αντιστοιχεί σε έναν κινητήρα:

\begin{tikzpicture}[scale=1]
\draw[->] (-0.25,0) -- (0.5,0) node[above,midway] {$r(s)$};
\draw (0.75,0) circle (0.25);
\draw (1,0) -- ++(0.5,0);
\draw (1.5,-0.5) rectangle (3,0.5) node[midway] {$κ$};
\draw (3,0) -- (3.75,0);
\draw (4,0) circle (0.25);
\draw (5,-0.5) rectangle (6.5,0.5) node[midway] {$\frac{1}{s(Js+b)}$};
\draw (4.25,0) -- (5,0);
\draw[->] (6.5,0) -- ++(1,0) node[above right] {$y$};

\draw[<-] (4,0.25)  node[right,xshift=1mm,yshift=1mm] {$+$} --++ (0,0.5)
node[above] {$d$};

\draw[->] (7,0) -- ++(0,-1.5) -- (0.75,-1.5) -- ++(0,1.25)
node[right,xshift=1mm,yshift=-1mm] {$-$};
\end{tikzpicture}

Το μετασχηματίζουμε στο ισοδύναμό του, ώστε να εφαρμόσουμε τους τύπους κλειστού βρόγχου:

\begin{tikzpicture}[scale=1]
\draw[->] (-0.25,0) -- (0.5,0) node[above,midway] {$r(s)$};
\draw (0.75,0) circle (0.25);
\draw (1,0) -- ++(0.5,0);
\draw (1.5,-0.5) rectangle (3,0.5) node[midway] {$κ$};
\draw (3,0) -- (3.75,0);
\draw (4,0) circle (0.25);
\draw (6,-0.5) rectangle (7.5,0.5) node[midway] {$\frac{1}{s(Js+b)}$};
\draw (4.25,0) -- (6,0);
\draw[->] (7.5,0) -- ++(1,0) node[above right] {$y$};

\draw[<-] (4,0.25)  node[right,xshift=1mm,yshift=1mm] {$+$} --++ (0,0.5)
node[above] {$d$};


\draw[->] (5,0) -- ++(0,-1.5) -- (3.5,-1.5);
\draw[->]  (2,-1.5) -- (0.75,-1.5) -- ++(0,1.25)
node[right,xshift=1mm,yshift=-1mm] {$-$};

\draw (2,-1.5-0.5) rectangle (3.5,-1.5+0.5) node[midway] {$\frac{1}{s(Js+b)}$};
\end{tikzpicture}

Τότε προκύπτει (για είσοδο \( \frac{1}{s} \), αφού αναζητούμε σφάλμα θέσης):
\begin{align*}
    T(s) &= \frac{1}{s(Js+b)}\frac{κ}{1+\frac{κ}{s(Js+b)}} 
    = \frac{κ}{s(Js+b)+κ}\\
    T_{d}(s) &= \frac{1}{s(Js+b)}\frac{1}{1+\frac{κ}{s(Js+b)}}
    = \frac{1}{s(Js+b)+κ}
     \\
    y(s) &= T(s)r(s) + T_d(s)d(s)
    \\ &= \frac{1}{s}\frac{κ}{s(Js+b)+κ} + \frac{1}{s}\frac{1}{s(Js+b)+κ}
    \\
    e(s) &= r(s) - y(s) \\
    e_{\mathrm{ssp}} &= \lim_{s\to 0} se(s)
    \\ &= \lim_{s \to 0} s\left[
    \frac{1}{s} - \frac{1}{s}\frac{κ}{s(Js+b)+κ} - \frac{1}{s}\frac{1}{s(Js+b)+κ}
    \right]
    \\
    &= \lim_{s\to 0} \left[
    1 - \frac{κ}{s(Js+b)+κ} - \frac{1}{s(Js+b)+κ}
    \right] \\
    &= 1-\frac{κ}{κ}-\frac{1}{κ} = -\frac{1}{κ}
\end{align*}
Δηλαδή βλέπουμε ότι η διαφορετική θέση της εισόδου διαταραχής επηρεάζει το σφάλμα θέσης του
συστήματος.

Αν, προσπαθώντας να μειώσουμε στο 0 το σφάλμα της εξόδου, προσθέσουμε έναν ολοκληρωτή
πριν από την είσοδο διαταραχής:

\begin{tikzpicture}[scale=1]
\draw[->] (-0.25,0) -- (0.5,0) node[above,midway] {$r(s)$};
\draw (0.75,0) circle (0.25);
\draw (1,0) -- ++(0.5,0);
\draw (1.5,-0.5) rectangle (3,0.5) node[midway] {$\sfrac{κ}{\color{cyan!70!black} s}$};
\draw (3,0) -- (3.75,0);
\draw (4,0) circle (0.25);
\draw (5,-0.5) rectangle (6.5,0.5) node[midway] {$\frac{1}{s(Js+b)}$};
\draw (4.25,0) -- (5,0);
\draw[->] (6.5,0) -- ++(1,0) node[above right] {$y$};

\draw[<-] (4,0.25)  node[right,xshift=1mm,yshift=1mm] {$+$} --++ (0,0.5)
node[above] {$d$};

\draw[->] (7,0) -- ++(0,-1.5) -- (0.75,-1.5) -- ++(0,1.25)
node[right,xshift=1mm,yshift=-1mm] {$-$};
\end{tikzpicture}

τότε το χαρακτηριστικό πολυώνυμο του συστήματος γίνεται:
\[
Js^3+bs^2+κ = 0
\]
που έχει ρίζες στο δεξί ημιεπίπεδο, άρα οδηγεί σε ασταθές σύστημα (θυμόμαστε από τα Κ3 ότι μόνο τα πολυώνυμα με θετικούς συντελεστές \textit{μπορεί} να οδηγήσουν σε ευστάθεια).

Για να διορθώσουμε αυτήν την ατέλεια, χρησιμοποιούμε έναν \textbf{ελεγκτή PI}
(Proportional \& Integral), δηλαδή πολλαπλασιάζουμε την είσοδό του \( e(t) \) 
με \( K_P e(t) \) και ολοκληρώνουμε με
\( K_I \int e(t) \).

\begin{infobox}{Ελεγκτής PI}
Σύμφωνα με την παραπάνω παράγραφο, η έξοδος \( u(t) \) ενός ελεγκτή PI είναι:
\[
u(t) = K_P\cdot e(t) + K_I \int e(t) \dif t
\]
(όπου \( e \) η είσοδος)
και, μετασχηματίζοντας κατά Laplace:
\[
u(s) = \left( K_P + \frac{K_I}{s} \right) =
\frac{K_P\left( s+\frac{K_I}{K_P} \right)}{s}
= \frac{K_P(s+z)}{s}
\]
όπου \( z = \frac{K_P}{K_I} \) μία σταθερή τιμή.
\end{infobox}

Στο συγκεκριμένο παράδειγμα, αντικαθιστούμε τη συνάρτηση μεταφοράς \( \sfrac{κ}{s}  \) με
τη συνάρτηση:
\[
K_p\left(\frac{s+z}{s}\right)
\]
όπου \( z = \frac{K_P}{K_I} \),
και το χαρακτηριστικό πολυώνυμο γίνεται:
\[
x(s) = Js^3 + bs^2 + K_p s + K_I = 0
\]
που μπορεί να είναι ευσταθές με κατάλληλη επιλογή των σταθερών.

\paragraph{Παράδειγμα} \hspace{0pt}

Στο σύστημα χωρίς βρόγχο:

\begin{tikzpicture}[scale=0.9]
\draw[->] (0,0) -- (1.5,0) node[above,midway] {$r(s)$};
\draw (1.5,-0.5) rectangle (3,0.5) node[midway] {$H_κ(s)$};
\draw[->] (3,0) -- ++(1.5,0) node[above,midway] {$y(s)$};
\end{tikzpicture}

\begin{align*}
	e_{\mathrm{ssp}} &= \lim_{s\to 0} se(s) = \lim_{s\to 0}\left(1-H_κ(s)\right) \\
	e_{\mathrm{ssv}} &= \lim_{s\to 0} \frac{1-H_κ(s)}{s} \\
	e_{\mathrm{ssa}} &= \lim_{s\to 0} \frac{1-H_κ(s)}{s^2}
\end{align*}

\paragraph{Παράδειγμα} \hspace{0pt}

\begin{tikzpicture}[scale=1]
\draw[->] (-0.25,0) -- (0.5,0) node[above,midway] {$r(s)$};
\draw (0.75,0) circle (0.25);
\draw (1,0) -- ++(0.5,0);
\draw (1.5,-0.5) rectangle (3,0.5) node[midway] {$H(s)$};
\draw (3,0) -- (3.75,0);
\draw[->] (3.25,0) -- ++(1,0) node[above right] {$y$};

\draw[->]  (3.6,0) -- ++(0,-1.5) -- (1.5,-1.5) -- (0.75,-1.5) -- ++(0,1.25) node[right,xshift=1mm,yshift=-1mm] {$-$};
\end{tikzpicture}

\[
H(s) = \frac{s+9}{s^2+7s+3}
\]

Τι σφάλμα θέσης έχει το παραπάνω σύστημα;

\begin{align*}
	e_{\mathrm{ssp}} &= \frac{1}{1+K_{\mathrm{pos}}} \\
	K_{\mathrm{pos}} &= \lim_{s\to 0} H(s) = \frac{9}{3} = 3 \implies \\
	e_{\mathrm{ssp}} &= \frac{1}{1+3} = \frac{1}{4}.
\end{align*}

\paragraph{Παράδειγμα} \hspace{0pt}

\begin{tikzpicture}[scale=1]
\draw[->] (-0.25,0) -- (0.5,0) node[above,midway] {$r(s)$};
\draw (0.75,0) circle (0.25);
\draw (1,0) -- ++(0.5,0);
\draw (1.5,-0.5) rectangle (3,0.5) node[midway] {$H(s)$};
\draw (3,0) -- (3.75,0);
\draw[->] (3.25,0) -- ++(1,0) node[above right] {$y$};

\draw[->]  (3.6,0) -- ++(0,-1) -- (1.5,-1) -- (0.75,-1) -- ++(0,0.75) node[right,xshift=1mm,yshift=-1mm] {$-$};
\end{tikzpicture}
Στο παραπάνω σχήμα, θέτουμε:

\[
H(s) = \frac{2(s+10)}{s(s+2)(s+5)}
\]

Τι σφάλματα έχει το παραπάνω σύστημα;
\subparagraph{}
\begin{alignat*}{2}
K_{\mathrm{pos}} &= \infty \\
K_{\mathrm v}	 &= \lim_{s\to 0} sH(s) &&= \frac{20}{10} = 2 \\
K_{\mathrm a}    &= 0 \\[2ex]
e_{\mathrm{ssp}} &= \frac{1}{1+\infty} &&= 0 \\
e_{\mathrm{ssv}} &= \frac{1}{K_{\mathrm v}} &&= \frac{1}{2} \\
e_{\mathrm{ssa}} &= \frac{1}{0} &&= \infty
\end{alignat*}

\paragraph{Παράδειγμα} \hspace{0pt}

\begin{tikzpicture}[scale=1]
\draw[->] (-0.25,0) -- (0.5,0) node[above,midway] {$r(s)$};
\draw (0.75,0) circle (0.25);
\draw (1,0) -- ++(0.5,0);
\draw (1.5,-0.5) rectangle (3,0.5) node[midway] {$H(s)$};
\draw (3,0) -- (3.75,0);
\draw[->] (3.25,0) -- ++(1,0) node[above right] {$y$};

\draw[->]  (3.6,0) -- ++(0,-1) -- (1.5,-1) -- (0.75,-1) -- ++(0,0.75) node[right,xshift=1mm,yshift=-1mm] {$-$};
\end{tikzpicture}

Στο παραπάνω σχήμα:
\[
H(s) = \frac{1.8κ}{s(s+3.3)}
\]

Ποιά πρέπει να είναι η σταθερά \( κ \) ώστε να ισχύει \( e_{\mathrm{ssv}} = 0.327 \);

\begin{align*}
	K_{\mathrm{v}} &= \lim_{s\to 0}
	sH(s) = \lim_{s \to 0} \frac{1.8κ}{s+3.3} = \frac{18κ}{33} \\
	e_{\mathrm{ssv}} &= \frac{1}{K_{\mathrm v}}
	= \frac{33}{18κ} = 0.327 \implies \\
	κ &= \frac{33}{18\cdot 0.327} \simeq 5.607
\end{align*}

\paragraph{Άσκηση}
Έστω ένα σύστημα με συνάρτηση μεταφοράς:
\[
H_{κ}(s) = \frac{κs+b}{s^2+as+b}
\]

Ποιά είναι τα σφάλματα του συστήματος, και ποιά είναι η συνάρτηση μεταφοράς, αν προέρχεται
από σύστημα κλειστού βρόγχου \textit{μοναδιαίας αρνητικής ανάδρασης};

\subparagraph{Σφάλματα}
\begin{alignat*}{2}
	e_{\mathrm{ssp}} &= \lim_{s\to 0}\left[1-H_κ(s)\right] = 1-\frac{b}{b} &&= 0 \\
	e_{\mathrm{ssv}} &= \lim_{s\to 0}\left[\frac{1-H_κ(s)}{s}\right] &&= \frac{a-κ}{b} \\
	e_{\mathrm{ssa}} &= \lim_{s\to 0}\left[\frac{1-Η_κ(s)}{s^2}\right] &&= \infty
\end{alignat*}
\subparagraph{Συνάρτηση μεταφοράς} \hspace{0pt}

\begin{tikzpicture}[scale=0.8,every node/.style={scale=.8}]
\draw[->] (-0.25,0) -- (0.5,0) node[above,midway] {$r(s)$};
\draw (0.75,0) circle (0.25);
\draw (1,0) -- ++(0.5,0);
\draw (1.5,-0.5) rectangle (3,0.5) node[midway] {$H(s)$};
\draw (3,0) -- (3.75,0);
\draw[->] (3.25,0) -- ++(1,0) node[above right] {$y$};

\draw[->]  (3.6,0) -- ++(0,-1.5) -- (1.5,-1.5) -- (0.75,-1.5) -- ++(0,1.25) node[right,xshift=1mm,yshift=-1mm] {$-$};
\end{tikzpicture}

Γνωρίζουμε ή βρίσκουμε ότι, για συστήματα κλειστού βρόγχου:
\[
H_κ(s) = \frac{H(s)}{1+H(s)}
\]
άρα:
\begin{align*}
	H_κ(1+H) &= H \implies \\
	H_κ + H_κH &= H \implies \\
	H(1-H_κ) &= H_κ \implies \\
	H &= \frac{H_κ}{1-H_κ} \implies \\
	H &= \frac{\frac{κs+b}{s^2+as+b}}{1-\frac{ks+b}{s^2+as+b}}
	= \frac{\frac{κs+b}{\cancel{s^2+as+b}}}{\frac{s^2+as+b-ks-b}{\cancel{s^2+as+b}}}
	\implies \\
	\Aboxed{H(s) &= \frac{κs+b}{s(s+a-κ)}}
\end{align*}

\paragraph{Παράδειγμα} \hspace{0pt}

\begin{tikzpicture}
\draw[->] (-0.1,0) -- (3,0) node[below right] {$t$};
\draw[->] (0,-0.1) -- (0,3);

\draw[dashed] (0,1.4) node[left] {$0.5$} -- ++(3,0);
\draw[dashed] (0,2.8) node[left] {$1$} -- ++(3,0);

\draw[very thick, blue!80!green]
plot[samples=10,variable=\x,domain=0:3,smooth]
({\x},{0.2/(0.01+2*\x)*sin(2*\x r)+1.4*(1-exp(-4*\x))})
node[right] {$y(t)$};

\foreach \i in {1,2,3}
\draw(\i*2.5/3,0.05) -- ++(0,-0.1) node[below] {$\i$};
\end{tikzpicture}

Έστω το σύστημα με την παραπάνω απόκριση στη βηματική συνάρτηση. Ποιός είναι ο τύπος του
συστήματος;

\subparagraph{} \hspace{0pt}

\begin{tikzpicture}
\draw[dashed] (0,1.4) node[left] {$0.5$} -- ++(3,0);
\draw[dashed] (0,2.8) node[left] {$1$} -- ++(3,0);

\draw[very thick, blue!80!green]
plot[samples=10,variable=\x,domain=0:3,smooth]
({\x},{0.2/(0.01+2*\x)*sin(2*\x r)+1.4*(1-exp(-4*\x))})
node[right] {$y(t)$};

\draw[very thick, blue!40!cyan]
(0,2.8) -- ++(3,0) node[right] {$r(s)$};

\draw[thin,<->,blue!40!cyan]
(1.5,2.7) -- ++(0,-1.2) node[midway,right,scale=.8] {$0.5$};

\foreach \i in {1,2,3}
\draw(\i*2.5/3,0.05) -- ++(0,-0.1) node[below] {$\i$};

\draw[->] (-0.1,0) -- (3,0) node[below right] {$t$};
\draw[->] (0,-0.1) -- (0,3);
\end{tikzpicture}

Η επιθυμητή έξοδος του συστήματος είναι 1, αλλά η έξοδός του στη μόνιμη κατάσταση είναι 0.5,
επομένως υπάρχει σταθερό σφάλμα \( 0.5 \). Άρα το σύστημα έχει τύπο \( 0 \).

Γνωρίζουμε ότι, αφού είναι τύπου \( 0 \), θα έχει άπειρο σφάλμα ταχύτητας. Πράγματι, αν
βάλουμε ως είσοδο τη συνάρτηση ράμπας, το σφάλμα όσο \( t \to \infty \) θα αυξάνεται όλο
και περισσότερο:

\begin{tikzpicture}
\draw[dashed] (0,2.6/3) node[left] {$1$} -- ++(3,0);
\draw[dashed] (0,2.6) node[left] {$3$} -- ++(3,0);

\begin{scope}
\pgfmathsetseed{22}
\clip  (0,0) --
plot[samples=10,variable=\x,domain=0.1:3,smooth]
({\x},{0.2*\x+0.1*rand}) -- (3,3) -- (0,0);
\fill[draw=yellow!50!black,opacity=.5,fill=yellow!50!white,postaction={pattern=vertical lines}] (0,0) -- (3,3) -- (3,0) -- cycle;
\end{scope}

\pgfmathsetseed{22}
\draw[very thick, blue!80!green] (0,0) --
plot[samples=10,variable=\x,domain=0.1:3,smooth]
({\x},{0.2*\x+0.1*rand})
node[right] {$y(t)$};

\draw[very thick, blue!40!cyan]
(0,0) -- (3,3) node[right] {$r(s)$};

\foreach \i in {1,2,3}
\draw(\i*2.5/3,0.05) -- ++(0,-0.1) node[below] {$\i$};

\draw[->] (-0.1,0) -- (3,0) node[below right] {$t$};
\draw[->] (0,-0.1) -- (0,3);
\end{tikzpicture}

\paragraph{Άσκηση} \hspace{0pt}

Ποιά είναι τα σφάλματα του παρακάτω συστήματος;

\begin{tikzpicture}[scale=1]
\draw[->] (-0.25,0) -- (0.5,0) node[above,midway] {$r(s)$};
\draw (0.75,0) circle (0.25);
\draw (1,0) -- ++(0.5,0);
\draw (1.5,-0.5) rectangle (3,0.5) node[midway] {$\displaystyle \frac{κ}{Js^2}$};
\draw (3,0) -- (3.75,0);
\draw[->] (3.25,0) -- ++(1,0) node[above right] {$y$};

\draw (3.6,0) -- ++(0,-1.5) -- (3,-1.5);
\draw (1.5,-2) rectangle (3,-1) node[midway] {$1+bs$};
\draw[->] (1.5,-1.5) -- (0.75,-1.5) -- ++(0,1.25) node[right,xshift=1mm,yshift=-1mm] {$-$};
\end{tikzpicture}

Επειδή δεν έχουμε μοναδιαία αρνητική ανάδραση, δεν μπορούμε να εφαρμόσουμε τους τύπους
σφάλματος για συστήματα μοναδιαίας αρνητικής ανάδρασης.

Βρίσκουμε τη συνάρτηση μεταφοράς κλειστού βρόγχου:
\begin{align*}
	y &= H(s)\left(r-G(s)y\right) = \dots = \frac{H(s)}{1+H(s)G(s)}
	= \frac{\sfrac{κ}{Js^2} }{1+bs}
	= \frac{κ}{Js^2 + κbs + κ}
\end{align*}

Επομένως μπορούμε να εφαρμόσουμε τους τύπους σφάλματος χωρίς βρόγχο:
\[
e_{\mathrm{ssp}} = 0 \hspace{100pt}
e_{\mathrm{ssv}} = b \hspace{100pt}
e_{\mathrm{ssa}} = \infty
\]

\paragraph{Άσκηση} \hspace{0pt}

\begin{tikzpicture}[scale=1]
\draw[->] (-0.25,0) -- (0.5,0) node[above,midway] {$r(s)$};
\draw (0.75,0) circle (0.25);
\draw (1,0) -- ++(0.5,0);
\draw (1.5,-0.5) rectangle (3,0.5) node[midway] {$H_1(s)$};
\draw[->] (3,0) -- (3.75,0);
\draw (4,0) circle (0.25);
\draw (5,-0.5) rectangle (6.5,0.5) node[midway] {$H_2(s)$};
\draw[->] (4.25,0) -- (5,0);
\draw[->] (6.5,0) -- ++(1,0) node[above right] {$y$};

\draw[<-] (4,0.25)  node[right,xshift=1mm,yshift=1mm] {$+$} --++ (0,0.5)
node[above] {$d(s)$};

\draw[->] (7,0) -- ++(0,-1.5) -- (0.75,-1.5) -- ++(0,1.25)
node[left,xshift=-1mm,yshift=-1mm] {$-$};
\end{tikzpicture}

\begin{enumgreekparen}
	\item
	\( \displaystyle
	H_1(s) = \frac{s+2}{s+1} \qquad
	H_2(s) = \frac{10}{s(s+6)}
	 \)
	\item
	\( \displaystyle
	H_1(s) = \frac{s+2}{s(s+1)} \qquad
	H_2(s) = \frac{10}{s+6}
	 \)
\end{enumgreekparen}
Να βρεθούν και στις δύο περιπτώσεις τα σφάλματα θέσης και ταχύτητας με διάφορες
περιπτώσεις διαταραχών.

\subparagraph{Λύση, αν δεν υπάρχει διαταραχή d(s)}
\begin{enumgreekparen}
\item Έχουμε μοναδιαία αρνητική ανάδραση και το σύστημα έχει έναν ολοκληρωτή, άρα είναι τύπου 1.
Το σφάλμα θέσης είναι \( e_{\mathrm{ssp}} = 0 \), και το σφάλμα ταχύτητας μία σταθερά
\(
\displaystyle e_{\mathrm{ssv}}=\frac{1}{K_{\mathrm v}} = \frac{1}{\lim_{s\to 0} sH(s)}
= \frac{1}{\lim_{s\to 0}s\frac{s+2}{s+1}\frac{10}{s(s+6)}}
=\frac{1}{\sfrac{20}{6} }=\frac{3}{10}
 \).
 
\item
Αντίστοιχα με το προηγούμενο ερώτημα, έχουμε:
\begin{align*}
K_{\mathrm{pos}} &=
\lim_{s\to 0} H(s) = \lim_{s\to 0}
\frac{s+2}{s(s+1)}\frac{10}{s+6} = \infty \\
K_{\mathrm{v}} &=
\lim_{s\to 0}sH(s) = \lim_{s\to 0}
\frac{s+2}{s+1}\frac{10}{s+6} = 2\cdot\frac{10}{6} = \frac{10}{3}
\intertext{Άρα:}
e_{\mathrm{ssp}} &= \frac{1}{1+K_{\mathrm{pos}}} = 0 \\
e_{\mathrm{ssv}} &= \frac{1}{K_{\mathrm v}} = \frac{3}{10}
\end{align*}
\end{enumgreekparen}


\subparagraph{Λύση με διαταραχή \( d(s) = \frac{A}{s} \)}
Υπολογίζουμε:
\begin{align*}
	y &= H_2\left( d+H_1(r-y) \right) \implies \dots \\
	\implies y &= \underbrace{\frac{H_1H_2}{1+H_1H_2}}_{y_1} r +
	\underbrace{\frac{H_2}{1+H_1H_2}d}_{y_2}
\end{align*}

Φαίνεται η επαλληλία στη απόκριση του συστήματος, την οποία θα μπορούσαμε να εκμεταλλευτούμε.

Θυμόμαστε ότι:
\[
e(s) = r(s)-y(s) \qquad \text{και} \qquad
e_{\mathrm{ss}} = \lim_{s\to 0} se(s)
\]

Για την εύρεση του αποτελέσματος, θα χρησιμοποιήσουμε τον ορισμό του σφάλματος.

\subparagraph{(α) σφάλμα θέσης}
Για το σφάλμα θέσης, θεωρούμε \( r(s) = \frac{1}{s} \),  και έχουμε:
\begin{align*}
	e_{\mathrm{ssp}} &= \lim_{s\to 0}s\left[
	\frac{1}{s} - \frac{H_1H_2}{1+H_1H_2} \frac{1}{s}
	- \frac{H_2}{1+H_1H_2}\frac{A}{s}
	\right]
	\\ &=
	\lim_{s\to 0} \left[
	1 - \frac{\frac{s+2}{s+1}\frac{10}{s(s+6)}}{1+\frac{s+2}{s+1}\frac{10}{s(s+6)}}
	- \frac{\frac{10}{s(s+6)}}{1+\frac{s+2}{s+1}\frac{10}{s(s+6)}}
	\right]
	\\ &= \lim_{s\to 0} \left[
	1-\frac{\frac{10(s+2)}{\cancel{s(s+1)(s+6)}}}{\frac{s(s+1)(s+6)+10(s+2)}{\cancel{s(s+1)(s+6)}}} - A
	\frac{\frac{10}{\cancel{s(s+6)}}}{\frac{s(s+1)(s+6)+10(s+2)}{\cancel{s(s+6)}(s+1)}}
	\right]
	\\ &= \lim_{s\to 0} \left[
	1- \frac{10(s+2)}{s(s+1)(s+6)+10(s+2)}
	-A\frac{10(s+1)}{s(s+1)(s+6)+10(s+2)}
	\right]
	\\ &= 1 - \frac{20}{20} - A\cdot\frac{10}{20}
	\\ &= -\frac{A}{2}.
\end{align*}
\subparagraph{(β) σφάλμα θέσης}
Αντίστοιχα με παραπάνω, βρίσκουμε:
\begin{align*}
    e_{\mathrm{ssp}} &= \lim_{s\to 0}s\left[
    \frac{1}{s} - \frac{H_1H_2}{1+H_1H_2} \frac{1}{s}
    - \frac{H_2}{1+H_1H_2}\frac{A}{s}
    \right]
    \\ &=
    \lim_{s\to 0} \left[
    1 - \frac{\frac{s+2}{s(s+1)}\frac{10}{s+6}}{1+\frac{s+2}{s(s+1)}\frac{10}{s+6}}
    - A\frac{\frac{10}{s+6}}{1+\frac{s+2}{s(s+1)}\frac{10}{s+6}}
    \right]
    \\ &=
    \lim_{s\to 0} \left[
    1-
    \frac{\frac{10(s+2)}{\cancel{s(s+1)(s+6)}}}{\frac{s(s+1)(s+6)+10(s+2)}{\cancel{s(s+1)(s+6)}}}
    - A\frac{\frac{10}{\cancel{s+6}}}{
    	\frac{s(s+1)(s+6)+10(s+2)}{s(s+1)\cancel{(s+6)}}
    	}
    \right]
    \\ &=
    \lim_{s\to 0} \left[
    1 - \frac{10(s+2)}{s(s+1)(s+6)+10(s+2)}
    - A \frac{10s(s+1)}{s(s+1)(s+6)+10(s+2)}
    \right]
    \\ &= 1 - \frac{20}{20} - A\cdot \frac{0}{20}
    \\ &= 0.
\end{align*}

Παρατηρούμε ότι η αλλαγή της θέσης του ολοκληρωτή \( \displaystyle \frac{1}{s} \) επηρεάζει
και το σφάλμα του συστήματος.

\subsubsection{Ασκήσεις (Παπαγεωργίου)}
\paragraph{Αντιστοιχίες ηλεκτρικού συστήματος σε μηχανικό σύστημα}

\[
\begin{array}{rcl}
	\text{τάση } V & \rightarrow & \text{δύναμη } F \\
	\text{ρεύμα } I & \rightarrow & \text{ταχύτητα } u \\
	\text{αντίσταση } R=\frac{V}{I} & \rightarrow & \text{αποσβεστήρας } B = \frac{F}{u}
	\quad
	\begin{circuitikz}[scale=0.7]
	\ctikzset{bipoles/length=.9cm}
	\draw (0,0) to[damper] (1,0);
	\end{circuitikz} \\
	\text{πυκνωτής } i_c=C\od{u_c}{t} & \rightarrow & \text{ελατήριο } \frac{1}{κ} \od{F}{t}
	\rightsquigarrow F=kx \\
	\text{πηνίο } u_L = L\od{i_L}{t} & \rightarrow & \text{αδράνεια } F = ma = m\od{u}{t}
\end{array}
\]

Ένα παράδειγμα μηχανικού συστήματος είναι τα \textbf{αμορτισέρ}, τα οποία ουσιαστικά
"κόβουν" τις υψηλές συχνότητες που μπορεί να οφείλονται σε ανομοιομορφίες του δρόμου, ώστε
να νιώθουμε άνετα μέσα σε ένα αυτοκίνητο.

\paragraph{Άσκηση 3.8} \hspace{0pt}

\begin{infobox}{}
	Οι αριθμοί των ασκήσεων συνήθως δίνονται από το βιβλίο του \textbf{Πετρίδη}.
\end{infobox}

\begin{circuitikz}
	\draw (2.5+0.2,-0.7) circle (0.2);
	\draw (3.5-0.2,-0.7) circle (0.2);
	
	\draw (5+0.2,-0.7) circle (0.2);
	\draw (6-0.2,-0.7) circle (0.2);
	
	\draw (1,-0.9) -- (8,-0.9);
	\fill[pattern=north east lines] (1,-0.9) rectangle (8,-1.5);
	
	\draw (8,-0.9) -- (8,1);
	\fill[pattern=north east lines] (8,-1.5) rectangle (9,1);
	
	\draw (0,0)
	to[spring={$K_1$}] (2,0)
	to[short] (2.5,0);
	;
	\draw (0,0.5) -- ++(0,-1);
	\draw (2.5,-0.5) rectangle ++(1,1) node[midway] {$m_1$};
	\draw (3.5,0.35) to[spring={$K_2$}] (5,0.35);
	\draw (3.5,-0.35) to[damper] (5,-0.35);
	\draw (5,-0.5) rectangle ++(1,1) node[midway] {$m_2$};
	\draw (6,0) to[damper={$B_2$}] (8,0);
	
	%\draw (3.5/2+5/2) node [label={[inner sep=1pt, fill=white,text=black, fill opacity=0.75, text opacity=1]above left:$(65, 35)$}] {};
	\draw ({(3.5+5)/2},-0.75) node[inner sep=1pt, fill=white,text=black, fill opacity=0.75, text opacity=1,below] {$B_1$};
	
	\draw[thick,orange!50!green,->] (-0.7,0) -- (0,0) node[midway,above] {$F$};
	\draw[thick,green!80!blue,->] (0,0.8) -- ++(0,-0.2) ++(0,0.1) -- ++(0.7,0)
	node[above,midway] {$I_0$};
	\draw[thick,green!80!blue,->] (2.5,0.8) -- ++(0,-0.2) ++(0,0.1) -- ++(0.7,0)
	node[above,midway] {$l_1$};
	\draw[thick,green!80!blue,->] (5,0.8) -- ++(0,-0.2) ++(0,0.1) -- ++(0.7,0)
	node[above,midway] {$l_2$};
\end{circuitikz}

Να υπολογιστεί η συνάρτηση μεταφοράς του παραπάνω συστήματος.
Σαν έξοδος θεωρείται η μετατόπιση της \( m_1 \).

\subparagraph{Λύση}
Βρίσκουμε το άθροισμα των δυνάμεων που ασκούνται σε κάθε μάζα, και εφαρμόζουμε το
νόμο του Νεύτωνα:
\begin{align*}
	\text{Για τη μάζα 2: } F = m_2\od[2]{l_2}{t}
	&=
	\underbrace{-K_2(l_2-l_1)}_{\text{ελατήριο}}
	-\underbrace{B_2\od{l_2}{t}}_{\text{αποσβ.}}
	-\underbrace{B_1\left(
	\od{l_2}{t}-\od{l_1}{t}
	\right)}_{\text{αποσβεστήρας}}
	\\
	\text{Για τη μάζα 1: } F = m_1\od[2]{l_1}{t}
	&=
	-K_1(l_1-l_0)-B_1\left(\od{l_1}{t}-\od{l_2}{t}\right)
	-K_2(l_1-l_2)
\end{align*}

Μετασχηματίζουμε τις δύο εξισώσεις κατά Laplace:
\begin{align}
	m_2s^2L_2 &= -K_2(L_2-L_1) - B_2sL_2 - B_1(sL_2-sL_1) \label{3.8.lap2} \\
	m_1s^2L_1 &= -K_1(L_1-L_0) - B_1(sL_1-sL_2) - K_2(L_1-L_2) \label{3.8.lap1}
\end{align}
Θυμόμαστε ότι δεν λαμβάνουμε υπ' όψιν τις αρχικές συνθήκες, αφού αναζητούμε τη συνάρτηση
μεταφοράς.

Αναζητούμε μια σχέση της μορφής \( H(s) = \frac{L_0(s)}{L_1(s)} \), άρα πρέπει να απαλείψουμε
το \( L_2 \) από τις δύο παραπάνω σχέσεις:
\begin{align*}
	\eqref{3.8.lap2} \implies L_2 &=
	L_1 \frac{K_2+B_1s}{m_2s^2+K_2+(B_1+B_2)s} \xRightarrow{\eqref{3.8.lap1}} \\
	\frac{L_1}{L_0}
	&= \frac{K_1m_2s^2+K_1K_2+K_1(B_1+B_2)s}{
		(m_1s+K_1+K_2+B_1s)\left(m_2s^2+K_2+(B_1+B_2)s\right)-(K_2+B_1s)^2}
\end{align*}

\paragraph{Άσκηση με brushed DC κινητήρα} \hspace{0pt}

Έστω ένας brushed DC κινητήρας:

\begin{tikzpicture}[scale=1.1]
\draw (0,0.7) -- (2,0.7);
\draw (0,0) -- (2,0);
\draw (2,0.35) ellipse (0.2 and 0.35);

\draw (2,0.35) ellipse (0.2/2 and 0.35/2);
\draw (2,0.35+0.35/2) -- ++ (-0.17,0);
\draw (2,0.35-0.35/2) -- ++ (-0.17,0);

\draw (0,0.7) arc (120:242:0.4);

\draw[fill=gray!50!white]
(2,0.35+0.07) to[bend right] (2,0.35-0.07)
-- ++(1.4,0)
-- ++(0,2*0.07)
-- cycle;
%(2,0.35+0.07) arc (120:242:0.082)

\begin{scope}[xshift=3.5cm]
%,fill=gray!10!white,path fading=north
\draw[top color=gray!5!white,bottom color=gray!10!white!90!blue] (0,0.7) arc (120:242:0.4) -- (2,0) -- (2,0.7)
-- (0,0.7);
\draw (0,0) -- (2,0);
\draw[fill=white] (2,0.35) ellipse (0.2 and 0.35);
\draw (0,0.7) arc (120:242:0.4);

\draw[->,black!80!gray] (2.3,0.9) to [bend left] node[midway,right] {$θ$} (2.3,-0.2);
\end{scope}

\draw[->,gray!30!black] (1,1.5) node[right] {στάτης}  to[bend left=5] (0.7,0.7);
\draw[->,gray!30!black] (3.5,1.5) node[right] {δρομέας}  to[bend right=10] (3,0.45);
\draw[->,gray!70!black!80!green] (2.4,1) node[above right] {$J_m$}  to[bend right=10] (2,0.45);
\end{tikzpicture}

με ισοδύναμο κύκλωμα δρομέα:

\begin{circuitikz}[american]
\draw
(0,0) to[V=$V_t$] (0,-2)
;
\draw
(0,0) to [R=$R_A$,i^={$i_A$}]
(2,0) to[L=$L_A$]
(4,0) to[V_=$V$,label={\small ηλεκτρεγερτική δύναμη λόγω περιστροφής}]
(4,-2) to[short] (0,-2)
;
\end{circuitikz}

Το ισοδύναμο κύκλωμα περιγράφεται από την εξίσωση:
\begin{equation}
V_t = R_Ai_A +L_a\od{i_A}{t}
+ K_v \od{\theta}{t} \label{ex.brusheddc.elec}
\end{equation}
και το μηχανικό ισοδύναμο:
\begin{equation}
\underbrace{(J_L+J_m)}_{\mathrm{\frac{Nms^2}{rad}}}
\underbrace{\od[2]{θ}{t}}_{\mathrm{\frac{rad}{s^2}}}
= \underbrace{K_T}_{\mathrm{\frac{Nm}{A}}} i_A \label{ex.brusheddc.mech}
\end{equation}
όπου \( J_L \) και \( J_m \) είναι οι ροπές αδράνειας του κάθε μέρους.

Στη συγκεκριμένη άσκηση θεωρούμε σαν έξοδο τη \textit{γωνία του κινητήρα}, και σαν είσοδο
την \textit{τάση του δρομέα} \( V_t \).

Λύνουμε το σύστημα:
\begin{align*}
	\eqref{ex.brusheddc.mech} \implies
	i_A &= \frac{(J_L+J_m)}{K_T} \od[2]{θ}{t}
	\implies \\
	I_A &= \frac{J_L+J_m}{K_T}s^2θ
	\xRightarrow{\eqref{ex.brusheddc.elec}}
	\\
	V_T &= \left[(R_A+\cancelto{0\mathclap{\raisebox{3ex}{\text{\tiny(συνήθως αγνοούμε το $L_A$)}}}}{L_A}s)
	\frac{J}{K_T}s^2+K_vs
	\right]θ \implies \\
	\frac{θ}{V} &= \frac{\sfrac{1}{K_v} }{\frac{R_AJ}{K_TK_v}s^2+s}
\end{align*}

Βέβαια σε πολλές περιπτώσεις μπορεί να τοποθετηθεί ένας \textbf{\textit{μειωτήρας}}
μεταξύ του \( J_m \) και του \( J_L \):

\begin{tikzpicture}
\draw (0,0.7) -- (2,0.7);
\draw (0,0) -- (2,0);
\draw (2,0.35) ellipse (0.2 and 0.35);

\draw (2,0.35) ellipse (0.2/2 and 0.35/2);
\draw (2,0.35+0.35/2) -- ++ (-0.17,0);
\draw (2,0.35-0.35/2) -- ++ (-0.17,0);

\draw (0,0.7) arc (120:242:0.4);

\draw[fill=gray!50!white]
(2,0.35+0.07) to[bend right] (2,0.35-0.07)
-- ++(1.4,0)
-- ++(0,2*0.07)
-- cycle;
%(2,0.35+0.07) arc (120:242:0.082)

\begin{scope}[xshift=3.5cm]
%,fill=gray!10!white,path fading=north
\draw[top color=gray!5!white,bottom color=gray!10!white!90!blue] (0,0.7) arc (120:242:0.4) -- (2,0) -- (2,0.7)
-- (0,0.7);
\draw (0,0) -- (2,0);
\draw[fill=white] (2,0.35) ellipse (0.2 and 0.35);
\draw (0,0.7) arc (120:242:0.4);

\draw[->,black!80!gray] (2.3,0.9) to [bend left] node[midway,right] {$θ$} (2.3,-0.2);
\end{scope}

\draw[fill=white] (2.45,0.35-0.25) rectangle ++(0.5,0.5);


\draw[->,gray!70!black!80!green] (2.4,1) node[above] {$J_m$}  to[bend right=10] (2,0.45);
\draw[->,gray!70!black!80!green] (3.2,0.9) node[above] {$J_L$}  to[bend right=10] (3.3,0.45);
\end{tikzpicture}

Ο μειωτήρας είναι το μηχανικό ισοδύναμο ενός ενισχυτή, που κατασκευάζεται ουσιαστικά με δύο
γρανάζια, ώστε να αυξηθεί η δύναμη μειώνοντας την ταχύτητα ή αντίστροφα:

\begin{tikzpicture}[scale=0.7]
% Source: https://tex.stackexchange.com/a/58735

% #1 number of teeths
% #2 radius intern
% #3 radius extern
% #4 angle from start to end of the first arc
% #5 angle to decale thesecond arc from the first
\newcommand{\gear}[5]{%
	\foreach \i in {1,...,#1} {%
		[rotate=(\i-1)*360/#1] (0:#2) arc (0:#4:#2) {[rounded corners=1.5pt]
			-- (#4+#5:#3) arc (#4+#5:360/#1-#5:#3)} -- (360/#1:#2)
	}}
	
	\draw[thick]
	\foreach \i in {1,2,...,10} {%
		[rotate=(\i-1)*36]
		(0:0.7) arc (0:18:0.7) {[rounded corners=2pt] -- ++(18: 0.3) arc (18:36:0.7+0.3) } -- ++(36: -0.3)
	};
	\draw[thick,xshift=3cm] \gear{22}{2}{2.3}{7}{2.6};
	
	\draw (-1,-1.3) node {$u_1F_1$};
	\draw (5.7,-1.3) node {$u_2F_2$};
\end{tikzpicture}


Τότε θα έχουμε:
\[
J = J_m + \left(
\frac{N_1}{N_2}
\right)^2 J_L
\]

Βέβαια ένα πραγματικό σύστημα θα περιλαμβάνει και δυναμική τριβή με τη μορφή αποσβεστήρα.

\paragraph{Παράδειγμα 3.5.2} \hspace{0pt}

\begin{circuitikz}
	\draw
	(0,0) to [R={$R_{\gamma f}$},o-] (2,0)
	to [L={$L_{\gamma f}$}] (2,-2)
	to [short,-o,i^={$i_{\gamma f}$}] (0,-2)
	to [open,v={$e_{\gamma f}$}] (0,0)
	;
\end{circuitikz}

\begin{circuitikz}[scale=1.4]
	\ctikzset{bipoles/length=1.2cm}
	
	\draw (0,1) node[elmech](motor1){};
	\draw (4,1) node[elmech](motor2){};
	
	\draw
	(motor1.north)
	-- (0,2)
	to[R] (1,2)
	to[L] (2,2)
	to[R] (3,2)
	to[L] (4,2)
	-- (motor2.north)
	;
	
	\draw
	(motor2.south)
	-- (4,0)
	to[short,i=$i_a$] (0,0)
	-- (motor1.south)
	;
	
	\draw
	([xshift=4mm]motor1.south)
	to[open,v=$e_\gamma$] ([xshift=4mm]motor1.north);
	
	\draw
	(6,0) to[L,l_=$L_F$]
	(6,2) to[R=$R_F$]
	(8,2) to[battery1=$V_F$]
	(8,0) -- (6,0);
	
	\draw[<-] (motor1) -- ++(-1,-1) node[below left]
	{
		$\underset{\mathclap{\substack{\downarrow\\ \text{σταθερό}}}}{n}
		=\od{\theta}{t}$
	};
	
	\def\l{0.4}
	\draw[xshift=5cm,rotate=45] (0,0) ellipse (0.22 and 0.14);
	\draw (5-0.17,0-0.14) -- ++(\l,-\l) node(C) {};
	\draw (5+0.17,0+0.14) -- ++(\l,-\l) node(D) {};
	\draw (C.center) to[bend right=65] node[midway,below] {$J,B,T_L$} (D.center);
	
	\def\d{0.04}
	\path (4,1) ++(\d,\d) -- ++(1,-1) node[] (A) {};
	\path (4,1) ++(-\d,-\d) -- ++(1,-1) node[] (B) {};
	
	\fill[white] (A.center) to[bend left] (B.center) -- ++(90+45:0.2) -- ++(2*\d,2*\d) -- cycle;
	
	\draw[<-] (A) ++ (45+70:0.6) node[above ] {$\omega$} to[bend left=60] ($(B) + (45+110:0.6)$);
	
	\draw (4,1) ++(\d,\d) -- ++(1,-1);
	\draw (4,1) ++(-\d,-\d) -- ++(1,-1);
	\draw (A.center) to[bend left] (B.center);
	
	\draw[->] (-0.5,0.5) ++ (45+90:0.2) to[bend left=60] ($(-0.5,0.5) + (45-90:0.2)$);
\end{circuitikz}

\begin{align*}
	e_{\gamma f} &=
	R_{\gamma f} \cdot i_{\gamma f}
	+ L_{γf} \od{i_{γf}}{t} \\
	e_γ &= K\cdot n \cdot i_{γf}\quad
	\left(
	i_{γf} = \frac{e_γ}{γ}
	\right) \\
	E_{γf} &= \left(
	\frac{R_{γf}}{γ}
	+ \frac{L_{γf}}{γ}s
	\right)E_γ \\
	J\od[2]{θ}{t} + B\od{θ}{t} + T_L &= K_m i_a
	\rightarrow (Js^2+Bs)Θ + T_L = K_mI_a \\
	L\od{i_a}{t} + Ri_a + K\od{θ}{t} &= e_γ
	\rightarrow L_sI_a + R\cdot I_A + R\cdot I_A + K_S Θ = E_γ \\
	(Js^2+Bs)Θ + T_L &= K_m \left(
	\frac{E_γ - K_sΘ}{Ls + R}
	\right) \\[2ex]
	\left( Js^2+Bs+\frac{K_mK_s}{Ls+R} \right)Θ
	+T_L &= \frac{K_m}{Ls+R}\left(
	\frac{1}{\frac{R_γf}{γ}}
	+\frac{L_γ f}{γ}s
	\right)E_{γf} \\
	\left(
	Js+B+\frac{K_mK}{Ls+R}
	\right)
	ω + T_L &= \frac{K_m}{Ls+R} \left(
	\frac{1}{\frac{R_γ f}{γ} + \frac{L_γ f}{γ}s}
	\right)E_{γf}
\end{align*}

Για \( T_L = 0 \):
\[
\frac{Ω}{E_{γf}} =
\frac{γK_m}{\left(
	(Js+B)(Ls+R)+KK_m
	\right)(Lγ_f s + R_{γf})}
\]

\paragraph{Τελεστικός ενισχυτής} \hspace{0pt}

\begin{center}
	\begin{circuitikz}[scale=2.5] \draw
		(0,0) node[op amp] (opamp) {}
		(opamp.+) to[short,-o] ++(-0.5,0)
		(opamp.-) to[short,-o] ++(-0.5,0)
		(opamp.out) to[short,-o] ++(0.5,0)
	;\end{circuitikz}
\end{center}

Ο τελεστικός ενισχυτής είναι ιδανικά ένας διαφορικός ενισχυτής άπειρου κέρδους.
Αν οι δύο είσοδοι είναι 0, τότε η έξοδος είναι 0, ενώ αν διαφέρουν, η είσοδος είναι άπειρη
(ή \( V_{cc} \) σε πραγματικό ενισχυτή).

Για τον ιδανικό τελεστικό ενισχυτή κάνουμε τις παραδοχές:
\begin{itemize}
	\item \( R_m = \infty \)
	\item \( V_{+} = V_{-} \) (για να έχουμε πεπερασμένη έξοδο)
\end{itemize}

\paragraph{Άσκηση Α35} \hspace{0pt}

\begin{circuitikz}[scale=1,american]
	\draw (0,0) node[op amp] (opamp) {};
	
	\draw (opamp.+) to[short,-*] ++(-0.5,0) node[below right] {$B$}
	to[short] ++(0,-1.2) node[ground] {};
	
	\draw (opamp.-) to[short,-*] ++(-0.5,0) node[above right] {$A$}
	to[short] ++(0,1.5)
	to[C] ++(3,0)
	to[short,-*] (1.3,0) node[above right] {$\Gamma$};
	
	\draw (opamp.-) ++ (-0.5,0) to[short] ++(0,2.5)
	to[R,i=$i_R$] ++(3,0) -- ++(0,-1);
	
	\draw (opamp.out) to[short,-o] ++ (3,0)
	to[open,v=$e_0$] ++(0,-1.5) to[short,o-] ++(0,-0.2) node[ground] {};
	
	\draw (opamp.-) to[R=$R$,i<=$I_i$,-o] ++(-4,0)
	to[open,v=$e_i$] ++(0,-2) to[short,o-] ++(0,-0.2) node[ground] {};
\end{circuitikz}

Μας ζητείται να βρούμε τον λόγο \( \displaystyle \frac{E_o}{E_i} \).

Χρησιμοποιούμε τη δεύτερη παραδοχή του ιδανικού τελεστικού, και παρατηρούμε ότι:
\[
V_B = V_A = 0
\]
άρα:
\begin{align*}
	V_{AΓ} + V_{ΓB} &= 0 \\
	\implies V_C + e_o &= 0 \\
	\implies V_C &= - e_0
\end{align*}

Υπολογίζουμε τα ρεύματα μέσω κυκλωματικών νόμων, και λαμβάνοντας υπ' όψιν ότι (λόγω της
άπειρης αντίστασης) δεν πηγαίνει ρεύμα στις εισόδους A και B του ενισχυτή:
\begin{align}
	I_i &= i_R + i_2 \label{ex.A35.cur} \\
	I_i &=\frac{e_i}{R} \label{ex.A35.ohm}
\end{align}
άρα:
\begin{align*}
	\eqref{ex.A35.cur} + \eqref{ex.A35.ohm}
	\implies \Aboxed{\frac{e_i}{R}&= -c\od{e_o}{t}-\frac{e_o}{R}}
	\intertext{και}
	i_R &= \frac{V_C}{R} = -\frac{e_o}{R}
\end{align*}

Μετασχηματίζοντας την παραπάνω εξίσωση κατά Laplace και λύνοντας, έχουμε:
\[
\boxed{\frac{E_o}{E_i} = \frac{1}{RCs+1}}
\]

\paragraph{Άσκηση 2.19} \hspace{0pt}

\begin{circuitikz}[scale=1]
	\draw (0,0) node[mixer] (m1) {};
	\draw (7,0) node[mixer] (m2) {};
	\draw (12,0) node[mixer] (m3) {};
	
	\draw (m1.west) node[inputarrow] {} node[above left] {$+$} -- ++(-1.5,0) node[above,pos=.7] {$u$};
	\draw (m1.east) to[twoport,t=$\frac{1}{s+6}$,i>_=$e$] (4,0)
	to[twoport,t=$\frac{s+3}{s+7}$,>] (5.5,0) 
	to[short,i^>=$w_1$] (m2.west) node[above,xshift=-1mm] {$+$};
	\draw (3.5,0) node[above left] {$A$} to[short,*-] ++(0,1.5) node(V) {};
	\draw (m2.north) node[inputarrow,rotate=-90] {} node[above right] {$+$} -- (m2.north |- 0,1.5) node[circ] {} -- (V.center);
	\draw (m2.east) to[twoport,i>^=$w$,t=$\frac{2}{s+5}$,-*] (10.5,0) node (B) {} node[above] {$B$};
	\draw (B.center) -- ++(0,-1.5) to[twoport,t=$\frac{1}{s}$] (m2 |- 0,-1.5)
	to[short] (m2.south) node[inputarrow,rotate=90] {} node[below right] {$-$};
	\draw (B.center) -- (m3.west) node[inputarrow] {};
	\draw (V) -- (m3 |- 0,1.5) -- (m3.north) node[inputarrow,rotate=-90] {} node[above right] {$+$};
	\draw (m3.east) -- ++(1,0) node[inputarrow] {} node[midway,above] {$y$};
	
	\draw (m3.east) ++(0.5,0) to[short,*-] ++(0,-3)
	to[twoport,t=$8$] (m1 |- 0,-3)
	-- (m1.south) node[inputarrow,rotate=90] {} node[below right] {$-$};
\end{circuitikz}

Θα μετασχηματίσουμε το παραπάνω σύστημα σε ένα ισοδύναμό του.

Αρχικά, θυμόμαστε την ισοδυναμία:

\begin{circuitikz}[scale=1]
	\draw (0,0) node[mixer] (m1) {};
	
	\draw (m1.west) node[inputarrow] {} node[above left] {$+$} -- ++(-1,0);
	\draw (m1.east) to[twoport,t=$H_1$] (3,0) to[short,*-] (4,0) node[inputarrow] {};
	\draw (3,0) -- (3,-2) to[twoport,t=$H_2$] (m1.east |- 0,-2) -- (m1 |- 0,-2) -- (m1.south)
	node[inputarrow,rotate=90] {} node[below right] {$\pm$};
	
	\draw (current bounding box.east) node[right,scale=2] {$\equiv$};
	\begin{scope}[xshift=5.2cm,yshift=-1cm]
		\ctikzset{bipoles/length=2.4cm}
		\draw (0,0) to[twoport,t=$\frac{H_1}{1\mp H_1H_2}$,>] (5,0) node[inputarrow] {};
	\end{scope}
\end{circuitikz}

Μετασχηματίζουμε το κομμάτι από \( w_1 \) ως \( B \):

\begin{circuitikz}[scale=1]
	\draw (0,0) node[mixer] (m1) {};
	
	\draw (m1.west) node[inputarrow] {} node[above left] {$+$} -- ++(-1,0) node[pos=.8,above] {$w_1$};
	\draw (m1.east) to[twoport,t=$\frac{2}{s+5}$] (3,0) node[above] {$B$}  to[short,*-] (4,0) node[inputarrow] {};
	\draw (3,0) -- (3,-2) to[twoport,t=$\frac{1}{s}$] (m1.east |- 0,-2) -- (m1 |- 0,-2) -- (m1.south)
	node[inputarrow,rotate=90] {} node[below right] {$-$};
	\draw (m1.north) node[inputarrow,rotate=-90] {} -- ++(0,1);
	
	\draw (current bounding box.east |- 0,-1) node[right,scale=2] {$\equiv$};
	\begin{scope}[xshift=7cm,yshift=-1cm]
		\draw (0,0) node[mixer] (m2) {};
		\draw (m2.west) node[inputarrow] {} node[above left] {$+$} -- ++(-1,0) node[pos=.8,above] {$w_1$};
		\draw (m2.north) node[inputarrow,rotate=-90] {} -- ++(0,1);
		\ctikzset{bipoles/length=2.4cm}
		\draw (m2.east) to[twoport,t=$\frac{2s}{s(s+5)+2}$,>] (4,0) node[inputarrow] {};
	\end{scope}
\end{circuitikz}

και το κομμάτι από \( A \) ως \( w_1 \) (αφού κάνουμε τις πράξεις και βρούμε \( \frac{w_1}{A} = 1+\frac{s+3}{s+7} \)):

\begin{circuitikz}[scale=1]
	\draw (0,0) node[mixer] (m1) {};
	
	\draw (m1.west) node[inputarrow] {} node[above left] {$+$};
	\draw (-4,0) to[twoport,t=$\frac{s+3}{s+7}$,>] (m1.west);
	\draw (-3.5,0) node[above left] {$A$} to[short,*-] ++(0,1.2) -- (m1 |- 0,1.2) -- (m1.north)
	node[inputarrow,rotate=-90] {} node[above right] {$+$};
	\draw (m1.east) -- ++(1,0) node[inputarrow] {} node[above] {$w_1$};
	
	\draw (2,0) node[right,scale=2] {$\equiv$};
	\begin{scope}[xshift=3.5cm]
		\ctikzset{bipoles/length=1.8cm}
		\draw (0,0) to[twoport,t=$\frac{2s+10}{s+7}$,>] (4,0) node[inputarrow] {};
	\end{scope}
\end{circuitikz}

Τα δύο παραπάνω μετασχηματισμένα διαγράμματα είναι συνδεδεμένα εν σειρά, άρα:

\begin{circuitikz}[scale=1]
	\draw (0,0) node[above] {$A$}
	to[twoport,t=$\frac{2s+10}{s+7}$,>] (2,0);
	\ctikzset{bipoles/length=2.2cm}
	\draw (2,0)
	to[twoport,t=$\frac{2s}{s(s+5)+2}$,>] (4,0);
	\ctikzset{bipoles/length=1.4cm}
	\draw (4,0) -- ++(0.7,0)
	node[above] {$B$} node[inputarrow] {};
	
	\draw (5,0) node[right,scale=2] {$\equiv$};
	\begin{scope}[xshift=6cm]
		\ctikzset{bipoles/length=3.4cm}
		\draw (0,0) to[twoport,t=$\frac{(2s+10)2s}{(s+7)\left(s(s+5)+2\right)}$,>] (5,0) node[inputarrow] {};
	\end{scope}
\end{circuitikz}

(όπου θέτουμε για ευκολία \( G(s) = \frac{(2s+10)2s}{(s+7)\left(s(s+5)+2\right)} \))
και το τμήμα του συστήματος εντός του βρόγχου ανάδρασης με το
\begin{circuitikz}[baseline,scale=.7]
	\ctikzset{bipoles/length=0.8cm}
	\draw (0,0) to[twoport,t=$8$,scale=.7] (2,0);
\end{circuitikz}
γίνεται:

\begin{circuitikz}[scale=1]
	\draw (0,0) node[mixer] (m1) {};
	
	\draw (m1.west) node[inputarrow] {} node[above left] {$+$};
	\draw (-3.5,0) to[twoport,t=$G(s)$,>] (m1.west);
	\draw (-3.5,0) -- (-4,0);
	\draw (-3.5,0) node[above left] {$A$} to[short,*-] ++(0,-1.2) -- (m1 |- 0,-1.2) -- (m1.south)
	node[inputarrow,rotate=90] {} node[below right] {$+$};
	\draw (m1.east) -- ++(1,0) node[inputarrow] {} node[above] {$w_1$};
	
	\draw (2,-0.5) node[right,scale=2] {$\equiv$};
	\begin{scope}[xshift=3.5cm,yshift=-0.5cm]
		\ctikzset{bipoles/length=2.2cm}
		\draw (0,0) to[twoport,t=$1+G(s)$,>] (4,0) node[inputarrow] {};
	\end{scope}
\end{circuitikz}

με \( 1+G(s) = \frac{\text{αριθμητής }N_3}{\text{παρονομαστής }D_3}
= \frac{
	(2s+10)2s+(s+7)\left[s(s+5)+2\right]}{
	(s+7)\left[s(s+5)+2\right]} \).

Άρα το τελικό σύστημα είναι:

\begin{circuitikz}[scale=1]
	\draw (0,0) node[mixer] (m1) {};
	
	\draw (m1.west) node[inputarrow] {} node[above left] {$+$} -- ++(-1,0) node[above] {$u$};
	\draw (m1.east) to[twoport,t=$H_1$] (2.5,0)
	to[twoport,t=$\frac{N_3}{D_3}$] (4,0) -- (4.5,0) node[above] {$y$} to[short,*-] (5.5,0) node[inputarrow] {};
	\draw (4.5,0) -- (4.5,-2) to[twoport,t=$8$] (m1.east |- 0,-2) -- (m1 |- 0,-1.5) -- (m1.south)
	node[inputarrow,rotate=90] {} node[below right] {$-$};
\end{circuitikz}

και επομένως προκύπτει από πράξεις:
\begin{align*}
	\frac{y}{u} &=
	\frac{\frac{N_3}{D_3(s+6)}}{1+8\frac{N_3}{D_3}\frac{1}{s+6}}
	\\ &= \frac{N_3}{D_3(s+6)+8N_3}
	\\ &= \frac{N_3}{D_3(s+6)+8N_3}
	\\ &= \frac{
		(2s+10)2s+(s+7)(s^2+5s+2)
		}{
		(s+14)(s+7)(s^2+5s+2)+16(2s+10)s
		}
\end{align*}

Η άσκηση αυτή μπορεί βεβαίως να λυθεί και αλγεβρικά. Συνοπτικά:
\begin{align}
	y &= A + B \label{ex.2.19.1} \\
	B &= \frac{2}{s+5}w \label{ex.2.19.2} \\
	A &= \frac{1}{s+6}e \label{ex.2.19.3} \\
	e &= u-8y \label{3x.2.19.4} \\
	w &= \left(1+\frac{s+3}{s+7}\right)A - \frac{1}{s}B \label{ex.2.19.5}
\end{align}
και λύνοντας το σύστημα μπορεί να βρεθεί η συνάρτηση μεταφοράς.

\paragraph{Ασκήσεις στα σφάλματα}
\begin{enumgreekparen}
	\item Τι σφάλμα θέσης έχει το σύστημα
	\( \displaystyle H(s) = \frac{s+9}{s^2+7s+3} \) αν συνδεθεί σε μοναδιαία αρνητική
	ανάδραση;
	
	\begin{circuitikz}[scale=1]
		\draw (0,0) node[mixer] (m1) {};
		
		\draw (m1.west) node[inputarrow] {} node[above left] {$+$} -- ++(-1,0);
		\draw (m1.east) to[twoport,t=$H(s)$] (3,0)  to[short,*-] (4.5,0) node[inputarrow] {};
		\draw (3,0) -- (3,-1.5) -- (m1 |- 0,-1.5) -- (m1.south)
		node[inputarrow,rotate=90] {} node[below right] {$-$};
	\end{circuitikz}
	
	\subparagraph{}
	Η συνάρτηση μεταφοράς είναι \( H_c = \frac{H}{1+H}
	= \frac{s+9}{s^2+7s+3+s+9} \). Το σφάλμα θέσης
	στη μόνιμη κατάσταση είναι το \( \lim_{s\to 0} s
	(1-H_C)\frac{1}{s} \), ή θυμόμαστε ότι \( e_{\mathrm{ss}} 
	= \frac{1}{1+K_P}
	\) όπου \( K_P = \lim_{s\to 0} t \).
	
	Ισχύει:
	\begin{align*}
		K_p &= \lim_{s\to 0} \cancel{8}
		\frac{s+9}{s^2+7s+3} = 3\\
		e_{\mathrm{ss}} &= \frac{1}{4} = 0.25\%
	\end{align*}
	\item
	Έστω συνάρτηση μεταφοράς που συνδέεται σε μοναδιαία αρνητική ανάδραση:
	\[
	H(s) = \frac{2(s+10)}{s(s^2+2)(s+5)}
	\]
	
	\subparagraph{}
	Τότε τα σφάλματα είναι:
	\begin{align*}
		e_{\mathrm{sv}} &= \lim_{s\to 0} \cancel{s}(1-H_c)\frac{1}{s^{\cancel{2}}} \\
		K_v &= \lim_{s\to 0} Hs \\
		e_{\mathrm{sv}} &= \frac{1}{K_v} \\
		e_{\mathrm{sa}} &= \lim_{s\to 0} \frac{1}{\cancel{2}}s(1-H_c)\frac{\cancel{2}}{s^3} \\
		K_a &= \lim_{s\to 0} H(s) \\
		e_{\mathrm{sa}} &= \frac{1}{K_a}
	\end{align*}
\end{enumgreekparen}

\subsection{Ταχύτητα}
Η δεύτερη προδιαγραφή που θα μελετήσουμε για τα συστήματα είναι η ταχύτητα, δηλαδή το πόσο
γρήγορα φτάνει ένα σύστημα στην επιθυμητή κατάσταση. Για να μελετήσουμε αυτήν την
προδιαγραφή θα χρησιμοποιήσουμε ως είσοδο τη \textbf{μοναδιαία βηματική συνάρτηση}
\( u(t) = \begin{cases}
0, &\quad t < 0\\
1, &\quad t > 0
\end{cases} \).

\subsubsection{Σε πρωτοβάθμια συστήματα}

Ως παράδειγμα, ας δούμε το παρακάτω σύστημα:

\begin{tikzpicture}
\draw (0,0) node[rectangle,draw,inner sep=10pt] (DM) {$\frac{k}{s+a}$};
\draw (DM.west) -- ++(-1,0) node[above right] {$r(s)$};
\draw (DM.east) -- ++(1,0) node[above left] {$y(s)$};
\end{tikzpicture}

με \(
H(s) = \frac{k}{s+a}
\).

Θυμόμαστε ότι το σφάλμα \( r(s)-y(s) \) στη μόνιμη κατάσταση (\( t\to \infty \))
είναι ορισμένο, και κάνοντας υπολογισμούς, η έξοδος στη μόνιμη κατάσταση προκύπτει:
\[
\lim_{s\to 0} sy(s)
=
\lim_{s\to 0} \cancel{s}\cancel{\frac{1}{s}}\frac{k}{s+a}
=
\frac{k}{a}
\]

και για να μηδενιστεί το σφάλμα, επιθυμούμε η έξοδος να είναι μοναδιαία, άρα πρέπει \( \frac{k}{a} = 1 \implies k=a \).

Σε μορφή σταθεράς χρόνου, η \( H \) γράφεται:
\[
H(s) = \frac{1}{τs+1}
\]
όπου
\[
τ = \frac{1}{a}
\]

Έχουμε θεωρήσει ότι \( k=a \), επομένως το σφάλμα είναι μηδενικό.

Για να βρούμε την έξοδο στο πεδίο του χρόνου, έχουμε ότι:
\[
y(s) = \frac{1}{τs+1}\frac{1}{s} = \frac{1}{s}-\frac{τ}{τs+1}
\]
και, χρησιμοποιώντας αντίστροφο μετασχηματισμό Laplace:
\[
y(t) = y_{\mathrm{ss}} \left( 1-e^{-\sfrac{t}{τ} } \right)
\]

Έχουμε μια αποσβεννύμενη απόκριση που μοιάζει ως εξής:

\begin{tikzpicture}[scale=2]
\draw[->] (-0.5,0) -- (3.5,0) node[below right] {$t$};
\draw (0,-0.5) -- (0,3);

\draw[very thick,orange!50!red]
(0,0) -- (0.8,0.8*2.5*1.5) node[midway,above,sloped] {$\sfrac{1}{\tau}$};

\draw[very thick,blue,opacity=.7]
plot [variable=\t,domain=0:6.5,samples=\gsamples]
(\t,{2.5*(1-exp(-1.5*\t))});

% not the real value to make the graph look better
\draw[dashed] (0.15,0) node[below,xshift=1mm] (t1) {$t_1$}-- ++(0,0.503709453101557)
-- (0,0.503709453101557) node[left] {$0.1y_{\mathrm{ss}}$};

\draw[dashed] (1.53505672866270,0) node[below] (t2) {$t_2$}-- ++(0,2.25)
-- (0,2.25) node[left] {$0.9y_{\mathrm{ss}}$};

\draw[dashed] (2.60801533695210,0) node[below] {$t_s$}-- ++(0,2.45)
-- (0,2.45);

\draw[dashed] (0,2.5) node[left] {$y_{\mathrm{ss}}$} -- (3.5,2.5);

\draw [decorate,decoration={brace,amplitude=10pt},green!20!black] (t2.south) -- (t1.south) node[below,midway,yshift=-4mm] {$t_r$};
\end{tikzpicture}

Θα ψάξουμε μετά από πόσο χρόνο έχει φτάσει η έξοδος στο 10\% και στο 90\% της επιθυμητής
τιμής:
\begin{gather*}
	1-e^{\sfrac{-t_1}{τ} }
	= 0.1 \implies t_1 = τ(\ln 10 - \ln 9)
	\\
	1-e^{\sfrac{-t_2}{τ} }
	= 0.9 \implies t_2 = τ(\ln 10)
\end{gather*}

\begin{defn}{Χρόνος ανόδου σε πρωτοβάθμιο σύστημα}{}
Υπολογίζουμε τον \textbf{χρόνο ανόδου} \( t_r \) (rise), δηλαδή τον χρόνο μεταξύ
της στιγμής που η έξοδος είναι στο 10\% και στο 90\% της επιθυμητής:
\[
t_r = t_2-t_1 \implies \boxed{t_r = τ\ln 9 \approx 2.2τ}
\]
\end{defn}

\begin{defn}{Χρόνος αποκατάστασης σε πρωτοβάθμιο σύστημα}{}
Ορίζουμε το \textbf{χρόνο αποκατάστασης} \( t_s \) (steady), ως εξής:
\[
1-e^{\sfrac{-t_s}{τ} } = 0.98
\implies \boxed{t_s \approx 4τ}
\]
\end{defn}

Επιβεβαιώνουμε δηλαδή ότι η σταθερά χρόνου \( τ \) σχετίζεται με την ταχύτητα απόκρισης
του συστήματος. Όσο μεγαλύτερη είναι η σταθερά χρόνου, τόσο μικρότερη είναι η ταχύτητά του.

Μάλιστα, αν βρούμε το σφάλμα ταχύτητας του συστήματος αυτού, θα δούμε ότι γίνεται μικρότερο,
όσο μεγαλώνει η ταχύτητά του.

\paragraph{Συνδυασμός συστημάτων}
Έστω δύο εν σειρά συστήματα:
\[
Y(s) = H(s)U(s)
\]
και πιο συγκεκριμένα:
\[
H(s) = \frac{N(s)}{D(s)} \qquad
U(s) = \frac{N_u(s)}{D_u(s)}
\]
επομένως, αν σπάσουμε σε κλάσματα όπως γνωρίζουμε από τα μαθηματικά:
\[
Y(s) = \frac{N}{D}\frac{N_u}{D_u}
= \frac{N_1}{D} + \frac{N_2}{D}
\]
που μπορεί να δώσει ένα αποτέλεσμα της μορφής:
\begin{gather*}
\frac{A}{s+p} + \cdots + \frac{B}{(s+p)^k} + \frac{C}{(s+a)^2+β}
\intertext{ή, αντίστοιχα}
(A_1+A_2t+\cdots)e^{-pt} + \dots
\end{gather*}

\paragraph{Παράδειγμα}
Έστω το σύστημα:
\[
H_z(s) = \frac{k(s+b)}{s+a}
\]
με τελική τιμή (είσοδος = \( \frac{1}{s} \)):
\[
y_{ss} = \lim_{s\to 0} sy(s) = \lim_{s\to 0}s\frac{k(s+b)}{s+a}\frac{1}{s} = \frac{kb}{a}
\]
και αρχική τιμή (για \( t=0 \)):
\[
y(0) = \lim_{t\to0} y(t) = \lim_{s\to \infty} sy(s) = k
\]
δηλαδή αυτό το σύστημα δεν ξεκινάει από μηδενική αρχική τιμή.

\begin{tikzpicture}[scale=1]
\draw (-2,0) -- (2,0);
\draw (0,-2) -- (0,2);

\draw[thick] (-1.5,0) node[circle,inner sep=2.7pt,draw,orange!80!black] {} node[below,yshift=-1mm] {$b$};
\draw[thick] (-0.75,0) node[cross=4pt,red!80!black] {} node[below,yshift=-1mm] {$\vphantom{b}a$};

\begin{scope}[xshift=3.5cm,yshift=-1cm]
\draw[->] (-0.5,0) -- (3.5,0) node[below right] {$t$};
\draw[->] (0,-0.5) -- (0,3) node[right] {$y(t)$};

\draw[very thick,blue,opacity=.7]
plot [variable=\t,domain=0:3.5,samples=\gsamples]
(\t,{0.5+1.5*exp(-1.5*\t)});
\end{scope}
\end{tikzpicture}

\paragraph{Μελέτη μηδενικών}
Θα εξετάσουμε την επίδραση των μηδενικών στην έξοδο του συστήματος.

Έστω ένα σύστημα με έξοδο στο χρόνο:
\begin{align*}
	y_z(t) &= \frac{kb}{a} + \frac{k(a-b)}{a}e^{-at}
	\intertext{Το οποίο αναλύουμε περαιτέρω:}
	&= \overbrace{\frac{kb}{a}}^{y_{\mathrm{ss}}}\left(1-e^{-at}\right)
	+ ke^{-at}
	\\ &=
	y_{\mathrm{ss}}\left( 1-e^{-at} \right) + \frac{\dot{y}(t)}{b}
\end{align*}

(Τοποθετήσαμε την παράγωγο στην τελευταία σχέση, επειδή μαθηματικά φαίνεται ότι
\( y(t) = \frac{kb}{a} (1-e^{-at}) \implies \dot y = kbe^{-at} \)).

Αναλύοντας κατά Laplace, προκύπτει:
\[
Y_z(s) =
\frac{kb}{a}\frac{1}{s} + \frac{k(a-b)}{a}\frac{1}{s+a}
= \frac{k(b+s)}{s(a+s)}
\]
και παρατηρούμε το μηδενικό στο \( -b \) και τους πόλους στο \( 0 \) και στο \( -a \).

Γραφικά:

\begin{tikzpicture}[scale=1]
\draw (-2,0) -- (2,0);
\draw (0,-2) -- (0,2);

\draw[thick] (-1,0) node[circle,inner sep=2.7pt,draw,orange!80!black,fill=white,fill opacity=.7] {} node[below,yshift=-2mm] {$-b$};

\begin{scope}[xshift=3.5cm,yshift=-1cm]
\draw[->] (-0.5,0) -- (4,0) node[below right] {$t$};
\draw[->] (0,-0.5) -- (0,3) node[right] {$y(t)$};

\draw[very thick,blue!20!green,opacity=.7]
plot [variable=\t,domain=0:3.5,samples=\gsamples]
(\t,{2*exp(-1*\t)})
(3.5,0) node[above] {$\dot y(t)$};
\draw[very thick,blue,opacity=.7]
plot [variable=\t,domain=0:3.5,samples=\gsamples]
(\t,{2*(1-exp(-1*\t))})
(3.5,2) node[below] {$y(t)$};
\draw[very thick,dashed,blue,opacity=.7]
plot [variable=\t,domain=0:3.5,samples=\gsamples]
(\t,{2*(1-exp(-3*\t))});

\draw[blue!70!black,<-] (1.8,2) to[bend left=30] ++ (0.5,0.6)
node[above,align=center,scale=.8]
{μηδενικό πιο κοντά\\σε φανταστικό άξονα};
\end{scope}
\end{tikzpicture}

Το μηδενικό \(-b\) της συνάρτησης εκφράζεται από την τιμή \( \frac{\dot y(t)}{b} \) στη συνάρτηση \( y_z(t) \).

Παρατηρούμε ότι αν υπάρχει μηδενικό, και όσο πιο κοντά πλησιάζει στον φανταστικό άξονα, τόσο
μεγαλύτερη επίδραση έχει στην έξοδο του συστήματος, αυξάνοντας την ταχύτητά του.

\paragraph{Ένα απλό πρόβλημα πρωτοβάθμιου συστήματος}
Έστω ένα σύστημα:
\[
G(s) = \frac{1}{s+1}
\]

Αυτό έχει πόλο στο \( -1 \), και σταθερά χρόνου \( τ = 1 \), άρα χρόνο αποκατάστασης \( t_s = 4\ \mathrm{sec} \),
κάτι που είναι πολύ μεγάλο και δεν μας αρέσει καθόλου. Επομένως, προσθέτουμε έναν βρόγχο ανάδρασης μαζί με τον
απλούστερο δυνατό ελεγκτή, που είναι το αναλογικό κέρδος \( \boxed{k} \):

\begin{tikzpicture}[scale=1]
\draw (0,0) node[circle,draw] (a1) {};
\draw (0.9,0) node[rectangle,draw,minimum height=20pt,minimum width=20pt] (b1) {$k$};
\draw (2.2,0) node[rectangle,draw] (b2) {$\frac{1}{s+1}$};

\draw[->] (-1,0) -- (a1.west);
\draw (a1.east) -- (b1.west);
\draw (b1.east) -- (b2.west);
\draw (b2.east) -- ++(1,0);
\draw[->] (3,0) -- ++(0,-1) -- (a1 |- 0,-1) -- (a1.south) node[below right] {$-$};
\end{tikzpicture}

Απαιτούμε χρόνο αποκατάστασης \( t_s \leq 1\ \mathrm{sec} \), και το ζητούμενο της άσκησης είναι η περιοχή τιμών
του \( k \) για τις οποίες ικανοποιείται αυτή η προδιαγραφή.

\subparagraph{Λύση}
Αρχικά βρίσκουμε τη συνάρτηση μεταφοράς του συστήματος:
\[
H(s) = \frac{\frac{k}{s+1}}{1+\frac{k}{s+1}}
= \frac{k}{s+(k+1)}
\]
Αυτό έχει σταθερά χρόνου \( τ = \frac{1}{k+1} \), επομένως:
\[
t_s \approx 4τ = \frac{4}{k+1}
\]

Πρέπει:
\[
t_s \leq 1 \implies \frac{4}{k+1} \leq 1 \implies k \geq 3.
\]

\subsubsection{Σε δευτεροβάθμια συστήματα}

\paragraph{Ένα παράδειγμα δευτεροβάθμιου συστήματος}
Έστω το σύστημα ενός περιστρεφόμενου κυλίνδρου με απόσβεση:

\begin{circuitikz}[scale=1]
	\tikzstyle{reverseclip}=[insert path={(3,0.5) --
		(3,-0.5) --
		(0.8,-0.5) --
		(0.8,0.5) --
		(3,0.5)}
	]
	
	\begin{scope}
		%[top color=gray!5!white,bottom color=gray!10!white!90!blue]
		\path[clip] (1,0) ellipse (0.3 and 0.4) [reverseclip];
		
		\fill[top color=gray!5!white,bottom color=gray!7!white!95!blue]
		plot [smooth,tension=1.5]
		coordinates {(1,0.4) (2,0) (1,-0.4)};
	\end{scope}
	
	\draw (1,0) ellipse (0.3 and 0.4);
	
	\draw
	plot [smooth,tension=1.5]
	coordinates {(1,0.4) (2,0) (1,-0.4)};
	
	\draw (2,0) to[damper,l=$B$] (3,0);
\end{circuitikz}
\[
u = J \ddot y + B\dot y
\]

Παραπάνω δίνεται η διαφορική εξίσωση του συστήματος, που θα μετασχηματίσουμε κατά Laplace:
\begin{gather*}
Js^2y(s) +Bsy(s) = u(s) \\
\frac{y(s)}{u(s)} = \frac{1}{s(Js+b)}
\end{gather*}

Αντίστοιχα με το προηγούμενο παράδειγμα, προσθέτουμε έναν ελεγκτή και έναν βρόγχο ανάδρασης:

\begin{tikzpicture}[scale=1]
\draw (0,0) node[circle,draw] (a1) {};
\draw (0.9,0) node[rectangle,draw,minimum height=20pt,minimum width=20pt] (b1) {$k$};
\draw (2.5,0) node[rectangle,draw] (b2) {$\frac{1}{s(Js+b)}$};

\draw[->] (-1,0) -- (a1.west) node[above,pos=.7] {$r$};
\draw (a1.east) -- (b1.west);
\draw (b1.east) -- (b2.west);
\draw[->] (b2.east) -- ++(1,0) node[above] {$y$};
\draw[->] (3.7,0) -- ++(0,-1) -- (a1 |- 0,-1) -- (a1.south) node[below right] {$-$};
\end{tikzpicture}

με έξοδο:
\[
\frac{y(s)}{r(s)}
= \frac{\frac{k}{s(Js+b)}}{1+\frac{k}{s(Js+b)}}
= \frac{k}{Js^2+Bs+k}
= \frac{\sfrac{k}{J} }{s^2 +\sfrac{B}{J}s + \sfrac{k}{J}   }
= \frac{\omega_n^2}{s^2+2ζ\omega_ns+\omega_n^2}
=
\boxed{
\frac{\omega_n^2}{(s+σ)^2 + \omega_d^2}
}
\]
όπου θέσαμε:
\begin{tcolorbox}[width=.6\textwidth]
	\vspace{-17pt}
\begin{alignat*}{3}
	\omega_n^2 &= \sfrac{k}{J} && \implies \omega_n &&= \sqrt{\frac{k}{J}}  \\
	2ζ\omega_n &= \sfrac{B}{J}, \quad ζ \leq 1 && \implies\ ζ &&= \frac{B}{2\sqrt{Jk}} \\
	s_{1,2}  &= -σ \pm j\omega_d && &&\text{(ρίζες παρ/τή)} \\[3ex]
	σ &= ζ\omega_n \\
	\omega_d &= \omega_n\sqrt{1-ζ^2}
\end{alignat*}
\end{tcolorbox}
\begin{defn}[width=.6\textwidth]{}{}
	\begin{itemize}
		\item \( \omega_n \): \textbf{φυσική συχνότητα}
		\item \( ζ \): \textbf{συντελεστής απόσβεσης}
		\item \( \omega_d \): \textbf{συχνότητα ταλαντώσεων}
	\end{itemize}
\end{defn}

Μετά από πράξεις που δεν παρουσιάζονται, η έξοδος του συστήματος ξεχωρίζεται στις δύο παρακάτω περιπτώσεις:
\begin{itemize}
	\item Αν έχουμε \textbf{δύο πραγματικούς πόλους} \( p1,\ p2 \):\[
	y(t) = y_{\mathrm{ss}}
	\left[
	1+\frac{p_1p_2}{p_1-p_2}
	\left(
	\frac{e^{p_1t}}{p_1}
	-
	\frac{e^{p_2t}}{p_2}
	\right)
	\right]
	\]
	
	\begin{tikzpicture}[scale=1]
	\draw (-2,0) -- (2,0);
	\draw (0,-1) -- (0,1);
	
	\draw[thick] (-1.5,0) node[cross=4pt,red!80!black] {} node[below,yshift=-1mm] {$-b$};
	\draw[thick] (-0.5,0) node[cross=4pt,red!80!black] {} node[below,yshift=-1mm] {$\vphantom{b}-a$};
	\end{tikzpicture}

	Δηλαδή η απόκριση του συστήματος αποτελείται από το άθροισμα δύο εκθετικά αποσβεννύμενων
	αποσβέσεων. Πρακτικά, παίζει ρόλο ο πόλος που προκαλεί την πιο αργή απόσβεση (ειδικά αν
	οι πόλοι απέχουν αρκετά μεγάλη απόσταση μεταξύ τους), καθώς η άλλη απόσβεση εξαλείφεται
	γρήγορα. Ο αργός πόλος επιβάλλει το ρυθμό του.
	\item Αν έχουμε \textbf{δύο μιγαδικές συζυγείς ρίζες} \( -σ \pm j\omega_d \):
	\begin{align*}
	y(t) &= y_{\mathrm{ss}}\left[
	1-e^{-ζ\omega_nt} \cdot
	\left(
	\cos \omega_dt + \frac{ζ}{\sqrt{1-ζ^2}} \sin \omega_d t
	\right)
	\right]
	\\ &= y_{\mathrm{ss}}\left[
	1 - \frac{e^{-ζ\omega_nt}}{\sqrt{1-ζ^2}}
	\cdot \sin\left(\omega_d t + \tan^{-1}\frac{\sqrt{1-ζ^2}}{ζ}\right)
	\right]
	\end{align*}

	
	Σε αυτήν την περίπτωση, η έξοδος του συστήματος μοιάζει κάπως έτσι:
	
	\begin{tikzpicture}[scale=1.7]
	\begin{scope}[xshift=-2cm,yshift=2cm,scale=0.7]
	\draw (-2,0) -- (1.5,0);
	\draw (0,-2) -- (0,2);
	
	\draw[dashed] (-1,1) -- ++ (1,0) node (xa) {};
	\draw[dashed] (-1,-1) -- ++ (1,0) node (xb) {};
	\draw[dashed] (0,0) -- (-1,1);
	
	\draw[thick] (-1,1) node[cross=4pt,red!80!black] {};
	\draw[thick] (-1,-1) node[cross=4pt,red!80!black] {};
	
	\draw[<->,orange!30!green!50!black] (xa.east) -- (xa.east |- 0,0) node[right,midway] {$\omega_d$};
	\draw[<->,orange!30!green!50!black] (-1,-0.1) -- ++(1,0) node[midway,below] {$ζ\omega_n$};
	
	\draw (-45+180:0.3) to[bend right] node[midway,left,yshift=1mm] {$θ$} (-0.3,0);
	\end{scope}
	
	\draw[->] (-0.5,0) -- (5,0) node[below right] {$t$};
	\draw[->] (0,-0.5) -- (0,4) node[below left] {$y(t)$};
	
	\draw[thick,dashed] (0,2) node[left] {$y_\mathrm{ss}$} -- ++(6,0);

	\draw[ultra thick,blue,opacity=.7]
	plot [variable=\t,domain=0:6,samples=\gsamples]
	(\t,{2*(1-1*exp(-0.8*\t)*cos(3.5*\t r))});
	
	\draw[very thick,cyan,opacity=.2]
	plot [variable=\t,domain=0:6,samples=\gsamples]
	(\t,{2*(1+1*exp(-0.8*\t))});
	
	\draw[dashed] (0.84,0) node[below,xshift=2mm] {$t_p$}-- ++(0,3)
	-- (0,3) node[left] {$M_p$};
	\draw[dashed] (0.44,0) node[below] {$t_r$} -- ++(0,2);
	
	\filldraw[thick,draw=orange,draw opacity=.5,fill=white!90!orange,fill opacity=.6]
	(0.84,3) circle (1.7pt);
	\filldraw[thick,draw=orange!50!green,draw opacity=.5,fill=orange!50!green!10!white,fill opacity=.6]
	(0.44,2) circle (1.7pt);
	
	% not real but used for a nicer graph
	\draw[dashed] (3.3,0) node[below] {$t_s$} -- ++(0,2);
	\draw[thick,yellow!50!blue] (3.3,{2*(1-1*exp(-0.8*3.3))}) -- (3.3,{2*(1+1*exp(-0.8*3.3))})
	node[midway,above,rotate=180,sloped,scale=.3] {2\%};
	\end{tikzpicture}
	
	Αποδεικνύεται επίσης ότι ισχύουν αρκετές σχέσεις, που παρουσιάζονται παρακάτω.

	\begin{theorem}{Χρόνος αποκατάστασης σε δευτεροβάθμιο σύστημα με συζυγείς πόλους}{}
		\[
		t_s \approx \frac{4}{ζ\omega_n}
		\]
		
		({\color{red!50!black}Προσοχή!} Αυτή η σχέση ισχύει μόνο όταν έχουμε δύο συζυγείς μιγαδικές ρίζες, και όχι πραγματικές!)
	\end{theorem}
		
	Ο χρόνος ανόδου ορίζεται διαφορετικά από τα πρωτοβάθμια συστήματα:
	\begin{defn}{Χρόνος ανόδου σε δευτεροβάθμιο σύστημα με συζυγείς πόλους}{}
		Ως \textbf{χρόνος ανόδου} ορίζεται ο χρόνος μέχρι η έξοδος να φτάσει την πρώτη
		φορά στην επιθυμητή τιμή.
		
		Αποδεικνύεται ότι είναι ίσος με:
		\[
		\boxed{t_r = \frac{π-θ}{\omega_d}}
		\]
		
		όπου \( θ \) η γωνία του ενός πόλου.
	\end{defn}
	
	\begin{defn}{Χρόνος Υπερύψωσης}{}
	Ο χρόνος \( t_p \) (peak) μέχρι η έξοδος να φτάσει στην πρώτη κορυφή της ημιτονοειδούς κυματομορφής είναι:
	\[
	\boxed{t_p = \frac{π}{\omega_d}}
	\]
	\end{defn}
	
	Βλέπουμε ότι αρκετές στιγμές η έξοδος έχει τιμή μεγαλύτερη της εξόδου. Η υπερύψωση αυτή
	μπορεί σε κάποιες περιπτώσεις να μην είναι επιθυμητή, επειδή για παράδειγμα δεν θέλουμε
	το ρεύμα σε ένα κύκλωμα να ξεπεράσει μια μέγιστη τιμή, ή ο βραχίονας ενός ρομπότ να
	φτάσει έξω από κάποια όρια.
	
	\begin{defn}{Υπερύψωση}{}
		Αποδεικνύεται ότι η πρώτη κορυφή φτάνει στην τιμή:
		\[
		M_p =
		e^{-\left(
			\frac{ζπ}{\sqrt{1-ζ^2}}
			\right)}
		= e^{-σ t_p}
		\]
	\end{defn}
	
	Αυξάνοντας τη σταθερά \( k \), μειώνουμε τους χρόνους \( t_p \) και \( t_r \), καθώς και την \( M_p \),
	αλλά αυξάνεται το \( ζ \) και η συχνότητα των αποσβεννύμενων ταλαντώσεων, δηλαδή το σύστημα γίνεται πιο "ζωηρό".
	
	Τέλος, για τη γωνία του πόλου, αποδεικνύεται ότι:
	\[
	\boxed{\cos θ = ζ}
	\]
\end{itemize}
	
\paragraph{Παραδείγματα συστήματος δεύτερης τάξης} \hspace{0pt}

\begin{enumgreekparen}
	\item
	\begin{tikzpicture}[scale=1,baseline]
	\tikzset{mbox/.style={rectangle,draw,minimum height=20pt,minimum width=17pt}}
	\draw (-1,0) node[circle,draw] (a1) {};
	\draw (0,0) node[circle,draw] (a2) {};
	\draw (0.9,0) node[mbox] (b1) {$k$};
	\draw (2.5,0) node[mbox] (b2) {$\frac{1}{Js+b}$};
	\draw (4.3,0) node[mbox] (b3) {$\frac{1}{s}$};
	\draw (1.6,-1) node[mbox] (b4) {$K_v$};
	
	\draw[->] (-2,0) -- (a1.west) node[above,pos=.7] {};
	\draw (a1.east) -- (a2.west);
	\draw (a2.east) -- (b1.west);
	\draw (b1.east) -- (b2.west);
	\draw (b2.east) -- (b3.west);
	\draw[->] (b3.east) -- ++(1,0) node[above] {$φ$};
	\draw (3.5,0) node[above] {$\omega$} -- ++(0,-1) -- (b4.east);
	\draw[->] (b4.west) -- (a2 |- 0,-1) -- (a2.south) node[below right] {$-$};
	\draw[->] (5,0) -- ++(0,-2) -- (a1 |- 0,-2) -- (a1.south) node[below right] {$-$};
	\end{tikzpicture}
	\item
	\begin{tikzpicture}[scale=1.2,baseline]
	\draw (0,0) node[circle,draw] (a1) {};
	\draw (1.5,0) node[rectangle,draw,minimum height=20pt,minimum width=20pt] (b1) {$k(1+K_vs)$};
	\draw (3.5,0) node[rectangle,draw] (b2) {$\frac{1}{s(Js+b)}$};
	
	\draw[->] (-1,0) -- (a1.west);
	\draw (a1.east) -- (b1.west);
	\draw (b1.east) -- (b2.west);
	\draw[->] (b2.east) -- ++(1,0);
	\draw[->] (4.5,0) -- ++(0,-1) -- (a1 |- 0,-1) -- (a1.south) node[below right] {$-$};
	\end{tikzpicture}
\end{enumgreekparen}

\subparagraph{(α)}
Εκτελούμε πράξεις για να βρούμε την απόκριση του συστήματος:
\begin{alignat*}{2}
	φ &= \frac{1}{s}\omega &&\implies\\
	sφ &= \omega &&\implies \numberthis \label{example.2.secorder.omega} \\
	sφ &= \frac{k}{Js+b}\cdot \left[r(s) - K_v \omega - \phi \right] &&\xRightarrow{\eqref{example.2.secorder.omega}} \\
	sφ &= \frac{k}{Js+b}\left[ r(s) - sK_vφ - φ \right] &&\implies \\
	sφ(Js+b) &= k\left[r(s) - sK_vφ - φ\right] &&\implies \\
	φ\left[ Js^2 + bs + k K_v s + k \right] &= kr(s) && \implies \\
	\frac{φ}{r} &= \frac{k}{Js^2 + (b+kK_v)s + k} && \implies \\
	H(s) &= \frac{\sfrac{k}{J} }{s^2
		+\left(\frac{B+kK_v}{J}\right)s + \sfrac{k}{J} 
	}
\end{alignat*}

\subparagraph{(β)}
Η συνάρτηση μεταφοράς είναι:
\[
H(s) =
\frac{\sfrac{K}{J}(1+K_vs) }{s^2 + \left(\frac{B+kK_v}{J}s\right)
	+ \sfrac{K}{J} }
\]

\subparagraph{}
Παρατηρούμε ότι τα δύο συστήματα έχουν παρόμοιες συναρτήσεις μεταφοράς, αλλά το
δεύτερο έχει ένα μηδενικό.

\paragraph{Παράδειγμα με μηδενικό}
Έστω μια συνάρτηση μεταφοράς που έχει μηδενικό στο \( -z \):
\begin{align*}
	H_z(s) &= \frac{b(s+z)}{s^2+a_1s+a_0}
	\\ &= y_{\mathrm{ss}}\frac{\sfrac{\omega_n^2}{z} (s+z) }{s^2+2ζ\omega_ns+\omega_n^2}
	\\ &= y_{\mathrm{ss}} \boxed{
		\frac{\omega_n^2}{s^2+2ζ\omega_ns+\omega_n^2}
		+\frac{s}{z}
		\left(
		    \frac{\omega_n^2}{s^2+2ζ\omega_n+\omega_n^2}
		\right)
		}
\end{align*}

Η τελευταία σχέση, αν παρατηρήσουμε ότι ο τελεστής \( s \) εκφράζει την \textbf{παραγώγιση}
μιας συνάρτησης στο πεδίο του χρόνου, βλέπουμε ότι ισχύει η σχέση
\( y_z(t) = y(t) + \frac{\dot y(t)}{z} \):
\[
\boxed{
	y_z(t) = \underbrace{y(t)}_{\text{έξοδος χωρίς μηδενικό}} +
	\underbrace{\frac{\dot y(t)}{z}}_{\text{επίδραση μηδενικού}}
	}
\]

\paragraph{Συνάρτηση χωρίς και με μηδενικό}
Έστω οι συναρτήσεις μεταφοράς:
\[
H_a = \frac{14}{(s+1)^2+6},\quad
H_b = \frac{7(s+2)}{(s+1)^2+6}
\]

με τελική έξοδο (είσοδος βηματική):
\begin{align*}
	y_{\mathrm{ss}_1} &= \lim_{s\to 0} s\frac{14}{(s+1)^2+6}\frac{1}{s} = 2 \\
	y_{\mathrm{ss}_2} &= \lim_{s\to 0} s\frac{7(s+2)}{(s+1)^2+6}\frac{1}{s} = 2
\end{align*}

Αν προσομοιώσουμε/εφαρμόσουμε τύπους, η έξοδος των συστημάτων θα είναι:

\def\gsamples{200}

\begin{tikzpicture}[scale=1.2]
\draw[->] (-1,0) -- (5,0) node[below] {$t$};
\draw (0,-0.5) -- (0,3.4);

\def\sqrtsix{2.45}

\draw[dashed] (4,0) node[below] {$4$} -- (4,2);
\draw[dashed] (1.27,0) node[below] {$1.27$} -- (1.27,2.55) -- (0,2.55) node[left] {$2.55$};
\draw[dashed] (0.75,3.19) -- (0,3.19) node[left] {$3.19$};
\draw[dashed,path fading=east] (0,2) -- (5,2);
\draw (0,2) node[left] {$2$};

\draw[very thick,blue!50!black]
plot[variable=\t,domain=0:5,samples=\gsamples]
(\t,{-1/3*(\sqrtsix*sin(\sqrtsix*\t r) + 6*cos(\sqrtsix*\t r))*exp(-\t) + 2});

\draw[very thick,blue!50!cyan]
plot[variable=\t,domain=0:5,samples=\gsamples]
(\t,{1/6*(5*\sqrtsix*sin(\sqrtsix*\t r) - 12*cos(\sqrtsix*\t r))*exp(-\t) + 2});

\draw (current bounding box.north east) node[minimum width=2mm,minimum height=2mm,rectangle,draw,fill=blue!50!black] (a) {};
\draw (a.south) ++(0,-0.5) node[minimum width=2mm,minimum height=2mm,rectangle,draw,fill=blue!50!cyan] (b) {};

\draw (a.east) node[right] {$ \frac{14}{(s+1)^2+6}\frac{1}{s}$};
\draw (b.east) node[right] {$ \frac{7(s+2)}{(s+1)^2+6}\frac{1}{s}$};
\end{tikzpicture}

Βλέπουμε ότι το σύστημα με το μηδενικό έχει πιο γρήγορη απόκριση, αλλά ταυτόχρονα αυξάνεται
και η υπερύψωση.

Συνήθως το \( ζ \) ως προδιαγραφή είναι ανάμεσα στο 0.4 και 0.8, ενώ αν δεν θέλουμε καθόλου
υπερύψωση, παίρνουμε \( ζ \geq 1 \), που οδηγεί πραγματικές ρίζες.

\paragraph{Άσκηση}
Ποιό από τα παρακάτω συστήματα θα εμφανίσει τη μεγαλύτερη υπερύψωση, ως απόκριση στη μοναδιαία
βηματική συνάρτηση;
\begin{enumgreekparen}
	\item \( \displaystyle \frac{8}{s^2+2s+1} \)
	\item \( \displaystyle \frac{48}{s^2+2s+16} \)
	\item \( \displaystyle \frac{48}{s^2+8s+16} \)
	\item \( \displaystyle \frac{8}{s^2+2s+4} \)
\end{enumgreekparen}

Η σωστή απάντηση είναι η (β).

\todo{why?}

\paragraph{Άσκηση}
Ποιό από τα παρακάτω συστήματα έχει ταχύτερη απόκριση;
\begin{enumgreekparen}
	\item \( \displaystyle \frac{s+10}{(s+2)(s+5)} \)
	\item \( \displaystyle \frac{s+1}{(s+2)(s+5)} \)
	\item \( \displaystyle \frac{10}{(s+2)(s+5)} \)
	\item \( \displaystyle \frac{s+3}{(s+2)(s+5)} \)
\end{enumgreekparen}

Το ταχύτερο σύστημα είναι αυτό με το πιο "σημαντικό" μηδενικό, δηλαδή αυτό που βρίσκεται πιο
κοντά στον φανταστικό άξονα, άρα το \textbf{(β)}, με μηδενικό \( z = -1 \).

\paragraph{Άσκηση}
Να συγκριθούν τα συστήματα ως προς ταχύτητα απόκρισης:
\begin{enumgreekparen}
	\item \( \displaystyle \frac{10}{(s+5)^2+3} \)
	\item \( \displaystyle \frac{10(s+2)}{(s+5)^2+3} \)
	\item \( \displaystyle \frac{10(s+3)}{(s+5)^2+3} \)
\end{enumgreekparen}

Το (β) είναι το (α) με πρόσθεση της παραγώγου του (α) διά 2, ενώ το (γ) είναι το (α) με
πρόσθεση της παραγώγου του (α) διά 3. Επομένως το \textbf{(β)}, που έχει πιο σημαντικό
μηδενικό, είναι ταχύτερο.

\paragraph{Άσκηση}
Έστω ένα σύστημα κλειστού βρόγχου:
\begin{tikzpicture}[scale=1.2,baseline]
\draw (0,0) node[circle,draw] (a1) {};
\draw (0.9,0) node[rectangle,draw,minimum height=20pt,minimum width=20pt] (b1) {$k$};
\draw (2.5,0) node[rectangle,draw] (b2) {$H(s)$};

\draw[->] (-1,0) -- (a1.west) node[above,pos=.7] {$r$};
\draw (a1.east) -- (b1.west);
\draw (b1.east) -- (b2.west);
\draw[->] (b2.east) -- ++(1,0) node[above] {$y$};
\draw[->] (3.7,0) -- ++(0,-1) -- (a1 |- 0,-1) -- (a1.south) node[below right] {$-$};
\end{tikzpicture}
όπου \( \displaystyle H(s) = \frac{10}{(s+6)(s+7)} \).

Θέλουμε να βρεθεί η περιοχή τιμών της σταθεράς \( k \), ώστε να ικανοποιούνται οι
παρακάτω προδιαγραφές:
\begin{align*}
	e_{\mathrm{ss}} &< 0.5 \\
	M_p &< 0.1 \\
	t_s &< 660\ \mathrm{ms}
\end{align*}

Έχουμε:
\todo{πράξεις}
\begin{enumerate}
	\item Για το \( e_{\mathrm{ss}} \):
	\begin{align*}
		e_{\mathrm{ss}} &= \frac{1}{1+K_{\mathrm{pos}}}
		= \frac{1}{1+\lim_{s\to 0}kH(s)}
		= \frac{1}{1+\lim_{s\to 0}k\frac{10}{(s+6)(s+7)}}
		= \frac{1}{1+k\frac{10}{6\cdot 7}} \\
		e_{\mathrm{ss}} &< 0.5 \implies \boxed{k > 4.2}
	\end{align*}
	\item Για το \( M_p < 0.1 \):
	\begin{align*}
		\ln M_p &= \frac{ζπ}{\sqrt{1-ζ^2}}\\
		\implies ζ &= \frac{\left\lvert \ln M_p \right\rvert}{
			\sqrt{\ln^2 M_p - π^2}
			}\\
		\implies ζ &> \frac{|\ln 0.1 |}{\sqrt{\ln^2 0.1 + π^2}}
		\implies \boxed{ζ > 0.59}
	\end{align*}
	
	Για να χρησιμοποιήσουμε το αποτέλεσμα που βρήκαμε για το \( ζ \), πρέπει να το
	εφαρμόσουμε επάνω στο χαρακτηριστικό πολυώνυμο του συστήματος, που μετά από πράξεις
	προκύπτει:
	\[
	s^2+
	\underbrace{13}_{ζ\omega_n}s
	+\underbrace{10k+42}_{\omega_n^2}
	\]
	
	Επομένως θέλουμε:
	\[
	ζ = \frac{13}{2\sqrt{42+10k}} > 0.59
	\implies \boxed{k < 8.3}
	\]
	
	\item
	Αν για τις παραπάνω προδιαγραφές υπολογίσουμε το χρόνο αποκατάστασης
	\[
	t_s \approx \frac{4}{ζ\omega_n}
	= \frac{4}{13} \simeq 0.3077 \simeq 308\ \mathrm{ms}
	\]
	φαίνεται ότι ήδη πληροί την προδιαγραφή \( t_s < 660\ \mathrm{ms} \).
	
	Σε διαφορετική περίπτωση, αν δηλαδή για αυτό το σύστημα μάς δινόταν μια προδιαγραφή
	που δεν ικανοποιούνταν (π.χ. \( t_s < 220\ \mathrm{ms} \)), δεν θα ήταν δυνατό να
	λυθεί η άσκηση με αυτό το σύστημα, αλλά θα έπρεπε να προστεθεί ένας ακόμα ελεγκτής.
\end{enumerate}

\subsubsection{Για συστήματα μεγαλύτερου βαθμού}
Ένα σύστημα μεγαλύτερου από 2 βαθμού μπορεί να έχει τη μορφή:
\todo{check}
\[
\frac{y(s)}{r(s)}
= \frac{k(s+z_1)\cdots(s+z_n)}{(s+p_1)\cdots(s+p_n)\left[
	(s+ζ_κ\omega_k)^2+(\omega_d)^2
	\right]}
\]

Αν αναλύσουμε σε απλά κλάσματα το παραπάνω, χωρίζοντας πραγματικούς και μιγαδικούς συζυγείς
πόλους, λαμβάνουμε:
\begin{align*}
	y(s) &= \frac{a}{s}
	+ \sum_{i}^{n} \frac{a_i}{s+p_i}
	+ \sum_{k}^q \frac{b_k(s+ζ_k\omega_k)+c_k\omega_{kd}}{s^2+2ζ_k\omega_k s + \omega_k^2}
\intertext{Και, μετασχηματίζοντας αντίστροφα κατά Laplace}
    y(t) &= a
    + \sum a_i e^{-p_it}
    + \sum b_k e^{-ζ_k\omega_k t}\cos \omega_d t
    + \sum c_k e^{-ζ_k\omega_k t}\sin \omega_d t
\end{align*}

\paragraph{Απλοποίηση συστημάτων}

Σε αυτό το σημείο κάνουμε μια σημαντική παρατήρηση: Αν έχουμε δύο μακρινούς πόλους,
για παράδειγμα \(-2\) και \( -20 \), που δίνουν αντίστοιχα αποκρίσεις:
\[
\mathlarger{e^{-2t}}, \quad \mathlarger{e^{-20t}}
\]
\todo{add a graph}%
Η πιο \textit{αργή} απόκριση \( e^{-2t} \) κυριαρχεί τη \textit{γρηγορότερη} απόκριση
\( e^{-20t} \), η οποία αποσβένεται γρήγορα και δεν επηρεάζει το τελικό αποτέλεσμα.

Επομένως, \textbf{μόνο όταν ασχολούμαστε με την ταχύτητα απόκρισης ενός συστήματος}, μπορούμε
προσεγγιστικά να απαλλαγούμε από τους γρήγορους πόλους, αυτούς δηλαδή που βρίσκονται
\textit{πιο μακριά από τον φανταστικό ημιάξονα}.

\paragraph{Παράδειγμα}
Στη συνάρτηση μεταφοράς \( \frac{10}{\left[(s+2)^2+5\right](s+10)} \) μπορούμε να
απαλλαχθούμε από τον πόλο \( p=-10 \), αλλά προσέχουμε ώστε \textbf{να παραμείνει ίδιο το
	DC κέρδος} του απλοποιημένου συστήματος. Δηλαδή η απλοποιημένη συνάρτηση μεταφοράς θα
προκύψει:
\[
\frac{1}{(s+2)^2+5}
\]
\todo{πράξεις}

\paragraph{Παράδειγμα}
Έστω η συνάρτηση μεταφοράς:
\[
H(s) = \frac{100(s+20)}{(s+2)(s+5)(s+50)(s+100)}
\]
\todo{πράξεις}

\paragraph{Παράδειγμα}
Έστω η συνάρτηση μεταφοράς:
\[
\frac{100(s+20)}{(s+2)(s+50)}
\]

Επειδή εδώ έχουμε μηδενικό, δεν μπορούμε να απαλλαγούμε από τον πόλο, διότι θα αλλάξει
εντελώς η απόκριση της συνάρτησης.

Επομένως γενικά προσέχουμε:
\begin{infobox}{Προϋποθέσεις απλοποίησης πόλου}
\begin{enumpar}
	\item Ίδιο DC κέρδος
	\item Προσοχή στην κανονικότητα του συστήματος \todo{συμπλ} (διαφορά αριθμού ριζών αριθμητή και παρονομαστή)
	\item Προσοχή σε μηδενικά κοντά σε πόλο
\end{enumpar}
\end{infobox}

\paragraph{Παράδειγμα με τρία συστήματα}
Να βρεθούν οι βηματικές αποκρίσεις των παρακάτω συστημάτων στο χρόνο:
\begin{align*}
	H_1 &= \frac{3}{s+3} \\
	H_2 &=\frac{60}{(s+3)(s+20)} \\
	H_3 &= \frac{18.75(s+3.2)}{(s+3)(s+20)}
\end{align*}

\begin{enumpar}
	\item
	Πολλαπλασιάζουμε τη συνάρτηση μεταφοράς με την είσοδο \( \frac{1}{s} \):
	\begin{align*}
		y(s) &= \frac{3}{(s+3)s} = \frac{A_1}{s} - \frac{B_1}{s+3}
		\\ &\overset{\cdots}{=} \frac{1}{s} - \frac{1}{s+3}
		\intertext{Άρα}
		y(t) &= 1-e^{-3t}
	\end{align*}
	
	\begin{tikzpicture}[scale=1.2]
	\draw[->] (-1,0) -- (5,0) node[below] {$t$};
	\draw (0,-0.5) -- (0,3.4);
	
	\def\sqrtsix{2.45}
	
	\draw (0,2.5) node[left] {$1$};
	\draw[dashed,path fading=east] (0,2.5) -- (4,2.5);
	\draw[dashed] (3,0) -- (3,2.5);
	\draw (2.2,0) node[circle,fill,inner sep=0.5pt] {} node[below] {$1$};
	\draw (4.4,0) node[circle,fill,inner sep=0.5pt] {} node[below] {$2$};
	
	\draw[blue!70!cyan,very thick]
	plot[smooth,samples=28,variable=\t,domain=0:5]
	(\t,{2.5*(1-exp(-1.5*\t)))});
	\end{tikzpicture}
	
	\item
	Εικάζουμε ότι η απόκριση του συστήματος \( H_2 \) είναι παρόμοια με αυτήν του
	\( H_1 \), αφού (αν κάνουμε τις πράξεις) έχουν ίδιο DC κέρδος, και απλώς προστέθηκε ένας
	μακρινός πόλος στη συνάρτηση μεταφοράς.
	\todo{Πράξεις και graph}
	
	\[
	y(t) = 1-1.17e^{-3t}+0.17e^{-20t}
	\]
	
	\item
	Εδώ έχουμε ένα μηδενικό \( z=-3.2 \) κοντά στον πόλο \( p_1=-3 \), επομένως λύνουμε
	αναλυτικά:
	\begin{align*}
		y(s) &= H_3\frac{1}{s} = \frac{A}{s} + \frac{B}{s+3} + \frac{Γ}{s+20}
		\\ & \overset{\cdots}{=} \frac{1}{s} - \frac{0.0735}{s+3} - \frac{0.92}{s+20}
		\intertext{Άρα}
		y(t) &= 1 - 0.0735e^{-3t} - 0.92e^{-20t}
	\end{align*}
	

	\begin{tikzpicture}[scale=1]
	\draw[->] (-1,0) -- (5,0) node[below] {$t$};
	\draw (0,-0.5) -- (0,3.4);
	
	\def\sqrtsix{2.45}
	
	\draw (0,2.5) node[left] {$1$};
	\draw[dashed] (3,0) -- (3,2.5);
	\draw (2.2,0) node[circle,fill,inner sep=0.5pt] {} node[below] {$1$};
	\draw (4.4,0) node[circle,fill,inner sep=0.5pt] {} node[below] {$2$};
	
	\draw[blue!70!cyan,very thick]
	plot[smooth,samples=28,variable=\t,domain=0:5]
	(\t,{2.5*(1-exp(-1.5*\t)))});
	
	\draw[green!70!cyan,very thick]
	plot[smooth,samples=28,variable=\t,domain=0:5]
	(\t,{2.5*(1-exp(-7*\t)))});
	
	\draw[dashed,path fading=east] (0,2.5) -- (4,2.5);
	\end{tikzpicture}
	
	Εδώ το μηδενικό επηρεάζει το αποτέλεσμα, και η γρήγορη απόκριση του \( (s+20) \)
	κυριαρχεί, οπότε δε θα ήταν σωστό να την αγνοήσουμε.
\end{enumpar}

\subsubsection{Ασκήσεις}
\paragraph{Άσκηση}
Έχουμε τέσσερις συναρτήσεις μεταφοράς:
\begin{align*}
G_1 &= \frac{16}{(s+2)(s+8)} \\
G_2 &= \frac{16}{5}\frac{s+5}{(s+2)(s+8)} \\
G_3 &= \frac{16}{2+\Delta_z}\frac{s+2+\Delta_z}{(s+2)(s+8)}\\
G_4 &= 16\frac{s+1}{(s+2)(s+8)}
\end{align*}
Να υπολογιστεί και να σχεδιαστεί η κρουστική απόκριση των παραπάνω συστημάτων.
\subparagraph{Λύση}
Ουσιαστικά θα πάρουμε έναν αντίστροφο μετασχηματισμό Laplace.

\todo{Να μπει κάπου ο ορισμός του DC κέρδους}

\paragraph{\( G_1 \)}

Δουλεύουμε πάνω στο \( G_1 \) κλασικά, σπάζοντας σε επιμέρους κλάσματα:
\begin{align*}
G_1 &= \frac{A}{s+2} + \frac{B}{s+8} \\
\intertext{και με κάποιον τρόπο βρίσκουμε τους συντελεστές:}
A &= \left. \frac{16}{s+8} \right|_{s=-2} = \frac{16}{6} \\
B &= \left. \frac{16}{s+2} \right|_{s=-8} = -\frac{16}{6}
\end{align*}

και ο αντίστροφος Laplace είναι:
\[
\mathscr{L}^{-1}\left\lbrace G_1 \right\rbrace
= \frac{16}{6}\left( e^{-2t} - e^{-8t} \right)
\]

\subparagraph{\( G_2 \)}
Αντίστοιχα:
\begin{align*}
G_2 &= \frac{A}{s+2} + \frac{B}{s+8} \\
A &= \left. \frac{16}{5} \frac{s+5}{s+8} \right|_{s=-2} = \frac{8}{5} \\
B &= \left. \frac{16}{5} \frac{s+5}{s+2} \right|_{s=-8} = \frac{8}{5}
\end{align*}
άρα:
\[
\mathscr{L}^{-1}\left\lbrace G_2 \right\rbrace = \frac{8}{5}\left( e^{-2t} + e^{-8t} \right)
\]

Παρατηρούμε και στις δύο παραπάνω απαντήσεις την επίδραση του κυρίαρχου, δηλαδή του πιο
αργού πόλου \( s=-2 \) (αντιστοιχεί στο \( e^{-2t} \)), ο οποίος είναι πιο κοντά στο 0.

\subparagraph{\( G_3 \)}
Έχουμε:
\begin{gather*}
	G_3 = \frac{A}{s+2} + \frac{B}{s+8} \\
	A = \left. \frac{16}{2+\Delta_z} \frac{s+2+\Delta_z}{s+8} \right|_{s=-2}
	= \frac{16}{6}\frac{\Delta_z}{2+Δ_z} \\
	B = \left. \frac{16}{2+Δ_z}\frac{s+2+Δ_z}{s+2} \right|_{s=-8} = -\frac{16}{6}
	+\frac{16}{6}\frac{Δ_z}{6} \\
	\mathscr{L}^{-1} \left\lbrace G_3 \right\rbrace =
	\frac{16}{2+Δ_z}\left(
	\frac{Δ_z}{6}e^{-2t} + \left(1-\frac{Δ_z}{6}\right)e^{-8t}
	\right)
\end{gather*}

Παρατηρούμε ότι όσο πιο μικρό (κοντά στο 0) είναι το \( Δ_z \), τόσο λιγότερη σημασία
έχει ο όρος \( e^{-2t} \), και επικρατεί ο γρηγορότερος \( e^{-8t} \). Αυτό μπορούμε να
το διαπιστώσουμε και από τη συνάρτηση μεταφοράς: Aν θεωρήσουμε ότι \( Δ_z \to 0 \),
τότε εξαφανίζονται τα \( (s+2) \) στον αριθμητή και παρονομαστή, και η συνάρτηση γίνεται
\( \frac{16}{2+0} \frac{1}{(s+8)} \).

Για παράδειγμα, αν θέσουμε:
\[
Δ_z = 0.1
\]
τότε η έξοδος θα γίνει:
\[
\frac{16}{6}(
\underbrace{0.047e^{-2t}}_{\mathclap{\text{μικρός όρος}}}
- 0.98e^{-8t}
)
\]

Στην ουσία δηλαδή, το μηδενικό στη θέση \( -(2+Δ_z) \to -2 \) ακυρώνει τον πόλο \( s=-2 \):

\begin{tikzpicture}[scale=1]
\draw[->] (0,-0.5) -- (0,3);
\draw[->] (0,0) -- (4.5,0) node[below] {t};

\draw[dashed] (2.7,2.5) -- ++(0,-2.5) node[below] {0.5};

\draw[thick,cyan!70!blue!70!black,mark position=0.5(a)]
plot[domain=0:4,variable=\x,samples=18,smooth]
(\x,{2.5/2*(2-exp(-4*\x)-exp(-2*\x))});
\draw[cyan!70!blue!50!black,<-]
([yshift=2pt]a) to[bend left] ++(0.5,0.5) node[above right] {$\frac{1}{s+2}$};
\draw[ultra thick,cyan!50!blue!70!black,mark position=0.55(b)]
plot[domain=0:4,variable=\x,samples=15,smooth]
(\x,{2.5*(1-exp(-1.5*\x))});
\draw[cyan!70!blue!50!black,<-]
([yshift=-2pt]b) to[bend left=20] ++(0.5,-0.5) node[below,xshift=-4mm] {$\frac{s+2+\Delta x^{(0.1)}}{(s+2)(s+8)}$};
\end{tikzpicture}

Μπορούμε επίσης να υπολογίσουμε το χρόνο αποκατάστασης του συστήματος:
\[
t_s = 4τ = 4\frac{1}{-p\mathrlap{\leftarrow \text{ πόλος}}}
\hphantom{\leftarrow \text{ πόλος}} = \frac{1}{2} = 0.5
\]

\paragraph{Άσκηση}
Ποιό από τα παρακάτω συστήματα έχει τη βηματική απόκριση που φαίνεται στο σχήμα;

\begin{tikzpicture}[scale=1]
\draw[->] (0,-0.5) -- (0,3);
\draw[->] (0,0) -- (6,0) node[below] {t};

\draw[dashed] (0,2) -- ++(3,0);

\draw[ultra thick,cyan!50!blue!70!black]
plot[domain=0:6,variable=\x,samples=20,smooth]
(\x,{1+1*(1-exp(-2*\x))});

\filldraw (0,1) circle(1pt) node[left] {$2$};
\filldraw (0,2) circle(1pt) node[left] {$4$};
\end{tikzpicture}

\begin{enumlatin}
	\item \( \displaystyle \frac{12}{s+2} \)
	\item \( \displaystyle \frac{2(s+6)}{s+3} \)
	\item \( \displaystyle \frac{2(s+3)}{s+2} \)
	\item \( \displaystyle \frac{6}{s+3} \)
\end{enumlatin}

Για το κάθε σύστημα ψάχνουμε αρχική και τελική τιμή, ώστε να επιβεβαιώσουμε την απόκριση:
\begin{enumlatin}
	\item \( \displaystyle \lim_{s\to \infty}
	\frac{1}{s} \frac{12}{s+3} s = 0
	 \) άρα δεν βολεύει.
	\item \( \displaystyle \lim_{s\to \infty} \frac{1}{s}\frac{2(s+6)}{s+3}s=2 \)
	και \( \displaystyle \lim_{s\to 0} \frac{1}{s}\frac{2(s+6)}{s+3}s=4  \), που είναι οι
	επιθυμητές τιμές.
\end{enumlatin}

Άρα η σωστή απάντηση είναι η \textbf{(β)}.

\paragraph{Άσκηση}
Ποιό από τα παρακάτω συστήματα έχει τη βηματική απόκριση που φαίνεται στο σχήμα;

\begin{tikzpicture}[scale=1]
\draw[->] (0,-0.5) -- (0,3);
\draw[->] (0,0) -- (6,0) node[below] {t};

\draw[dashed] (0,2) -- ++(3,0);

\draw[ultra thick,cyan!50!blue!70!black]
plot[domain=0:6,variable=\x,samples=20,smooth]
(\x,{0.75+1.25*(1-exp(-\x))});

\filldraw (0,0.75) circle(1pt) node[left] {$1$};
\filldraw (0,2) circle(1pt) node[left] {$2$};
\end{tikzpicture}

\begin{enumlatin}
	\item \( \displaystyle \frac{4}{s+2} \)
	\item \( \displaystyle \frac{6}{s+2} \)
	\item \( \displaystyle \frac{3(s+2)}{s+3} \)
	\item \( \displaystyle \frac{s+4}{s+2} \)
\end{enumlatin}

Αντίστοιχα με παραπάνω, υπολογίζουμε αρχική και τελική τιμή για κάθε σύστημα. Η σωστή απάντηση είναι η
\textbf{(δ)}, αφού ισχύει \( \lim_{s\to \infty} \frac{s+4}{s+2} = 1 \) και \( \lim_{s\to 0} \frac{s+4}{s+2} = 2 \).

\paragraph{Άσκηση}
Ποιά συνάρτηση μοιάζει με την παρακάτω απόκριση;

\begin{tikzpicture}[scale=1]
\draw[->] (0,-0.5) -- (0,4);
\draw[->] (0,0) -- (5,0) node[below] {t};

\draw[ultra thick,cyan!50!blue!70!black,mark position=0.9(b),mark position=0.57(a)]
plot[domain=0:5,variable=\x,samples=14,smooth]
(\x,{3.5*(1-exp(-0.8*\x))});

\draw[dashed] (a |- 0,0) node[circle,fill,inner sep=1pt] {} node[below] {$0.6$} |- (a -| 0,0) node[circle,fill,inner sep=1pt] {} node[left] {$9$};
\draw[dashed] (b |- 0,0) node[circle,fill,inner sep=1pt] {} node[below] {$1.2$} |- (b -| 0,0) node[circle,fill,inner sep=1pt] {} node[left] {$10$};
\end{tikzpicture}

\begin{enumlatin}
    \item \( \displaystyle \frac{10}{s+5} \)
	\item \( \displaystyle \frac{20}{s+2} \)
	\item \( \displaystyle \frac{50}{s+2} \)
	\item \( \displaystyle \frac{50}{s+5} \)
\end{enumlatin}

Η τελική τιμή της συνάρτησης είναι το 10, επομένως οι απαντήσεις (α) και (γ) απορρίπτονται, αφού το όριό τους στο \( \infty \)
δεν είναι 10.

Για να βρούμε τη σωστή απάντηση, μπορούμε να βρούμε τον αντίστροφο μετασχηματισμό Laplace των εναπομείναντων συναρτήσεων, ή
πιο εύκολα, να παρατηρήσουμε ότι στα \( \SI{1.2}{s} \) η συνάρτηση φτάνει πολύ κοντά στην τελική της τιμή, δηλαδή
για τη σταθερά χρόνου ισχύει
\( 4τ \approx 1.2 \implies τ \approx 0.3 \). Αυτή η σταθερά χρόνου αντιστοιχεί σε μια συνάρτηση μεταφοράς \( \frac{α}{0.3s+1}
\approx \frac{α'}{s+3.33} \). Άρα η σωστή απάντηση είναι η \textbf{(δ)}.

\paragraph{Άσκηση}
Ποιό από τα παρακάτω συστήματα έχει τη μεγαλύτερη ταχύτητα ανόδου;

\begin{enumlatin}
	\item \( \displaystyle \frac{s+10}{(s+2)^2+5} \)
	\item \( \displaystyle \frac{s+3}{(s+2)^2+5} \)
	\item \( \displaystyle \frac{s+10}{(s+2)^2+10} \)
	\item \( \displaystyle \frac{s+3}{(s+2)^2+10} \)
\end{enumlatin}

\subparagraph{Λύση}
Θυμίζουμε ότι οι παρονομαστές των παραπάνω συναρτήσεων έχουν γραφεί με τη μορφή
\( (s+σ)^2 + \omega_d^2 \), και έχουν λύση \underline{\( -σ\pm j\omega_d \)}.

Η ταχύτητα ανόδου εξαρτάται από τον χρόνο ανόδου.

Θυμόμαστε επίσης ότι για μεγαλύτερο \( \omega_d \) ή μηδενικό \( z \) κοντά στον
φανταστικό άξονα, έχουμε μικρότερο χρόνο ανόδου \( t_{r_1} \),
ενώ ο χρόνος ανόδου \( t_{r_1} \) ορίζεται ως ο χρόνος που απαιτείται μέχρι η απόκριση να φτάσει από
το 10\% ως το 90\% (προσοχή στη διαφορά από το χρόνο \( t_r\) μέχρι την πρώτη φορά που φτάνουμε στην επιθυμητή τιμή):

\begin{tikzpicture}[scale=1.4]
\draw[->] (-0.5,0) -- (5,0) node[below right] {$t$};
\draw[->] (0,-0.5) -- (0,4) node[below left] {$y(t)$};

\draw[thick,dashed] (0,2) node[left] {$y_\mathrm{ss}$} -- ++(6,0);

\draw[dashed] (0.1,-0.1) |- (0,0.2) node[left,scale=.7] {$0.1y_{\mathrm{ss}}$};
\draw[dashed] (0.4,-0.1) |- (0,1.8) node[left,scale=.7] {$0.9y_{\mathrm{ss}}$};

\draw[ultra thick,blue,opacity=.7]
plot [variable=\t,domain=0:6,samples=\gsamples]
(\t,{2*(1-1*exp(-0.8*\t)*cos(3.5*\t r))});

\draw[very thick,cyan,opacity=.2]
plot [variable=\t,domain=0:6,samples=\gsamples]
(\t,{2*(1+1*exp(-0.8*\t))});
\draw[very thick,cyan,opacity=.2]
plot [variable=\t,domain=0:6,samples=\gsamples]
(\t,{2*(1-1*exp(-0.8*\t))});

\draw[dashed] (0.84,0) node[below,xshift=2mm] {$t_p$}-- ++(0,3)
-- (0,3) node[left] {$M_p$};
\draw[dashed] (0.44,0) node[below] {$t_r$} -- ++(0,2);

\filldraw[thick,draw=orange,draw opacity=.5,fill=white!90!orange,fill opacity=.6]
(0.84,3) circle (1.7pt);
\filldraw[thick,draw=orange!50!green,draw opacity=.5,fill=orange!50!green!10!white,fill opacity=.6]
(0.44,2) circle (1.7pt);

\draw[dashed] (3.3,0) node[below] {$t_s\approx\frac{4}{τ}=\frac{4}{ζ\omega_n}$} -- ++(0,2);

\draw[gray] ([yshift=-1mm]0.1,0) -- ++(0,-0.6) node (a) {};
\draw[gray] ([yshift=-1mm]0.4,0) -- ++(0,-0.6) node (b) {};

\draw[<->] ([xshift=1pt]a.north) -- ([xshift=-1pt]b.north) node[below,midway,scale=.7] {$t_{r_1}$};
\end{tikzpicture}

Η σωστή απάντηση είναι η \textbf{(δ)}, η οποία έχει το μεγαλύτερο \( \omega_d \) και το
μηδενικό \( z=-3 \) που είναι κοντινότερο στον άξονα.

(Βέβαια γρήγορη ταχύτητα ανόδου δεν σημαίνει και γρήγορη απόκριση).

\paragraph{Άσκηση}
Πόσο πρέπει να είναι τα \( k_p \) ώστε να έχουμε μοναδιαίο DC κέρδος, και ποιός είναι ο
χρόνος αποκατάστασης του κάθε συστήματος;

\begin{enumlatin}
	\item \( \displaystyle \frac{k_{p_1}}{s+3} \)
	\item \( \displaystyle \frac{k_{p_2}}{(s+3)(s+20)} \)
\end{enumlatin}

\subparagraph{Απάντηση}
\begin{enumlatin}
	\item \( k_{p_1} = 3 \), η χρονική σταθερά είναι \( τ=\frac{1}{3} \), άρα
	ο χρόνος αποκατάστασης προκύπτει
	\( t_s = \frac{4}{3} \).
	\item \( k_{p_2} = 60 \), ο κυρίαρχος πόλος είναι ο πιο κοντινός (\( s=-3 \)), και
	επομένως \( τ=\frac{1}{4} \) και \( t_s = \frac{4}{3} \).
\end{enumlatin}

Σε περίπτωση που μας δινόταν \( \displaystyle \frac{K_{p_2}(s+3.1)}{(s+3)(s+20)} \), το
μηδενικό στο -3.1 θα αλληλοαναιρούνταν με τον πόλο στο -3, και επομένως ο κυρίαρχος πόλος θα
ήταν στο -20.

\paragraph{Άσκηση}
Έστω η συνάρτηση μεταφοράς:
\[
\frac{18}{s^2+11s+18}
\]
Ποιός είναι ο χρόνος αποκατάστασής της;

\begin{attnbox}{}
	Έχουμε \textbf{δύο πραγματικές ρίζες}, οπότε δεν είναι σωστό να ακολουθήσουμε τη
	μεθοδολογία με τα \( σ \) και \( \omega_d \).
\end{attnbox}

Οι πόλοι του συστήματος βγαίνουν -2 και -9, ο κυρίαρχος είναι ο -2, επομένως \( t_s=\frac{4}{2} = 2 \).



\subsection{Εύρος ζώνης}
Το \textbf{εύρος ζώνης} είναι μία προδιαγραφή του συστήματος που σχετίζεται με το πεδίο
της συχνότητας. Πιο συγκεκριμένα, μπορεί να απαιτήσει να έχουμε ένα συγκεκριμένο
\textbf{εύρος ζώνης} \( \omega_b \), για παράδειγμα \( \omega_b > a \).

Η μελέτη αυτής της προδιαγραφής μπορεί να μελετηθεί με τη βοήθεια των
\textbf{διαγραμμάτων Bode} (προφ. "Μπόντε"), τα οποία εκφράζουν το πλάτος και τη φάση
για κάθε συχνότητα.

Ο έλεγχος γίνεται δίνοντας ως είσοδο στο σύστημα ένα μεγάλο εύρος συχνοτήτων:

\begin{circuitikz}
	\draw (0,0) node[rectangle,draw,minimum height=7mm,minimum width=5mm]
	(h) {$H(s)$};
	
	\draw (-1,0) node[oscillator] (osc) {};
	
	\draw (h.east) to[short,-o] ++(1,0);
	\draw (h.west) to[short,-*] (osc.east);
\end{circuitikz}

και στη συνέχεια κατασκευάζουμε δύο διαγράμματα, ένα για το πλάτος ως συνάρτηση της
συχνότητας, και ένα για τη φάση, ως εξής:

\begin{tikzpicture}
\draw[->] (0,-5.5) -- (0,3) node[above left] {$|A|$};
\draw[->] (-1,0) -- (6,0) node[below right] {$\omega$};

\draw[very thick,blue!50!cyan!70!black,name path=FR] plot[smooth,tension=1]
coordinates {(0,1.5) (1.5,1.6) (3,2.5) (4.5,1.4) (6,0.7) };

\path[name path=l] (0,1.5*0.707)-- ++(6,0);

\path [name intersections={of=FR and l,by=G}];
\draw[dashed] (G) -- (G -| 0,0) node[left] {$0.707$};
\draw (0,1.5) node[left] {$1$};
\draw[dashed] (3,2.5) -- (3,0);
\draw[dashed] (G) -- (G |- 0,0) node[below] {$\omega_b$};

\draw[->] (0,-2.2) node[above left] {$φ$} -- ++(6,0) node[right] {$\omega$};

\draw[very thick,blue!50!cyan!50!green!70!black,name path=FR,yshift=-4cm] plot[smooth,variable=\x,domain=0:6,samples=18]
(\x,{3*atan((-\x +3)*3)/180});

\draw (0,-2.6) node[left] {$0\degree$};
\draw[dashed] (0,{(-2.6-5.3)/2}) node[left] {$45\degree$} -- ++(3,0);
\draw (0,-5.3) node[left] {$90\degree$};
\end{tikzpicture}

Τα περισσότερα συστήματα είναι χαμηλοπερατά.

Όσο μεγαλύτερο είναι το εύρος ζώνης ενός συστήματος, τόσο ταχύτερη είναι η απόκριση του
συστήματος. Αυτό μπορούμε να το διαπιστώσουμε αν σκεφτούμε την τετραγωνική κυματομορφή,
που έχει συχνότητες που εκτείνονται μέχρι το άπειρο (αν θυμηθούμε τις σειρές Fourier),
επομένως ένα σύστημα που θέλει να την αναπαραστήσει, όσο μεγαλύτερο εύρος ζώνης έχει,
τόσο πιο γρήγορα θα αποκριθεί στην είσοδο.

\begin{tikzpicture}
\draw[very thick,blue!50!cyan!]
plot[smooth,tension=.6]
%plot[smooth,variable=\x,domain=0:6,samples=12]
%(\x,{4/pi*1/1*sin(1*pi*\x r) + 4/pi*1/3*sin(3*pi*\x r)});
coordinates {(0,0) (0.5,1) (0.8,0.7) (1,0) (1.5,0) (2,0) (2.3,0.8) (2.9,1) (3,0) (4,0) (4.2,1) (4.8,1) (4.9,0) (5.2,0)};

\draw[very thick] plot[const plot]
coordinates {(0,0) (0,1) (1,0) (2,1) (3,0) (4,1) (5,0) };
\end{tikzpicture}

Επομένως, είναι επιθυμητό να έχουμε μεγάλο εύρος ζώνης για να αυξήσουμε την ταχύτητα, αλλά
πρέπει να λάβουμε σε αυτήν την περίπτωση υπ' όψιν και τον θόρυβο, ο οποίος βρίσκεται κυρίως
στις υψηλότερες συχνότητες.

Τα διαγράμματα Bode γίνονται συνήθως σε λογαριθμική κλίμακα, που ορίζεται ως εξής:
\[
20\log\left\lvert \frac{\omega}{\omega_0} \right\rvert
\]

\begin{tikzpicture}
\draw[->] (0,-4) -- (0,1.5);
\draw[->] (0,0) -- (6,0);

\draw[very thick,blue!50!cyan!70!black,name path=FR,mark position=0.9(c)] plot[smooth,tension=0.7,yscale=1.2]
coordinates {(0,0) (1.7,0.7)  (4.5,-0.3) (6,-0.5)};

\draw (0,0) node[left] {0 dB};
\draw[dashed] (c) -- (c -| 0,0) node[left] {-3 dB};

\path[name path=l] (0,1.5*0.707)-- ++(6,0);

\draw[->] (0,-2.2) -- ++(6,0);

\draw[very thick,blue!50!cyan!50!green!70!black,name path=FR,yshift=-3cm] plot[smooth,variable=\x,domain=0:6,samples=18]
(\x,{1.5*atan((-\x +3)*3)/180});
\end{tikzpicture}

και ορίζουμε το εύρος ζώνης να είναι η απόσταση μεταξύ των συχνοτήτων στις οποίες η έξοδος
έχει τιμή \(-3\) dB, που αντιστοιχεί στην τιμή 0.707, αν το πλάτος 1 αντιστοιχεί σε 0 dB.

Προσοχή ότι η λογαριθμική κλίμακα στον οριζόντιο άξονα των συχνοτήτων δηλώνει ότι στην
αρχή των αξόνων δεν έχουμε τη συχνότητα 0, αλλά απλώς μία μικρή συχνότητα.

\subsubsection{Σε πρωτοβάθμια συστήματα}
Ας ασχοληθούμε με το σύστημα \( \displaystyle H(s)=\frac{1}{1+sτ} \).

Στην απόκριση συχνότητας μάς ενδιαφέρει η έξοδος του συστήματος σε ημιτονοειδείς όρους,
δηλαδή για \( s = j\omega  \), οπότε:
\[
H(j\omega ) = \frac{1}{1+j\omega τ}
\]

Η απόκριση φάσης είναι:
\[
φ(\omega) = \arg\left( H(j\omega) \right) = \arg\left(\frac{1-j\omega τ}{(1+j\omega τ)(1-j\omega τ)}\right)
= -\tan^{-1}
\frac{
	\overbrace{
	\frac{-\omega τ}{(1+j\omega τ)(1-j\omega τ)}
}^{\mathclap{\text{φανταστικό μέρος}}}
}{
	\underbrace{\frac{1}{(1+j\omega τ)(1-j\omega τ)}}_{\mathclap{\text{πραγματικό μέρος}}}
} =
-\tan^{-1}(\omega τ)
\]

Η απόκριση συχνότητας τότε θα είναι:
\begin{align*}
	20\log\left\lvert H(j\omega ) \right\rvert &=
	-20\log\left\lvert 1+j\omega τ \right\rvert \\
	&= -20\log \left( 1+\omega^2 τ^2 \right)^{\frac{1}{2}}
	\\ &= -10\log(1+\omega^2τ^2)
\end{align*}

Ο σκοπός όμως είναι την παραπάνω έκφραση να γνωρίζουμε προσεγγιστικά πώς μπορεί να
παρουσιαστεί σε διάγραμμα.

Ψάχνουμε τις ασύμπτωτες για την παραπάνω σχέση, και έχουμε:
\[
\begin{cases}
\text{για } \omega \ll \frac{1}{τ} &\quad =0\ \mathrm{dB} \\
\text{για } \omega \gg \frac{1}{τ} &\quad =-20\log \omega τ = \underbrace{-20\log\omega}_{\omega_c=\frac{1}{\tau}} -20\log\tau
\end{cases}
\]
(όπου ονομάζουμε την \( \omega_c = \frac{1}{\tau} \) χαρακτηριστική συχνότητα του συστήματος)

Έτσι, έχουμε δύο ασύμπτωτες ευθείες, και μπορούμε να προσεγγίσουμε τη συνάρτηση:

\begin{tikzpicture}[scale=1.2]
\draw[->] (0,-3) -- (0,2);
\draw[->] (0,0) -- (4,0) node[below] {$\omega$};

\draw[ultra thick,blue!50!cyan!70!black,mark position=0.67(c)]
plot[variable=\x,domain=0:4,samples=18,smooth,yshift=-1pt]
(\x,{-10*0.06*ln(1+    ( (exp(ln(10)*(\x-3)))/1  )^2     )});

\draw[ultra thick,orange!90!blue!90!black,opacity=.7]
(0,0) -- (3,0) -- ++(-68:3);

\draw (current bounding box.north east) node[minimum width=2mm,minimum height=2mm,rectangle,draw,fill=orange!90!blue!90!black] (a) {};
\draw (a.south) ++(0,-0.5) node[minimum width=2mm,minimum height=2mm,rectangle,draw,fill=blue!50!cyan!70!black] (b) {};

\draw (a.east) node[right] {προσεγγιστική απόκριση};
\draw (b.east) node[right] {πραγματική απόκριση};
\end{tikzpicture}

(Η χειρότερη προσέγγιση ίσως γίνεται στο σημείο όπου τέμνονται οι ευθείες, στο οποίο η
πραγματική τιμή είναι \( -10\log2 = -3\ \mathrm{dB} \)).

Για το σχεδιασμό του διαγράμματος φάσης, έχουμε:
\[
\begin{cases}
\text{για } \omega = 0 &\quad = 0\\
\text{για } \omega =\infty &\quad = -90\degree\\
\text{για } \omega = \omega_c &\quad = -45\degree
\end{cases}
\]

\begin{tikzpicture}
\draw[->] (0,-3.5) -- (0,2);
\draw[->] (0,0) -- (4,0) node[below] {$\omega$};

\draw[ultra thick,blue!50!cyan!70!black,mark position=0.67(c)]
plot[variable=\x,domain=0:3.5,samples=18,smooth,yshift=-1pt]
(\x,{-10*0.02*ln(1+    ( (exp(ln(10)*(\x-2)))/1  )^2     )});

\draw[ultra thick,orange!90!blue!90!black,opacity=.7]
(0,0) -- (2,0) -- ++(-40:2);

\draw (current bounding box.north east) node[minimum width=2mm,minimum height=2mm,rectangle,draw,fill=orange!90!blue!90!black] (a) {};
\draw (a.south) ++(0,-0.5) node[minimum width=2mm,minimum height=2mm,rectangle,draw,fill=blue!50!cyan!70!black] (b) {};
%\draw (a.east) ++(0.25,0) node[minimum width=2mm,minimum height=2mm,rectangle] (a) {};
\draw (b.east) ++(0.25,0) node[minimum width=2mm,minimum height=2mm,rectangle,draw,fill=blue!50!cyan!50!green!70!black] (b) {};

\draw (a.east) node[right] {προσεγγιστική απόκριση};
\draw (b.east) node[right] {πραγματική απόκριση};

\def\plh{-2.2}
\def\ptl{0.15}

\draw[->] (0,\plh) -- ++(4.5,0);

\begin{scope}[every node/.style={scale=.8,yshift=2pt}]
\draw (0,\plh) node[above right] {$\sfrac{0.01}{τ}$};
\draw (1,\plh) node[above] {$\sfrac{0.1}{τ}$} ++(0,\ptl/2) -- ++(0,-\ptl);
\draw (2,\plh) node[above] {$\sfrac{1}{τ}$} ++(0,\ptl/2) -- ++(0,-\ptl);
\draw (3,\plh) node[above] {$\sfrac{10}{τ}$} ++(0,\ptl/2) -- ++(0,-\ptl);
\draw (4,\plh) node[above] {$\sfrac{100}{τ}$} ++(0,\ptl/2) -- ++(0,-\ptl);

\draw[dashed] (0,-2.7) node[left] {$-45\degree$} -| (2,0);
\draw (0,-2.7-0.5) node[left] {$-90\degree$};
\draw[->] (0,-2.7-0.5) -- ++(4.5,0);
\end{scope}

\draw[very thick,blue!50!cyan!50!green!70!black,name path=FR,yshift=-2.7cm] plot[smooth,variable=\x,domain=0:4.4,samples=18]
(\x,{1*atan((-\x +2)*5)/180});
\end{tikzpicture}

Επίσης, μπορούμε να κάνουμε μια γραμμική προσέγγιση αυτής της εξόδου στη φάση, θεωρώντας
οριζόντια έξοδο μία δεκάδα πριν και μετά από τη χαρακτηριστική συχνότητα, και γραμμική
ανάμεσα:

\begin{tikzpicture}[scale=1.2]
\draw[->] (0,-3.5) -- (0,2);
\draw[->] (0,0) -- (4,0) node[below] {$\omega$};

\draw[ultra thick,blue!50!cyan!70!black,mark position=0.67(c)]
plot[variable=\x,domain=0:3.5,samples=18,smooth,yshift=-1pt]
(\x,{-10*0.02*ln(1+    ( (exp(ln(10)*(\x-2)))/1  )^2     )});

\draw[ultra thick,orange!90!blue!90!black,opacity=.7]
(0,0) -- (2,0) -- ++(-40:2);

\draw (current bounding box.north east) node[minimum width=2mm,minimum height=2mm,rectangle,draw,fill=orange!90!blue!90!black] (a) {};
\draw (a.south) ++(0,-0.5) node[minimum width=2mm,minimum height=2mm,rectangle,draw,fill=blue!50!cyan!70!black] (b) {};
\draw (a.east) ++(0.25,0) node[minimum width=2mm,minimum height=2mm,rectangle,draw,fill=magenta] (a) {};
\draw (b.east) ++(0.25,0) node[minimum width=2mm,minimum height=2mm,rectangle,draw,fill=blue!50!cyan!50!green!70!black] (b) {};

\draw (a.east) node[right] {προσεγγιστική απόκριση};
\draw (b.east) node[right] {πραγματική απόκριση};

\def\plh{-2.2}
\def\ptl{0.15}

\draw[->] (0,\plh) -- ++(4.5,0);

\begin{scope}[every node/.style={scale=.8,yshift=2pt}]
\draw (0,\plh) node[above right] {$\sfrac{0.01}{τ}$};
\draw (1,\plh) node[above] {$\sfrac{0.1}{τ}$} ++(0,\ptl/2) -- ++(0,-\ptl);
\draw (2,\plh) node[above] {$\sfrac{1}{τ}$} ++(0,\ptl/2) -- ++(0,-\ptl);
\draw (3,\plh) node[above] {$\sfrac{10}{τ}$} ++(0,\ptl/2) -- ++(0,-\ptl);
\draw (4,\plh) node[above] {$\sfrac{100}{τ}$} ++(0,\ptl/2) -- ++(0,-\ptl);

\draw[dashed] (0,-2.7) node[left] {$-45\degree$} -| (2,0);
\draw (0,-2.7-0.5) node[left] {$-90\degree$};
\draw[->] (0,-2.7-0.5) -- ++(4.5,0);
\end{scope}

\draw[very thick,blue!50!cyan!50!green!70!black,name path=FR,yshift=-2.7cm] plot[smooth,variable=\x,domain=0:4.4,samples=18]
(\x,{1*atan((-\x +2)*5)/180});

\draw[gray,dashed] (3,\plh) -- ++(0,-1);

\draw[ultra thick,magenta,yshift=-2.7cm]
(0,0.5) -- (1,0.5) -- (3,-0.5) -- (4.4,-0.5);
\end{tikzpicture}

δηλαδή:
\[
φ = -45-45\log|\omega_t|
\]

Αν είχαμε μια συνάρτηση με ένα κέρδος \( κ \neq 1 \), τότε στο διάγραμμα μέτρου απλώς
προστίθεται μια σταθερά \( 20\log κ \), και το διάγραμμα θα ανεβεί προς τα πάνω ή προς τα
κάτω, ανάλογα με το αν \( κ > 1 \) ή \( κ < 1 \) αντίστοιχα.

Με τα διαγράμματα Bode μπορούμε επομένως να βρούμε την έξοδο του συστήματος σε ημιτονοειδή
είσοδο. Για την απόκριση ταχύτητας, ενδιαφέρουν περισσότερο τα διαγράμματα πλάτους και όχι
φάσης.

\subsubsection{Σε δευτεροβάθμια συστήματα}
Θυμίζουμε τη συνάρτηση μεταφοράς δευτεροβάθμιων συστημάτων:
\[
H(s) = \frac{\omega_n^2}{s^2+2ζ\omega_n s + \omega_n^2}
\]

Όπως παραπάνω, για να πάρουμε την απόκριση συχνότητας θεωρούμε \( s = j\omega  \):
\begin{align*}
	H(j\omega )&= \frac{1}{1-\left(\frac{\omega }{\omega_n}\right)^2
		+j2ζ\frac{\omega}{\omega_n}} \\
	\left\lvert H(j\omega ) \right\rvert_{\mathrm dB}
	&= -10\log\left[
	\left(1-\frac{\omega^2}{\omega_n^2}\right)^2+4ζ^2\frac{\omega^2}{\omega_n^2}
	\right]
\end{align*}

Κάνουμε προσέγγιση με ασυμπτώτους:
\[
\begin{cases}
	\frac{\omega}{\omega_n} \ll 1 &\quad 0\ \mathrm{dB} \\
	\frac{\omega}{\omega_n} \gg 1 &\quad -40\log\frac{\omega}{\omega_n} \\
%	\omega = \omega_n &\quad 0 \ \mathrm{dB}
\end{cases}
\]

\begin{tikzpicture}[scale=1.2]
\draw[->] (0,-2) -- (0,1);
\draw[->] (0,0) -- (4,0) node[below] {$\omega$};

\draw[ultra thick,orange!90!blue!80!black,opacity=1]
(0,0) -- (2,0) -- ++(-40:3)
node[midway,right] {-40 dB/dec};

\filldraw (2,0) circle (1pt) node[above] {$\omega_n$};

\draw (current bounding box.north east) node[minimum width=2mm,minimum height=2mm,rectangle,draw,fill=orange!90!blue!80!black] (a) {};

\draw (a.east) node[right] {προσεγγιστική απόκριση};
\end{tikzpicture}

Στην πραγματικότητα, η απόκριση συχνότητας μοιάζει ως εξής, και εξαρτάται από το \( ζ \):

\begin{tikzpicture}[scale=1.2]
\draw[->] (0,-2) -- (0,1.5);
\draw[->] (0,0) -- (4,0) node[below] {$\omega$};

%\draw[ultra thick,blue!50!cyan!70!black,mark position=0.67(c)] plot[variable=\x,domain=0:4,samples=18,smooth,yshift=-1pt] (\x,{-10*0.06*ln(1+    ( (exp(ln(5)*(\x-3)))/1  )^2     ) +10*0.02*ln(1+    ( (exp(ln(10/4)*(\x-1)))/1  )^2     ) });

\draw[blue!50!cyan!70!black]
(2.25,0) -- ++(0,1);

\draw[ultra thick,blue!50!cyan!70!black,mark position=0.67(c)]
plot[smooth] coordinates {(0,0) (1.5,0.05) (2.5,0.1) (4.4,-1.8)};
\draw[ultra thick,blue!50!cyan!70!black,mark position=0.67(c)]
plot[smooth] coordinates {(0,0) (1.5,0.2) (2.5,0.5) (4.4,-1.75)};
\draw[ultra thick,blue!50!cyan!70!black,mark position=0.67(c)]
plot[smooth,tension=.6] coordinates {(0,0) (1.5,0.4) (2.5,0.9) (4.4,-1.7)};

\filldraw[blue!50!cyan!70!black]
(2.25,0) circle (1pt) node[below right,scale=.9] {$\omega_r$};

\draw[ultra thick,orange!90!blue!80!black,opacity=1]
(0,0) -- (2,0) -- ++(-40:3)
node[midway,below left] {-40 dB/dec};

\filldraw (2,0) circle (1pt) node[above] {$\omega_n$};

\draw (current bounding box.north east) node[minimum width=2mm,minimum height=2mm,rectangle,draw,fill=orange!90!blue!80!black] (a) {};
\draw (a.south) ++(0,-0.5) node[minimum width=2mm,minimum height=2mm,rectangle,draw,fill=blue!50!cyan!70!black] (b) {};

\draw (a.east) node[right] {προσεγγιστική απόκριση};
\draw (b.east) node[right] {πραγματική απόκριση};
\end{tikzpicture}

Παρατηρούμε μία υπερύψωση κοντά στο \( \omega_n \) (αλλά όχι ακριβώς επάνω του), που αυξάνεται
όσο μειώνεται το \( ζ \), και συχνά δεν είναι επιθυμητή.

Αν παραγωγίσουμε την έξοδο για να βρούμε το ακρότατο (θέτουμε \( u=\frac{\omega}{\omega_n} \) για ευκολία):
\[
\od{H}{u} = \frac{1}{\sqrt{1-u^2+(2ζu)^2}}
\]
η οποία μηδενίζεται για \( \frac{\omega}{\omega_n} = u = \sqrt{1-2ζ^2} \). Αυτό ορίζεται
αν \( ζ<0.707 \), δηλαδή υπάρχει υπερύψωση μόνο όταν \( \boxed{ζ<0.707} \).

\begin{theorem}{Εύρος ζώνης δευτεροβάθμιου συστήματος}{}
	Το \emph{εύρος ζώνης} (μέχρι τα -3 dB) αποδεικνύεται ότι είναι:
	\[
	\omega_b = \omega_n \left[
	1-2ζ^2+\sqrt{2-4ζ^2+4ζ^4}
	\right]^{\sfrac{1}{2} }
	\]
\end{theorem}

\begin{theorem}{Συχνότητα συντονισμού δευτεροβάθμιου συστήματος}
	Η \textbf{συχνότητα συντονισμού} \( \omega_r \), δηλαδή η συχνότητα κατά την οποία
	έχουμε υπερύψωση, είναι:
	\[
	\omega_r = \omega_n\sqrt{1-2ζ^2} \qquad ζ < 0.707
	\]
	ενώ το πλάτος της υπερύψωσης:
	\[
	M_r = \frac{1}{2ζ\sqrt{1-ζ^2}}
	\]
\end{theorem}

Όσον αφορά τη φάση,
\[
\phi(\omega) = -\tan^{-1}\frac{2ζ\frac{\omega}{\omega_n}}{1-\frac{\omega^2}{\omega_n^2}}
\]
Ασυμπτωτικά:
\[
\begin{cases}
\frac{\omega}{\omega_n} = 0 &\quad φ = 0\\
\frac{\omega}{\omega_n} = 1 &\quad φ = -\sfrac{π}{2} \\
\frac{\omega}{\omega_n} \to \infty &\quad φ=-π
\end{cases}
\]

και συμπληρώνουμε το διάγραμμα Bode:

\begin{tikzpicture}[scale=1.2]
\draw[->] (0,-4) -- (0,1.5);
\draw[->] (0,0) -- (4,0) node[below] {$\omega$};
\draw[->] (0,-2) -- ++(4.5,0);

%\draw[ultra thick,blue!50!cyan!70!black,mark position=0.67(c)] plot[variable=\x,domain=0:4,samples=18,smooth,yshift=-1pt] (\x,{-10*0.06*ln(1+    ( (exp(ln(5)*(\x-3)))/1  )^2     ) +10*0.02*ln(1+    ( (exp(ln(10/4)*(\x-1)))/1  )^2     ) });

\draw[dashed] (2,0) -- ++(0,-3) -- ++(-2,0);

\draw[blue!50!cyan!70!black]
(2.25,0) -- ++(0,1);

\draw[ultra thick,blue!50!cyan!70!black,mark position=0.67(c)]
plot[smooth] coordinates {(0,0) (1.5,0.05) (2.5,0.1) (4.4,-1.8)};
\draw[ultra thick,blue!50!cyan!70!black,mark position=0.67(c)]
plot[smooth] coordinates {(0,0) (1.5,0.2) (2.5,0.5) (4.4,-1.75)};
\draw[ultra thick,blue!50!cyan!70!black,mark position=0.67(c)]
plot[smooth,tension=.6] coordinates {(0,0) (1.5,0.4) (2.5,0.9) (4.4,-1.7)};

\filldraw[blue!50!cyan!70!black]
(2.25,0) circle (1pt) node[below right,scale=.9] {$\omega_r$};

\draw[ultra thick,orange!90!blue!80!black,opacity=1]
(0,0) -- (2,0) -- ++(-40:3)
node[midway,below left] {-40 dB/dec};

\filldraw (2,0) circle (1pt) node[above] {$\omega_n$};

\draw (current bounding box.north east) node[minimum width=2mm,minimum height=2mm,rectangle,draw,fill=orange!90!blue!80!black] (a) {};
\draw (a.south) ++(0,-0.5) node[minimum width=2mm,minimum height=2mm,rectangle,draw,fill=blue!50!cyan!70!black] (b) {};
\draw (a.east) ++(0.25,0) node[minimum width=2mm,minimum height=2mm,rectangle,draw,fill=magenta] (a) {};

\draw (a.east) node[right] {προσεγγιστική απόκριση};
\draw (b.east) node[right] {πραγματική απόκριση};

\draw[ultra thick,magenta,name path=FR,yshift=-3cm] plot[smooth,variable=\x,domain=0:4.4,samples=18]
(\x,{1.5*atan((-\x +2)*5)/180});
\end{tikzpicture}

Βέβαια η πραγματική συμπεριφορά εξαρτάται πάλι από το \( ζ \).

\paragraph{Άσκηση}
Έστω ένα σύστημα:

\begin{circuitikz}
	\draw (0,0) node[circle,inner sep=2mm,draw] (sum) {};
	\draw (1.5,0) node[rectangle,minimum height=5mm,draw] (H) {$\frac{k}{s(s+a)}$};
	
	\draw[<-] (sum) -- ++(-1.5,0);
	\draw (sum) -- (H);
	\draw[->] (H) -- ++(2,0) node[midway] (P) {};
	\draw[->] (P.center) -- ++(0,-1)  -| (sum);
\end{circuitikz}

Ζητείται να βρεθούν οι σταθερές \( k \) και \( a \), ώστε η υπερύψωση συχνότητας να είναι
\( Μ_{\mathrm{pr}} = 1.04 \) για \( \omega_r = 11.54 \ \sfrac{\mathrm{rad}}{s} \). Και από
εκεί, ζητείται να προσδιοριστεί ο χρόνος αποκατάστασης \( t_s \) και το εύρος ζώνης
\( \omega_b \).
\subparagraph{Λύση}
Από τους παραπάνω τύπους, έχουμε:
\begin{align*}
	\frac{1}{2ζ\sqrt{1-ζ^2}} = 1.04 \implies
	4ζ^4-4ζ^2+0.92 = 0 \implies ζ^2 = \begin{cases}
	0.64 \\ 0.352
	\end{cases} \implies \cancel{ζ=0.8 } \text{ ή } \boxed{ζ=0.6}
\end{align*}

Απορρίπτουμε την περίπτωση \( ζ=0.8 \), αφού για να υπάρχει υπερύψωση στη συχνότητα πρέπει
\( ζ < 0.707 \). Άρα \( ζ = 0.6 \).

Τότε υπολογίζουμε ακόμα ότι
\( \omega_n = \frac{\omega_r}{\sqrt{1-2ζ^2}} \).

Η συνάρτηση μεταφοράς του συστήματος βγαίνει μετά από πράξεις \( \displaystyle 
\frac{k}{s^2+as+k}  = \frac{\omega_n^2}{s^2+2ζ\omega_n s + \omega_n^2} \).

Άρα \( k=21.67^2=469.59 \), και \( a=2ζ\omega_n = 2\cdot 0.6\cdot 21.67 =26.00 \).

\todo{ts,\omega_b}

\subsubsection{Σε συστήματα μεγαλύτερου βαθμού}

Γενικά ισχύει:
\[
H(s) = \frac{G}{s^n}\frac{(τ_js+1)\cdots}{(t_is+1)\cdots}
\]

Αυτό αντιστοιχεί, όπως έχουμε δει παραπάνω, σε ένα άθροισμα πιο απλών κλασμάτων. Επομένως,
για να βρούμε το διάγραμμα Bode, απλώς προσθέτουμε τις επιμέρους αποκρίσεις.

Επίσης, αν έχουμε ένα σύστημα \( \frac{1}{τs+1} \), και το αντιστρέψουμε, \(\frac{τs+1}{1} \),
	η έξοδος θα είναι αντίθετη, από την ιδιότητα του λογαρίθμου.
	
\begin{tikzpicture}[scale=0.89]
\draw[->] (0,-3) -- (0,1.5);
\draw[->] (0,0) -- (4.5,0) node[below] {$\omega$};
\draw[->] (0,-2) -- ++(4.5,0);

\draw[ultra thick,blue!50!cyan!70!black,mark position=0.67(c)]
plot[variable=\x,domain=0:4,samples=18,smooth]
(\x,{-10*0.02*ln(1+    ( (exp(ln(10)*(\x-2.2)))/1  )^2     )});
\draw[very thick,blue!50!cyan!50!green!70!black,name path=FR,yshift=-2.7cm] plot[smooth,variable=\x,domain=0:4.4,samples=18]
(\x,{1*atan((-\x +2)*5)/180});

\draw[very thick,>->] (5,-1) -- ++(0.7,0);

\begin{scope}[xshift=6cm]
\draw[->] (0,-3) -- (0,1.5);
\draw[->] (0,0) -- (4.5,0) node[below] {$\omega$};
\draw[->] (0,-2) -- ++(4.5,0);

\draw[ultra thick,blue!50!cyan!70!black,mark position=0.67(c)]
plot[variable=\x,domain=0:4,samples=18,smooth]
(\x,{10*0.02*ln(1+    ( (exp(ln(10)*(\x-2.2)))/1  )^2     )});
\draw[very thick,blue!50!cyan!50!green!70!black,name path=FR,yshift=-1.4cm] plot[smooth,variable=\x,domain=0:4.4,samples=18]
(\x,{-1*atan((-\x +2)*5)/180});
\end{scope}

\end{tikzpicture}
	
\paragraph{}
Για το απλό σύστημα \( \displaystyle H(s)=\frac{G}{s} \), έχουμε:
\begin{align*}
	|H|_{\mathrm{dB}} &= 20\log G - 20\log\omega \\
	φ &= -\sfrac{π}{2} 
\end{align*}

και το διάγραμμα Bode γίνεται (σε συνδυασμό με το \( H(s) = Gs \)):

\begin{tikzpicture}[scale=1]
\draw[->] (0,-3) -- (0,1.5);
\draw[->] (0,0) node[left] {0 dB} -- (4.5,0) node[below] {$\omega$};
\draw[->] (0,-1.5) -- ++(4.5,0);

\draw[ultra thick,orange!90!blue!80!black]
(0,1.4) -- (4,-0.7) node[pos=.3,above,sloped] {-20 dB/dec};
\draw[ultra thick,orange!50!blue!80!black]
(0,-0.7) -- (4,1.4) node[pos=.08,below right] {+20 dB/dec};
\draw[ultra thick,magenta,yshift=-2.4cm] (0,0) node[left] {$\ang{90}$} -- ++(4.5,0);

\filldraw (2.68,0) circle(1pt) node[above right] {$G$};
\filldraw (1.34,0) circle(1pt) node[below right] {$\frac{1}{G}$};
\end{tikzpicture}

Γενικά, μπορούμε να συνθέσουμε οποιαδήποτε συνάρτηση μεταφοράς με τα παρακάτω στοιχεία:
\begin{itemize}
	\item \( H_0 = Gs^ν \)
	\item \( \displaystyle H_1 = \left( \frac{s}{ρ}+1 \right) \) για \( p>0 \)
	\item \( \displaystyle H_2 = \left( \frac{s^2}{\omega_n^2} + \frac{2Js}{\omega_n} + 1 \right) \) για \( \omega_n > 0 \) και \( 0<ζ<1 \).
\end{itemize}

Για παράδειγμα, \( \frac{s(s+3)}{(s+4)\left( (s+5)^2+4 \right)} \). \todo{solve}

\todo{add warning above that log != ln}
Για καθεμία από αυτές, έχουμε:
\begin{itemize}
	\item
	\( \displaystyle \left\lvert H_0(j\omega)\right\rvert_{\mathrm{dB}} 
	= 20\log G\omega^ν = 20\log G + 20ν\log ω
	\)
	και:\\
	\( \phi_0 = ν\cdot 90\degree \)
	
	Και το αντίστοιχο διάγραμμα Bode είναι:
	
	\begin{tikzpicture}[scale=1]
	\draw[->] (0,-3) -- (0,3.5) node[left,scale=1.2] {$H_0$};
	\draw[->] (0,0) -- (4.5,0) node[above] {$\omega$};
	\draw[->] (0,-2.5) -- ++(4.5,0) node[above] {$\omega$};
	
	\draw[ultra thick,orange!90!blue!80!black]
	(0,0) -- (4.5,3/4*4.5);
	
	\draw (0,2*3/4) node[left] (a) {};
	\draw (0,3*3/4) node[left] (b) {};
	\draw[<->] (a) -- (b) node[midway,left] {20 dB};
	
	
	\foreach \x in {1,2,...,4} {
		\pgfmathsetmacro\result{10^\x/10}
		\draw (\x,-0.1) node[below] {$\pgfmathprintnumber\result$} -- ++(0,0.2);
		\draw[dashed] (\x,0) |- (0,\x*3/4);
	}
	\draw (0,-0.1) node[below right] {$0.1$};
	
	\draw[ultra thick,magenta]
	(0,-1.7) -- ++(4.5,0);
	\end{tikzpicture}
	
	όπου παρατηρούμε ότι έχουμε αύξηση της απόκρισης κατά 20 dB ανά δεκάδα.
	
	Η ευθεία αυτή θα έχει αντίθετη κλίση αν το \( H_0(s) \) βρίσκεται στον παρονομαστή
	
	\item
	\( \displaystyle \left\lvert H_1(jω)\right\rvert_{\mathrm{dB}} \approx
	\begin{cases}
	0 \qquad & \text{αν } \omega < p \\
	20\log ω -20\log p \qquad & \text{αν} \omega > p
	\end{cases} \)\\
	και:\\
	\( φ_1 \approx \begin{cases}
	0\degree \qquad & \text{αν } \omega < 0.1p \\
	45\degree + 45\degree \log\frac{\omega}{p} \qquad & \text{αν } 0.1p < \omega < 10p \\
	90\degree \qquad & \text{αν } \omega > 10p
	\end{cases} \)
	
	Άρα η απόκριση γίνεται:
	
	\begin{tikzpicture}[scale=1]
	\draw[->] (0,-3) -- (0,3.5) node[left,scale=1.2] {$H_1$};
	\draw[->] (0,0) -- (4.5,0) node[above] {$\omega$};
	\draw[->] (0,-2.7) -- ++(4.5,0) node[above] {$\omega$};
	
	\draw[gray,dashed] (2,-2.7) -- (2,0);
	
	\draw[ultra thick,orange!90!blue!80!black]
	(0,0) -- (2,0) -- (4,2) node[orange!90!blue!40!black,above] {για $ρ=10$};
	
	\draw[dashed] (3,0) |- (0,1) node[left] {20 dB};
	
	\foreach \x in {1,2,...,4} {
		\pgfmathsetmacro\result{10^\x/10}
		\draw (\x,-0.1) node[below] {$\pgfmathprintnumber\result$} -- ++(0,0.2);
		%\draw[dashed] (\x,0) |- (0,\x*3/4);
	}
	\draw (0,-0.1) node[below right] {$0.1$};
	
	\draw[ultra thick,magenta,yshift=-2.7cm]
	(0,0) -- (1,0) -- (3,1.5) -- ++(1.2,0);
	
	\foreach \x in {1,2,...,3} {
		\pgfmathsetmacro\result{10^\x/10}
		\draw (\x,-2.7-0.1) node[below] {$\pgfmathprintnumber\result$} -- ++(0,0.2);
		%\draw[dashed] (\x,0) |- (0,\x*3/4);
	}
	
	\draw[dashed,yshift=-2.7cm] (2,0) |- (0,1.5/2) node[left] {$\ang{45}$};
	\draw[dashed,yshift=-2.7cm] (3,0) |- (0,1.5) node[left] {$\ang{90}$};
	\draw[yshift=-2.7cm] node[left] {$\ang{0}$};
	
	\draw[magenta!50!black,fill=white,opacity=.8,fill opacity=.3] (2,-2.7+1.5/2) circle(3pt);
	\end{tikzpicture}
	
	\item \( 
	\displaystyle H_2 = \frac{s^2}{\omega_n^2} + \frac{2J}{\omega_n}s+1
	 \) και αποδεικνύεται ότι:
	 \begin{align*}
	 	\left\lvert H_2(j\omega )\right\rvert_{\mathrm{dB}}
	 	&\approx \begin{cases}
	 	0&\qquad \text{αν } \omega < \omega_n \\
	 	40\log \omega - 40\log \omega_n &\qquad \text{αν } \omega > \omega_n
	 	\end{cases} \\
	 	φ_2 &\approx \begin{cases}
	 	0\degree &\qquad \text{αν } \omega< 0.1\omega_n\\
	 	90\degree +90\degree\log\frac{\omega}{\omega_n} &\qquad 
	 	\text{αν }0.1\omega_n < \omega < 10\omega_n \\
	 	180\degree &\qquad \text{αν } \omega>10\omega_n
	 	\end{cases}
	 \end{align*}
	 
	 \begin{tikzpicture}[scale=1]
	 \draw[->] (0,-3.9) -- (0,3.5) node[left,scale=1.2] {$H_2$};
	 \draw[->] (0,0) -- (4.5,0) node[above] {$\omega$};
	 \draw[->] (0,-3.6) -- ++(4.5,0) node[above] {$\omega$};
	 
	 \draw[gray,dashed] (2,-3.6) -- (2,0);
	 
	 \draw[ultra thick,orange!90!blue!80!black]
	 (0,0) -- (2,0) -- (4,4);
	 
	 \draw[dashed] (3,0) |- (0,2) node[left] {40 dB};
	 
	 \foreach \x in {1,2,...,4} {
	 	\pgfmathsetmacro\result{10^\x/10}
	 	\draw (\x,-0.1) node[below] {$\pgfmathprintnumber\result$} -- ++(0,0.2);
	 	%\draw[dashed] (\x,0) |- (0,\x*3/4);
	 }
	 \draw (0,-0.1) node[below right] {$0.1$};
	 
	 \draw[ultra thick,magenta,yshift=-3.6cm]
	 (0,0) -- (1,0) -- (3,2.2) -- ++(1.2,0);
	 
	 \foreach \x in {1,2,...,3} {
	 	\pgfmathsetmacro\result{10^\x/10}
	 	\draw (\x,-3.6-0.1) node[below] {$\pgfmathprintnumber\result$} -- ++(0,0.2);
	 	%\draw[dashed] (\x,0) |- (0,\x*3/4);
	 }
	 
	 \draw[dashed,yshift=-3.6cm] (2,0) |- (0,2.2/2) node[left] {$\ang{90}$};
	 \draw[dashed,yshift=-3.6cm] (3,0) |- (0,2.2) node[left] {$\ang{180}$};
	 \draw[yshift=-3.6cm] node[left] {$\ang{0}$};
	 
	 \draw[magenta!50!black,fill=white,opacity=.8,fill opacity=.3] (2,-3.6+2.2/2) circle(3pt);
	 \end{tikzpicture}
\end{itemize}

\subsubsection{Ασκήσεις}

\begin{comment}
	\paragraph{Παράδειγμα}
	Έστω η συνάρτηση μεταφοράς \( \displaystyle\frac{(s+3)}{(s+1)(s+7)} \). Πώς μπορούμε να
	κατασκευάσουμε το διάγραμμα Bode της;
	
	\subparagraph{Λύση}
	Αρχικά προσπαθώ να τη μετασχηματίσω σε συνδυασμό των απλών μορφών που παρουσιάστηκαν παραπάνω:
	\begin{align*}
	\frac{(s+3)}{(s+1)(s+7)} &=
	\frac{3\left(\frac{s}{3}+1\right)}{\left(\frac{s}{1} + 1\right)\cdot 7\left(\frac{s}{7} + 1\right)}
	\\ &= \frac{3}{7} \frac{H_{1a}}{H_{1b}\cdot H_{1c}}
	\intertext{όπου οι συναρτήσεις \( H_1 \) είναι της μορφής \( \left(\frac{s}{ρ}+1\right)
	\), με \( p_a=3 \), \( p_b=1 \) και \( p_c=7 \)}
	&= \frac{3}{7}\cdot \left( \frac{s}{3}+1 \right)
	\cdot \left(\frac{1}{\frac{s}{7}+1}\right)\cdot \frac{1}{s+1}
	\end{align*}
\end{comment}

\todo{continue or remove}

\paragraph{Άσκηση}
Να βρεθεί η απόκριση συχνότητας του \( \displaystyle
H(s) = \frac{40(s+1)}{s^2(s^2+s+4)} \)
\subparagraph{Απάντηση}

Έχουμε:
\[
H(s) = \underset{\substack{\downarrow\\H_0}}{(10)}
\cdot
\underset{\substack{\downarrow\\H_1\\p=1}}{(s+1)}
\cdot
\underset{\substack{\downarrow\\H_0^{-1}\\ν=2}}{\frac{1}{s^2}}
\cdot
\underset{\substack{\downarrow\\H_2^{-1}\\ \omega_n=2}}{\frac{1}{\frac{s^2}{4}+\frac{s}{4}+1}}
= G_1\cdot G_3 \cdot G_2 \cdot G_4
\]

και εργαζόμαστε για κάθε συνάρτηση ξεχωριστά, σχεδιάζοντας το διάγραμμα Bode της, ώστε
να τα συνδυάσουμε στο τέλος:
\begin{itemize}
	\item \( |G_1|_{\mathrm{dB}} = 20\log10 +\cancelto{0}{20\cdot 0 \cdot\log \omega }  \)
	\\ \( φ =\frac{νπ}{2} = 0 \)
	
	
	\begin{tikzpicture}[scale=1,xscale=0.8]
	\def\phasedown{-1.5}
	
	\draw[->] (0,\phasedown-0.3) -- (0,3);
	\draw[->] (0,0) -- (4.5,0) node[above] {$\si{\radian/\second}$};
	
	\draw[ultra thick,orange!90!blue!80!black]
	(0,1.5) -- ++(4,0);
	
	\draw[dashed] (0,1.5) node[left] {20 dB};
	
	%\foreach \x in {1,2,...,3} {
	%	\pgfmathsetmacro\result{10^\x/10}
	%	\draw (\x,-3.6-0.1) node[below] {$\pgfmathprintnumber\result$} -- ++(0,0.2);
	%	%\draw[dashed] (\x,0) |- (0,\x*3/4);
	%}
	
	\begin{scope}[yshift=\phasedown cm]
	\draw[->] (0,0) -- ++(4.5,0);
	
	%\draw[dashed] (2,0) |- (0,2.2/2) node[left] {$\ang{90}$};
	%\draw[dashed] (3,0) |- (0,2.2) node[left] {$\ang{180}$};
	\draw node[left] {$\ang{0}$};
	
	\draw[ultra thick,magenta,path fading=east] (0,0) -- ++(4,0);
	\end{scope}
	\end{tikzpicture}
	
	\item \( |G_3|_{\mathrm{dB}} \approx \begin{cases}
    0 \qquad & \text{για } \omega < 1 \\
    20\log\omega - \cancelto{0}{20\log 1} &\text{για } \omega > 1
	\end{cases} \)\\
	\( 
	φ_3 \approx  \begin{cases}
	0\degree \qquad & \text{αν } \omega < 0.1p \\
	45\degree + 45\degree \log\frac{\omega}{p} \qquad & \text{αν } 0.1p < \omega < 10p \\
	90\degree \qquad & \text{αν } \omega > 10p
	\end{cases}
	 \)
	
	\begin{tikzpicture}[scale=1,xscale=0.8]
	\pgfplotsset{
		/pgf/number format/freq/.style={
			fixed,
			%  fixed zerofill,
			%  precision=0,
		},
	}
	\def\phasedown{-2}
	
	\draw[->] (0,\phasedown-0.3) -- (0,3);
	\draw[->] (0,0) -- (4.5,0) node[above] {$\si{\radian/\second}$};
	
	\draw[gray,dashed] (2,\phasedown) -- (2,0);
	\draw[gray,dashed] (3,\phasedown) -- (3,0);
	
	\draw[ultra thick,orange!90!blue!80!black]
	(0,0) -- ++(2,0) -- ++(1,1.5) -- ++(1,1.5);
	
	\draw[dashed] (0,1.5) node[left] {20 dB} -| (3,0);
	
	\foreach \x in {1,2,...,3} {
		\pgfmathsetmacro\result{10^\x/100}
		\draw (\x,-0.1) node[below] {$\pgfmathprintnumber[freq]\result$} -- ++(0,0.2);
		%\draw[dashed] (\x,0) |- (0,\x*3/4);
	}
	\foreach \x in {1,2,...,3} {
		\pgfmathsetmacro\result{10^\x/100}
		\draw (\x,\phasedown-0.1) node[below] {$\pgfmathprintnumber[freq]\result$} -- ++(0,0.2);
		%\draw[dashed] (\x,0) |- (0,\x*3/4);
	}
	
	\begin{scope}[yshift=\phasedown cm]
	\draw[->] (0,0) -- ++(4.5,0);
	
	\draw[dashed] (2,0) |- (0,1/2) node[left] {$\ang{45}$};
	\draw[dashed] (3,0) |- (0,1) node[left] {$\ang{90}$};
	%\draw node[left] {$\ang{0}$};
	
	\draw[ultra thick,magenta] (0,0) -- (1,0) -- (3,1) -- (4,1);
	\end{scope}
	\end{tikzpicture}
	
	\item \( |G_2|_{\mathrm{dB}} = \cancelto{0}{20\log 1} + 20(-2)\log \omega = -40\log \omega  \) \\
	\( φ=\frac{νπ}{2} = -π \)
	
	\begin{tikzpicture}[scale=1,xscale=0.8]
	\pgfplotsset{
		/pgf/number format/freq/.style={
			fixed,
			%  fixed zerofill,
			%  precision=0,
		},
	}
	\def\phasedown{-1.5}
	\def\phaseextra{-1.5}
	
	\draw[->] (0,\phasedown+\phaseextra) -- (0,3);
	\draw[->] (0,0) -- (4.5,0) node[above] {$\si{\radian/\second}$};
	
	%\draw[gray,dashed] (2,\phasedown) -- (2,0);
	%\draw[gray,dashed] (3,\phasedown) -- (3,0);
	
	\draw[ultra thick,orange!90!blue!80!black]
	(0,2.5) -- (3.5,-1) node[midway, above right] {-40 dB/dec};
	
	\draw[dashed] (0,2.5) node[left] {40 dB};
	
	\foreach \x in {1,2,...,3} {
		\pgfmathsetmacro\result{10^\x/100}
		\draw (\x,-0.1) node[below] {$\pgfmathprintnumber[freq]\result$} -- ++(0,0.2);
		%\draw[dashed] (\x,0) |- (0,\x*3/4);
	}
	\foreach \x in {1,2,...,3} {
		\pgfmathsetmacro\result{10^\x/100}
		\draw (\x,\phasedown-0.1) node[below] {$\pgfmathprintnumber[freq]\result$} -- ++(0,0.2);
		%\draw[dashed] (\x,0) |- (0,\x*3/4);
	}
	
	\begin{scope}[yshift=\phasedown cm]
	\draw[->] (0,0) -- ++(4.5,0);
	
	%\draw[dashed] (2.5,0) |- (0,1/2) node[left] {$\ang{45}$};
	%\draw[dashed] (3,0) |- (0,1) node[left] {$\ang{90}$};
	\draw node[left] {$\ang{0}$};
	\draw (0,-1) node[left] {$\ang{-180}$};
	
	\draw[ultra thick,magenta] (0,-1) -- (4.5,-1);
	\end{scope}
	\end{tikzpicture}
	
	\item \( |G_4|_{\mathrm{dB}} \approx \begin{cases}
	0\qquad&\text{για } \omega  < 2 \ \mathrm{rad/s}
	\\ -40\log\omega + 40\log 2 \qquad &\text{για } \omega > 2
	\end{cases} \) \\
	\( φ =  \begin{cases}
	0\degree &\qquad \text{αν } \omega< 0.2_n\\
	90\degree +90\degree\log\frac{\omega}{2} &\qquad 
	\text{αν }0.1\omega_n < \omega < 20 \\
	180\degree &\qquad \text{αν } \omega>20
	\end{cases} \)
	
	\begin{tikzpicture}[scale=1,xscale=0.8]
	\pgfplotsset{
		/pgf/number format/freq/.style={
			fixed,
			%  fixed zerofill,
			%  precision=0,
		},
	}
	\def\phasedown{-3}
	\def\phaseextra{-1.5}
	
	\draw[->] (0,\phasedown+\phaseextra) -- (0,1);
	\draw[->] (0,0) -- (4.5,0) node[above] {$\si{\radian/\second}$};
	
	%\draw[gray,dashed] (2,\phasedown) -- (2,0);
	%\draw[gray,dashed] (3,\phasedown) -- (3,0);
	
	\draw[ultra thick,orange!90!blue!80!black]
	(0,0) -- (1.3010,0) -- (4.5,-2) node[midway, above right] {-40 dB/dec};
	
	%\draw[dashed] (0,2.5) node[left] {40 dB};
	
	\foreach \x in {1,1.3010,2.3010} {
		\pgfmathsetmacro\result{10^\x/10}
		\draw (\x,-0.1) node[below] {$\pgfmathprintnumber[freq]\result$} -- ++(0,0.2);
	}
	\foreach \x in {1,1.3010,2.3010} {
		\pgfmathsetmacro\result{10^\x/10}
		\draw (\x,\phasedown-0.1) node[below] {$\pgfmathprintnumber[freq]\result$} -- ++(0,0.2);
	}
	
	\begin{scope}[yshift=\phasedown cm]
	\draw[->] (0,0) -- ++(4.5,0);
	
	\draw[dashed] (1.3010,0) |- (0,-1.5/2) node[left] {$\ang{-90}$};
	\draw[dashed] (2.3010,0) |- (0,-1.5) node[left] {$\ang{-180}$};
	\draw node[left] {$\ang{0}$};
	
	\draw[ultra thick,magenta] (0,0) -- (0.3010,0) -- (2.3010,-1.5) -- (4.5,-1.5);
	\end{scope}
	\end{tikzpicture}
\end{itemize}

Τελικά, για να βρούμε τη σωστή απάντηση, προσθέτουμε όλα τα παραπάνω διαγράμματα
(αφού ο πολλαπλασιασμός γίνεται πρόσθεση μετά την εφαρμογή του λογαρίθμου):

\begin{tikzpicture}[scale=1,xscale=1]
\pgfplotsset{
	/pgf/number format/freq/.style={
		fixed,
		%  fixed zerofill,
		%  precision=0,
	},
}
\def\phasedown{-3}
\def\phaseextra{-4}

\draw[->] (0,\phasedown+\phaseextra) -- (0,3);
\draw[->] (0,0) -- (4.5,0) node[above] {$\si{\radian/\second}$};

\foreach \x in {1,1.3010,2,2.3010,3,3.3010} {
	\draw[densely dotted] (\x,\phasedown) -- (\x,0);
}

\def\at{20}
\draw[ultra thick,orange!90!blue!80!black]
(0,2.5) 
-- ++(-2*\at:{1/cos(2*\at)}) node(c1) {} node[above right] {-40 dB/dec}
-- ++(-2*\at:{1/cos(2*\at)}) node(c2) {}
-- ++(-1*\at:{0.3010/cos(1*\at)}) node(c3) {} node[midway,above right] {-20 dB/dec}
-- ++(-3*\at:{1/cos(3*\at)}) node(c4) {} node[pos=1,right] {-60 dB/dec}
;

\foreach \p in {c1,c2,c3,c4} {
	\draw[densely dashed] (\p.center) -- (\p.center |- 0,0) (\p.center) -- (\p.center -| 0,0);
}
\draw (c1 -| 0,0) node[left] {60 dB};
\draw (c2 -| 0,0) node[left,yshift=1mm] {20 dB};
\draw (c3 -| 0,0) node[left,yshift=-1mm] {14 dB};

\foreach \x in {1,2,2.3010,3.3010} {
	\pgfmathsetmacro\result{10^\x/100}
	\draw (\x,-0.1) node[below] {$\pgfmathprintnumber[freq]\result$} -- ++(0,0.2);
	%\draw[dashed] (\x,0) |- (0,\x*3/4);
}
\foreach \x in {1.3010,2,2.3010,3} {
	\pgfmathsetmacro\result{10^\x/100}
	\draw (\x,\phasedown-0.1) node[below] {$\pgfmathprintnumber[freq]\result$} -- ++(0,0.2);
	%\draw[dashed] (\x,0) |- (0,\x*3/4);
}

\begin{scope}[yshift=\phasedown cm]
\draw[->] (0,0) -- ++(4.5,0);
\def\at{40}

\draw[ultra thick,magenta]
(0,0) -- (1,0)
-- ++(1*\at:{0.3010/cos(1*\at)}) node(c1) {} node[above,scale=.9] {$\ang{45}$/dec}
-- ++(-1*\at:{1.7000/cos(1*\at)}) node(c3) {} node[pos=.6,below left] {-$\ang{45}$/dec}
-- ++(-2*\at:{0.3010/cos(2*\at)}) node(c4) {} node[midway,right,xshift=1mm] {-$\ang{90}$/dec}
-- ++(1,0)
;

\foreach \p in {c1,c3,c4} {
	\draw[dashed,gray] (\p.center) -- (\p.center |- 0,0) (\p.center) -- (\p.center -| 0,0);
}

\draw[every node/.style={scale=.8}]
(c1 -| 0,0) node[left] {$-\ang{153}$}
(c3 -| 0,0) node[left] {$-\ang{243}$}
(c4 -| 0,0) node[left] {$-\ang{270}$}
;
\end{scope}

\end{tikzpicture}

\paragraph{Άσκηση}
Ποιά είναι η συνάρτηση μεταφοράς της οποίας το προσεγγιστικό διάγραμμα Bode φαίνεται στο
παρακάτω σχήμα;

\begin{tikzpicture}[scale=1.2]
\draw[->] (0,-0.2) -- (0,3);
\draw[->] (-0.2,0) -- (4.5,0) node[below] {$\omega$};

\draw[ultra thick,blue!50!cyan!70!black]
(0,2) -- (1,2) -- (1.3,1.5) -- (3,1.5) -- (4,{-0.5/3});

\draw (0,2) node[left] {$20\log 0.8$};
\draw[dashed] (1,2) -- (1,0) node[below,scale=.9,xshift=-1mm] {$0.5$};
\draw[dashed] (0,1.5) node[left] {$-14\ \mathrm{dB}$} -| (1.3,0) node[below,scale=.9,xshift=0.5mm] {$2$};
\draw[dashed] (3,1.5) -- (3,0) node[below] {$10$};

\draw[blue!50!black] (1.2,1.7) node[above right] {$-20 \ \mathrm{db/dec}$};
\draw[blue!50!black] (3.5,0.8) node[above right] {$-20 \ \mathrm{db/dec}$};
\end{tikzpicture}

Επιέξετε μία από τις πιθανές απαντήσεις
\begin{enumgreekparen}
	\item \( \displaystyle \frac{10(s+1)(s+2)}{(s+0.5)(5s+10)(s+10)} \)
	\item \( \displaystyle \frac{10(s+0.5)}{(s+2)(s+10)} \)
	\item \( \displaystyle \frac{10(s+1)(s+2)}{(s+0.5)(5s+5)(s+10)} \)
	\item \( \displaystyle \frac{10(s+10)}{(s+0.5)(s+2)} \)
\end{enumgreekparen}

\subparagraph{Λύση}
Κανονικά, φέρνουμε όλες τις συναρτήσεις μεταφοράς στη μορφή (π.χ. για την \textbf{(α)}):
\[
\frac{10\left(\frac{s}{1}+1\right)2\left(\frac{s}{2}+1\right)}{
	0.5\left(\frac{s}{0.5}+1\right)10\left(\frac{s}{2}+1\right)10\left(\frac{s}{10}+1\right)}
\]
και ελέγχουμε τις θέσεις των πόλων και των μηδενικών για τα σημεία μείωσης του κέρδους. Για
παράδειγμα, για το \textbf{(γ)}:
\[
\text{(γ)} =
\frac{10\cancel{(s+1)2\left(\frac{s}{2}+1\right)}}{0.5\left(\frac{s}{0.5}+1\right)
	5\cancel{(s+1)}10\left(\frac{s}{10}+1\right)}
\]
Στην παραπάνω συνάρτηση παρατηρούμε ότι απλοποιούνται ο πόλος και το μηδενικό για \( s=-1 \),
και μένουν ένας πόλος με συντελεστή \( \frac{1}{0.5} \) (άρα υπάρχει πτώση \( 20
\si{\decibel}/\mathrm{dec} \) από τη συχνότητα 0.5 και μετά), ένα μηδενικό με συντελεστή
\( \frac{1}{2} \) (άρα υπάρχει άνοδος που ακυρώνει την πτώση από το 2 και μετά), και ένας
πόλος με συντελεστή \( \frac{1}{10} \) (άρα υπάρχει ξανά πτώση από το 10 και μετά).

Εναλλακτικά, σε αυτήν την άσκηση, μπορώ να υπολογίσω το κέρδος της κάθε συνάρτησης
μεταφοράς, που πρέπει να είναι \( 20\log 0.8 \SI{}{\decibel} \), άρα να έχει μέτρο \( 0.8 \).

Έτσι, για κάθε απάντηση έχουμε:
\begin{enumgreekparen}
	\item \( \displaystyle \frac{10\cdot 1\cdot 2}{0.5\cdot 10 \cdot 10} 
	= 0.4 \xrightarrow{\si{\decibel}} 20\log 0.4 \neq 20\log 0.8 \)
	\item \( \displaystyle \frac{10\cdot 0.5}{2\cdot 10} = 0.25 \nrightarrow 20\log 0.8 \)
	\item \( \displaystyle \frac{10\cdot 1\cdot 2}{0.5\cdot 5\cdot 10} = 0.8
	\rightarrow 20\log 0.8 \quad \checkmark \)
	\item \( \displaystyle \frac{10\cdot 10}{0.5\cdot 2} = 100
	\nrightarrow 20\log 0.8 \)
\end{enumgreekparen}

\paragraph{Άσκηση}
Τι τιμή έχει το προσεγγιστικό διάγραμμα μέτρου και φάσης στη συνάρτηση μεταφοράς
\( \displaystyle \frac{100}{(s+1)^2(s+10)} \) στη συχνότητα \( \omega = 1\ \sfrac{\mathrm{rad}}{\mathrm{s}} \);

\begin{attnbox}{}
	Εδώ μας ζητείται η τιμή του προσεγγιστικού διαγράμματος, όχι η ακριβής τιμή.
\end{attnbox}
Αν ζητούνταν η ακριβής τιμή, θα θέταμε \( s=j\omega = j\cdot 1 \), και υπολογίζαμε το
μέτρο και το όρισμα του μιγαδικού που θα προέκυπτε από τη συνάρτηση μεταφοράς.

Αρχικά, μετασχηματίζουμε ελαφρά τη συνάρτηση μεταφοράς:
\[
\frac{100}{(s+1)^2)(s+10)}
\xrightarrow{\quad}
\frac{100}{1\left(\frac{s}{1}+1\right)^2 10 \left(\frac{s}{10}+1\right)}
= \frac{10}{\left(\frac{s}{1}+1\right)\left(\frac{s}{10}+1\right)}
\]

Εδώ έχουμε έναν πόλο για \( \omega  = \SI{10}{\radian/\second} \), ο οποίος ξεκινάει μία
πτώση \( \SI{20}{\decibel/\decade} \) σε εκείνη τη συχνότητα, και έναν διπλό πόλο
\( \omega  = \SI{1}{\decibel/\decade} \), του οποίου η επίδραση είναι μια πτώση κατά
\( 2\cdot20 = \SI{40}{\decibel/\decade} \). Άρα το προσεγγιστικό διάγραμμα μοιάζει κάπως έτσι:

\begin{tikzpicture}[scale=1.4]
\draw[gray!70!white] (0,1.5) node[left,scale=.8] {$0 \ \mathrm{dB}$} -- ++(4.5,0);
\draw[->] (0,-1.5) -- (0,3);
\draw[->] (-0.2,0.3) -- (5,0.3) node[right] {$\omega\ \mathsmaller{(\mathrm{rad/s})}$};

\draw[ultra thick,blue!50!cyan!70!black]
(0,2) -- (2,2) -- (4,0) -- (5,-1.5);
\fill[blue!50!cyan!70!black]
(2,2) circle (2pt) node[above right,scale=0.8] {$20\ \mathrm{dB}$}
(4,0) circle (2pt);

\draw (0,2) node[left,align=right] {$20\log 10 $\\$=20\ \mathrm{dB}$};
\draw[dashed] (2,2) -- ++(0,-1.7) node[below] {$p=1$};
\draw[dashed] (4,0) -- (4,0.3) node[above] {$10$};

\draw[blue!50!black] (3,1) node[above right] {$-40 \ \mathrm{db/dec}$};
\draw[blue!50!black] (4.7,-1) node[above right] {$-60 \ \mathrm{db/dec}$};
\end{tikzpicture}

Το διάγραμμα φάσης είναι λίγο πιο δύσκολο, επειδή η απόκριση επηρεάζεται μία δεκάδα πριν
και μία δεκάδα μετά από τη συχνότητα του κάθε πόλου.

Έτσι, για \( p=1 \) (διπλός πόλος), η επίδραση ξεκινάει από την \( \omega = 0.1p =
\SI{0.1}{\radian/\second}  \), με μία κλίση \( 2\cdot/(\ang{-45}) = \ang{-90}/\si{\decade} \),
μέχρι και την \( \omega = 10p = \SI{10}{\radian/\second} \).

\begin{tikzpicture}[scale=1,yscale=0.8]
\draw[dashed] (1,-4) -- ++(0,2);
\draw[dashed] (2,-4) -- ++(0,2);
\draw[dashed] (3,-4) -- ++(0,2);

\draw (0,-1.5) -- (0,-4.5);
\draw[->] (0,-2) -- ++(4,0);
\draw[->] (0,-4) -- ++(4,0);

\draw[ultra thick,magenta!90!black,yshift=-2cm,yscale=.8,every node/.style={inner sep=3pt}]
(0,0) -- (1,0)
-- (3,-2) node[midway,above right,fill=white] {$\ang{-90} \mathrm{/dec}$}
-- (4,-2);

\draw (1,-4) node[below] {$0.1$};
\draw (2,-4) node[below] {$1$};
\draw (3,-4) node[below] {$10$};
\end{tikzpicture}

Αντίστοιχα, για τον απλό πόλο \( p=10 \), ξεκινάμε από \( \SI{1}{\radian/\second} \) μέχρι
τα \( \SI{100}{\radian/\second} \) με κλίση \( -\ang{45}/\si{\decade} \).

\begin{tikzpicture}[scale=1,yscale=0.8]
\draw[dashed] (1,-4) -- ++(0,2);
\draw[dashed] (2,-4) -- ++(0,2);
\draw[dashed] (3,-4) -- ++(0,2);

\draw (0,-1.5) -- (0,-4.5);
\draw[->] (0,-2) -- ++(4,0);
\draw[->] (0,-4) -- ++(4,0);

\draw[ultra thick,magenta!90!black,yshift=-2cm,yscale=.6,every node/.style={inner sep=1pt}]
(0,0) -- (1,0)
-- (3,-2) node[midway,above right,yshift=-2pt,xshift=2pt,fill=white] {$\ang{-45} \mathrm{/dec}$}
-- (4,-2);

\draw (1,-4) node[below] {$1$};
\draw (2,-4) node[below] {$10$};
\draw (3,-4) node[below] {$100$};
\end{tikzpicture}

Προσθέτοντας τα δύο παραπάνω διαγράμματα για τις φάσεις, έχουμε:

\begin{tikzpicture}[scale=1.4]
\draw[dashed] (1,-4) -- ++(0,4.3);
\draw[dashed] (2,-4) -- ++(0,4.3);
\draw[dashed] (3,-4) -- ++(0,4.3);

\draw[gray!70!white] (0,1.5) node[left,scale=.8] {$0 \ \mathrm{dB}$} -- ++(4.5,0);
\draw[->] (0,-4.5) -- (0,3);
\draw[->] (-0.2,0.3) -- (5,0.3) node[right] {$\omega\ \mathsmaller{(\mathrm{rad/s})}$};

\draw[ultra thick,blue!50!cyan!70!black]
(0,2) -- (2,2) -- (3,0) -- (3.5,-1.5);
\fill[blue!50!cyan!70!black]
(2,2) circle (2pt) node[above right,scale=0.8] {$20\ \mathrm{dB}$}
(3,0) circle (2pt);

\draw (0,2) node[left,align=right] {$20\log 10 $\\$=20\ \mathrm{dB}$};
\draw[dashed] (2,2) -- ++(0,-1.7) node[below,fill=white,yshift=-1pt] {$p=1$};
\draw[dashed] (3,0) -- (3,0.3) node[above] {$10$};

\draw[blue!50!black] (3,1) node[above right] {$-40 \ \mathrm{db/dec}$};
\draw[blue!50!black] (3.5,-1) node[above right] {$-60 \ \mathrm{db/dec}$};

\draw[->] (0,-2) -- ++(5,0);
\draw[->] (0,-4) node[left] {$\ang{-270}$} -- ++(5,0);

\draw[ultra thick,magenta!90!black,yshift=-2cm,yscale=2/3]
(0,0) -- (1,0)
-- (2,-1) node[midway,below left,fill=white] {$\ang{-90} \mathrm{/dec}$}
-- (3,-2.5) node[midway,above right,fill=white] {$\ang{-135} \mathrm{/dec}$}
-- (4,-3) node[midway,above right,fill=white] {$\ang{-45} \mathrm{/dec}$}
-- ++(0.5,0);

\draw (1,-4) node[below] {$0.1$};
\draw (2,-4) node[below] {$1$};
\draw (3,-4) node[below] {$10$};
\draw (4,-4) node[below] {$100$};
\end{tikzpicture}

Γενικά είναι εύκολο να εργαζόμαστε με συναρτήσεις μεταφοράς των οποίων οι πόλοι και τα
μηδενικά είναι μεταξύ τους πολλαπλασιασμένα με παράγοντες του 10.

\paragraph{Άσκηση}
Ποιά από τις παρακάτω συναρτήσεις μεταφοράς έχει διάγραμμα Bode κέρδους με κλίση
\( \SI{-60}{\decibel/\decade} \) στο \( \omega = \SI{30}{\radian/\second} \);

\begin{enumgreek}
	\item \( \displaystyle \frac{s+70}{(s+1)(s+2)(s+4)(s+6)} \)
	\item \( \displaystyle \frac{s+60}{(s+80)(s+90)(s^2+18s+100)} \)
	\item \( \displaystyle \frac{s+4}{(s+2)(s^2+18s+100)} \)
	\item \( \displaystyle \frac{s+4}{(s+6)(s+9)(s^2+18s+100)} \)
\end{enumgreek}

\subparagraph{Λύση}
Ας μελετήσουμε κάθε συνάρτηση ξεχωριστά:
\begin{enumgreek}
	\item Ο όρος \( s+70 \) είναι πιο μακριά από τα \( \SI{30}{\radian/\second} \), άρα
	τον αγνοούμε. Στον παρονομαστή έχουμε 4 όρους πριν από το 30, άρα κλίση
	\( -4\cdot 20 = \SI{-80}{\decibel/\decade} \), που δεν συμφωνεί με την εκφώνηση.
	\item Οι όροι \( s+60 \), \( s+80 \) και \( s+90 \) είναι πολύ μακριά και δεν μας
	απασχολούν. Ο όρος \( s^2+18s+100 \) είναι ένα πολυώνυμο με μιγαδικές ρίζες και
	\( \sqrt{100} = 10\omega_n \), άρα έχει κλίση \( \SI{-40}{\decibel/\decade} \)
	(αφού \( 30 > 10 = \omega_n \)), και
	δεν συμφωνεί με την εκφώνηση.
	\item Μετά από το \( \omega = \SI{10}{\radian/\second} \) συνεισφέρουν το
	μηδενικό με \( \SI{+20}{\decibel/\decade} \), ο απλός πόλος με \( \SI{-20}{\decibel/\decade} \), και το πολυώνυμο με \( \SI{-40}{\decibel/\decade} \).
	Συνολικά έχουμε \( +20 -20 - 40 = \SI{-40}{\decibel/\decade} \), άρα δεν συμφωνούμε με την
	εκφώνηση.
	\item Έχουμε μηδενικό (\( \SI{+20}{\decibel/\decade} \)) και τέσσερις πόλους
	(\( -20\cdot 4 = \SI{-80}{\decibel/\decade} \)), άρα συνολικά \( \SI{60}{\decibel/\decade} \), που είναι και η ζητούμενη απάντηση.
\end{enumgreek}

Άρα το σωστό αποτέλεσμα είναι το \textbf{(δ)}.

\paragraph{Άσκηση}
Δίνεται το διάγραμμα Bode πλάτους, καθώς και η βηματική απόκριση ενός συστήματος. Να βρεθεί
η αντίστοιχη συνάρτηση μεταφοράς, και το σφάλμα θέσης αν την τοποθετήσουμε σε σύστημα κλειστού
βρόγχου μοναδιαίας αρνητικής ανάδρασης.

\begin{minipage}{15cm}
	\centering
	\raisebox{-0.5\height}{
\begin{tikzpicture}[scale=1,xscale=1]
\pgfplotsset{
	/pgf/number format/freq/.style={
		fixed,
		%  fixed zerofill,
		%  precision=0,
	},
}
\def\phasedown{-3}
\def\phaseextra{-4}

\draw[->] (0,-5) -- (0,3);
\draw[->] (0,0) -- (4.5,0) node[above] {$\si{\radian/\second}$};


\def\at{60}
\draw[ultra thick,orange!90!blue!80!black]
(0,1)
-- ++(1.6021,0) node(c1) {}
-- ++(-1*\at:{1/cos(1*\at)}) node(c2) {} node[near start,above right] {-40 dB/dec}
-- ++(-1.2*\at:{1/cos(1.2*\at)}) node(c3) {} node[near start,below left,fill=white] {-60 dB/dec}
-- ++(-1.2*\at:0.2)
;

\foreach \p in {c1,c2,c3} {
	\draw[densely dashed] (\p.center) -- (\p.center |- 0,0) (\p.center) -- (\p.center -| 0,0);
}
\draw (c1 -| 0,0) node[left] {20 dB};
\draw (c2 -| 0,0) node[left,yshift=1mm] {-20 dB};
\draw (c3 -| 0,0) node[left,yshift=-1mm] {-80 dB};

\foreach \x in {1.6021,2.6021,3.6021} {
	\pgfmathsetmacro\result{10^\x/10}
	\draw (\x,-0.1) node[below] {$\pgfmathprintnumber[freq]\result$} -- ++(0,0.2);
	%\draw[dashed] (\x,0) |- (0,\x*3/4);
}
\end{tikzpicture}}
\hspace{2cm}
\raisebox{-0.5\height}{
\begin{tikzpicture}
\draw (0,-0.2) -- (0,3);
\draw[->] (-0.2,0) -- (5,0) node[right] {$t$};

\draw[path fading=east,dashed] (0,2) -- (3.5,2);
\draw (0,2) node[left] {$1$};

\draw[very thick,blue!80!cyan!80!black]
plot[domain=0:5,variable=\t,smooth,samples=18]
(\t,{2*(1-exp(-1.4*\t))});
\end{tikzpicture}}
\end{minipage}

\subparagraph{Λύση}
Παρατηρούμε μία πτώση κατά \( \SI{-40}{\decibel/\decade} \) στη συνάρτηση μεταφοράς. Οπότε,
προκύπτει το ερώτημα αν αυτή προέρχεται από δύο απλούς πραγματικούς πόλους, ή από δύο
μιγαδικούς συζυγείς.

Παρατηρούμε ότι η βηματική απόκριση δεν παρουσιάζει υπερύψωση, αλλά είναι υπεραποσβεννύμενη.
Επομένως δεν έχουμε μιγαδικούς πόλους, αλλά δύο πραγματικούς.

Υπολογίζουμε το κέρδος \( G \) του συστήματος: αφού το διάγραμμα Bode ξεκινάει από την τιμή
\( \SI{20}{\decibel} = 20\log G \implies G = 10 \). Η συνάρτηση μεταφοράς έχει μια κλίση
\( \SI{-40}{\decibel/\decade} \) που ξεκινάει από \( \omega = \SI{4}{\radian/\second} \),
άρα στον παρονομαστή της έχει δύο όρους της μορφής \( \left(\frac{s}{4}+1\right) \). Επίσης,
έχει μια επιπλέον κλίση \( \SI{-20}{\decibel/\decade} \) από το \( \omega = \SI{40}{\radian/\second} \) και μετά, άρα έναν όρο της μορφής \( \left(\frac{s}{40} + 1\right) \).

Επομένως, τελικά η συνάρτηση γίνεται:
\[
H(s) = 10 \frac{1}{\left(\frac{s}{4}+1\right)^2\left(\frac{s}{40}+1\right)}
\]

Αν το τοποθετήσουμε σε κλειστό βρόγχο:

\begin{tikzpicture}
\draw (2,0) node[rectangle,draw] (h) {$\displaystyle 10\frac{1}{\left(\frac{s}{4}+1\right)^2\left(\frac{s}{40}+1\right)}$};
\draw (h.west) ++(-1,0) node[circle,draw,minimum size=5mm] (mix) {};

\draw[<-] (mix) -- ++(-1,0);
\draw[->] (mix) -- (h);
\draw[->] (h.east) -- ++(2,0) node[midway] (fb) {};
\draw[->] (fb.center) -- ++(0,-1) -| (mix);
\end{tikzpicture}

Το σύστημα έχει τύπο 0 (αφού δεν υπάρχουν ολοκληρωτές), άρα και πεπερασμένο σφάλμα θέσης:
\begin{align*}
	K_{\mathrm{p}} &= \lim_{s\to 0} \cancel{s}\frac{1}{\cancel{s}}H(s) = 10 \\
	e_{\mathrm{ss}} &= \frac{1}{1+K_{\mathrm{p}}} = 0.09
\end{align*}

\paragraph{Παράδειγμα}
Να βρεθεί η συνάρτηση μεταφοράς με το παρακάτω προσεγγιστικό διάγραμμα Bode:

\begin{tikzpicture}[scale=1,xscale=1]
\pgfplotsset{
	/pgf/number format/freq/.style={
		fixed,
		%  fixed zerofill,
		%  precision=0,
	},
}
\def\phasedown{-3}
\def\phaseextra{-4}

\draw[->] (0,-3) -- (0,3);
\draw[->] (0,0) -- (4.5,0) node[above] {$\si{\radian/\second}$};
\draw[opacity=.8] (0,0) node[above left] {dB};


\def\at{30}
\draw[ultra thick,orange!90!blue!80!black]
(0,2)
-- (2.5,0) node[midway,above right] {-20 dB/dec}
-- (3.5,-1) node(c1) {}
-- (4.1,-3) node[midway,above right] {-40 dB/dec}
;

\foreach \p in {c1} {
	\draw[densely dashed] (\p.center) -- (\p.center |- 0,0);
}

\foreach \x in {2.5,3.5} {
	\pgfmathsetmacro\result{5*\x-7.5}
	\draw (\x,0.1) node[above] {$\pgfmathprintnumber[freq]\result$} -- ++(0,-0.2);
	%\draw[dashed] (\x,0) |- (0,\x*3/4);
}
\end{tikzpicture}

\subparagraph{Λύση}
Έχουμε πτώση \( \SI{-20}{\decibel/\decade} \) από τη συχνότητα \( \omega = 0 \), άρα έχουμε
έναν πόλο στο 0, δηλαδή έναν ολοκληρωτή. Μετά από τα \( \omega = \SI{10}{\radian/\second} \),
εμφανίζεται και μία δεύτερη πτώση \( \SI{-20}{\decibel/\decade} \) που υποδηλώνει πόλο
στο 10.

Θυμόμαστε ότι το DC κέρδος του ολοκληρωτή \( \frac{G}{s^N} \) είναι και το σημείο τομής
του με τον οριζόντιο άξονα, δηλαδή σε αυτήν την περίπτωση το 5. Άρα η συνάρτηση
μεταφοράς έχει τη μορφή:
\[
H(s) = \frac{5}{s\left(\frac{s}{10}+1\right)} = \frac{50}{s(s+10)}
\]

\subsection{Ευστάθεια}
Για να ελέγξουμε την ευστάθεια του συστήματος χρησιμοποιούμε τη
\textbf{συνάρτηση δέλτα}. Θυμόμαστε ότι ένα σύστημα είναι ευσταθές
αν έχει πόλους μόνο στο αριστερό ημιεπίπεδο, και οριακά ευσταθές
αν έχει συζυγείς πόλους (ή πραγματικό πόλο) πολλαπλότητας 1 επάνω
στον φανταστικό άξονα:
\\*
\begin{tikzpicture}[scale=1,xscale=1]
\def\xp{8cm}
\def\yp{5cm}

\tikzstyle{pole}=[cross=4pt,red!80!black,very thick]
\draw[->] (-1.5,0) -- (5,0) node[below] {$t$};
\draw (0,-2) -- (0,2);
\draw (2,-2) -- (2,2);

\draw (-0.5,0) node[pole] {};

\draw[very thick,blue!50!cyan!80!white,xshift=2cm]
plot[variable=\x,domain=0:3,samples=18,smooth]
(\x,{-1.5+3*exp(-1.1*\x)});

\begin{scope}[xshift=\xp]
\draw[->] (-1.5,0) -- (5,0) node[below] {$t$};
\draw (0,-2) -- (0,2);
\draw (2,-2) -- (2,2);

\draw (0.5,1) node[pole] {};
\draw (0.5,-1) node[pole] {};

\draw[very thick,blue!50!cyan!80!white,xshift=2cm]
plot[variable=\x,domain=0:3,samples=\gsamples/2.5,smooth]
(\x,{sin(7*\x r)*(-1+exp(\x/3.5))});
\end{scope}

\begin{scope}[yshift=-\yp]
\draw[->] (-1.5,0) -- (5,0) node[below] {$t$};
\draw (0,-2) -- (0,2);
\draw (2,-2) -- (2,2);

\draw (-0.5,1) node[pole] {};
\draw (-0.5,-1) node[pole] {};

\draw[very thick,blue!50!cyan!80!white,xshift=2cm]
plot[variable=\x,domain=0:3,samples=\gsamples/2.5,smooth]
(\x,{cos(9*\x r)*(1.5*exp(-\x))});
\end{scope}

\begin{scope}[xshift=\xp,yshift=-\yp]
\draw[->] (-1.5,0) -- (5,0) node[below] {$t$};
\draw (0,-2) -- (0,2);
\draw (2,-2) -- (2,2);

\draw (0,0) node[pole] {};

\draw[very thick,blue!50!cyan!80!white,xshift=2cm]
(0,1) -- (3,1);
\end{scope}
\end{tikzpicture}

Θα ασχοληθούμε επίσης με τη \textbf{σχετική ευστάθεια}, που δηλώνει
το πόσο ευσταθές είναι ένα σύστημα σε σχέση με ένα άλλο. Για παράδειγμα,
αν έχουμε δύο συστήματα
\( A \) και \( B \) με πόλους στο αριστερό ημιεπίπεδο:

\begin{tikzpicture}[scale=.7]
\fill[green,path fading=west,fill opacity=.4] (-3,3) rectangle (0,-3);

\tikzstyle{pole}=[cross=4pt,red!80!black,very thick]

\draw (-3,0) -- (3,0);
\draw (0,-3) -- (0,3);

\draw (-1,2) node[pole] (pa1) {};
\draw (-1,-2) node[pole] (pa2) {};
\draw (-2,2) node[pole] (pb1) {};
\draw (-2,-2) node[pole] (pb2) {};

\draw[dashed] (pa1.center) -- (pa2.center);
\draw[dashed] (pb1.center) -- (pb2.center);

\draw (pa1.north) node[above] {A};
\draw (pb1.north) node[above] {B};
\end{tikzpicture}

Τότε παρατηρούμε ότι και τα δύο είναι ευσταθή, αλλά το \( A \) έχει
πόλους πιο κοντά στον φανταστικό άξονα, άρα μπορούμε να θεωρήσουμε
ότι είναι "λιγότερο ευσταθές".

Έστω δύο συστήματα με συναρτήσεις μεταφοράς κλειστού βρόγχου \( T_1 \) και \( T_2 \):

\begin{tikzpicture}
\tikzstyle{transfer}=[rectangle,draw,minimum height=9mm]

\draw (2,0) node[transfer] (h1) {$\displaystyle H_1(s)$};
\draw (h1.west) ++(-1,0) node[circle,draw,minimum size=5mm] (mix) {};
\draw (h1.east) ++(1.2,0) node[transfer] (h2) {$\displaystyle H_2(s)$};

\draw[<-] (mix) -- ++(-1,0);
\draw[->] (mix) -- (h1);
\draw (h1) -- (h2);
\draw[->] (h2.east) -- ++(1,0) node (end) {} node[midway] (fb) {};
\draw[->] (fb.center) -- ++(0,-1) -| (mix);
\draw (mix.south) node[below left] {$-$};

\draw(-1,0) node[left] {$\displaystyle T_1(s) = \frac{H_1H_2}{1+H_1H_2}$};

\begin{scope}[yshift=-2cm]
\draw (3,0) node[transfer] (h1) {$\displaystyle H_1(s)$};
\draw (mix |- 0,0) node[circle,draw,minimum size=5mm] (mix) {};
\draw (h1.south) ++(0,-1) node[transfer] (h2) {$\displaystyle H_2(s)$};

\draw[<-] (mix) -- ++(-1,0);
\draw[->] (mix) -- (h1);
\draw[->] (h1.east) -- (end |- 0,0) node[midway] (fb) {};
\draw[->] (fb.center) |- (h2) -| (mix);
\draw (mix.south) node[below left] {$-$};

\draw(-1,0) node[left] {$\displaystyle T_1(s) = \frac{H_1}{1+H_1H_2}$};
\end{scope}
\end{tikzpicture}

και τα δύο έχουν τον ίδιο παρονομαστή (άρα και χαρακτηριστικό πολυώνυμο) στη συνάρτηση
μεταφοράς, επομένως είναι ισοδύναμα ως προς την ευστάθεια.

Επομένως μπορούμε να μελετάμε τα παρακάτω ισοδύναμα συστήματα, με συνάρτηση μεταφοράς ανοιχτού
βρόγχου \( A(s) = H_1H_2 \):
\\*
\begin{tikzpicture}
\tikzstyle{transfer}=[rectangle,draw,minimum height=9mm,minimum width=10mm]

\draw (2,0) node[transfer] (h1) {$\displaystyle H_1$};
\draw (h1.west) ++(-1,0) node[circle,draw,minimum size=5mm] (mix) {};
\draw (h1.east) ++(1.2,0) node[transfer] (h2) {$\displaystyle H_2$};

\draw[<-] (mix) -- ++(-1,0);
\draw[->] (mix) -- (h1);
\draw (h1) -- (h2) node[midway] (center) {};
\draw[->] (h2.east) -- ++(0.5,0) node (end) {} node[midway] (fb) {};
\draw[->] (fb.center) -- ++(0,-1.5) -| (mix);
\draw (mix.south);

\draw[->,thick] (center.center |- 0,-1.5/2) arc [start angle=120,end angle=-170,radius=2mm];

\begin{scope}[yshift=-2.5cm]
\draw (3,0) node[transfer] (h1) {$\displaystyle H_1$};
\draw (mix |- 0,0) node[circle,draw,minimum size=5mm] (mix) {};
\draw (h1.south) ++(0,-1) node[transfer] (h2) {$\displaystyle H_2$};

\draw[<-] (mix) -- ++(-1,0);
\draw[->] (mix) -- (h1);
\draw[->] (h1.east) -- (end |- 0,0) node[midway] (fb) {};
\draw[->] (fb.center) |- (h2) -| (mix);
\draw (mix.south);
\end{scope}
\end{tikzpicture}

Για τη μελέτη της ευστάθειας, ορίζουμε το \textbf{περιθώριο φάσης} και το \textbf{περιθώριο κέρδους}.
\begin{defn}{Περιθώριο φάσης}{}
	\textbf{Περιθώριο φάσης} ονομάζεται η απόσταση της φάσης της συνάρτησης μεταφοράς
	από τις \( \ang{180} \), στο σημείο στο οποίο το πλάτος της είναι \( \SI{0}{\decibel} \)
	ή 1:

	\begin{tikzpicture}[scale=1.2]
	\pgfplotsset{
		/pgf/number format/freq/.style={
			fixed,
		},
	}
	\def\phasedown{-2.2}
	\def\phaseextra{-2}
	
	\draw[->] (0,\phasedown+\phaseextra) -- (0,2);
	\draw[->] (0,0) -- (4.5,0) node[above] {$\si{\radian/\second}$};
	
	\def\at{20}
	\draw[ultra thick,orange!90!blue!90!black]
	(0.7,1.2) -- (2.8,0)
	.. controls (3.5,-0.5) and (4,-1) .. (4,-2);
	
	\draw[dashed] (2.8,0) -- ++(0,\phasedown-1.215);
	
	\draw (0,0) node[above left,orange!90!blue!50!black] {$20\log\left|A(j\omega)\right|$};
	
	\filldraw[orange!90!blue!90!black,fill opacity=.6] (2.8,0) circle (1mm)
	node[above,orange!90!blue!40!black,yshift=1mm,opacity=1] {$\omega_c$};
	
	\begin{scope}[yshift=\phasedown cm]
	\draw[->] (0,0) -- ++(4.5,0);
	
	\draw[dashed] (2.8,-1.215) -- +(-0.7,0) node[midway] (mp) {} -- +(0.7,0);
	\draw[<-,thick] (mp.center) -- (mp.center |- 0,-1.7) node[midway,left] {$\phi_M$};
	
	\draw (0,-0.3) node[left,scale=.8] {$\ang{-90}$};
	\draw[dashed] (0,-1.7) node[left,scale=.8] {$\ang{-180}$} -- ++(4,0);
	\draw (0,-1) node[left,blue!50!cyan!50!green!50!black] {$\measuredangle\ A(j\omega)$};
	
	\draw[very thick,blue!50!cyan!50!green!70!black]
	plot[smooth,variable=\x,domain=0:4.4,samples=30]
	(\x,{-1+1.5*atan((-\x +2.7)*5)/180});
	
	\filldraw[very thin,blue!50!cyan!50!green!70!black,fill opacity=.4] (2.8,-1.215) circle (.6mm);
	\end{scope}
	
	\end{tikzpicture}
	
	Το περιθώριο φάσης συμβολίζεται \( \phi_M \) (margin) και έχει τύπο:
	\[
	\boxed{\phi_M = \ang{180} + \measuredangle A(j\omega_c)}
	\]
	όπου \underline{\( \omega_c \) είναι το σημείο τομής του πλάτους με τον άξονα των
		\( \SI{0}{\decibel} \)}.
\end{defn}

Για να καταλάβουμε ποιοτικά τη σημασία της απόστασης της φάσης από τις \( \ang{180} \),
δίνουμε ένα ημιτονοειδές σήμα σε ένα σύστημα μοναδιαίας αρνητικής ανάδρασης:

\begin{tikzpicture}
\tikzstyle{transfer}=[rectangle,draw,minimum height=9mm]

\draw (2,0) node[transfer] (h1) {$\displaystyle A(j\omega)$};
\draw (h1.west) ++(-1,0) node[circle,draw,minimum size=5mm] (mix) {};

\draw[<-] (mix) -- ++(-1,0) node[above] {$\sim$};
\draw[->] (mix) -- (h1);
\draw (h1.east) -- ++(1,0) node (end) {} node[midway] (fb) {} node[above right] {$\sim$};
\draw[->] (fb.center) -- ++(0,-1) -| (mix);
\draw (mix.south) node[below left] {$-$};
\end{tikzpicture}

Αν η συνάρτηση \( A(j\omega) \) προκαλεί σε εκείνο το σημείο διαφορά φάσης \( \ang{180} \),
τότε στην έξοδο εμφανίζεται αρχικά ένα αντίθετο ημίτονο (αφού \( \sin(t-\ang{180})=-\sin(t)
 \)). Αυτό όμως αφαιρείται από την είσοδο λόγω της ανάδρασης, άρα η είσοδος διπλασιάζεται,
και αυτή η διαδικασία επαναλαμβάνεται συνεχώς, οδηγώντας σε συνεχώς αυξανόμενη και άρα
ασταθή έξοδο.

Για την ευστάθεια θέλουμε όσο το δυνατόν πιο \textbf{θετικό περιθώριο φάσης}.

Επίσης, για το εύρος ζώνης \( \omega_b \) της συνάρτησης (δηλαδή το εύρος μεταξύ των
συχνοτήτων στις οποίες το κέρδος είναι \( \SI{-3}{\decibel} \)), ισχύει:
\[
\boxed{\omega_b > \omega_c}
\]

\begin{defn}{Περιθώριο κέρδους}{}
	Ως περιθώριο κέρδους ορίζουμε τον αριθμό:
	\[
	g_M = \frac{1}{\left\lvert A(j\omega_1) \right\rvert}
	\]
	όπου \( \omega_1 \) είναι η συχνότητα στην οποία η συνάρτηση μεταφοράς εισάγει
	διαφορά φάσης \( \ang{180} \).
	
	Για να έχουμε ευστάθεια, θέλουμε \( g_M > 1 \), ή, εκφρασμένο σε \( \si{\decibel} \):
	\[
	\boxed{-20\log\left\lvert A(j\omega_1) \right\rvert > 0}
	\]
	
	Δηλαδή το περιθώριο φάσης είναι η απόσταση (από κάτω προς τα επάνω) του πλάτους της
	\( A \), όταν η φάση της είναι \( \ang{180} \).
	
    \begin{tikzpicture}[scale=1.2]
    \pgfplotsset{
    	/pgf/number format/freq/.style={
    		fixed,
    	},
    }
    \def\phasedown{-1.2}
    \def\phaseextra{-2}
    
    \draw[->] (0,\phasedown+\phaseextra) -- (0,2);
    \draw[->] (0,1.1) -- (4.5,1.1) node[above] {$\si{\radian/\second}$};
    
    \def\at{20}
    \draw[ultra thick,orange!90!blue!90!black]
    (0.2,1.4) -- (1.5,1.4)
    .. controls (3,1.4) and (4,-1) .. (4,-1);
    
    \draw[dashed] (2.82,0.775) -- (2.82,\phasedown-1.48);
    
    \draw (0,0) node[above left,orange!90!blue!50!black] {$20\log\left|A(j\omega)\right|$};
    
    \filldraw[orange!90!blue!50!black,fill opacity=.4] (2.82,0.775) circle (.65mm) node (circ) {};
    \draw[very thick,->] ([yshift=.65mm]circ.center) -- (2.82,1.1) node[midway,right,yshift=-2mm,xshift=2mm] {$20\log g_M$};
    
    \begin{scope}[yshift=\phasedown cm]
    \draw[->] (0,0) -- ++(4.5,0);
    
    \draw[dashed] (0,-1.48) node[left,scale=.8] {$\ang{-180}$} -- ++(4,0);
    \draw (0,-1) node[left,blue!50!cyan!50!green!50!black] {$\measuredangle\ A(j\omega)$};
    
    \draw[very thick,blue!50!cyan!50!green!70!black]
    plot[smooth,variable=\x,domain=0:4.4,samples=30]
    (\x,{-1+1.5*atan((-\x +2.5)*5)/180});
    
    \filldraw[blue!50!cyan!50!green!70!black,fill opacity=.4] (2.82,-1.48) circle (1mm) node[opacity=1,below,yshift=-1mm] {$\omega_1$};
    \end{scope}
    
    \end{tikzpicture}
\end{defn}

Παρατηρούμε πως όσο αυξάνεται το πλάτος της συνάρτησης μεταφοράς, τόσο πιο κοντά στον
άξονα των \( \SI{0}{\decibel} \) φτάνει
(και μειώνεται το περιθώριο κέρδους), και το σύστημα γίνεται ασταθές μόλις τον ξεπεράσει.

\paragraph{Εφαρμογή}
Για το παρακάτω σύστημα:

\begin{tikzpicture}
\tikzstyle{transfer}=[rectangle,draw,minimum height=9mm,minimum width=10mm]

\draw (2,0) node[transfer] (h1) {$\displaystyle H_i(s)$};
\draw (h1.west) ++(-1,0) node[circle,draw,minimum size=5mm] (mix) {};
\draw (h1.east) ++(1.2,0) node[transfer] (h2) {$\displaystyle H(s)$};
\draw (h2.east) ++(1.5,0) node[circle,draw,minimum size=5mm] (d) {};

\draw[<-] (mix) -- ++(-1,0);
\draw[->] (mix) -- (h1);
\draw (h1) -- (h2) node[midway] (center) {} -- (d);
\draw[<-] (d) -- ++(0,1.2) node[right] {$d$};
\draw[->] (d.east) -- ++(1,0) node (end) {} node[midway] (fb) {} node[right] {$y$};
\draw[->] (fb.center) -- ++(0,-1.5) -| (mix);
\draw (mix.south);

\draw (h2.north) node[above] {$H(s)=\frac{6}{s+6}$};
\end{tikzpicture}

Να σχεδιαστεί ο ελεγκτής \( H_i(s) \) ώστε να τηρούνται ταυτόχρονα οι προδιαγραφές:
\begin{enumpar}
	\item Για βηματική είσοδο διαταραχής \( d=s(t) \), να έχουμε \( \displaystyle
	\lim_{t\to \infty} y(t) = 0 \).
	\item Για το εύρος ζώνης να ισχύει \( \omega_b > \SI{35}{\radian/\second} \)
	\item Για τα περιθώρια να ισχύει \( g_M = \infty\) και \( \phi_M > \ang{70} \)
\end{enumpar}

\subparagraph{Λύση}
Θυμόμαστε ότι η συνάρτηση \( H_i(s) \) θα πρέπει να περιέχει έναν ολοκληρωτή
για να καλύπτει την πρώτη προδιαγραφή, άρα
να έχει τη μορφή \( \underbrace{\frac{κ}{s}}_{\mathclap{\text{ελεγκτής } I}} \) ή
τη μορφή \( \underbrace{\frac{κ(s+z)}{s}}_{\mathclap{\text{ελεγκτής } PI}} \).

Η προδιαγραφή \( \omega_b > \SI{35}{\radian/\second} \) σημαίνει ότι \( \omega_c >
 \SI{35}{\radian/\second} \).
 
Ας ξεκινήσουμε να μελετάμε την \( \frac{κ}{s} \).
Η απαίτηση \( g_M = \infty \) ικανοποιείται, αφού η συνάρτηση μεταφοράς δεν φτάνει ποτέ
στις \( \ang{180} \), άρα έχει άπειρο περιθώριο κέρδους.

Για την απαίτηση \( \omega_c > 35 \), μελετάμε τις συναρτήσεις μεταφοράς.
Η \( H_i(s) \) έχει πλάτος (σε \( \si{\decibel} \)) \( 20\left[\log κ - \log\omega\right] \),
και η \( H(s) \) έχει πλάτος \( 20\left[-\log\frac{\omega}{6}\right] \). Επομένως πρέπει:
\begin{align*}
	20\left[\log κ - \log \omega_c - \log\frac{\omega_c}{6}\right] = 0
	\implies \log\left( \frac{κ}{\frac{\omega_c^2}{6}} \right) = 0
\end{align*}
άρα:
\begin{align*}
	\frac{6κ}{\omega_c^2} &= 1 \implies \omega_c = \sqrt{6}κ \\
	\omega_c &>35 \implies κ > \frac{35^2}{6} \implies \underline{κ > 204.16}
\end{align*}

Ας εξετάσουμε και το περιθώριο φάσης στη συχνότητα \( \omega_c \):
\begin{align*}
	\phi_M &> \ang{70} \implies \\
	\ang{180} - \ang{90} - \tan^{-1}\frac{\omega_c}{6} &> \ang{70} \implies \\
	\ang{90} - \ang{80.2} &\ngtr \ang{70}
\end{align*}
κάτι που δεν ισχύει, επομένως δεν μπορούμε να χρησιμοποιήσουμε τον ελεγκτή I, και πρέπει
να τοποθετήσουμε τον PI. Με τον ίδιο συλλογισμό, θα οδηγηθούμε σε ένα αποτέλεσμα:
\[
κ \geq 5.8\qquad z=10
\]

\paragraph{Παράδειγμα}
Έστω η συνάρτηση μεταφοράς με ένα μηδενικό:
\[
\frac{s-a}{a} = (1-sτ)
\]
η οποία για τις φανταστικές συχνότητες γίνεται:
\[
1-j\omega t \quad \text{ή} \quad 1+j\omega t
\]
και έχει διάγραμμα Bode:

\todo{Graph 92}

Παρατηρούμε ότι το μέτρο παραμένει σταθερό ανεξάρτητα από το
πρόσημο του \( a \), αλλά η φάση αλλάζει προσανατολισμό ανάλογα
με αυτό το πρόσημο, δηλαδή αν ο πόλος βρίσκεται αριστερά ή δεξιά
του φανταστικού άξονα.

\subparagraph{}
Γενικότερα, αν έχουμε ένα σύστημα βαθμού αριθμητή \( m \) και
παρονομαστή \( n \):
\[
\frac{N(s)\qquad{\mathsmaller{n}}}{D(s) \qquad \mathsmaller{m} }
\]
ισχύει:
\todo{ToDO για \omega -> 0, \infty}
και αυτά ονομάζονται \textit{Συστήματα μη ελάχιστης φάσης}.

\paragraph{Σύστημα με καθυστέρηση}
Έστω ένα σύστημα που βάζει καθυστέρηση \( T \) δευτερολέπτων
στην είσοδο \( u \):
\[
y(t) = u(t-T)
\]

Αν πάρουμε το μετασχηματισμό Laplace έχουμε:
\[
y(s) = e^{-sT} u(s)
\]

Δηλαδή το σύστημα είναι μία πολύ απλή συνάρτηση:
\todo{Graph 93}

και για να κατασκευάσουμε το διάγραμμα Bode, θεωρούμε \( s=j\omega \),
και:
\begin{align*}
	H(j\omega) &= e^{-j\omega T}
	\\
	\left\lvert H(j\omega) \right\rvert &= 1 \\
	\measuredangle H(j\omega) &= - \omega T
\end{align*}

Γενικότερα, αν είχαμε ένα σύστημα \( H_1(j\omega) \) και εφαρμόζαμε
μία καθυστέρηση σε αυτό, θα είχαμε:
\begin{align*}
	H(j\omega) &= e^{-j\omega T} H_1(j\omega) \\
	\left\lvert H(j\omega) \right\rvert &= \left\lvert H_1(j\omega)
	\right\rvert \\
	\measuredangle H(j\omega) &= -\omega T + \measuredangle H_1(j\omega)
\end{align*}
Δηλαδή εισάγεται μία καθυστέρηση φάσης στο σύστημα.

\paragraph{Παράδειγμα}
Έστω ότι δίνεται μία συνάρτηση μεταφοράς με χρονική καθυστέρηση:
\[
A(s) = e^{-sT} A_1(s)
\]

όπου ισχύει:
\[
A_1(s) = \frac{15(s+3)}{(s+1)(s+7)}
\]

Πόση μπορεί να είναι η χρονική καθυστέρηση μέχρι το σύστημα
να γίνει ασταθές;

\subparagraph{Λύση}
\todo{Πράξεις}
Με πράξεις βρίσκουμε ότι το σημείο στο οποίο η \( A_1 \) (και
ισοδύναμα η \( A \)) έχουν μέτρο 0 dB είναι:
\[
\omega_c = 13.6 \cong 15
\]

Για αυτήν την τιμή υπολογίζουμε:
\begin{align*}
	\measuredangle A_1(j\omega_c) \cong -71
\end{align*}

Τότε, το περιθώριο φάσης
για την \( A_1 \) (η απόσταση από τις \( \ang{180} \)) είναι:
\[
\theta_M\left( A_1(j\omega c) \right)
= \ang{180} - \ang{71} = \ang{109}
= \frac{109π}{180} \ \mathrm{rad}
\]

και, λόγω της καθυστέρησης, το περιθώριο φάσης για την \( A \)
γίνεται:
\[
\theta_M\left( A(j\omega_c) \right)
= -\omega_c T + \theta_M \left( A_1(j\omega_c) \right)
\cong -\omega_c T + 1.9024
\]
(προσοχή στο ότι πρέπει να λαμβάνουμε τις μονάδες σε rad!)

Τελικά, προκύπτει ότι:
\[
T < \SI{0.1396}{\second},
\]
δηλαδή η καθυστέρηση του συστήματος πρέπει να είναι μικρότερη από
0.1396 δευτερόλεπτα για να έχουμε ευστάθεια.

\paragraph{Παράδειγμα}
Έστω η συνάρτηση μεταφοράς ανοιχτού βρόχου:
\[
A(s) = \frac{300κ}{(s+2)(s+3)(s+5)}
\]

Για το σύστημα \underline{κλειστού} βρόχου:
\begin{enumparen}
	\item Ποιά τιμή πρέπει να έχει το \( κ \) ώστε να έχουμε ευστάθεια;
	\item Ποιά τιμή πρέπει να έχει το \( κ \) ώστε το περιθώριο κέρδους
	να είναι \( g_M = 3 \);
\end{enumparen}

\subparagraph{Λύση}
Η συνάρτηση μεταφοράς κλειστύ βρόχου προκύπτει:
\[
s^3+10s^2+31s+30+300κ = 0
\]

Επομένως μπορούμε να εφαρμόσουμε το \textbf{κριτήριο ευστάθειας Routh}
για να μελετήσουμε την ευστάθεια του συστήματος:
\[
\begin{array}{r|ccc}
s^3 & 1 & 31 & 0 \\
s^2 & 10 & 30+300κ & 0 \\
s^1 & 28-30κ & 0 & \\
s^0 & 30+300κ & &
\end{array}
\]

Θυμόμαστε ότι για να έχουμε ευστάθεια πρέπει όλοι οι συντελεστές
της πρώτης στήλης να είναι θετικοί:
\begin{align*}
	28 - 30κ &\geq 0 \quad \implies \boxed{κ \leq 0.93} \\
	30 + 300κ &\geq 0 \qquad \text{(ισχύει επειδή το $κ$ είναι θετικό)}
\end{align*}

Για το δεύτερο ερώτημα, θυμόμαστε ότι το περιθώριο κέρδους αναφέρεται
στην οριακή τιμή με την οποία μπορούμε να πολλαπλασιάσουμε το κέρδος του συστήματος για να έχουμε ευστάθεια, δηλαδή:
\[
g_M κ = 0.93 \implies \boxed{κ=0.31}
\]

\subsubsection{Διαγράμματα Nyquist}
Έστω η συνάρτηση μεταφοράς:
\[
\frac{1}{1+sτ} \quad \rightarrow \quad
\frac{1}{1+j\omega τ}
\]
με μέτρο \( \displaystyle \frac{1}{\sqrt{1+\omega^2τ^2}} \)
και φάση \( -\tan^{-1} \omega τ \).

Το διάγραμμα Nyquist είναι μία αναπαράσταση της συνάρτησης μεταφοράς στο
μιγαδικό επίπεδο.

Για \( \omega \to 0 \), το μέτρο γίνεται 1 και η φάση 0.\\
Για \( \omega \to \infty \), το μέτρο γίνεται 0 και η φάση \( \sfrac{π}{2} \).\\
Για \( \omega = \frac{1}{τ} \) (χαρακτηριστική συχνότητα), το μέτρο γίνεται
\( \frac{1}{\sqrt{2}} \), και η φάση \( -\frac{π}{4} \).

Μπορούμε επομένως να κατασκευάσουμε ένα διάγραμμα τοποθετώντας τα τρία παραπάνω
σημεία στο μιγαδικό επίπεδο:
\todo{Graph 94}

\paragraph{Παράδειγμα}
Έστω η συνάρτηση μεταφοράς με ολοκληρωτή:
\[
A(j\omega) =
\frac{1}{j\omega (1+j\omega τ)}
= \frac{1}{\omega^2 τ +j\omega }
= \frac{\omega^2 τ - j\omega }{\Re^2 + \Im^2}
= - \frac{τ}{\omega^2τ^2 + 1} - j\frac{1}{\omega (\omega^2τ^2+1)}
\]

Τότε, αντίστοιχα με παραπάνω, μπορούμε να βρούμε τα σημεία \todo{find points} και να σχεδιάσουμε το
διάγραμμα Nyquist:
\todo{Graph 95}

\paragraph{}
Γενικά για τα διαγράμματα Nyquist έχει σημασία το \textbf{κρίσιμο σημείο} \( \mathbf{-1+j0} \).
Αν το διάγραμμα δεν περνάει στα αριστερά του σημείου αυτού, τότε το σύστημα
είναι ευσταθές.

Πιο συγκεκριμένα, έστω μια συνάρτηση μεταφοράς \( A(j\omega) \) και το διάγραμμα
Nyquist της:
\todo{Graph 96}

Στο παραπάνω διάγραμμα, το περιθώριο κέρδους εκφράζεται από τον αριθμό με τον
οποίο μπορούμε να πολλαπλασιάσουμε το κέρδος της συνάρτησης στο σημείο τομής με τον
πραγματικό άξονα μέχρι να συναντήσουμε το κρίσιμο σημείο, ενώ το περιθώριο
φάσης εκφράζει το πόσο μπορούμε να περιστρέψουμε τη φάση της συνάρτησης για να
έχουμε ευστάθεια, και αντιστοιχεί στη γωνία \( \phi_M \).

\paragraph{Παράδειγμα}

\todo{Graph 97}
Στο παραπάνω διάγραμμα φάσης το περιθώριο κέρδους είναι \( \frac{1}{0.5} = 2 \),
δηλαδή μπορούμε να πολλαπλασιάσουμε το κέρδος με τον αριθμό 2 μέχρι να φτάσουμε
στο κρίσιμο σημείο.

\paragraph{Συνήθη διαγράμματα Nyquist}
\todo{Graph 100}

\paragraph{Γενικό κριτήριο Nyquist}
Παραπάνω παρουσιάστηκε το απλοποιημένο κριτήριο Nyquist, για το οποίο παίρναμε
τιμές του \( \omega \) μεταξύ του 0 και του \( +\infty \).

Για την εφαρμογή του γενικευμένου κριτηρίου Nyquist, δεν θεωρούμε πάντα ότι \( s=j\omega \),
αλλά παίρνουμε τιμές του \( s \) από το 0 μέχρι το \( +j\infty \), μετά συνεχίζουμε με
ένα ημικύκλιο άπειρης ακτίνας μέχρι το \( \omega = -j\infty \), για να επιστρέψουμε ξανά
στο 0.
\todo{Graph 101}

Τότε το διάγραμμα Nyquist γίνεται:
\todo{Graph 102}

και τα παραπάνω συνήθη διαγράμματα:
\todo{Graph 103}

\paragraph{}
Για το διάγραμμα Nyquist (στη γενικευμένη μορφή) ισχύει:
\[
\mathbf{N = Z - P}
\]

όπου:
\begin{itemize}
	\item \( N \) ο αριθμός των περικυκλώσεων γύρω από το \( -1 \) κατά την
	ωρολογιακή φορά
	\item \( Z \) 
	είναι ο αριθμός των \textbf{ασταθών πόλων} του συστήματος \textbf{κλειστού βρόχου}
	(δηλαδή οι ρίζες της χαρακτηριστικής εξίσωσης \( 1+A(s) \))
	\item \( P \) 
	είναι ο αριθμός των \textbf{ασταθών πόλων} του συστήματος \textbf{ανοιχτού βρόχου}
	(δηλαδή οι πόλοι της \( A(s) \))
\end{itemize}
(όπου με τον όρο \textit{ασταθείς πόλοι} εννοούμε τους πόλους που βρίσκονται στο δεξί
ημιεπίπεδο)

Το ζητούμενο συνήθως είναι να βρούμε τον αριθμό των ασταθών πόλων του συστήματος
\textit{κλειστού βρόχου}, ενώ, όπως αναφέραμε παραπάνω, δεν μπορούμε να χρησιμοποιήσουμε
το απλοποιημένο κριτήριο, επειδή η \( A(s) \) έχει και ασταθείς πόλους.

\subparagraph{Για να βρούμε τον αριθμό \( N \) των περικυκλώσεων}
Για να βρούμε τον αριθμό των περικυκλώσεων, παίρνουμε μια ημιευθεία με οποιαδήποτε κλίση από το κρίσιμο σημείο \( (-1) \).

Για κάθε σημείο τομής της \textit{ημιευθείας} με το \textit{διάγραμμα}, αντιστοιχίζουμε
έναν αριθμό, ανάλογα με τη φορά του διαγράμματος σε σχέση με την ευθεία: \( -1 \) αν
το διάγραμμα τείνει να στρέψει την ευθεία κατά την \textit{αντιωρολογιακή} φορά,
και \( +1 \) αν το διάγραμμα τείνει να στρέψει την ευθεία κατά την \textit{ωρολογιακή} φορά:

\todo{Graph 103}

Το ζητούμενο \( N \) είναι το άθροισμα των αριθμών που αντιστοιχούμε στο κάθε σημείο τομής,
δηλαδή:
\[
N = -1 + 1 = 0
\]

\subsubsection{Ασκήσεις}
\paragraph{Παράδειγμα}
Να σχεδιαστεί το πλήρες διάγραμμα Nyquist και να εξεταστεί η ευστάθεια του συστήματος:
\todo{Graph 104}

\subparagraph{Λύση}
Κατ' αρχάς, παίρνω τη συνάρτηση ανοιχτού βρόχου:
\[
A(s) = \frac{50}{(s+2)(s-6)}
\]

και ξεκινώ την κατασκευή του διαγράμματος.

\begin{itemize}
	\item Για \( \omega  = 0 \):\[
	A(0) = \frac{50}{-12} = -4.16
	\]
	\item Για \( \omega \to \infty \): \[
	A(\infty) = 0
	\]
	
	Μάλιστα, με δύο πόλους πολλαπλότητας 1, η φάση του διαγράμματος στο \( +\infty \)
	καταλήγει στις \( \ang{-180} \), για πολύ μεγάλο \( \omega  \), όπως γνωρίζουμε από
	τα διαγράμματα Bode.
	\item Για να βρούμε αν το διάγραμμα βρίσκεται στο επάνω ή στο κάτω ημιεπίπεδο,
	κάνουμε τις πράξεις για \( s= j \omega  \):
	\begin{align*}
		A(j\omega ) &= \frac{50}{(j\omega +2)(j\omega -6)}
		\\ &= \frac{-50(\omega^2+12) + 200j\omega }{(\omega^2+12)^2 + 16\omega^2}
	\end{align*}
	Παρατηρούμε ότι το φανταστικό μέρος (\( \frac{200\omega}{(\omega^2+12)^2} + 16\omega^2 \)) είναι θετικό, άρα το απλό διάγραμμα Nyquist βρίσκεται στην επάνω μεριά
	του επιπέδου:
	
	\todo{Graph 105}
\end{itemize}

Για να σχεδιάσουμε το πλήρες διάγραμμα Nyquist, λαμβάνουμε υπ' όψιν και τα αρνητικά
\( \omega \), δηλαδή σχεδιάζουμε και το συζυγές του παραπάνω διαγράμματος:
\todo{Graph 106}

Όσον αφορά το ημικυκλικό κομμάτι που παρουσιάσαμε στη θεωρία, σε εκείνο ισχύει
ότι για \( s \to \infty: \left|A(s)\right| = 0 \), οπότε αντιστοιχεί στο σημείο \( 0 \).

Για να βρούμε αν το σύστημα κλειστού βρόχου είναι ασταθές, πρώτα θεωρούμε μια ημιευθεία
γύρω από το -1:
\todo{Graph 107}

Αυτή τέμνει το διάγραμμα Nyquist σε ένα σημείο με ωρολογιακή φορά, οπότε αντιστοιχούμε σε
αυτόν τον αριθμό 1. Επομένως, για το δείκτη Nyquist ισχύει:
\[
N = +1
\]

Η συνάρτηση ανοιχτού βρόχου έχει έναν πόλο στο δεξί ημιεπίπεδο (τον \( +6 \)):
\[
P = 1
\]

Άρα τελικά ο αριθμός των πόλων του κλειστού βρόχου υπολογίζεται:
\[
N = Z-P \implies \boxed{Z = N+P = 2}
\]

Δηλαδή το σύστημα κλειστού βρόχου έχει δύο ασταθείς πόλους, και επομένως είναι ασταθές.

\paragraph{Άσκηση}
Δίνεται το παρακάτω διάγραμμα Nyquist:
\todo{Graph 108}

Ποιό είναι το περιθώριο κέρδους; Σε ποιό διάστημα ανήκει το περιθώριο φάσης;
\subparagraph{Λύση}
Το περιθώριο κέρδους είναι άπειρο, αφού δεν υπάρχει αριθμός με τον οποίο μπορούμε
να πολλαπλασιάσουμε το κέρδος της συνάρτησης για να φτάσει το σημείο -1.

Για το περιθώριο φάσης, παίρνουμε το σημείο τομής της συνάρτησης με το μοναδιαίο κύκλο:
\todo{Graph 109}

Παρατηρούμε από το διάγραμμα ότι ισχύει \( \ang{90} < \phi_M < \ang{150} \).

\subparagraph{}
Διαισθητικά, αν περιστρέψουμε το διάγραμμα κατά το περιθώριο φάσης, δηλαδή κατά
\( \phi_M \), τότε η καμπύλη θα συμπέσει με το κρίσιμο σημείο -1, άρα θα αρχίσουμε να
έχουμε αστάθεια.

Αντίστοιχα, αν "φουσκώσουμε" (κάνουμε scaling) την καμπύλη, πολλαπλασιάζοντας το μέτρο
της με έναν αριθμό που είναι το περιθώριο κέρδους, τότε μια καμπύλη πάλι θα συμπέσει
με το -1, και θα υπάρχει αστάθεια. Στη συγκεκριμένη άσκηση βέβαια, όσο κι αν μεγαλώσουμε
την καμπύλη, αυτή δεν πρόκειται να φτάσει στο -1.

\paragraph{Άσκηση}
Δίνεται το διάγραμμα Bode της \( A(s) \):
\todo{Graph}
Ποιό είναι το περιθώριο φάσης;

\subparagraph{Λύση}
\todo{Graph}
Αν \( \omega_c \) είναι η συχνότητα στην οποία η συνάρτηση έχει μηδενικό κέρδος, θα έχουμε:
\[
\phi_M = \measuredangle A(j\omega_c) - (\ang{-180}) = \ang{180} + \measuredangle A(j\omega_c)
\]

Από το διάγραμμα παρατηρούμε ότι \( \measuredangle A(j\omega_c) \in (\ang{-102},\ \ang{-92}) \), ή, προσεγγιστικά,
\( \measuredangle A(j\omega_c) = \ang{-98} \).

Άρα:
\[
\phi_M = \ang{180} - \ang{98} = \ang{102}
\]

\paragraph{Άσκηση}
Δίνεται το διάγραμμα Bode της \( A(s) \):
\todo{Graph}

Είναι το σύστημα ευσταθές;

\subparagraph{Λύση}
Για το \textbf{περιθώριο φάσης} ισχύει:
\[
\phi_M = \ang{180} + \measuredangle A(j\omega_c)
\]

Όμως για τη γωνία \( \measuredangle A(j\omega_c) \) παρατηρούμε από το διάγραμμα ότι ισχύει \( \measuredangle A(j\omega_c) < \ang{-180} \). Επομένως:
\[
\phi_M < 0
\]
άρα το σύστημα είναι \textit{ασταθές}.

\paragraph{Άσκηση}
Ποιό είναι το περιθώριο κέρδους για το παρακάτω διάγραμμα Bode;
\todo{Graph}

\subparagraph{Λύση}
Για το περιθώριο κέρδους ισχύει:
\[
g_m = 0 - \left|A(\omega_1)\right|_{\mathrm{dB}}
\]
όμως για τη συχνότητα \( \omega_1 \) στην οποία η φάση είναι \( \ang{-180} \) ισχύει περίπου \( \left|A(\omega_1)\right|_{\mathrm{dB}} = \SI{-45}{\decibel} \), άρα:
\[
g_m = 0 - (-45) = \SI{45}{\decibel}
\]

\paragraph{Άσκηση}
Να εξεταστεί η ευστάθεια του συστήματος με χαρακτηριστική συνάρτηση:
\[
s^4+8s^3+18s^2+16s+5=0
\]

\subparagraph{Λύση}
Εδώ μπορούμε να εφαρμόσουμε το κριτήριο ευστάθειας Routh:

\[
\begin{array}{r|rrc}
s^4 & 1 & 18 & 5\\
s^3 & 8 & 16 & 0\\
s^2 & \frac{8\cdot18-1\cdot16}{8} = 16 & \frac{8\cdot5-1\cdot0}{8} = 5\\
s^1 & 13.5 & 0 \\
s^0 & 5
\end{array}
\]

Εφ' όσον δεν υπάρχει αλλαγή προσήμου στην πρώτη στήλη, το σύστημα είναι ευσταθές.

\paragraph{Άσκηση}
Να εξεταστεί η ευστάθεια του συστήματος με χαρακτηριστική εξίσωση:
\[
s^5+s^4+2s^3+2s^2+3s+5=0
\]

\subparagraph{Λύση}
Εφαρμόζουμε κριτήριο Routh:
\begin{multicols*}{2}
	\[
	\begin{array}{r|ccc}
	s^5 & \textcolor{green!50!black}{1} & 2 & 3 \\
	s^4 & \textcolor{green!50!black}{1} & 2 & 5\\
	s^3 & \textcolor{green!50!black}{ε} & -2 & \\
	s^2 & \textcolor{green!50!black}{\frac{2ε+2}{ε}} & 5 &\\
	s^1 & \textcolor{red!50!black}{\frac{4ε-4-5ε^2}{2ε+2}} & 0 \\
	s^0 & \textcolor{green!50!black}{5}
	\end{array}
	\]
	\columnbreak

	Στην 3η σειρά προκύπτει 0. Για αυτό θεωρούμε έναν αριθμό \( ε>0 \) αλλά πολύ μικρό. Όταν προκύπτει 0 στην πρώτη θέση στήλης,
	τότε έχουμε πόλο στον φανταστικό άξονα ή στο δεξί ημιεπίπεδο.
	
	Ο πρώτος όρος της προτελευταίας γραμμής είναι αρνητικός, αφού:
	\[
	\lim_{\epsilon\to0^+} \frac{-4ε-4-5ε^2}{2ε+2} = -2
	\]
	
	Οι υπόλοιποι όροι της πρώτης στήλης είναι θετικοί.
\end{multicols*}

Αφού έχουμε 2 εναλλαγές προσήμου στην 1\textsuperscript{η} στήλη, έχουμε 2 ρίζες στο δεξί ημιεπίπεδο, και το σύστημα είναι ασταθές.

\paragraph{Άσκηση}
Έστω το σύστημα:

\begin{tikzpicture}
\draw (2,0) node[rectangle,draw] (h) {$\displaystyle \frac{k}{s(s^2+s+1)(s+4)}$};
\draw (h.west) ++(-1,0) node[circle,draw,minimum size=5mm] (mix) {};

\draw[<-] (mix) -- ++(-1,0);
\draw[->] (mix) -- (h);
\draw[->] (h.east) -- ++(2,0) node[midway] (fb) {};
\draw[->] (fb.center) -- ++(0,-1) -| (mix);
\end{tikzpicture}

Είναι το σύστημα ευσταθές για κάποιο εύρος \( k \), και ποιό είναι αυτό;

\subparagraph{Λύση}
Αρχικά υπολογίζουμε τη συνάρτηση μεταφοράς του συστήματος (αν θεωρήσουμε ότι η συνάρτηση μεταφοράς \textit{ανοιχτού} βρόχου είναι
\(G\)):
\[
G_{\text{ΣΜΚΒ}}(s) = \frac{G}{1+G} = \frac{k}{s(s^2+s+1)(s+4)+k} = \frac{k}{s^4+5s^3+5s^2+4s+k}
\]

Και εφρμόζουμε το κριτήριο Routh:
\[
\begin{array}{r|ccc}
s^4&1&5&k\\
s^3&5&4&0\\
s^2&\frac{21}{5}&\\
s^1&\frac{84-25k}{21} & 0\\
s^0&k
\end{array}
\]

Για να είναι ευσταθές το σύστημα θέλουμε να μην υπάρχει καμία εναλλαγή προσήμου στην πρώτη στήλη, δηλαδή:
\[
\left.
\begin{aligned}
84-25k > 0 \implies k &< 3.36 \\
k&>0
\end{aligned}
\right\rbrace \implies 0 < k < 3.36
\]

\paragraph{Άσκηση}
Δίνεται το παρακάτω σύστημα:

\begin{tikzpicture}
	\tikzstyle{transfer}=[rectangle,draw,minimum height=9mm]
	\draw (2,0) node[transfer] (h1) {$\displaystyle \frac{5}{s(s+2)}$};
	\draw (0,0) node[circle,draw,minimum size=5mm] (mix) {};
	\draw (h1.south) ++(0,-1) node[transfer] (h2) {$\displaystyle \frac{4}{s+5}$};
	
	\draw[<-] (mix) -- ++(-1,0);
	\draw[->] (mix) -- (h1);
	\draw[->] (h1.east) -- +(1,0) node[midway] (fb) {};
	\draw[->] (fb.center) |- (h2) -| (mix);
	\draw (mix.south);
\end{tikzpicture}

Να προσδιοριστούν το περιθώριο φάσης και το περιθώριο κέρδους.

\subparagraph{Λύση}
\begin{enumgreekparen}
	\item Για το περιθώριο φάσης βρίσκουμε τη συνάρτηση μεταφοράς ανοιχτού βρόχου και σχεδιάζουμε το διάγραμμα Bode:
	\[
	G(s) = \frac{5}{s(s+2)}\frac{4}{s+5} = \frac{20}{s(s+2)(s+5)}
	\]
	
	\todo{Graph}
	
	Βρίσκουμε την \( \omega_c \) ώστε \( \left|A(j\omega_c)\right| = 1 \):
	\[
	G(s) = \frac{20}{j\omega\left(\frac{j\omega}{2}+1\right)\left(\frac{j\omega}{5}+1\right)\cdot 2\cdot5}
	\]
	
	\begin{enumroman}
		\item Για \( \omega_c \ll 2\):
		\[
		G(s) \approx \frac{20}{2\cdot 5 j\omega}
		\]
		Και \( |G| = 1 \implies \omega_c = 2 \), άτοπο αφού υποθέσαμε ότι \( \omega_c \ll 2 \).
		
		\item
		Για \( \omega_c \ll 5 \):
		\[
		G(s) \approx \frac{20}{2.5(j\omega)\left(\frac{j\omega}{2}+1\right)}
		\]
		
		Λύνουμε την εξίσωση \( G(s) = 1 \) και έχουμε:
		\begin{align*}
			\left| G(s) \right| &= 1 \implies \\
			\left|\frac{20}{5s(s+2)}\right| &= 1 \implies \\
			\frac{4}{\left| -\omega^2 + 2j\omega \right|} &= 1 \implies \\
			16 &= \left|-\omega^2+2j\omega\right|^2 \implies \\
			\omega^4+4\omega^2&=16 \implies \underline{\omega_c = 1.57}
		\end{align*}
	\end{enumroman}

	Επομένως, για \( \omega_c = \SI{1.57}{\radian/\second} \), έχουμε (για τον κάθε πόλο):
	\[
	\left.
	\begin{aligned}
	\measuredangle \frac{1}{s} &= \ang{-90} \\
	\measuredangle \frac{1}{s+2} &= \measuredangle \frac{1}{1.57j+2}
	= \tan^{-1} \left(\frac{1.57}{2}\right) = \ang{-38} \\
	\measuredangle \frac{1}{s+5} &= \ang{-17}
	\end{aligned}
	\right\rbrace \xRightarrow{+} \measuredangle A(j\omega_c) = \ang{-145}
	\]
	
	Άρα τελικά:
	\[
	\phi_M = \measuredangle A(j\omega_c) - (\ang{-180}) = \ang{35}
	\]
	
	Μάλιστα, μπορούμε να σχεδιάσουμε αναλυτικά το προσεγγιστικό διάγραμμα Bode, υπολογίζοντας την τιμή για \( \omega=1 \),
	η οποία προκύπτει \( 20\log 2 \si{\decibel} \):
	\todo{Graph}
	
	Συγκεκριμένα, έχουμε:
	\[
	\begin{cases}
	\text{για } \omega \in [1,2] \implies &\quad |A|_{\mathrm{dB}} = 20\log2 - 20\log(\omega) \\
	\text{για } \omega \in [2,5] \implies &\quad |A|_{\mathrm{dB}} = 20\log2 - 20\log(\omega) -20\log\left(\frac{\omega}{2}\right) \\
	\text{για } \omega \in [5,\infty) \implies &\quad |A|_{\mathrm{dB}} = 20\log2 - 20\log(\omega) -20\log\left(\frac{\omega}{2}\right) -20\log\left(\frac{\omega}{5}\right) \\
	\end{cases}
	\]
	
	\item Για να βρούμε το περιθώριο κέρδους, πρώτα ψάχνουμε τη συχνότητα \( \omega_a \) για την οποία ισχύει \( \measuredangle
	A(j\omega_a) = \ang{-180} \):
	\begin{align*}
		\left(\measuredangle\frac{1}{s}\right)+ \left(\measuredangle\frac{1}{s+2}\right)
		+ \left(\measuredangle\frac{1}{s+5}\right) &= \ang{-180}\\
		-\ang{90} + \measuredangle \frac{1}{j\omega_a+2} + \measuredangle \frac{1}{j\omega_a+5} &= \ang{-180} \\
		-\ang{90} -\tan^{-1}\left(\frac{\omega_a}{2}\right)-\tan^{-1}\left(\frac{\omega_a}{5}\right) &= \ang{-180} \\
		-\tan^{-1}\left(\frac{\omega_a}{2}\right)-\tan^{-1}\left(\frac{\omega_a}{5}\right) &= \ang{-90}
	\end{align*}
	
	Για να προχωρήσουμε χρησιμοποιούμε την ταυτότητα:
	\[
	\infoboxed{
		\tan^{-1}(x) + \tan^{-1}(y) = \tan^{-1} \left( \frac{x+y}{1-xy} \right)	
	}
	\]
	
	Άρα:
	\[
	\tan^{-1}\left(\frac{\frac{7\omega_a}{10}}{1-\frac{\omega_a^2}{10}}\right)
	= \tan^{-1}\left(\frac{\frac{7\omega_a}{10}}{\frac{10-\omega_a^2}{10}}\right)
	= \tan^{-1}\left(\frac{7\omega_a}{10-\omega_a^2}\right) = \ang{90}
	\]
	
	Για να τείνει η γωνία στις \( \ang{90} \) πρέπει:
	\[
	\frac{7\omega_a}{10-\omega_a^2} \to \infty \implies 10-\omega_a^2 = 0 \implies \underline{\omega_a = \sqrt{10} \approx 3.16}
	\]
	
	Εφ' όσον \( \omega_a \in (2,5] \) παίρνουμε την κατάλληλη προσέγγιση για το πλάτος της συνάρτησης:
	\[
	\left| A(j\omega_a) \right|_{\mathrm{dB}} = 20\log 2-20\log(3.16)-20\log\left(\frac{3.16}{2}\right) \approx
	\SI{-7.95}{\decibel}
	\]
	
	Άρα τελικά:
	\[
	g_m = - |A| = \SI{7.95}{\decibel}
	\]
	
	\subparagraph{}
	Εναλλακτικά, για να βρούμε το περιθώριο κέρδους μπορούμε να χρησιμοποιήσουμε το κριτήριο Routh, και να ψάξουμε ποιός
	είναι ο μέγιστος όρος με τον οποίο μπορούμε να πολλαπλάσιασουμε τη συνάρτηση ανοιχτού βρόχου και να παραμείνουμε στην
	ευστάθεια.
	
	Για αυτό χρειάζεται να θεωρήσουμε το σύστημα:
	\todo{Graph}
	με συνάρτηση μεταφοράς:
	\[
	H_c = \frac{G}{1+GF} = \frac{5k(s+5)}{s(s+2)(s+5)+k\cdot 20}
	\]
	και χαρακτηριστική εξίσωση:
	\[ s(s+2)(s+5)+20k = 0 \iff s^3+7s^2+10s+20k = 0 \]
	
	Εφαρμόζουμε το κριτήριο Routh:
	\[
	\begin{array}{r|cc}
	s^3&1&10 \\
	s^2&7&20k \\
	s^1 & \frac{70-20k}{7} & 0\\
	s^0 & 20k
	\end{array}
	\]
	
	Θέλουμε \(
		\begin{cases}
		\frac{70-20k}{7}> 0 \implies 70-20k > 0 \implies k<\frac{7}{2}\\
		20k > 0 \implies k>0
		\end{cases}
	\), δηλαδή \( k_{\max} = \frac{7}{2} \).
	
	Επομένως:
	\[
	g_m = k_{\max} = 3.5 \to 20\log\left( \frac{7}{2} \right) = 10.88
	\]
	\todo{looks wrong}
\end{enumgreekparen}

\paragraph{Άσκηση}
Να σχεδιαστεί διαγραμμα Nyquist και να εξεταστεί η ευστάθεια του συστήματος κλειστού βρόχου, αν η συνάρτηση ανοιχτού βρόχου είναι:
\[
A(s) = \frac{s+2}{(s+3)(s+5)}
\]
\subparagraph{Λύση}
Πρώτα βρίσκουμε τις ακραίες τιμές:
\begin{align*}
	A(0) &= 0.133 + j0 \\
	\left|A(\infty)\right| &= 0 + j0
\end{align*}

Αφού έχουμε ένα μηδενικό (\( \ang{+90} \)) και δύο πόλους (\( 2\cdot(\ang{-90}) \)), η γωνία στο \( \infty \) γίνεται:
\[
\phi(\omega\to\infty) = \ang{-90}
\]

Για να ελέγξουμε από ποιά μεριά είναι το διάγραμμα, θεωρούμε:
\begin{align*}
\frac{s+2}{s^2+8s+15} &= \frac{j\omega+2}{-\omega^2+8j\omega+15} = \frac{(j\omega+2)(15-\omega^2-8j\omega)}{(-\omega^2+15+8j\omega)(15-\omega^2-8j\omega)}\\
&=\frac{15j\omega-j\omega^3-8j^2\omega^2+30-2\omega^2-16j\omega}{(15-\omega^2)^2+(8\omega)^2}
=\underbrace{\frac{30+6\omega^2}{\omega^4+34\omega^2+225}}_{>0}
\underbrace{-j\frac{\omega(\omega^2+1)}{\omega^4+34\omega^2+225}}_{<0}
\end{align*}

Με τη μελέτη της μορφής του μιγαδικού, μπορούμε να καταλάβουμε ότι το μηδέν προσεγγίζεται από κάτω. Εναλλακτικοί τρόποι
προσέγγισής του είναι οι εξής:
\todo{Graph}

Τελικά, το διάγραμμα Nyquist είναι:
\todo{Graph}

Αφού δεν περικυκλώνεται το -1, το σύστημα είναι ευσταθές.

\paragraph{Άσκηση}
Να σχεδιαστεί το διάγραμμα Nyquist:
\[
G(s) = \frac{10(s+2)}{(s+3)^2+25}
\]
\subparagraph{Λύση}
Έχουμε:
\[
A(j\omega) = \frac{10\left( 4\omega^2+68-j\omega(\omega^2-22) \right)}{\omega^4-32\omega^2+1156}
\]
και επιπλέον:
\begin{alignat*}{2}
	\omega=0 : &&\quad A(j\omega) &= 0.588 \\
	\omega\to \infty : &&\quad \left|A(j\omega)\right| &= 0
\end{alignat*}

Θα ψάξουμε ακόμα τα σημεία τομής με τον πραγματικό άξονα:
\[
\Im\left[A(j\omega)\right] = 0 \implies \omega(\omega^2-22) = 0 \implies \begin{cases}
\omega &= 0\\
\omega &= \sqrt{22} = 4.69
\end{cases}
\]

Τέλος, παρατηρούμε ότι \( \Re\left[A(j\omega)\right] = 1.66 \), ότι για \( \omega \in [0,\ 4.69] \implies \omega^2-22 < 0
\implies \Im\left[A(j\omega)\right] > 0 \), άρα προσεγγίζουμε από επάνω, και ότι στον φανταστικό άξονα πηγαίνουμε μόνο όταν
\( \Re\left[A(j\omega)\right] = 0 \implies \omega \to \infty \):
\todo{Graph}

Αφού το -1 δεν περικυκλώνεται, έχουμε \( N= 0\). Επειδή δεν υπάρχουν πόλοι δεξιά, ισχύει \( P = 0\). Άρα τελικά \( Z=0\), και
το σύστημα είναι ευσταθές.

\paragraph{Άσκηση}
Να σχεδιαστεί το διάγραμμα Nyquist και να εξεταστεί η ευστάθεια:
\todo{Graph}
\subparagraph{Λύση}
Η (ισοδύναμη) συνάρτηση μεταφοράς είναι:
\[
A(s) = \frac{6}{s+1}\frac{7}{s^2+5s+2}\frac{8}{s+2} = \frac{336}{s^4+8s^3+19s^2+16s+4}
\]
άρα:
\[
A(j\omega) = \frac{336}{\omega^4-j8\omega^3-19\omega^2+j16\omega+4}
= \frac{336\left[ (\omega^4-19\omega^2+4)+j(8\omega^3-16\omega) \right]}{\omega^8 + 26\omega^6 + 113\omega^4 + 104\omega^2+16}
\]

Αντίστοιχα με την προηγούμενη άσκηση, υπολογίζουμε:
\begin{align*}
	A(0) &= 84\\
	\left|A(\infty)\right| &= 0 \\
	\Im\left[ A(j\omega) \right] &= 0 \implies 8\omega^3-16\omega = 0 \implies \omega = 0 \text{ ή } \omega=\sqrt{2} \\
	\Im\left[A(j\omega)\right] &<0 \quad \text{(άρα προσεγγίζουμε από κάτω)} \\
	A(j\sqrt{2}) &= -11.2 \\
	\phi(\infty) &= -\ang{90}-\ang{90}-\ang{90}-\ang{90} = \ang{-360}
\end{align*}

Άρα το διάγραμμα γίνεται:
\todo{Graph}

Παρατηρούμε ότι έχουμε \( N=2 \) και \( P=0 \), άρα \( Z=2\), δηλαδή το σύστημα είναι ασταθές.

\section{Γεωμετρικός τόπος ριζών}
Έστω ότι έχουμε το σύστημα:
\todo{Graph 110}

Τότε η συνάρτηση μεταφοράς έχει τη μορφή
\[
H_p(s) = \frac{\prod_{i}(s+z_i)}{s^N \prod_j (s+p_j)}
= \frac{N(s)}{D(s)}
\]

Για το συγκεκριμένο σύστημα, η συνάρτηση μεταφοράς ανοιχτού βρόχου
είναι:
\[
H_p(s) = \frac{κ}{s(s+10+κb)}
\]

Για να βρούμε το \textbf{γεωμετρικό τόπο ριζών}, χρειάζεται να
βρούμε μια εξίσωση της μορφής:
\[
\boxed{1+ K_\gamma H_p(s) = 0}
\]

Για το συγκεκριμένο πρόβλημα μπορούμε να εργαστούμε με \( K_γ = κ \)
ή \( Κ_\gamma = b \), δηλαδή:
\begin{enumgreekparen}
	\item Για \( K_γ = κ,\quad b=\mathrm{const.} \):
	\[
	1 + Kb \cdot \frac{(s+\sfrac{1}{b} )}{s(s+10)} = 0
	\]
	\item Για \( K_γ = b,\quad κ=\mathrm{const.}\):
	\[
	1 + bk \frac{s}{s^2+10s+κ} = 0
	\]
\end{enumgreekparen}

%todo{Reorganise}

Επομένως, έχουμε τη μορφή:
\[
1+K_γ\frac{N(s)}{D(s)} = 0
\implies
D(s) + K_γN(s) = 0
\]

Όταν στην παραπάνω εξίσωση θέσουμε το \( K_γ = 0 \), τότε αποκτά
τη μορφή \( D(s) = 0\), δηλαδή οι λύσεις της είναι οι πόλοι που αντιστοιχούν
στο χαρακτηριστικό πολυώνυμο.

Αντίστοιχα, αν θέσουμε \( Κ_γ \to \infty \), τότε αποκτά τη μορφή
\( N(s) \), δηλαδή οι λύσεις είναι τα μηδενικά της συνάρτησης μεταφοράς.

Για όλες τις υπόλοιπες τιμές του \( κ \), η λύση της παραπάνω
εξίσωσης αποκτά μια τιμή στο μιγαδικό επίπεδο. Ο γεωμετρικός τόπος
των ριζών αυτών μπορεί να σχεδιαστεί:
\todo{Graph 112}

Αυτό σημαίνει ότι, επιλέγοντας τις τιμές του \( κ \), μπορούμε να
έχουμε πόλους για το σύστημα μόνο στα συγκεκριμένα σημεία που
φαίνονται παραπάνω και ανήκουν στο γεωμετρικό τόπο. Παρ' όλα αυτά,
μπορούμε να μελετήσουμε πώς η μεταβολή του \( κ \) επηρεάζει
τις θέσεις των πόλων του συστήματος. Για παράδειγμα, στο συγκεκριμένο
διάγραμμα, παρατηρούμε πως όσο αυξάνεται το \( κ \to \infty \),
τότε, ακολουθώντας τα βελάκια, ο ένας πόλος πηγαίνει στο \( \infty \),
αλλά ο άλλος πηγαίνει όλο και πιο δεξιά, καθιστώντας το σύστημα
πιο αργό.

Αντίστοιχα, για το \( b \) με την εξίσωση \( 1+bκ\frac{s}{s^2+10s+κ}=0 \), έχουμε τον εξής
γεωμετρικό τόπο:
\todo{Graph 113}

\paragraph{Κανόνες}
Για να διευκολυνθούμε στο σχεδιασμό των διαγραμμάτων γεωμετρικών
τόπων, ακολουθούμε μερικούς \textbf{κανόνες}:
\begin{itemize}
	\item Το πλήθος των πόλων της \( D(s) \) είναι το πλήθος των κλάδων \( n \)
	\item Το πλήθος των πόλων της γης είναι 2
	\todo{remove}
	\item Το πλήθος των μηδενικών είναι ίσο με το πλήθος των κλάδων που
	τείνουν σε ένα μηδενικό \( m \)
	\item Η διαφορά του πλήθους των πόλων και των μηδενικών είναι το
	πλήθος των κλάδων που τείνουν στο άπειρο
	\item Τα τμήματα του \textit{πραγματικού} άξονα που ανήκουν στο
	γεωμετρικό τόπο είναι αυτά που αφήνουν δεξιά τους περιττό πλήθος
	πόλων και μηδενικών.
	\item
	Το σημείο τομής των ασυμπτώτων με τον πραγματικό άξονα είναι το:
	\begin{align*}
		σ_A &= \frac{\sum p_i - \sum z_i}{n-m}
		\intertext{Και η γωνία σε εκείνο το σημείο είναι}
		\theta_i &= \frac{(2i+1)\cdot 180}{n-m}
		\qquad i=0,1,\dots,n-m-1
	\end{align*}
	\item
	Για να βρούμε τα σημεία σύγκλισης (\textbf{σημεία απόσχισης}) δύο καμπυλών, βρίσκουμε τη λύση της παραγώγου:
	\[
	\od{H_p(s)}{s} = 0
	\]
	και απορρίπτουμε τυχόν λύσεις που δεν ανήκουν στο γεωμετρικό τόπο.
	
	Στο συγκεκριμένο πρόβλημα, για \( \frac{1}{b}=20 \), βρίσκουμε
	\( s^2+40s+200 =0 \), άρα \( s = -5.35 \) και \( s=-34.14 \) για το πρώτο διάγραμμα,
	και \( s=-6 \) για το δεύτερο διάγραμμα.
	
	\item Για τις γωνίες αναχώρησης από τα μηδενικά και τους πόλους ισχύει:
	\[
	\sum \measuredangle(s-z_i) - \sum\measuredangle(s-p_i) = \ang{180}
	\]
	Δηλαδή το άθροισμα όλων των γωνιών αναχώρησης από μηδενικά μείον πόλους
	είναι \( \ang{180} \).
	
	Για να βρούμε τη γωνία αναχώρησης από έναν πόλο, βρίσκουμε τις γωνίες από τους υπόλοιπους
	πόλους και μηδενικά, και εφαρμόζουμε τον τύπο:
	\todo{Graph 114}
	
	Συγκεκριμένα, στο παραπάνω παράδειγμα, βρίσκουμε \( \theta_x = \ang{-123.5} \):
	\todo{Graph 115}
	
	\item Επειδή έχουμε συζυγείς μιγαδικές ρίζες, έχουμε συμμετρία ως προς τον οριζόντιο
	άξονα.

\end{itemize}

\paragraph{Παράδειγμα}
Θέλουμε να βρούμε το γεωμετρικό τόπο των ριζών για την:
\[
A(s) = \frac{κ}{s(s+2)(s+4)}
\]
\subparagraph{Λύση}
Η εξίσωση που πρέπει να λύσουμε είναι:
\[
1+k\frac{1}{s(s+2)(s+4)} = 0
\]

Αρικά τοποθετούμε στο μιγαδικό επίπεδο τους πόλους:
\todo{Graph 116}

Εφ' όσον έχουμε 3 πόλους, έχουμε και 3 κλάδους, και επειδή δεν υπάρχει κανένα μηδενικό,
οι κλάδοι αυτοί πηγαίνουν στο άπειρο.

Το σημείο τομής των ασυμπτώτων είναι:
\[
σ_A = \frac{-4-2-0}{3-0} = -2
\]
και οι ασύμπτωτες έχουν γωνία (από τον τύπο \( \theta_i = \frac{(2i+1)\cdot180}{n-m} \)):
\begin{align*}
	\theta_1 &= \ang{60} \\
	\theta_2 &= \ang{-60} \\
	\theta_3 &= \ang{180}
\end{align*}
επομένως στο διάγραμμα μπορούμε να προσθέσουμε τις ασύμπτωτες και τα κομμάτια του
πραγματικού άξονα:
\todo{Graph 117}
Επίσης, λύνοντας την εξίσωση \( \od{A}{s} = 0 \), βρίσκουμε το σημείο απόσχισης στη θέση
\( s \approx -0.85 \), και μπορούμε να σχεδιάσουμε το γεωμετρικό τόπο:
\todo{Graph 118}

\paragraph{}
Για να έχουμε \textbf{άπειρο περιθώριο κέρδους}, θέλουμε να μην εμφανίζεται κανένα κομμάτι
του γεωμετρικού τόπου στο δεξί ημιεπίπεδο. Για να το διορθώσουμε αυτό στη συνάρτηση
του παραδείγματος, μπορούμε να αλλάξουμε τη διεύθυνση των ασυμπτώτων, προσθέτοντας ένα
μηδενικό, για παράδειγμα στη θέση \( \mu = -5 \):
\[
A(s) = \frac{κ(s+μ)}{s(s+2)(s+4)}
\]

Τότε ο γεωμετρικός τόπος θα αποκτήσει αυτήν τη μορφή:
\todo{Graph 119}

\end{document}
