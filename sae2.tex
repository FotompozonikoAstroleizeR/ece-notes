\documentclass[11pt,a4paper,notitlepage,fleqn,final]{article}

\usepackage{amsmath}
\usepackage{amsfonts}
\usepackage{amssymb}
\usepackage{libs/commath2}
\usepackage[table]{xcolor}
\usepackage[hidelinks,draft=false]{hyperref}
\usepackage[skins,theorems]{tcolorbox}
\usepackage{titlesec}
\usepackage{tikz}
\usepackage{libs/circuitikz} % use our own recent version to make sure some bugs are fixed
\usepackage{pgfplots}
\usepackage{mathtools}
\usepackage[makeroom]{cancel}
\usepackage{mathrsfs}
\usepackage{wrapfig}
%\usepackage{subcaption}
%\usepackage{floatrow}
\usepackage{esint}
\usepackage{enumitem}
%\usepackage{bm}
\usepackage{relsize}
\usepackage{xfrac}
\usepackage{comment}
\usepackage{siunitx}
\usepackage{multicol}
%\usepackage{MnSymbol}
\usepackage[obeyDraft,disable]{todonotes}
%\usepackage{morefloats} % oh no!
%\usepackage[linesnumbered,lined]{algorithm2e}
\usepackage{glossaries}
\usepackage{xifthen}


\pgfplotsset{compat=1.13}
\usetikzlibrary{arrows.meta}
\usetikzlibrary{patterns}
\usetikzlibrary{decorations.pathmorphing}
\usetikzlibrary{decorations.markings}
\usetikzlibrary{backgrounds}
\usetikzlibrary{shapes.misc}
\usetikzlibrary{shapes.multipart}
\usetikzlibrary{shadows.blur}
\usetikzlibrary{fadings}
\usetikzlibrary{intersections}
\usetikzlibrary{arrows.meta}
\usetikzlibrary{calc}
\usetikzlibrary{matrix}
\usetikzlibrary{positioning}
\usetikzlibrary{shapes}
\usetikzlibrary{shadings}

\tcbuselibrary{breakable}
\tcbuselibrary{skins}
\tcbuselibrary{xparse}

\tikzset{cross/.style={cross out, draw,
        minimum size=2*(#1-\pgflinewidth),
        inner sep=0pt, outer sep=0pt}}
\tikzset{
    mark position/.style args={#1(#2)}{
        postaction={
            decorate,
            decoration={
            	post length=1mm, % ??? Magic to fix "Dimension
            	pre length=1mm, % ???  too large" errors.
                markings,
                mark=at position #1 with \coordinate (#2);
            }
        }
    }
}
\tikzset{
	arrow at/.style args={#1}{
		postaction={
			decorate,
			decoration={
				post length=1mm, % ??? Magic to fix "Dimension
				pre length=1mm, % ???  too large" errors.
				markings,
				mark=at position #1 with {\arrow{>}};
			}
		}
	}
}
\makeatletter
\tikzset{
  use path for main/.code={%
    \tikz@addmode{%
      \expandafter\pgfsyssoftpath@setcurrentpath\csname tikz@intersect@path@name@#1\endcsname
    }%
  },
  use path for actions/.code={%
    \expandafter\def\expandafter\tikz@preactions\expandafter{\tikz@preactions\expandafter\let\expandafter\tikz@actions@path\csname tikz@intersect@path@name@#1\endcsname}%
  },
  use path/.style={%
    use path for main=#1,
    use path for actions=#1,
  }
}
\makeatother

\pgfmathdeclarefunction{sinc}{1}{%
	\pgfmathparse{abs(#1)<0.01 ? int(1) : int(0)}%
	\ifnum\pgfmathresult>0 \pgfmathparse{1}\else\pgfmathparse{sin(#1 r)/#1}\fi%
}
\pgfmathdeclarefunction{gauss}{2}{%
	\pgfmathparse{1/(#2*sqrt(2*pi))*exp(-((x-#1)^2)/(2*#2^2))}%
}

\usepackage[left=2cm,right=2cm,top=2cm,bottom=2cm]{geometry}

%\usepackage[no-math]{fontspec}
%\usepackage{fontspec}
\usepackage{mathspec}
%\usepackage{newtxtext,newtxmath}
%\usepackage{unicode-math}
%\setmainfont{texgyretermes-regular.otf}
%\setsansfont{texgyreheros-regular.otf}
%\newfontfamily\greekfont[Script=Greek]{Linux Libertine O}
%\newfontfamily\greekfontsf[Script=Greek]{Linux Libertine O}
\usepackage{polyglossia}
%\newfontfamily\greekfont[Script=Greek]{texgyretermes-regular.otf}
\newfontfamily\greekfontsf[Script=Greek]{texgyreheros-regular.otf}
\newfontfamily\greekfonttt[Script=Greek]{Latin Modern Mono}
%\usepackage[greek]{babel}
\setdefaultlanguage{greek}
\setotherlanguage{english}

%\usepackage[utf8]{inputenc}
%\usepackage[greek]{babel}


%\usepackage{tkz-euclide} % loads  TikZ and tkz-base
%\usetkzobj{angles} % important you want to use angles

\newlist{enumparen}{enumerate}{1}
\setlist[enumparen]{label=(\arabic*)}
\newlist{enumpar}{enumerate}{1}
\setlist[enumpar]{label=\arabic*)}

\newlist{enumgreek}{enumerate}{1}
\setlist[enumgreek]{label=\alph*.}
\newlist{enumgreekparen}{enumerate}{1}
\setlist[enumgreekparen]{label=(\alph*)}
\newlist{enumgreekpar}{enumerate}{1}
\setlist[enumgreekpar]{label=\alph*)}


\newlist{enumroman}{enumerate}{1}
\setlist[enumroman]{label=(\roman*)}

\newlist{enumlatin}{enumerate}{1}
\setlist[enumlatin]{label=(\alph*)}

\newlist{invitemize}{itemize}{1}
\setlist[invitemize]{noitemsep,label=}

\input{libs/fiximplies}
\input{libs/sphere}

\makeatletter
\let\anw@true\anw@false

%\newcommand{\attnboxed}[1]{\textcolor{red}{\fbox{\normalcolor\m@th$\displaystyle#1$}}}
\makeatother
\tcbset{highlight math style={enhanced,colframe=red,colback=white,%
        arc=0pt,boxrule=1pt,shrink tight,boxsep=1.5mm,extrude by=0.5mm}}
\newcommand{\attnboxed}[1]{\tcbhighmath[colback=red!5!white,drop fuzzy shadow,arc=0mm]{#1}}
\newcommand{\infoboxed}[1]{%
	\tcbhighmath[colframe=blue!50!white,colback=blue!5!white,arc=0mm]{#1}}
\titleformat{\section}{\bf\Large}{Κεφάλαιο \thesection}{1em}{}
\newtcolorbox{attnbox}[1]{colback=red!5!white,%
    colframe=red!75!black,fonttitle=\bfseries,title=#1}
\newtcbox{quickattnbox}[1]{colback=red!5!white,%
	colframe=red!75!black,fonttitle=\bfseries,title=#1}
\newtcolorbox{infobox}[1]{colback=blue!5!white,%
    colframe=blue!75!black,fonttitle=\bfseries,title=#1}

\tcbset{frogbox/.style={enhanced jigsaw,%
		overlay first={\foreach \x in {0cm} {
				\begin{scope}[shift={([xshift=-0.2cm]title.west)}]
					\draw[very thick,green!65!black!50!white,latex-] (0,0) -- ++(-1.5,0);
\end{scope}}}}}
\tcbset{frogtitle/.style={
attach boxed title to top left=
{xshift=0mm,yshift=-0.50mm},
boxed title style={skin=enhancedfirst jigsaw,
	bottom=0mm,
	interior style={fill=none,
		left color=green!20!black,
		right color=gray}}
}}
\DeclareTColorBox{exercise}{ O{} }{
	enhanced jigsaw,
	breakable,parbox=false,
	%title style={left color=gray!50!white!50!green,opacity=.5,right color=white},
	subtitle style={%boxrule=1pt,
		colback=yellow!50!red!25!white,fontupper=\bfseries},
	coltitle=black,colbacktitle=green!90!black!25!white,colframe=black,
	frame hidden,
	boxrule=0mm,
	%boxrule=1mm,
	leftrule=0.8pt,toprule=0.8pt,rightrule=0pt, %reserve space
	borderline west={0.8pt}{0pt}{white!25!black},%---- draw line
	borderline north={0.8pt}{0pt}{white!25!black},%---- draw line
	interior hidden,
	%frame style={left color=black,right color=white},
	sharp corners=all,
	%frogbox, %TODO: frogbox
	before lower={\tcbsubtitle[before skip=\baselineskip]{Λύση}},lower separated=false,
	before title={\textbf{Άσκηση\ifthenelse{\isempty{#1}}{}{: }}},
	title={\ifthenelse{\isempty{#1}}{\hspace{0pt}}{#1}}%
}

\AtBeginDocument{%
\let\arg\relax
\let\Re\relax
\let\Im\relax
\DeclareMathOperator{\arg}{Arg}
\DeclareMathOperator{\Re}{Re}
\DeclareMathOperator{\Im}{Im}
}
\DeclareMathOperator{\sinc}{sinc}
\DeclareMathOperator{\sgn}{sgn}
\DeclareMathOperator{\erf}{erf}
\DeclareMathOperator{\cov}{cov}
\DeclareMathOperator{\atand}{atan2}

\newenvironment{absolutelynopagebreak}
{\par\nobreak\vfil\penalty0\vfilneg
	\vtop\bgroup}
{\par\xdef\tpd{\the\prevdepth}\egroup
	\prevdepth=\tpd}

\DeclareSIUnit \voltampere { VA } %apparent power 
\DeclareSIUnit \var { VAr } %volt-ampere reactive - idle power 
\DeclareSIUnit \decade { dec } %decade

% Global amount of samples
% Set to a higher value (e.g. 200) for nicer graphs
% Set to a low value (e.g. 10) for performance
% NOTE: Check the sample variables below for further measurements
\newcommand*{\gsamples}{200}

% Equals command as a workaround for CircuiTikZ bug
% not allowing the = sign in labels
\newcommand*{\equals}{=}

\newcommand{\nesearrow}{%
	\,%
	\smash{\raisebox{-1.1ex}
		{$%
			\stackrel{\displaystyle\nearrow}{\displaystyle\searrow}%
			$}}%
}
\newcommand{\degree}{^{\circ}} % not great
\newcommand\numberthis{\addtocounter{equation}{1}\tag{\theequation}} % add an equation number to a number-less math environment

% Provided commands
\providecommand\dif{d}
\providecommand\od[2]{\frac{#1}{#2}}

\newtcbtheorem[number within=section,list inside=thm]{theorem}{Θεώρημα}%
{colback=green!5,colframe=green!35!black,colbacktitle=green!35!black,fonttitle=\bfseries,enhanced,attach boxed title to top left={yshift=-2mm,xshift=-7mm},width=.9\textwidth,arc=.7mm}{th}
\newtcbtheorem[number within=section,list inside=defn]{defn}{Ορισμός}%
{colback=blue!5,colframe=cyan!35!black,colbacktitle=blue!35!black,fonttitle=\bfseries,enhanced,attach boxed title to top left={yshift=-2mm,xshift=-2mm}}{def}

% Locus plot utilities
\tikzset{locus/.style={orange!50!red!70!brown}}
\tikzset{locuspole/.style={draw=red!30!black,cross,inner sep=2.5pt,fill=white,fill opacity=.6,thick,label={[below]-90:#1}}}
\tikzset{locuszero/.style={draw=red!30!black,circle,inner sep=2pt,fill=white,fill opacity=.6,thick,label={[below]-90:#1}}}
\tikzset{locusbreak/.style={rounded corners=1.5pt,inner sep=2pt,draw,top color=brown,bottom color=black,fill opacity=.8,label={[below]-90:#1}}}

% New plotting utilities
\def\lowsamples{18}
\def\hisamples{40}
\def\timecolour{blue!50!cyan}

\tikzstyle{timecolour}=[\timecolour]



\title{ΣΑΕ 2
	\\
	{ 
		\normalsize Συστήματα Αυτομάτου Ελέγχου II
		\\
		\normalsize Σημειώσεις από τις παραδόσεις
	}}
\date{Φεβρουάριος 2018
	\\
	{ 
		\small Τελευταία ενημέρωση: \today
	}
}
\author{
	Για τον κώδικα σε \LaTeX, ενημερώσεις και προτάσεις:
	\\
	\url{https://github.com/kongr45gpen/ece-notes}}

\setallmainfonts(Digits,Latin,Greek){Asana Math}
\setmainfont{Noto Serif}
\setsansfont{Ubuntu}
\usepackage{polyglossia}
\newfontfamily\greekfont[Script=Greek,Scale=1.00]{Liberation Serif}

\hypersetup{pdftitle = {ΣΑΕ 2}}


\begin{document}
\maketitle

\hrule
\vspace{50pt}

\begin{infobox}{Λάθη \& Διορθώσεις}
	Οι τελευταίες εκδόσεις των σημειώσεων βρίσκονται στο Github
	(\url{https://github.com/kongr45gpen/ece-notes/raw/master/sae1=2.pdf}) ή
	στη διεύθυνση \url{http://helit.org/ece-notes/sae2.pdf}.
	
	Περιέχουν διορθώσεις σε λάθη και τυχόν βελτιώσεις.
	
	\tcblower
	
	Μπορείτε να ενημερώνετε για οποιοδήποτε λάθος και πρόταση
	μέσω PM στο forum, issue στο Github, ή οποιουδήποτε άλλου τρόπου!
\end{infobox}
	
Το μάθημα περιλαμβάνει ένα μικρό προαιρετικό εργαστήριο. Η επιλογή γίνεται με βάση
προαιρετικής προόδου (μπαίνουν οι πρώτοι 18, εφ' όσον έχουν γράψει βαθμό \( \geq 6 \)).

Εφ' όσον δοθεί, η πρόοδος συμμετέχει κατά 25\% στον τελικό βαθμό
(και το υπόλοιπο 75\% στις εξετάσεις), για αυτούς που συμμετάσχουν
σε αυτήν. Διαφορετικά, μετράν οι εξετάσεις κατά 100\%. Το εργαστήριο μετράει προσθετικά
με μέγιστο βαθμό \( +3 \), εφ' όσον ο βαθμός της εξέτασης είναι από 4 και πάνω.

Αν δηλωθεί η πρόοδος, ο βαθμός της μετράει όπως παραπάνω, και δεν υπάρχει δυνατότητα να
ακυρωθεί, ακόμα και αν ο φοιτητής δεν την παραδώσει.

\newpage

\tableofcontents

\newpage

\subsection{Εισαγωγή}
Στα Συστήματα Αυτόματου Ελέγχου 2 οι αναλύσεις γίνονται στο πεδίο του \textbf{χρόνου}
και όχι της \textbf{συχνότητας}. Αυτό επιτρέπει να μελετηθούν συστήματα μη γραμμικά,
καθώς και συστήματα με περισσότερες από μία εισόδους και εξόδους.

Γενικότερα, τα ΣΑΕ έχουν εφαρμογές σε πολυάριθμους τομείς, όπως η αυτοκίνηση (φρένα
ABS, σύστημα πρόσφυσης, διατήρηση ευστάθειας σε πλαγιολίσθηση, \textellipsis), έλεγχος
κινητήρων, έλεγχος υπερμεγέθων τηλεσκοπίων, κατανομή πρόσβασης σε δίκτυα internet και
τηλεφωνικά, διαχείριση συστημάτων ενέργειας (για διανομή, ασφάλεια, αξιοπιστία, π.χ.
απόσβεση διαταραχών μετά από κεραυνό)\textellipsis

Θυμόμαστε ότι \textbf{\textit{σύστημα}} είναι οποιαδήποτε λειτουργική μονάδα που διεγείρεται
από κάποιες εισόδους, και επιστρέφει κάποιες εξόδους.

\begin{tikzpicture}
\draw (0,0) node[rectangle,draw,inner sep=15pt,minimum width=20pt] (r) {$S$};
\draw[<-] (r.west) -- ++(-2,0) node[above right] {είσοδος};
\draw[->] (r.east) -- ++(2,0) node[above left] {έξοδος};
\end{tikzpicture}

Ένα \textbf{ελεγχόμενο σύστημα} είναι τέτοιο ώστε να φροντίζουμε η έξοδος \( y \) να έχει μια
επιθυμητή τιμή \( r \), και το οποίο συχνά περιλαμβάνει και μια είσοδο διαταραχής
\( d \), που δεν μπορούμε να ελέγξουμε.

\begin{tikzpicture}
\draw (0,0) node[rectangle,draw,inner sep=15pt,minimum width=20pt] (r) {$S$};
\draw[<-] (r.west) ++ (0,-0.3) -- ++(-2,0) node[above right] {$u$};
\draw[<-] (r.west) ++ (0,0.3) -- ++(-2,0) node[above right] {$r$};
\draw[<-] (r.north) -- ++(0,1) node[midway,right] {$d(t)$};
\draw[->] (r.east) -- ++(2,0) node[above left] {$y$};
\end{tikzpicture}

Μια ακόμα χρήσιμη έννοια που μάθαμε είναι αυτή της ανάδρασης, στην οποία η έξοδος του συστήματος ανατροφοδοτείται στο σύστημα ως είσοδος, ίσως αφού περαστεί από έναν
ελεγκτή \( C \). Σε αυτά μπορούμε να προσθέσουμε μια μετρητική διάταξη \( M \) και έναν
ενεργοποιητή (actuator) που να μετατρέπει την έξοδο σε μορφή αποδεκτή από το σύστημα. Αυτά
είναι και τα \textbf{συστήματα κλειστού βρόχου}.

\begin{tikzpicture}
\draw (0,0) node[rectangle,draw,inner sep=15pt,minimum width=20pt] (r) {$S$};
\draw (-3,0) node[rectangle,draw,inner sep=10pt,minimum width=10pt] (c) {$C$};
\draw (-1.5,0) node[rectangle,draw,minimum height=30pt,minimum width=7pt] (e) {$E$};

\draw (-0.5,-1.5) node[rectangle,draw,minimum width=30pt,inner sep=8pt] (d) {$M$};

\draw[->] (e) -- (r) node[midway, above] {$u$};
\draw[->] (c) -- (e);
\draw[<-] (c.west) ++ (0,0.3) -- ++(-2,0) node[above right] {$r$};
\draw[<-] (r.north) -- ++(0,1) node[midway,right] {$d(t)$};
\draw[->] (r.east) -- ++(2,0) node[above left] {$y$} node[midway] (m) {};

\draw[->] (m.center) node[circ] {} |- (d) -- ++(-3.5,0) |- (c);
\end{tikzpicture}

Στόχος των μαθημάτων είναι ο σχεδιασμός του ελεγκτή \( C \) ώστε να ικανοποιούνται
συγκεκριμένες προδιαγραφές. Χρειάζεται βέβαια και μια διαισθητική κατανόηση των εννοιών.
Για παράδειγμα, αν έχουμε προδιαγραφή το σύστημα να έχει έξοδο \( 1 \) στη μόνιμη κατάσταση,
είναι προτιμότερο να φτάσει σε αυτήν με υπεραποσβεννύμενη απόκριση, παρά με ταλαντώσεις.
\todo{Graph 1}

Για τη \textbf{μοντελοποίηση} των συστημάτων μπορούμε είτε να υπολογίσουμε και να
αναλύσουμε τη φυσική λειτουργία του συστήματος, είτε να μελετήσουμε σύνολα εισόδων και
εξόδων ώστε να προβλέψουμε τη συμπεριφορά τους.

Στην πραγματικότητα βέβαια δεν θα μας δίνονται οι μαθηματικές προδιαγραφές, αλλά οι
φυσικές προδιαγραφές του συστήματος.

Υπάρχει μάλιστα η περίπτωση τα λειτουργικά κομμάτια των παραπάνω διατάξεων να μην
είναι συνδεδεμένα φυσικά μεταξύ τους, αλλά να βρίσκονται σε απόσταση, εισάγοντας
χρονικές καθυστερήσεις στη μεταφορά των σημάτων (π.χ. drones). Άλλα προβλήματα μπορεί
να είναι ο κβαντισμός των σημάτων (π.χ. για ασύρματη μεταφορά δεδομένων), περιορισμοί του
hardware ή του software.

\paragraph{Παράδειγμα}
Έστω ένα αυτοκίνητο μάζας \( m \) που κινείται σε δρόμο κλίσης \( \phi \) με ταχύτητα
\( y(t) \). Στο αυτοκίνητο ασκείται δύναμη οδήγησης \( u(t) \) και δύναμη του αέρα ανάλογη
με την ταχύτητα, με συντελεστή \( a \) (αεροδυναμικός συντελεστής).
Οι τριβές μεταξύ αυτοκινήτου και οδοστρώματος θεωρούνται αμελητέες.

Επιθυμούμε να σχεδιάσουμε έναν ελεγκτή \( u(t) \) που να ελέγχει τη δύναμη οδήγησης, ώστε
το αυτοκίνητο να κινείται με σταθερή ταχύτητα.

\subparagraph{Λύση}
\todo{Graph 2}

Από το νόμο του Νεύτωνα \( \sum F = m\ddot y \)έχουμε:
\[
m\dot y = u - ay -mg\sin\phi
\]
δηλαδή φτάσαμε σε μία διαφορική εξίσωση που μοντελοποιεί το πρόβλημα.

Παρατηρούμε μάλιστα τον όρο \( \left[-mg\sin\phi\right] \), ο οποίος δεν εξαρτάται από μεταβλητές που
μπορούμε να επηρεάσουμε, αλλά μόνο από την κλίση του δρόμου, που πιθανώς αλλάζει. Δηλαδή
αποτελεί την \textbf{είσοδο διαταραχών}.

Η επιθυμητή έξοδος του συστήματος που δίνεται ως είσοδο είναι \( y(t) = r \), και εδώ
θεωρούμε για παράδειγμα ότι \( r = \SI{25}{\meter/\second} \).

\begin{itemize}
	\item Ισχυρίζομαι ότι μπορώ να λύσω το πρόβλημα \textbf{χωρίς κλειστό βρόχο}, δηλαδή
	χωρίς να φτάνει στον ελεγκτή \( C \) η έξοδος \( y \):
	\[
	u(t) = kr(t)
	\]
	
	Και για απλότητα στους υπολογισμούς θεωρώ \( \phi = 0 \).
	
	Τότε, από το παραπάνω μοντέλο του συστήματος, προκύπτει η διαφορική εξίσωση:
	\[
	m\dot y = -ay + k\cdot25
	\]
	με λύση:
	\[
	y(t) = 25\frac{k}{a}\left(
	1-e^{-\frac{a}{m} t}
	\right),\quad t\geq 0
	\]
	ή, αν επιλέξουμε το κέρδος του ελεγκτή να είναι \( k = a \), η ταχύτητα θα γίνει
	όντως \( \SI{25}{\meter/\second} \). Όμως ο χρόνος αποκατάστασης εξαρτάται από τα
	\( a \) και \( m \), που είναι παράμετροι του ελεγχόμενου συστήματος, και δεν μπορούν
	να επηρεαστούν. Δηλαδή ο ρυθμός με τον οποίο συγκλίνουμε στο 25 εξαρτάται μόνο από τις
	παραμέτρους του ελεγχόμενου συστήματος. Συνήθως έχουμε μεγάλη μάζα \( m \) για το
	αυτοκίνητο και μικρό αεροδυναμικό συντελεστή. Επομένως το πηλίκο \( \frac{a}{m} \) είναι
	μικρό και η απόκριση λογικά αργή.
	
	Ακόμα δεν είναι γνωστή τις περισσότερες φορές η τιμή του συντελεστή \( a \). Αν για
	παράδειγμα ανοίξουμε ένα παράθυρο, μεταβάλλεται ανάλογα με το πόσο το έχουμε ανοίξει.
	Γενικότερα στη διαδικασία της σχεδίασης \textbf{δεν μπορούμε να χρησιμοποιήσουμε
		μεγέθη που δεν είναι γνωστά}.
	
	Επίσης, στην περίπτωση που θεωρήσουμε ότι \( \phi \neq 0 \), η λύση θα προκύψει μετά
	από πράξεις:
	\[
	y(t) = \underbrace{\left(\frac{25k}{a} - \frac{mg\sin\phi}{a}\right)}_{\text{θέλουμε } = 25}
	\left(  1-e^{-\frac{a}{m} t} \right)
	\]
	επομένως:
	\[
	k = \frac{25a + mg\sin\phi}{25}
	\]
	το οποίο πάλι δεν μπορεί να υπολογιστεί εύκολα, αφού η κλίση του δρόμου \( \phi \) κάθε
	φορά δεν είναι γνωστή.
	\item Χρησιμοποιούμε \textbf{αναλογικό ελεγκτή κλειστού βρόχου}:
	\[
	u(t) = k \cdot \Big( r(t) - y(t) \Big), \qquad k > 0 
	\]
	
	Τότε η λύση της διαφορικής εξίσωσης θα προκύψει, μετά από πράξεις:
	\[
	y(t) = \left(
	\frac{25k - mg\sin\phi}{a+k}
	\right)\left( 1-e^{-\left(\frac{a+k}{m}\right)t} \right)
	\]
	
	Όσο μεγαλώνουμε το \( k \), ο παράγοντας στον εκθέτη αυξάνεται, άρα μεγαλώνει η ταχύτητα
	σύγκλισης στη μόνιμη τιμή. Όσον αφορά τον πρώτο όρο, για \( k \) που φτάνει στο
	\( \infty \), έχουμε:
	\[
	\lim_{k \to \infty} \left( \frac{mg\sin\phi}{a+k} \right) = 0
	\]
	και ο πρώτος όρος γίνεται 25.
	
	Η μικρή διαφορά που υπάρχει για πεπερασμένες τιμές του \( k \) είναι ουσιαστικά το
	σφάλμα θέσης, που μπορούμε να εξαλείψουμε με έναν ολοκληρωτή.
\end{itemize}

\end{document}
