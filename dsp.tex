% !TeX program = xelatex
\documentclass[11pt,a4paper,notitlepage,fleqn]{article}

\usepackage{amsmath}
\usepackage{amsfonts}
\usepackage{amssymb}
\usepackage{libs/commath2}
\usepackage[table]{xcolor}
\usepackage[hidelinks,draft=false]{hyperref}
\usepackage[skins,theorems]{tcolorbox}
\usepackage{titlesec}
\usepackage{tikz}
\usepackage{libs/circuitikz} % use our own recent version to make sure some bugs are fixed
\usepackage{pgfplots}
\usepackage{mathtools}
\usepackage[makeroom]{cancel}
\usepackage{mathrsfs}
\usepackage{wrapfig}
%\usepackage{subcaption}
%\usepackage{floatrow}
\usepackage{esint}
\usepackage{enumitem}
%\usepackage{bm}
\usepackage{relsize}
\usepackage{xfrac}
\usepackage{comment}
\usepackage{siunitx}
\usepackage{multicol}
%\usepackage{MnSymbol}
\usepackage[obeyDraft,disable]{todonotes}
%\usepackage{morefloats} % oh no!
%\usepackage[linesnumbered,lined]{algorithm2e}
\usepackage{glossaries}
\usepackage{xifthen}


\pgfplotsset{compat=1.13}
\usetikzlibrary{arrows.meta}
\usetikzlibrary{patterns}
\usetikzlibrary{decorations.pathmorphing}
\usetikzlibrary{decorations.markings}
\usetikzlibrary{backgrounds}
\usetikzlibrary{shapes.misc}
\usetikzlibrary{shapes.multipart}
\usetikzlibrary{shadows.blur}
\usetikzlibrary{fadings}
\usetikzlibrary{intersections}
\usetikzlibrary{arrows.meta}
\usetikzlibrary{calc}
\usetikzlibrary{matrix}
\usetikzlibrary{positioning}
\usetikzlibrary{shapes}
\usetikzlibrary{shadings}

\tcbuselibrary{breakable}
\tcbuselibrary{skins}
\tcbuselibrary{xparse}

\tikzset{cross/.style={cross out, draw,
        minimum size=2*(#1-\pgflinewidth),
        inner sep=0pt, outer sep=0pt}}
\tikzset{
    mark position/.style args={#1(#2)}{
        postaction={
            decorate,
            decoration={
            	post length=1mm, % ??? Magic to fix "Dimension
            	pre length=1mm, % ???  too large" errors.
                markings,
                mark=at position #1 with \coordinate (#2);
            }
        }
    }
}
\tikzset{
	arrow at/.style args={#1}{
		postaction={
			decorate,
			decoration={
				post length=1mm, % ??? Magic to fix "Dimension
				pre length=1mm, % ???  too large" errors.
				markings,
				mark=at position #1 with {\arrow{>}};
			}
		}
	}
}
\makeatletter
\tikzset{
  use path for main/.code={%
    \tikz@addmode{%
      \expandafter\pgfsyssoftpath@setcurrentpath\csname tikz@intersect@path@name@#1\endcsname
    }%
  },
  use path for actions/.code={%
    \expandafter\def\expandafter\tikz@preactions\expandafter{\tikz@preactions\expandafter\let\expandafter\tikz@actions@path\csname tikz@intersect@path@name@#1\endcsname}%
  },
  use path/.style={%
    use path for main=#1,
    use path for actions=#1,
  }
}
\makeatother

\pgfmathdeclarefunction{sinc}{1}{%
	\pgfmathparse{abs(#1)<0.01 ? int(1) : int(0)}%
	\ifnum\pgfmathresult>0 \pgfmathparse{1}\else\pgfmathparse{sin(#1 r)/#1}\fi%
}
\pgfmathdeclarefunction{gauss}{2}{%
	\pgfmathparse{1/(#2*sqrt(2*pi))*exp(-((x-#1)^2)/(2*#2^2))}%
}

\usepackage[left=2cm,right=2cm,top=2cm,bottom=2cm]{geometry}

%\usepackage[no-math]{fontspec}
%\usepackage{fontspec}
\usepackage{mathspec}
%\usepackage{newtxtext,newtxmath}
%\usepackage{unicode-math}
%\setmainfont{texgyretermes-regular.otf}
%\setsansfont{texgyreheros-regular.otf}
%\newfontfamily\greekfont[Script=Greek]{Linux Libertine O}
%\newfontfamily\greekfontsf[Script=Greek]{Linux Libertine O}
\usepackage{polyglossia}
%\newfontfamily\greekfont[Script=Greek]{texgyretermes-regular.otf}
\newfontfamily\greekfontsf[Script=Greek]{texgyreheros-regular.otf}
\newfontfamily\greekfonttt[Script=Greek]{Latin Modern Mono}
%\usepackage[greek]{babel}
\setdefaultlanguage{greek}
\setotherlanguage{english}

%\usepackage[utf8]{inputenc}
%\usepackage[greek]{babel}


%\usepackage{tkz-euclide} % loads  TikZ and tkz-base
%\usetkzobj{angles} % important you want to use angles

\newlist{enumparen}{enumerate}{1}
\setlist[enumparen]{label=(\arabic*)}
\newlist{enumpar}{enumerate}{1}
\setlist[enumpar]{label=\arabic*)}

\newlist{enumgreek}{enumerate}{1}
\setlist[enumgreek]{label=\alph*.}
\newlist{enumgreekparen}{enumerate}{1}
\setlist[enumgreekparen]{label=(\alph*)}
\newlist{enumgreekpar}{enumerate}{1}
\setlist[enumgreekpar]{label=\alph*)}


\newlist{enumroman}{enumerate}{1}
\setlist[enumroman]{label=(\roman*)}

\newlist{enumlatin}{enumerate}{1}
\setlist[enumlatin]{label=(\alph*)}

\newlist{invitemize}{itemize}{1}
\setlist[invitemize]{noitemsep,label=}

\input{libs/fiximplies}
\input{libs/sphere}

\makeatletter
\let\anw@true\anw@false

%\newcommand{\attnboxed}[1]{\textcolor{red}{\fbox{\normalcolor\m@th$\displaystyle#1$}}}
\makeatother
\tcbset{highlight math style={enhanced,colframe=red,colback=white,%
        arc=0pt,boxrule=1pt,shrink tight,boxsep=1.5mm,extrude by=0.5mm}}
\newcommand{\attnboxed}[1]{\tcbhighmath[colback=red!5!white,drop fuzzy shadow,arc=0mm]{#1}}
\newcommand{\infoboxed}[1]{%
	\tcbhighmath[colframe=blue!50!white,colback=blue!5!white,arc=0mm]{#1}}
\titleformat{\section}{\bf\Large}{Κεφάλαιο \thesection}{1em}{}
\newtcolorbox{attnbox}[1]{colback=red!5!white,%
    colframe=red!75!black,fonttitle=\bfseries,title=#1}
\newtcbox{quickattnbox}[1]{colback=red!5!white,%
	colframe=red!75!black,fonttitle=\bfseries,title=#1}
\newtcolorbox{infobox}[1]{colback=blue!5!white,%
    colframe=blue!75!black,fonttitle=\bfseries,title=#1}

\tcbset{frogbox/.style={enhanced jigsaw,%
		overlay first={\foreach \x in {0cm} {
				\begin{scope}[shift={([xshift=-0.2cm]title.west)}]
					\draw[very thick,green!65!black!50!white,latex-] (0,0) -- ++(-1.5,0);
\end{scope}}}}}
\tcbset{frogtitle/.style={
attach boxed title to top left=
{xshift=0mm,yshift=-0.50mm},
boxed title style={skin=enhancedfirst jigsaw,
	bottom=0mm,
	interior style={fill=none,
		left color=green!20!black,
		right color=gray}}
}}
\DeclareTColorBox{exercise}{ O{} }{
	enhanced jigsaw,
	breakable,parbox=false,
	%title style={left color=gray!50!white!50!green,opacity=.5,right color=white},
	subtitle style={%boxrule=1pt,
		colback=yellow!50!red!25!white,fontupper=\bfseries},
	coltitle=black,colbacktitle=green!90!black!25!white,colframe=black,
	frame hidden,
	boxrule=0mm,
	%boxrule=1mm,
	leftrule=0.8pt,toprule=0.8pt,rightrule=0pt, %reserve space
	borderline west={0.8pt}{0pt}{white!25!black},%---- draw line
	borderline north={0.8pt}{0pt}{white!25!black},%---- draw line
	interior hidden,
	%frame style={left color=black,right color=white},
	sharp corners=all,
	%frogbox, %TODO: frogbox
	before lower={\tcbsubtitle[before skip=\baselineskip]{Λύση}},lower separated=false,
	before title={\textbf{Άσκηση\ifthenelse{\isempty{#1}}{}{: }}},
	title={\ifthenelse{\isempty{#1}}{\hspace{0pt}}{#1}}%
}

\AtBeginDocument{%
\let\arg\relax
\let\Re\relax
\let\Im\relax
\DeclareMathOperator{\arg}{Arg}
\DeclareMathOperator{\Re}{Re}
\DeclareMathOperator{\Im}{Im}
}
\DeclareMathOperator{\sinc}{sinc}
\DeclareMathOperator{\sgn}{sgn}
\DeclareMathOperator{\erf}{erf}
\DeclareMathOperator{\cov}{cov}
\DeclareMathOperator{\atand}{atan2}

\newenvironment{absolutelynopagebreak}
{\par\nobreak\vfil\penalty0\vfilneg
	\vtop\bgroup}
{\par\xdef\tpd{\the\prevdepth}\egroup
	\prevdepth=\tpd}

\DeclareSIUnit \voltampere { VA } %apparent power 
\DeclareSIUnit \var { VAr } %volt-ampere reactive - idle power 
\DeclareSIUnit \decade { dec } %decade

% Global amount of samples
% Set to a higher value (e.g. 200) for nicer graphs
% Set to a low value (e.g. 10) for performance
% NOTE: Check the sample variables below for further measurements
\newcommand*{\gsamples}{200}

% Equals command as a workaround for CircuiTikZ bug
% not allowing the = sign in labels
\newcommand*{\equals}{=}

\newcommand{\nesearrow}{%
	\,%
	\smash{\raisebox{-1.1ex}
		{$%
			\stackrel{\displaystyle\nearrow}{\displaystyle\searrow}%
			$}}%
}
\newcommand{\degree}{^{\circ}} % not great
\newcommand\numberthis{\addtocounter{equation}{1}\tag{\theequation}} % add an equation number to a number-less math environment

% Provided commands
\providecommand\dif{d}
\providecommand\od[2]{\frac{#1}{#2}}

\newtcbtheorem[number within=section,list inside=thm]{theorem}{Θεώρημα}%
{colback=green!5,colframe=green!35!black,colbacktitle=green!35!black,fonttitle=\bfseries,enhanced,attach boxed title to top left={yshift=-2mm,xshift=-7mm},width=.9\textwidth,arc=.7mm}{th}
\newtcbtheorem[number within=section,list inside=defn]{defn}{Ορισμός}%
{colback=blue!5,colframe=cyan!35!black,colbacktitle=blue!35!black,fonttitle=\bfseries,enhanced,attach boxed title to top left={yshift=-2mm,xshift=-2mm}}{def}

% Locus plot utilities
\tikzset{locus/.style={orange!50!red!70!brown}}
\tikzset{locuspole/.style={draw=red!30!black,cross,inner sep=2.5pt,fill=white,fill opacity=.6,thick,label={[below]-90:#1}}}
\tikzset{locuszero/.style={draw=red!30!black,circle,inner sep=2pt,fill=white,fill opacity=.6,thick,label={[below]-90:#1}}}
\tikzset{locusbreak/.style={rounded corners=1.5pt,inner sep=2pt,draw,top color=brown,bottom color=black,fill opacity=.8,label={[below]-90:#1}}}

% New plotting utilities
\def\lowsamples{18}
\def\hisamples{40}
\def\timecolour{blue!50!cyan}

\tikzstyle{timecolour}=[\timecolour]



\title{ΨΕΣ
	\\
	{ 
		\normalsize Ψηφιακή Επεξεργασία Σήματος
		\\
		\normalsize Σημειώσεις από τις παραδόσεις\footnote{Όπως διδάσκονται στο τμήμα \textit{Ηλεκτρολόγων Μηχανικών και Μηχανικών Υπολογιστών} στο \textit{Αριστοτέλειο Πανεπιστήμιο Θεσσαλονίκης}.}
	}}
\date{Οκτώβριος 2018
	\\
	{ 
		\small Τελευταία ενημέρωση: \today
	}
}
\author{
	Για τον κώδικα σε \LaTeX, ενημερώσεις και προτάσεις:
	\\
	\url{https://github.com/kongr45gpen/ece-notes}}

\setallmainfonts(Digits,Latin,Greek){Asana Math}
\setmainfont{Noto Serif}
\setsansfont{Ubuntu}
\usepackage{polyglossia}
\newfontfamily\greekfont[Script=Greek,Scale=1.00]{Liberation Serif}

\hypersetup{pdftitle = {ΣΑΕ 2}}

\let\mytodo\todo
\renewcommand{\todo}[1]{\par\mytodo[inline,noline]{#1}}


\begin{document}
\maketitle

\hrule
\vspace{50pt}

\begin{infobox}{Λάθη \& Διορθώσεις}
	Οι τελευταίες εκδόσεις των σημειώσεων βρίσκονται στο Github
	(\url{https://github.com/kongr45gpen/ece-notes/raw/master/dsp.pdf}) ή
	στη διεύθυνση \url{http://helit.org/ece-notes/dsp.pdf}.
	
	Περιέχουν διορθώσεις σε λάθη και τυχόν βελτιώσεις.
	
	\tcblower
	
	Μπορείτε να ενημερώνετε για οποιοδήποτε λάθος και πρόταση
	μέσω PM στο forum, issue στο Github, ή οποιουδήποτε άλλου τρόπου!
\end{infobox}

\section{Εισαγωγή}
	
Στο μάθημα της \textbf{Ψηφιακής Επεξεργασίας Σήματος} ασχολούμαστε
με προβλήματα όπως το εξής:

Παλιότερα, αν μας ζητούνταν να κατασκευάσουμε ένα φίλτρο (π.χ.
ζωνοπερατό από 20 Hz - 500 kHz), μπορούσαμε πολύ εύκολα να αγοράσουμε μια σακούλα με
όλα τα εξαρτήματα (πυκνωτές, αντιστάσεις, \textellipsis) από ένα
κατάστημα και να φτιάξουμε το κύκλωμα.

Αν αργότερα μας ζητούσαν να φτιάξουμε ένα φίλτρο 20 Hz - 300 kHz, θα
έπρεπε να ξαναπάμε στο κατάστημα και να αγοράσουμε ξανά νέα σακούλα και νέα εξαρτήματα
και να ξανακατασκευάσουμε το κύκλωμα.

Από τη στιγμή όμως που εφευρέθηκαν οι υπολογιστές, φανταστήκαμε να τους
χρησιμοποιήσουμε και για να πραγματοποιήσουμε την παραπάνω διαδικασία και
να αλλάζουμε προδιαγραφές όσο συχνά θέλουμε χωρίς να τρέχουμε στο κατάστημα.

Αυτή η διαδικασία απαιτεί τα εξής:
\begin{itemize}
	\item Πρέπει να μετατρέψουμε το \textit{πραγματικό} \textit{αναλογικό}
	σήμα σε μια μορφή που αναγνωρίζει αυτός ο υπολογιστής. Δηλαδή να μετατραπεί
	από αναλογικό σε ψηφιακό. Αυτό γίνεται με δύο βήματα:
	\begin{enumerate}
		\item \textbf{Δειγματοληψία}. Πρώτα πρέπει με έναν συγκεκριμένο τρόπο
		να λάβουμε \textit{δείγματα} σε \textit{διακριτές} στιγμές του αρχικού
		αναλογικού σήματος. Αφού ο υπολογιστής δεν μπορεί να αποθηκεύσει
		άπειρες τιμές, αναγκαστικά θα λάβουμε ένα πεπερασμένο εύρος τους.
		
		\begin{tikzpicture}
		\def\points{(0,0) (0.2,0.2) (0.4,0.5) (0.6,0.7)
			(0.8,0.9) (1,0.7) (1.2,0.6) (1.4,1) (1.6,0.8) (1.8,0.8) (2,1.1) (2.2,1.4)
			(2.4, 1.2) (2.6,1.1) (2.8,1) (3,0.95)
		}
		\def\pointsc{(0,0),(0.2,0.2),(0.4,0.5),(0.6,0.7),(0.8,0.9),(1,0.7),(1.2,0.6),(1.4,1),(1.6,0.8),(1.8,0.8),(2,1.1),(2.2,1.4),(2.4, 1.2),(2.6,1.1),(2.8,1),(3,0.95)
		}
		\def\pointsr{(0,0),(0.2,0.21),(0.4,0.42),(0.6,0.63),(0.8,0.84),(1,0.63),(1.2,0.63),(1.4,1.05),(1.6,0.84),(1.8,0.84),(2,1.05),(2.2,1.47),(2.4, 1.26),(2.6,1.05),(2.8,1.05),(3,0.84)
		}
		
		\draw[->] (0,0) -- (3,0) node[below] {$t$};
		\draw[->] (0,0) -- (0,2) node[left] (xt) {$x(t)$} node[right,align=left,scale=.7] {αναλογικό\\σήμα};
		\draw (xt.south) node[below,scale=.7,yshift=2mm] {$\in\mathbb R$};
		
		\draw[\timecolour,thick]
		plot [smooth] coordinates \points;
		
		\foreach \p in \pointsc
		\filldraw[red,top color=white,bottom color=blue,opacity=.4] \p circle(0.05);
		
		\draw[<->,thick,opacity=.9]
		(3.75,1) -- ++(1,0);
		
		\begin{scope}[xshift=5.5cm]
		\draw[->] (0,0) -- (3,0) node[below] {$n\in\mathbb N$};
		\draw[->] (0,0) -- (0,2) node[left] (xn) {$x(n)$} node[right,align=left,scale=.7] {διακριτό\\σήμα};
		
		\draw (xn.south) node[below,scale=.7,yshift=2mm] {$\in\mathbb R$};
		
		\foreach \p in \pointsc
		\filldraw[draw=black,top color=white,bottom color=blue,opacity=.4] \p circle(0.05);
		\end{scope}
		
		\end{tikzpicture}
		
		Αυτό το σήμα ονομάζεται \textbf{διακριτό} (\textbf{discrete}).
		\item \textbf{Κβάντωση}. Στον αναλογικό κόσμο, οι πεπερασμένες τιμές έχουν
		άπειρη ακρίβεια. Όμως στον υπολογιστή δεν μπορούμε να αποθηκεύσουμε άπειρα
		δεκαδικά ψηφία, αλλά πρέπει να τα αποθηκεύσουμε σε πεπερασμένες στάθμες
		που μας επιτρέπεται να χρησιμοποιηθούν.
		
		\begin{tikzpicture}
		\def\points{(0,0) (0.2,0.2) (0.4,0.5) (0.6,0.7)
			(0.8,0.9) (1,0.7) (1.2,0.6) (1.4,1) (1.6,0.8) (1.8,0.8) (2,1.1) (2.2,1.4)
			(2.4, 1.2) (2.6,1.1) (2.8,1) (3,0.95)
		}
		\def\pointsc{(0,0),(0.2,0.2),(0.4,0.5),(0.6,0.7),(0.8,0.9),(1,0.7),(1.2,0.6),(1.4,1),(1.6,0.8),(1.8,0.8),(2,1.1),(2.2,1.4),(2.4, 1.2),(2.6,1.1),(2.8,1),(3,0.95)
		}
		\def\pointsr{(0,0),(0.2,0.21),(0.4,0.42),(0.6,0.63),(0.8,0.84),(1,0.63),(1.2,0.63),(1.4,1.05),(1.6,0.84),(1.8,0.84),(2,1.05),(2.2,1.47),(2.4, 1.26),(2.6,1.05),(2.8,1.05),(3,0.84)
		}
		
		\draw[->] (0,0) -- (3,0) node[below] {$t$};
		\draw[->] (0,0) -- (0,2) node[left] (xt) {$x(t)$} node[right,align=left,scale=.7] {αναλογικό\\σήμα};
		\draw (xt.south) node[below,scale=.7,yshift=2mm] {$\in\mathbb R$};
		
		\draw[blue!50!cyan!80!brown,thick]
		plot [smooth] coordinates \points;
		
		\foreach \p in \pointsc
		\filldraw[red,top color=white,bottom color=blue,opacity=.4] \p circle(0.05);
		
		\draw[<->,thick,opacity=.9]
		(3.75,1) -- ++(1,0);
		
		\begin{scope}[xshift=5.5cm]
		\draw[->] (0,0) -- (3,0) node[below] {$n\in\mathbb N$};
		\draw[->] (0,0) -- (0,2) node[left] (xn) {$x(n)$} node[right,align=left,scale=.7] {διακριτό\\σήμα};
		
		\draw (xn.south) node[below,scale=.7,yshift=2mm] {$\in\mathbb R$};
		
		\foreach \y in {0,0.21,...,1.6}
		\draw[blue,opacity=.5,densely dashed] (0,\y) -- ++(3,0);
		
		\foreach \p in \pointsc
		\filldraw[draw=black,top color=white,bottom color=blue,opacity=.4] \p circle(0.05);
		\end{scope}
		
		\draw[->,thick,opacity=.9]
		(9.5,1) -- ++(1,0);
		
		\begin{scope}[xshift=11.25cm]
		\draw[->] (0,0) -- (3,0) node[below] {$n\in\mathbb N$};
		\draw[->] (0,0) -- (0,2) node[left] (xn) {$x(n)$} node[right,align=left,scale=.7] {ψηφιακό\\σήμα};
		
		\draw (xn.south) node[below,scale=.9,yshift=1mm] {$\in\mathbb Z$};
		
		\foreach \y in {0,0.21,...,1.6}
		\draw[blue,opacity=.5,densely dashed] (0,\y) -- ++(3,0);
		
		\foreach \p in \pointsr
		\filldraw[draw=black,top color=white!50!red,bottom color=blue,opacity=.4] \p circle(0.05);
		\end{scope}
		
		\end{tikzpicture}
		
		Αυτό το σήμα ονομάζεται \textbf{ψηφιακό} (\textbf{digital}).
	\end{enumerate}

	Σε αυτό το μάθημα, παρά τον τίτλο του, θα ασχοληθούμε με \textbf{διακριτά σήματα}, και
	όχι ψηφιακά.
	
	Για να επεξεργαστούμε τα σήματα, υποθέτουμε ότι θα υπάρχει μια συσκευή που
	μετατρέπει το \textit{Αναλογικό σήμα} σε \textit{Διακριτό σήμα} με τέτοιον τρόπο
	ώστε να μπορούμε να γυρίσουμε πίσω και να μπορούμε να το επεξεργαστούμε με έναν
	ασφαλή τρόπο που θα επιστρέψει σίγουρα σωστό αποτέλεσμα. Αυτό το εξασφαλίζει το
	\textbf{θεώρημα δειγματοληψίας} (Nyquist-Shannon), σύμφωνα με το οποίο για να
	αναπαραστήσουμε ένα ζωνοπερατό σήμα, αρκεί να το δειγματοληπτήσουμε με συχνότητα διπλάσια
	της μέγιστης συχνότητας του σήματος.
	
	Στην πραγματικότητα βέβαια, το θεώρημα δειγματοληψίας απαιτεί να δειγματοληπτούμε για άπειρο
	χρόνο, κάτι μη πραγματικά εφικτό. Μάλιστα, τα πραγματικά σήματα είναι \textit{χρονοπερατά}, άρα
	\textit{μη ζωνοπερατά}, επομένως εν γένει δεν εφαρμόζεται το θεώρημα δειγματοληψίας. Όμως η παραπάνω
	διαδικασία μπορούμε να πούμε ότι δίνει προσεγγιστικά ορθό αποτέλεσμα.
	
	Ακόμα, υπάρχουν σήματα που είναι από τη φύση τους ψηφιακά, όπως τιμές χρηματιστηρίου, ακολουθίες,
	δεδομένα, followers στο instagram, \textellipsis
	
	\item Παραμένει η ίδια η \textbf{επεξεργασία} του σήματος. Αυτή γίνεται με κώδικα ενός αλγορίθμου
	που εκτελεί πράξεις, δεδομένου ότι έχει επεξεργαστεί το σήμα σωστά.
\end{itemize}

Σχηματικά:

\begin{tikzpicture}

\pgfdeclarelayer{foreground}
\pgfsetlayers{main,foreground}

\draw[fill=yellow,draw opacity=.7,fill opacity=.15] (2.2,1.6) rectangle (5.2,-0.7) node[below,opacity=.7,circle,draw,inner sep=1pt,outer sep=2pt] {$A$};
\draw[fill=yellow,draw opacity=.7,fill opacity=.15] (-0.4,-2.3) rectangle (8.39,-3.7) node[below,opacity=.7,circle,draw,inner sep=1pt,outer sep=2pt] {$B$};

\draw (-2,0.5) node[scale=.9,align=center] {αναλογικός\\κόσμος};

\draw (0,0) --(1,0);
\draw (0,0) --(0,1);
\draw[\timecolour] plot[variable=\x,domain=0:1,samples=10,smooth]
(\x,{\x*((1+rand*0.3))});

\ctikzset{bipoles/length=.6cm}

\begin{pgfonlayer}{foreground}
\begin{scope}[xshift=3cm,yshift=1cm,local bounding box=scope1]
\draw (-0.1,0) node {};
\draw (0,0) to[L] ++(0.5,0);
\draw (0,-0.3) to[C] ++(0.5,0);
\draw (0,-0.7) to[R] ++(0.5,0);
\draw (0,-1.1) to[D] ++(0.5,0);

\draw (1.2,0) node {$H(\omega)$};
\draw (1.2,-0.4) node {$H(s)$};
\draw (1.2,-0.8) node {$h(t)$};
\draw (1.2,-1.2) node {???};
\end{scope}
\end{pgfonlayer}
\draw[fill=white] (scope1.north west) rectangle (scope1.south east);

\draw[<-] (scope1.west) -- ++(-1,0) node[left] {$x(t)$};
\draw[->] (scope1.east) -- ++(1,0) node[right] {$y(t)$};

\begin{scope}[xshift=6.5cm]
\draw (0,0) --(1.1,0);
\draw (0,0) --(0,1);
\draw[\timecolour] plot[variable=\x,domain=0:1,samples=10,smooth]
(\x,{(1-\x)*((0.9+rand*0.3))});
\end{scope}

\begin{scope}[yshift=-3cm]
\draw (-2,0) node[scale=.9,align=center] {ψηφιακός\\κόσμος};

\draw (1,0) node[rectangle,draw] (b1) {A$\to$D converter};
\draw (4,0) node[rectangle,scale=.8,draw,align=center] (b2)
{Algorithm\\Code\\ALU+Memory};
\draw (7,0) node[rectangle,draw] (b3) {D$\to$A converter};

\draw[<-] (b1.west) -- ++(-0.2,0) node[left,scale=.8] {$x(t)$};
\draw[->] (b1) -- (b2);
\draw[->] (b2) -- (b3);
\draw[->] (b3.east) -- ++(0.15,0) node[right,scale=.8] {$y(t)$};
\end{scope}
\end{tikzpicture}

Θέλουμε η διαδικασία \( A \) που είναι ένα \textbf{πραγματικό, φυσικό} φίλτρο ή κύκλωμα, να βγάζει
περίπου ίδιο αποτέλεσμα με μια διαδικασία \( B \) που υλοποιούμε ψηφιακά, δηλαδή:
\[
A \simeq B
\]
ή, ισοδύναμα για τα σήματα \( x(t) \) και \( y(t) \):
\begin{align*}
	A\left[x(t)\right] &\simeq B\left[x(t)\right] \\
	y_{\mathrm{analog}}(t) &\simeq y_{\mathrm{digital}}(t)
\end{align*}

\subsection{Ιδιότητες}
Όταν μιλάμε για διακριτό σήμα, μιλάμε ουσιαστικά για μία ακολουθία, όπως τη γνωρίζουμε από τα μαθηματικά.

\paragraph{Πράξεις}
Οι πράξεις σημάτων ορίζονται όπως και στις ακολουθίες. Ασχολούμαστε με τιμές ίδιων \textbf{δεικτών} (indices) \( n \):
\begin{enumerate}
	\item \textbf{Πρόσθεση}: \( x(n) + y(n) = z(n) \)
	\item \textbf{Πολλαπλασιασμός ακολουθιών}: \( x(n) \cdot y(n) = z(n) \)
	\item \textbf{Πολλαπλασιασμός αριθμού-ακολουθίας}: \( a\in\mathbb R,\ z(n) = a\cdot x(n) \)
\end{enumerate}

\paragraph{Άθροισμα γεωμετρικής προόδου}
Η γεωμετρική πρόοδος έχει σημαντική θέση στην ψηφιακή επεξεργασία σήματος (αφού στο αναλογικό σήμα
είχαμε το σημαντικό \( e^{j\omega t} \), εδώ είναι σημαντικό το \( e^{j\omega n} \) που εκφράζει συνεχή πολλαπλασιασμό με το \( e^{j\omega } \)):
\[
\sum_{n=0}^{\infty} a^n
= \begin{cases}
\displaystyle \frac{1}{1-a} &,\quad |a|<1\\
\text{αποκλίνει}&,\quad |a|>1
\end{cases}
\]
(για \( a=1 \) αποκλίνει, και για \( a=-1 \) ταλαντεύεται. Επίσης, η παραπάνω σχέση ισχύει και για
\( a\in\mathbb C \)).

\textbf{Προσοχή} ότι η παραπάνω σχέση ισχύει μόνο όταν το \( n \) ξεκινάει από το 0. Για παράδειγμα,
\( \sum_{n=1}^{\left(\frac{1}{2}\right)^n} = \frac{1}{1-\frac{1}{2}}-1 = 1 \).

Για \textbf{πεπερασμένο αριθμό όρων}:
\[
\sum_{n=0}^{N} a^n
= \frac{1-a^{n+1}}{1-a}
\]

\subsubsection{Χρήσιμοι τύποι ακολουθιών}
\paragraph{Περιοδική ακολουθία}
Η περιοδική ακολουθία περιέχει όρους που επαναλαμβάνονται, όπως και μία περιοδική συνάρτηση.
Μαθηματικά:
\[
\exists N \in \mathbb Z: \ \forall n \in \mathbb Z: \ x(n) = x(n+N)
\]
δηλαδή η περιοδική μας ακολουθία έχει μια \textbf{περίοδο \( \mathbb N \)} που είναι ακέραιος αριθμός.

\textbf{Προσοχή!} Δεδομένου ότι η \( \cos(\omega t) \) είναι \textit{περιοδική}, θα μπορούσε κάποιος
να φαντασεί ότι και η \( \cos(\omega n) \) είναι \textit{περιοδικό διακριτό σήμα}. Αν το δούμε μαθηματικά:
Έστω \( \exists N \in \mathbb Z:\ \cos(\omega n)=\cos\left( \omega (n+N) \right)
\implies \omega n = \omega n + \omega N + kπ \implies N = \frac{π}{\omega } \notin \mathbb Z \).

\begin{tikzpicture}[scale=.8]
\def\freq{0.5}

\draw (0,0) node[left,align=right,scale=.8] {περιοδικό\\ψηφιακό\\σήμα};

\draw (0,0) -- (10,0);
\draw (0,-1.5) -- (0,1.5);

\begin{scope}[green!50!black,opacity=.6]
\draw (0.5,1) -- (0.5,-1.5);
\draw (1,0) -- (1,-1.5);
\draw[<->] ([xshift=-1mm]1,-1.3) -- ([xshift=1mm]0.5,-1.3) node[below,yshift=-3mm,scale=.6,midway] {$f_s=2\;\mathrm{Hz}$};
\end{scope}

\draw[thick,timecolour]
plot[domain=0:9,samples=\lowsamples,smooth] (\x,{sin(\freq*2*pi*\x r)});

\foreach \x in {0.5,1,...,9} {
	\def\sample{sin(\freq*2*pi*\x r)}
	\draw[dashed,orange] (\x,0) -- (\x,{\sample});
	\filldraw[fill opacity=.8,fill=orange!50!white] (\x,{\sample}) circle (2.5pt);
}

\begin{scope}[yshift=-4cm]
\draw (0,0) node[left,align=right,scale=.8] {\emph{μη} περιοδικό\\ψηφιακό\\σήμα};

\draw (0,0) -- (10,0);
\draw (0,-1.5) -- (0,1.5);

\begin{scope}[green!50!black,opacity=.6]
\draw (0.55,{sin(\freq*2*pi*0.55 r)}) -- (0.55,-1.5);
\draw (1.1,{sin(\freq*2*pi*1.1 r)}) -- (1.1,-1.5);
\draw[<->] ([xshift=-1mm]1.1,-1.3) -- ([xshift=1mm]0.55,-1.3) node[below,yshift=-3mm,scale=.6,midway] {$f_s=1.28\sqrt{2}\;\mathrm{Hz}$};
\end{scope}

\draw[thick,timecolour]
plot[domain=0:9,samples=\lowsamples,smooth] (\x,{sin(\freq*2*pi*\x r)});

\foreach \x in {0.55,1.1,...,9} {
	\def\sample{sin(\freq*2*pi*\x r)}
	\draw[dashed,red] (\x,0) -- (\x,{\sample});
	\filldraw[fill opacity=.8,fill=red!50!white] (\x,{\sample}) circle (2.5pt);
}
\end{scope}
\end{tikzpicture}

Πρακτικά, δειγματοληπτούμε σε διαφορετικά σημεία, άσχετα ίσως από την περίοδο του σήματος. Γενικότερα,
όταν δειγματοληπτούμε περιοδικά αναλογικά σήματα, δεν θα παίρνουμε πάντα περιοδικά διακριτά πίσω.

Μάλιστα, κάτι άλλο περίεργο όταν δειγματοληπτούμε είναι πως για διαφορετικά αναλογικά σήματα, μπορεί
να πάρουμε το ίδιο ψηφιακό!

\begin{tikzpicture}
\draw (0,1.5) -- (0,-1.5);
\draw[->] (0,0) -- (5,0) node[below] {$t$};

\draw[\timecolour!40!black,very thick]
plot[samples=\lowsamples,smooth,domain=0:4,variable=\x]
(\x,{1.1*sin((\x*1.25*pi) r)});
\draw[\timecolour,very thick]
(0,0) -- (1/4*1.6,1) -- (3/4*1.6,-1) -- (5/4*1.6,1) -- (7/4*1.6,-1)
-- (9/4*1.6,1);
\end{tikzpicture}

Στο παραπάνω σχήμα, δειγματοληπτώντας στις κορυφές και στα μηδενικά, θα πάρουμε το ίδιο πράγμα από
τα δύο σήματα.

\paragraph{Άρτιες \& Περιττές ακολουθίες}
\begin{align*}
	\text{άρτια (even)} &\quad \forall n \in \mathbb Z \ x_e(n) = x_e(-n) \\
	\text{περιττή (odd)} &\quad \forall n \in \mathbb Z \ -x_o(n) = x_o(-n)
\end{align*}

Μπορούμε να μετατρέψουμε οποιαδήποτε ακολουθία σε ένα άρτιο και ένα περιττό μέρος:
\begin{align*}
	x_e(n) &= \frac{x(n) + x(-n)}{2}\\
	x_o(n) &= \frac{x(n) - x(-n)}{2}
\end{align*}

\subsubsection{Χαρακτηριστικά Μεγέθη}
\begin{enumerate}
	\item \textbf{Μέση τιμή}:
	\( \displaystyle
	\overline{x(n)} = \frac{\displaystyle \sum_{n=0}^{N} x(n)}{N+1}
	 \)
	\item \textbf{Ενεργός τιμή}:
	\( 
	\displaystyle
	\overline{\overline{x(n)}} = \left[
	\frac{\displaystyle \sum_{n=0}^{N} x^2(n)}{N+1}
	\right]^{\sfrac{1}{2} }
	 \)
	\item \textbf{Στιγμιαία ισχύς}:
	\( 
	\displaystyle
	p(n) = x^2(n)
	 \)
	\item \textbf{Μέση Ισχύς}:
	\( 
	\displaystyle
	p = \overline{p(n)} = \frac{\displaystyle \sum_{n=0}^{N} x^2(n)}{N+1}
	 \)
	\item \textbf{Ενέργεια}:	\todo{Check}
	\( 
	\displaystyle
	W = \sum_{n=0}^{N} x^2(n) = (N+1)p
	 \)
\end{enumerate}

\pagebreak[3]

\subsubsection{Χρήσιμες ακολουθίες}
\begin{enumpar}
	\item \textbf{Εκθετική ακολουθία}: \todo{Box this in a cool way}
	\begin{minipage}{.5\textwidth}
		\begin{knowledgebox}{Εκθετική ακολουθία}
		\[ \displaystyle
		x(n) = Ae^{sn} = Aa^{(\sigma + j \omega )n}
		\]
		\end{knowledgebox}

		για την οποία λαμβάνουμε τις εξής περιπτώσεις για τις σταθερές:
		\begin{itemize}
			\item \( a=e \) και \( s = \sigma < 0 \):
			\[
			x(n) = Ae^{-|σ|n}
			\]
			(γεωμετρική πρόοδος με λόγο \( e^{-|σ|} \))
			\item \( a=e \) και \( s = \pm j\omega  \):
			\[
			x(n) = A\left[
			cos(\omega n) \pm j \sin(\omega n)
			\right]
			\]
			
			Είναι περιοδική \textit{μόνο} εάν \( \frac{\pi}{\omega } \in \mathbb Q \)
		\end{itemize}
	\end{minipage}
	\begin{minipage}{.5\textwidth}
		\begin{tikzpicture}
		\tikzstyle{sample}=[bottom color=orange,top color=\timecolour,fill opacity=.7,rounded corners=2.5pt,thick]
		
		\draw[->] (-3,0) -- (3.5,0);
		\draw[->] (0,0) -- (0,2)  node[right] {$x(n)$};
		
		\def\sf{0.7}
		\def\sx{0.11}
		\def\sy{0.09}
		
		\foreach \x in {-4,-3,...,4} {
			\draw (\x*\sf-0.1,0.1) -- (\x*\sf+0.1,-0.1);
			\draw (\x*\sf,-0.1) node[below,scale=.9] {$\x$};
		}
		
		\foreach \x in {-4,-3,...,4} {
			\filldraw[sample] (\x*\sf + \sx,{exp(\x/3)/2 + \sy}) rectangle ++(-\sx*2,-\sy*2);
		}
		\end{tikzpicture}
	\end{minipage}
    \item \textbf{Δέλτα του Kronecker}:

    \begin{minipage}{.5\textwidth}
    	\begin{knowledgebox}{Δέλτα του Kronecker}
    		\[ \displaystyle
    		\delta(n) = \begin{cases}
    		1 &\quad n=0\\
    		0 &\quad n\neq 0
    		\end{cases}
    		\]
    	\end{knowledgebox}
    \end{minipage}
	\begin{minipage}{.5\textwidth}
		\begin{tikzpicture}
		\tikzstyle{sample}=[bottom color=orange,top color=\timecolour,fill opacity=.7,rounded corners=2.5pt,thick]
		
		\draw[->] (-3,0) -- (3.5,0);
		\draw[->] (0,0) -- (0,2)  node[right] {$\delta(n)$};
		
		\def\sf{0.7}
		\def\sx{0.11}
		\def\sy{0.09}
		
		\foreach \x in {-4,-3,...,4} {
			\draw (\x*\sf-0.1,0.1) -- (\x*\sf+0.1,-0.1);
			\draw (\x*\sf,-0.1) node[below,scale=.9] {$\x$};
		}
		
		\foreach \x in {1,2,...,4} {
			\filldraw[sample] (\x*\sf + \sx,0 + \sy) rectangle ++(-\sx*2,-\sy*2);
			\filldraw[sample] (-\x*\sf + \sx,0 + \sy) rectangle ++(-\sx*2,-\sy*2);
		}
		\filldraw[sample] ( + \sx,1 + \sy) rectangle ++(-\sx*2,-\sy*2);
		\draw (0+0.15,1) node[right] {$1$};
		
		\end{tikzpicture}
	\end{minipage}
    \item \textbf{Βηματική ακολουθία step}:
    
    \begin{minipage}{.5\textwidth}
    \begin{knowledgebox}{Βηματική ακολουθία}
    	\[
    	u(n) = \begin{cases}
    	1 &\quad n \geq 0\\
    	0 &\quad n < 0
    	\end{cases}
    	\]
    \end{knowledgebox}
    
	\end{minipage}
	\begin{minipage}{.5\textwidth}
		\begin{tikzpicture}
		\tikzstyle{sample}=[bottom color=orange,top color=\timecolour,fill opacity=.7,rounded corners=2.5pt,thick]
		
		\draw[->] (-3,0) -- (3.5,0);
		\draw[->] (0,0) -- (0,2)  node[right] {$\mathrm u(n)$};
		
		\def\sf{0.7}
		\def\sx{0.11}
		\def\sy{0.09}
		
		\foreach \x in {-4,-3,...,4} {
			\draw (\x*\sf-0.1,0.1) -- (\x*\sf+0.1,-0.1);
			\draw (\x*\sf,-0.1) node[below,scale=.9] {$\x$};
		}
		
		\foreach \x in {1,2,...,4} {
			\filldraw[sample] (\x*\sf + \sx,1 + \sy) rectangle ++(-\sx*2,-\sy*2);
			\filldraw[sample] (-\x*\sf + \sx,0 + \sy) rectangle ++(-\sx*2,-\sy*2);
		}
		\filldraw[sample] ( + \sx,1 + \sy) rectangle ++(-\sx*2,-\sy*2);
		\draw (0-\sx,1) node[left] {$1$};
		
		\end{tikzpicture}
	\end{minipage}
    
    Μάλιστα, ισχύει ότι:
    \[
    u(n) = \sum_{m=-\infty}^{n} δ(m)
    \]
    κάτι που μας θυμίζει αντίστοιχα από το αναλογικό σήμα ότι \( u(t) = \int_{-\infty}^{t} δ(τ)\dif τ \)
\end{enumpar}

\subsubsection{Συνέλιξη}
Μπορούμε να μετατρέψουμε τη \textbf{συνέλιξη} του αναλογικού σήματος στο ψηφιακό.

Στο αναλογικό, θυμόμαστε ότι, σύμφωνα με τον ορισμό της συνέλιξης:
\[
x(t) = x(t) * δ(t) = \int_{-\infty}^{\infty} x(τ)δ(t-τ) \dif τ
\]
και στο ψηφιακό, μπορούμε να έχουμε κάτι αντίστοιχο: \todo{Add a cool box}
\[
x(n) = \sum_{-\infty}^{\infty} x(m) δ(n-m)
\]

\subsubsection{Σύστημα}
Στον αναλογικό κόσμο, ένα σύστημα ήταν ένα κουτί που έπαιρνε σήματα εισόδου, τα επεξεργαζόταν, και έβγαζε σήματα εξόδου. Μαθηματικά, είναι μια απεικόνιση συναρτήσεων \( x(t) \) εισόδου σε συναρτήσεις εξόδους.

Επομένως, μπορούμε να ορίσουμε το \textbf{ψηφιακό σύστημα} ως ένα σύστημα που απεικονίζει ακολουθίες
σε ακολουθίες.

 \begin{tikzpicture}[scale=.8]
\draw (0,0) node[rectangle,inner sep=18pt,draw] (s) {Σύστημα};
\draw[->] (s.east) -- ++(1,0) node[right] {$y(n)$};
\draw[<-] (s.west) -- ++(-1,0) node[left] {$x(n)$};

\draw[->,gray!50!brown!50!black] (s) ++(0.2,-2) node[below] {Νόμος $T$} to[bend left] (s);
\end{tikzpicture}

Αντίστοιχα, ένα \textbf{υβριδικό σύστημα} απεικονίζει συναρτήσεις σε ακολουθίες. Δηλαδή έχει είσοδο αναλογικό σήμα, και έξοδο ψηφιακό.

\paragraph{Γραμμικό Σύστημα}
Μπορούμε σε αυτό το σημείο να δώσουμε τον ορισμό του \textbf{γραμμικού συστήματος} που συναντάμε συνέχεια,
για ένα ψηφιακό σύστημα \( T \). Έστω οι έξοδοι \( y_1(n) = T\left[x_1(n)\right] \),
\( y_2(n) = T\left[x_2(n)\right] \). Το σύστημα είναι γραμμικό ανν:
\begin{gather*}
\forall x_1,x_2\quad,\quad \forall a_1,a_2\in\mathbb C:\\
a_1T\left[x_1(n)\right]+a_2T\left[x_2(n)\right] = T\left[a_1x_1(n)+a_2x_2(n)\right]
\end{gather*}

Παραδείγματα:
\begin{itemize}
	\item Το \( y(n) = ax(n) + b \) \textit{δεν} είναι γραμμικό, λόγω του \( b \).
	\item Το \( y(n) = nx(n) \) \textit{είναι} γραμμικό.
\end{itemize}

\paragraph{Αμετάβλητο Κατά τη Μετατόπιση Σύστημα (ΑΚΜ)}
\begin{align*}
	y(n) &= T\left[x(n)\right] \\
	y(n-k) &= T\left[x(n-k)\right]
\end{align*}
δηλαδή, αν το ενοχλήσουμε τη στιγμή 2 ή τη στιγμή 50, θα δώσει την ίδια έξοδο, ξεκινώντας αντίστοιχα
από τη στιγμή 2 ή τη στιγμή 50.

Παραδείγματα:
\begin{itemize}
	\item Το \( y(n) = ax(n) + b \) \textit{είναι} είναι αμετάβλητο κατά τη μετατόπιση.
	\item Το \( y(n) = nx(n) \) \textit{δεν} είναι αμετάβλητο κατά τη μετατόπιση, λόγω του όρου \( n \).
\end{itemize}

Το σύστημα που προκύπτει από μία διαφορική εξίσωση είναι αμετάβλητο κατά τη μετατόπιση όταν οι συντελεστές των παραγώγων του δεν εξαρτώνται από το χρόνο.

\paragraph{Λίγη προσοχή}
Χρειάζεται κάποια προσοχή στο χειρισμό ΑΚΜ και γραμμικών συστημάτων.

Έστω τα συστήματα που εκφράζουν το νόμο του \emph{Ohm} (\( V=IR \)) σε μια \emph{αντίσταση}:
\begin{align*}
	y_1(t) &= R(t) \cdot x(t)\\
	y_2(t) &= R(x) \cdot x(t)
\end{align*}

Στο πρώτο σύστημα η αντίσταση εξαρτάται από το χρόνο (π.χ. διάβρωση), και στο δεύτερο εξαρτάται από την είσοδο (π.χ. αύξηση θερμοκρασίας \( \implies \) αλλαγή αντίστασης για μεγαλύτερα ρεύματα).

Το πρώτο σύστημα είναι γραμμικό αλλά \emph{όχι} ΑΚΜ, αφού η \( R(t) \) εξαρτάται από το χρόνο.
Το δεύτερο σύστημα είναι ΑΚΜ αλλά \emph{όχι} γραμμικό, αφού η \( R(x) \) εξαρτάται από την είσοδο.

Είναι \textbf{λάθος} να πούμε πως έστω \( R(t)=x(t) \implies y_1(t)=x^2(t)\ \text{(μη γραμμικό)} \), καθώς
η \( R(t) \) είναι μια παράμετρος του συστήματος που δεν μπορεί να είναι ίση με τις διαφορετικές πιθανές
εισόδους του. Παρομοίως, είναι \emph{λάθος} να θεωρήσουμε ότι \( R(x) = R(x(t)) = R(t) \implies y_2(t) = R(t)x(t) \ \text{(μη ΑΚΜ)} \) (δηλαδή ότι αφού η \( R \) εξαρτάται από το \( x \) και το \( x \) εξαρτάται από το χρόνο, άρα η \( R \) εξαρτάται μόνο από το χρόνο).



\end{document}
