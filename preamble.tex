\usepackage{amsmath}
\usepackage{amsfonts}
\usepackage{amssymb}
\usepackage{libs/commath2}
\usepackage[table]{xcolor}
\usepackage[hidelinks,draft=false]{hyperref}
\usepackage[skins,theorems]{tcolorbox}
\usepackage{titlesec}
\usepackage{tikz}
\usepackage{libs/circuitikz} % use our own recent version to make sure some bugs are fixed
\usepackage{pgfplots}
\usepackage{mathtools}
\usepackage[makeroom]{cancel}
\usepackage{mathrsfs}
\usepackage{wrapfig}
%\usepackage{subcaption}
%\usepackage{floatrow}
\usepackage{esint}
\usepackage{paralist}
\usepackage{enumitem}
%\usepackage{bm}
\usepackage{relsize}
\usepackage{xfrac}
\usepackage{comment}
\usepackage{siunitx}
\usepackage{multicol}
%\usepackage{MnSymbol}
\usepackage[obeyDraft,textsize=tiny]{todonotes}
%\usepackage{morefloats} % oh no!
%\usepackage[linesnumbered,lined]{algorithm2e}
\usepackage{glossaries}
\usepackage{xifthen}
\usepackage{tocloft}


\pgfplotsset{compat=1.13}
\usetikzlibrary{arrows.meta}
\usetikzlibrary{patterns}
\usetikzlibrary{decorations.pathmorphing}
\usetikzlibrary{decorations.markings}
\usetikzlibrary{backgrounds}
\usetikzlibrary{shapes.misc}
\usetikzlibrary{shapes.multipart}
\usetikzlibrary{shadows.blur}
\usetikzlibrary{fadings}
\usetikzlibrary{intersections}
\usetikzlibrary{arrows.meta}
\usetikzlibrary{calc}
\usetikzlibrary{matrix}
\usetikzlibrary{positioning}
\usetikzlibrary{shapes}
\usetikzlibrary{shadings}

\tcbuselibrary{breakable}
\tcbuselibrary{skins}
\tcbuselibrary{xparse}

\tikzset{cross/.style={cross out, draw,
        minimum size=2*(#1-\pgflinewidth),
        inner sep=0pt, outer sep=0pt}}
\tikzset{
    mark position/.style args={#1(#2)}{
        postaction={
            decorate,
            decoration={
            	post length=1mm, % ??? Magic to fix "Dimension
            	pre length=1mm, % ???  too large" errors.
                markings,
                mark=at position #1 with \coordinate (#2);
            }
        }
    }
}
\tikzset{
	arrow at/.style args={#1}{
		postaction={
			decorate,
			decoration={
				post length=1mm, % ??? Magic to fix "Dimension
				pre length=1mm, % ???  too large" errors.
				markings,
				mark=at position #1 with {\arrow{>}};
			}
		}
	}
}
\makeatletter
\tikzset{
  use path for main/.code={%
    \tikz@addmode{%
      \expandafter\pgfsyssoftpath@setcurrentpath\csname tikz@intersect@path@name@#1\endcsname
    }%
  },
  use path for actions/.code={%
    \expandafter\def\expandafter\tikz@preactions\expandafter{\tikz@preactions\expandafter\let\expandafter\tikz@actions@path\csname tikz@intersect@path@name@#1\endcsname}%
  },
  use path/.style={%
    use path for main=#1,
    use path for actions=#1,
  }
}
\makeatother

\pgfmathdeclarefunction{sinc}{1}{%
	\pgfmathparse{abs(#1)<0.01 ? int(1) : int(0)}%
	\ifnum\pgfmathresult>0 \pgfmathparse{1}\else\pgfmathparse{sin(#1 r)/#1}\fi%
}
\pgfmathdeclarefunction{gauss}{2}{%
	\pgfmathparse{1/(#2*sqrt(2*pi))*exp(-((x-#1)^2)/(2*#2^2))}%
}

\usepackage[left=2cm,right=2cm,top=2cm,bottom=2cm]{geometry}

%\usepackage[no-math]{fontspec}
%\usepackage{fontspec}
\usepackage{mathspec}
%\usepackage{newtxtext,newtxmath}
%\usepackage{unicode-math}
%\setmainfont{texgyretermes-regular.otf}
%\setsansfont{texgyreheros-regular.otf}
%\newfontfamily\greekfont[Script=Greek]{Linux Libertine O}
%\newfontfamily\greekfontsf[Script=Greek]{Linux Libertine O}
\usepackage{polyglossia}
%\newfontfamily\greekfont[Script=Greek]{texgyretermes-regular.otf}
\newfontfamily\greekfontsf[Script=Greek]{texgyreheros-regular.otf}
\newfontfamily\greekfonttt[Script=Greek]{Latin Modern Mono}
%\usepackage[greek]{babel}
\setdefaultlanguage{greek}
\setotherlanguage{english}

%\usepackage[utf8]{inputenc}
%\usepackage[greek]{babel}


%\usepackage{tkz-euclide} % loads  TikZ and tkz-base
%\usetkzobj{angles} % important you want to use angles

\newlist{enumparen}{enumerate}{1}
\setlist[enumparen]{label=(\arabic*)}
\newlist{enumpar}{enumerate}{1}
\setlist[enumpar]{label=\arabic*)}

\newlist{enumgreek}{enumerate}{1}
\setlist[enumgreek]{label=\alph*.}
\newlist{enumgreekparen}{enumerate}{1}
\setlist[enumgreekparen]{label=(\alph*)}
\newlist{enumgreekpar}{enumerate}{1}
\setlist[enumgreekpar]{label=\alph*)}


\newlist{enumroman}{enumerate}{1}
\setlist[enumroman]{label=(\roman*)}

\newlist{enumlatin}{enumerate}{1}
\setlist[enumlatin]{label=(\alph*)}

\newlist{invitemize}{itemize}{1}
\setlist[invitemize]{noitemsep,label=}

\input{libs/fiximplies}
\input{libs/sphere}

\makeatletter
\let\anw@true\anw@false

%\newcommand{\attnboxed}[1]{\textcolor{red}{\fbox{\normalcolor\m@th$\displaystyle#1$}}}
\makeatother
\tcbset{highlight math style={enhanced,colframe=red,colback=white,%
        arc=0pt,boxrule=1pt,shrink tight,boxsep=1.5mm,extrude by=0.5mm}}
\newcommand{\attnboxed}[1]{\tcbhighmath[colback=red!5!white,drop fuzzy shadow,arc=0mm]{#1}}
\newcommand{\infoboxed}[1]{%
	\tcbhighmath[colframe=blue!50!white,colback=blue!5!white,arc=0mm]{#1}}
\titleformat{\section}{\bf\Large}{Κεφάλαιο \thesection}{1em}{}
\newtcolorbox{attnbox}[1]{colback=red!5!white,%
    colframe=red!75!black,fonttitle=\bfseries,title=#1}
\newtcbox{quickattnbox}[1]{colback=red!5!white,%
	colframe=red!75!black,fonttitle=\bfseries,title=#1}
\newtcolorbox{infobox}[1]{colback=blue!5!white,%
    colframe=blue!75!black,fonttitle=\bfseries,title=#1}

\tcbset{frogbox/.style={enhanced jigsaw,%
		overlay first={\foreach \x in {0cm} {
				\begin{scope}[shift={([xshift=-0.2cm]title.west)}]
					\draw[very thick,green!65!black!50!white,latex-] (0,0) -- ++(-1.5,0);
\end{scope}}}}}
\tcbset{frogtitle/.style={
attach boxed title to top left=
{xshift=0mm,yshift=-0.50mm},
boxed title style={skin=enhancedfirst jigsaw,
	bottom=0mm,
	interior style={fill=none,
		left color=green!20!black,
		right color=gray}}
}}
\DeclareTColorBox{exercise}{ O{} }{
	enhanced jigsaw,
	breakable,parbox=false,
	%title style={left color=gray!50!white!50!green,opacity=.5,right color=white},
	subtitle style={%boxrule=1pt,
		colback=yellow!50!red!25!white,fontupper=\bfseries},
	coltitle=black,colbacktitle=green!90!black!25!white,colframe=black,
	frame hidden,
	boxrule=0mm,
	%boxrule=1mm,
	leftrule=0.8pt,toprule=0.8pt,rightrule=0pt, %reserve space
	borderline west={0.8pt}{0pt}{white!25!black},%---- draw line
	borderline north={0.8pt}{0pt}{white!25!black},%---- draw line
	interior hidden,
	%frame style={left color=black,right color=white},
	sharp corners=all,
	%frogbox, %TODO: frogbox
	before lower={\tcbsubtitle[before skip=\baselineskip]{Λύση}},lower separated=false,
	before title={\textbf{Άσκηση\ifthenelse{\isempty{#1}}{}{: }}},
	title={\ifthenelse{\isempty{#1}}{\hspace{0pt}}{#1}}%
}

\AtBeginDocument{%
\let\arg\relax
\let\Re\relax
\let\Im\relax
\DeclareMathOperator{\arg}{Arg}
\DeclareMathOperator{\Re}{Re}
\DeclareMathOperator{\Im}{Im}
}
\DeclareMathOperator{\sinc}{sinc}
\DeclareMathOperator{\sgn}{sgn}
\DeclareMathOperator{\erf}{erf}
\DeclareMathOperator{\cov}{cov}
\DeclareMathOperator{\atand}{atan2}

\newenvironment{absolutelynopagebreak}
{\par\nobreak\vfil\penalty0\vfilneg
	\vtop\bgroup}
{\par\xdef\tpd{\the\prevdepth}\egroup
	\prevdepth=\tpd}

\DeclareSIUnit \voltampere { VA } %apparent power 
\DeclareSIUnit \var { VAr } %volt-ampere reactive - idle power 
\DeclareSIUnit \decade { dec } %decade

% Link colours
\hypersetup{colorlinks,linkcolor={blue!40!black!95!green},citecolor={blue!50!black},urlcolor={cyan!70!black}}

% Global amount of samples
% Set to a higher value (e.g. 200) for nicer graphs
% Set to a low value (e.g. 10) for performance
% NOTE: Check the sample variables below for further measurements
\newcommand*{\gsamples}{200}

% Equals command as a workaround for CircuiTikZ bug
% not allowing the = sign in labels
\newcommand*{\equals}{=}

\newcommand{\nesearrow}{%
	\,%
	\smash{\raisebox{-1.1ex}
		{$%
			\stackrel{\displaystyle\nearrow}{\displaystyle\searrow}%
			$}}%
}
\newcommand{\degree}{^{\circ}} % not great
\newcommand\numberthis{\addtocounter{equation}{1}\tag{\theequation}} % add an equation number to a number-less math environment

% Provided commands
\providecommand\dif{d}
\providecommand\od[2]{\frac{#1}{#2}}

\newtcbtheorem[number within=section,list inside=thm]{theorem}{Θεώρημα}%
{colback=green!5,colframe=green!35!black,colbacktitle=green!35!black,fonttitle=\bfseries,enhanced,attach boxed title to top left={yshift=-2mm,xshift=-7mm},width=.9\textwidth,arc=.7mm}{th}
\newtcbtheorem[number within=section,list inside=defn]{defn}{Ορισμός}%
{colback=blue!5,colframe=cyan!35!black,colbacktitle=blue!35!black,fonttitle=\bfseries,enhanced,attach boxed title to top left={yshift=-2mm,xshift=-2mm}}{def}

\makeatletter
\def\tcb@cnt@theoremautorefname{Θεώρημα}
\def\tcb@cnt@defnautorefname{Ορισμός}
\makeatother

% Locus plot utilities
\tikzset{locus/.style={orange!50!red!70!brown}}
\tikzset{locuspole/.style={draw=red!30!black,cross,inner sep=2.5pt,fill=white,fill opacity=.6,thick,label={[below]-90:#1}}}
\tikzset{locuszero/.style={draw=red!30!black,circle,inner sep=2pt,fill=white,fill opacity=.6,thick,label={[below]-90:#1}}}
\tikzset{locusbreak/.style={rounded corners=1.5pt,inner sep=2pt,draw,top color=brown,bottom color=black,fill opacity=.8,label={[below]-90:#1}}}

% Lecture specifications

\newcommand{\listlecturename}{Κατάλογος Διαλέξεων}
\newlistof[chapter]{lecture}{toclec}{\listlecturename}
\renewcommand{\cfttoclectitlefont}{\normalfont\Large\bfseries}

\newcommand{\lecture}[2]{%
	\refstepcounter{lecture}
	\addcontentsline{toclec}{lecture}{\protect\numberline{\thelecture}Διάλεξη #2}
	\hypertarget{lecture_#1}{}
	\nointerlineskip \vspace{.4\baselineskip}%\hspace{\fill}
	\centerline{%\color{#1}
		%\resizebox{1.1\linewidth}{\height}
		\smash{{%
				{\begin{tikzpicture}[xscale=2,baseline={([yshift=0ex]current bounding box.north)}]
					\draw[brown,path fading=west] (0,0) -- (10.1,0);
					\draw[yellow!70!brown,path fading=east] (0,0) -- (10.1,0);
					\draw[>-,brown!90!yellow,>={LaTeX[scale=2]},draw opacity=1] (9.9,0) -- ++(0,-0.01);
					\draw (9.9,-0.4) node[rectangle,align=center,scale=.7,brown!50!black,below]
					{Διάλεξη #1\textsuperscript{η}\\#2};
					\end{tikzpicture}}}}}%
	%\hspace{\fill}
	\par\nointerlineskip \vspace{.5\baselineskip}
}

% Note: Latex requires a configuration flag for PDF named destinations to be stored:
% In dvipdfmx.cfg search for Dvipdfmx Compatibility Flags, and add this line after %C  0x0000:
%     C  0x0010
%

% New plotting utilities
\def\vlowsamples{4}
\def\lowsamples{18}
\def\midsamples{28}
\def\hisamples{40}
\def\timecolour{blue!50!cyan}

\tikzstyle{timecolour}=[\timecolour]

