% !TeX program = xelatex
\documentclass[11pt,a4paper,notitlepage,fleqn,final]{article}

\usepackage{amsmath}
\usepackage{amsfonts}
\usepackage{amssymb}
\usepackage{libs/commath2}
\usepackage[table]{xcolor}
\usepackage[hidelinks,draft=false]{hyperref}
\usepackage[skins,theorems]{tcolorbox}
\usepackage{titlesec}
\usepackage{tikz}
\usepackage{libs/circuitikz} % use our own recent version to make sure some bugs are fixed
\usepackage{pgfplots}
\usepackage{mathtools}
\usepackage[makeroom]{cancel}
\usepackage{mathrsfs}
\usepackage{wrapfig}
%\usepackage{subcaption}
%\usepackage{floatrow}
\usepackage{esint}
\usepackage{enumitem}
%\usepackage{bm}
\usepackage{relsize}
\usepackage{xfrac}
\usepackage{comment}
\usepackage{siunitx}
\usepackage{multicol}
%\usepackage{MnSymbol}
\usepackage[obeyDraft,disable]{todonotes}
%\usepackage{morefloats} % oh no!
%\usepackage[linesnumbered,lined]{algorithm2e}
\usepackage{glossaries}
\usepackage{xifthen}


\pgfplotsset{compat=1.13}
\usetikzlibrary{arrows.meta}
\usetikzlibrary{patterns}
\usetikzlibrary{decorations.pathmorphing}
\usetikzlibrary{decorations.markings}
\usetikzlibrary{backgrounds}
\usetikzlibrary{shapes.misc}
\usetikzlibrary{shapes.multipart}
\usetikzlibrary{shadows.blur}
\usetikzlibrary{fadings}
\usetikzlibrary{intersections}
\usetikzlibrary{arrows.meta}
\usetikzlibrary{calc}
\usetikzlibrary{matrix}
\usetikzlibrary{positioning}
\usetikzlibrary{shapes}
\usetikzlibrary{shadings}

\tcbuselibrary{breakable}
\tcbuselibrary{skins}
\tcbuselibrary{xparse}

\tikzset{cross/.style={cross out, draw,
        minimum size=2*(#1-\pgflinewidth),
        inner sep=0pt, outer sep=0pt}}
\tikzset{
    mark position/.style args={#1(#2)}{
        postaction={
            decorate,
            decoration={
            	post length=1mm, % ??? Magic to fix "Dimension
            	pre length=1mm, % ???  too large" errors.
                markings,
                mark=at position #1 with \coordinate (#2);
            }
        }
    }
}
\tikzset{
	arrow at/.style args={#1}{
		postaction={
			decorate,
			decoration={
				post length=1mm, % ??? Magic to fix "Dimension
				pre length=1mm, % ???  too large" errors.
				markings,
				mark=at position #1 with {\arrow{>}};
			}
		}
	}
}
\makeatletter
\tikzset{
  use path for main/.code={%
    \tikz@addmode{%
      \expandafter\pgfsyssoftpath@setcurrentpath\csname tikz@intersect@path@name@#1\endcsname
    }%
  },
  use path for actions/.code={%
    \expandafter\def\expandafter\tikz@preactions\expandafter{\tikz@preactions\expandafter\let\expandafter\tikz@actions@path\csname tikz@intersect@path@name@#1\endcsname}%
  },
  use path/.style={%
    use path for main=#1,
    use path for actions=#1,
  }
}
\makeatother

\pgfmathdeclarefunction{sinc}{1}{%
	\pgfmathparse{abs(#1)<0.01 ? int(1) : int(0)}%
	\ifnum\pgfmathresult>0 \pgfmathparse{1}\else\pgfmathparse{sin(#1 r)/#1}\fi%
}
\pgfmathdeclarefunction{gauss}{2}{%
	\pgfmathparse{1/(#2*sqrt(2*pi))*exp(-((x-#1)^2)/(2*#2^2))}%
}

\usepackage[left=2cm,right=2cm,top=2cm,bottom=2cm]{geometry}

%\usepackage[no-math]{fontspec}
%\usepackage{fontspec}
\usepackage{mathspec}
%\usepackage{newtxtext,newtxmath}
%\usepackage{unicode-math}
%\setmainfont{texgyretermes-regular.otf}
%\setsansfont{texgyreheros-regular.otf}
%\newfontfamily\greekfont[Script=Greek]{Linux Libertine O}
%\newfontfamily\greekfontsf[Script=Greek]{Linux Libertine O}
\usepackage{polyglossia}
%\newfontfamily\greekfont[Script=Greek]{texgyretermes-regular.otf}
\newfontfamily\greekfontsf[Script=Greek]{texgyreheros-regular.otf}
\newfontfamily\greekfonttt[Script=Greek]{Latin Modern Mono}
%\usepackage[greek]{babel}
\setdefaultlanguage{greek}
\setotherlanguage{english}

%\usepackage[utf8]{inputenc}
%\usepackage[greek]{babel}


%\usepackage{tkz-euclide} % loads  TikZ and tkz-base
%\usetkzobj{angles} % important you want to use angles

\newlist{enumparen}{enumerate}{1}
\setlist[enumparen]{label=(\arabic*)}
\newlist{enumpar}{enumerate}{1}
\setlist[enumpar]{label=\arabic*)}

\newlist{enumgreek}{enumerate}{1}
\setlist[enumgreek]{label=\alph*.}
\newlist{enumgreekparen}{enumerate}{1}
\setlist[enumgreekparen]{label=(\alph*)}
\newlist{enumgreekpar}{enumerate}{1}
\setlist[enumgreekpar]{label=\alph*)}


\newlist{enumroman}{enumerate}{1}
\setlist[enumroman]{label=(\roman*)}

\newlist{enumlatin}{enumerate}{1}
\setlist[enumlatin]{label=(\alph*)}

\newlist{invitemize}{itemize}{1}
\setlist[invitemize]{noitemsep,label=}

\input{libs/fiximplies}
\input{libs/sphere}

\makeatletter
\let\anw@true\anw@false

%\newcommand{\attnboxed}[1]{\textcolor{red}{\fbox{\normalcolor\m@th$\displaystyle#1$}}}
\makeatother
\tcbset{highlight math style={enhanced,colframe=red,colback=white,%
        arc=0pt,boxrule=1pt,shrink tight,boxsep=1.5mm,extrude by=0.5mm}}
\newcommand{\attnboxed}[1]{\tcbhighmath[colback=red!5!white,drop fuzzy shadow,arc=0mm]{#1}}
\newcommand{\infoboxed}[1]{%
	\tcbhighmath[colframe=blue!50!white,colback=blue!5!white,arc=0mm]{#1}}
\titleformat{\section}{\bf\Large}{Κεφάλαιο \thesection}{1em}{}
\newtcolorbox{attnbox}[1]{colback=red!5!white,%
    colframe=red!75!black,fonttitle=\bfseries,title=#1}
\newtcbox{quickattnbox}[1]{colback=red!5!white,%
	colframe=red!75!black,fonttitle=\bfseries,title=#1}
\newtcolorbox{infobox}[1]{colback=blue!5!white,%
    colframe=blue!75!black,fonttitle=\bfseries,title=#1}

\tcbset{frogbox/.style={enhanced jigsaw,%
		overlay first={\foreach \x in {0cm} {
				\begin{scope}[shift={([xshift=-0.2cm]title.west)}]
					\draw[very thick,green!65!black!50!white,latex-] (0,0) -- ++(-1.5,0);
\end{scope}}}}}
\tcbset{frogtitle/.style={
attach boxed title to top left=
{xshift=0mm,yshift=-0.50mm},
boxed title style={skin=enhancedfirst jigsaw,
	bottom=0mm,
	interior style={fill=none,
		left color=green!20!black,
		right color=gray}}
}}
\DeclareTColorBox{exercise}{ O{} }{
	enhanced jigsaw,
	breakable,parbox=false,
	%title style={left color=gray!50!white!50!green,opacity=.5,right color=white},
	subtitle style={%boxrule=1pt,
		colback=yellow!50!red!25!white,fontupper=\bfseries},
	coltitle=black,colbacktitle=green!90!black!25!white,colframe=black,
	frame hidden,
	boxrule=0mm,
	%boxrule=1mm,
	leftrule=0.8pt,toprule=0.8pt,rightrule=0pt, %reserve space
	borderline west={0.8pt}{0pt}{white!25!black},%---- draw line
	borderline north={0.8pt}{0pt}{white!25!black},%---- draw line
	interior hidden,
	%frame style={left color=black,right color=white},
	sharp corners=all,
	%frogbox, %TODO: frogbox
	before lower={\tcbsubtitle[before skip=\baselineskip]{Λύση}},lower separated=false,
	before title={\textbf{Άσκηση\ifthenelse{\isempty{#1}}{}{: }}},
	title={\ifthenelse{\isempty{#1}}{\hspace{0pt}}{#1}}%
}

\AtBeginDocument{%
\let\arg\relax
\let\Re\relax
\let\Im\relax
\DeclareMathOperator{\arg}{Arg}
\DeclareMathOperator{\Re}{Re}
\DeclareMathOperator{\Im}{Im}
}
\DeclareMathOperator{\sinc}{sinc}
\DeclareMathOperator{\sgn}{sgn}
\DeclareMathOperator{\erf}{erf}
\DeclareMathOperator{\cov}{cov}
\DeclareMathOperator{\atand}{atan2}

\newenvironment{absolutelynopagebreak}
{\par\nobreak\vfil\penalty0\vfilneg
	\vtop\bgroup}
{\par\xdef\tpd{\the\prevdepth}\egroup
	\prevdepth=\tpd}

\DeclareSIUnit \voltampere { VA } %apparent power 
\DeclareSIUnit \var { VAr } %volt-ampere reactive - idle power 
\DeclareSIUnit \decade { dec } %decade

% Global amount of samples
% Set to a higher value (e.g. 200) for nicer graphs
% Set to a low value (e.g. 10) for performance
% NOTE: Check the sample variables below for further measurements
\newcommand*{\gsamples}{200}

% Equals command as a workaround for CircuiTikZ bug
% not allowing the = sign in labels
\newcommand*{\equals}{=}

\newcommand{\nesearrow}{%
	\,%
	\smash{\raisebox{-1.1ex}
		{$%
			\stackrel{\displaystyle\nearrow}{\displaystyle\searrow}%
			$}}%
}
\newcommand{\degree}{^{\circ}} % not great
\newcommand\numberthis{\addtocounter{equation}{1}\tag{\theequation}} % add an equation number to a number-less math environment

% Provided commands
\providecommand\dif{d}
\providecommand\od[2]{\frac{#1}{#2}}

\newtcbtheorem[number within=section,list inside=thm]{theorem}{Θεώρημα}%
{colback=green!5,colframe=green!35!black,colbacktitle=green!35!black,fonttitle=\bfseries,enhanced,attach boxed title to top left={yshift=-2mm,xshift=-7mm},width=.9\textwidth,arc=.7mm}{th}
\newtcbtheorem[number within=section,list inside=defn]{defn}{Ορισμός}%
{colback=blue!5,colframe=cyan!35!black,colbacktitle=blue!35!black,fonttitle=\bfseries,enhanced,attach boxed title to top left={yshift=-2mm,xshift=-2mm}}{def}

% Locus plot utilities
\tikzset{locus/.style={orange!50!red!70!brown}}
\tikzset{locuspole/.style={draw=red!30!black,cross,inner sep=2.5pt,fill=white,fill opacity=.6,thick,label={[below]-90:#1}}}
\tikzset{locuszero/.style={draw=red!30!black,circle,inner sep=2pt,fill=white,fill opacity=.6,thick,label={[below]-90:#1}}}
\tikzset{locusbreak/.style={rounded corners=1.5pt,inner sep=2pt,draw,top color=brown,bottom color=black,fill opacity=.8,label={[below]-90:#1}}}

% New plotting utilities
\def\lowsamples{18}
\def\hisamples{40}
\def\timecolour{blue!50!cyan}

\tikzstyle{timecolour}=[\timecolour]



\title{Ανεφάρμοστα Μαθηματικά I
	\\
	{ 
		\normalsize Σημειώσεις από τις παραδόσεις.}
	}
\date{Απρίλης 2018
	\\
	{ 
		\small Τελευταία ενημέρωση: \today
	}
}
\author{
	Για τον κώδικα σε \LaTeX, ενημερώσεις και προτάσεις:
	\\
	\url{https://github.com/kongr45gpen/ece-notes}}

\setallmainfonts(Digits,Latin,Greek){Asana Math}
\setmainfont{Noto Serif}
\setsansfont{Ubuntu}
\usepackage{polyglossia}
\newfontfamily\greekfont[Script=Greek,Scale=1.00]{Liberation Serif}
\usepackage{amsthm}

\hypersetup{pdftitle = {Ανεφάρμοστα Μαθηματικά 1}}
\newcommand{\truncateit}[1]{\truncate{0.8\textwidth}{#1}}
\usepackage[numbers]{natbib}
\usepackage[fit]{truncate}


\newtcbtheorem[number within=section,list inside=thm]{corollary}{Πόρισμα}%
{colback=red!5,colframe=red!35!black,colbacktitle=red!35!black,fonttitle=\bfseries,enhanced,attach boxed title to top left={yshift=-2mm,xshift=-7mm},width=.9\textwidth,arc=.7mm}{th}

\newtcbtheorem[number within=section,list inside=thm]{lemma}{Λήμμα}%
{colback=orange!5,colframe=orange!35!black,colbacktitle=orange!35!black,fonttitle=\bfseries,enhanced,attach boxed title to top left={yshift=-2mm,xshift=-7mm},width=.9\textwidth,arc=.7mm}{th}

\newtcbtheorem[number within=section,list inside=thm]{claim}{Ισχυρισμός}%
{colback=purple!5,colframe=purple!35!black,colbacktitle=purple!35!black,fonttitle=\bfseries,enhanced,attach boxed title to top left={yshift=-2mm,xshift=-7mm},width=.9\textwidth,arc=.7mm}{th}

\newtcbtheorem[number within=section,list inside=thm]{proposition}{Πρόταση}%
{colback=gray!5,colframe=gray!35!black,colbacktitle=gray!35!black,fonttitle=\bfseries,enhanced,attach boxed title to top left={yshift=-2mm,xshift=-7mm},width=.9\textwidth,arc=.7mm}{th}

\newtcbtheorem[number within=section,list inside=thm]{conjecture}{Εικασία}%
{colback=lime!5,colframe=lime!35!black,colbacktitle=orange!35!black,fonttitle=\bfseries,enhanced,attach boxed title to top left={yshift=-2mm,xshift=-7mm},width=.9\textwidth,arc=.7mm}{th}

\newtcbtheorem[number within=section,list inside=thm]{question}{Ερώτημα}%
{colback=magenta!5,colframe=magenta!35!black,colbacktitle=magenta!35!black,fonttitle=\bfseries,enhanced,attach boxed title to top left={yshift=-2mm,xshift=-7mm},width=.9\textwidth,arc=.7mm}{th}

\newtcbtheorem[number within=section,list inside=thm]{definition}{Ορισμός}%
{colback=cyan!5,colframe=cyan!35!black,colbacktitle=cyan!35!black,fonttitle=\bfseries,enhanced,attach boxed title to top left={yshift=-2mm,xshift=-7mm},width=.9\textwidth,arc=.7mm}{th}

\newtcbtheorem[number within=section,list inside=thm]{example}{Παράδειγμα}%
{colback=brown!5,colframe=brown!35!black,colbacktitle=brown!35!black,fonttitle=\bfseries,enhanced,attach boxed title to top left={yshift=-2mm,xshift=-7mm},width=.9\textwidth,arc=.7mm}{th}

\newtcbtheorem[number within=section,list inside=thm]{notation}{Συμβολισμός}%
{colback=red!5,colframe=green!35!black,colbacktitle=yellow!35!black,fonttitle=\bfseries,enhanced,attach boxed title to top left={yshift=-2mm,xshift=-7mm},width=.9\textwidth,arc=.7mm}{th}

\begin{document}
\maketitle

\hrule
\vspace{50pt}

\begin{infobox}{Λάθη \& Διορθώσεις}
	Οι τελευταίες εκδόσεις των σημειώσεων βρίσκονται στο Github
	(\url{https://github.com/kongr45gpen/ece-notes/raw/master/unapplied.pdf}) ή
	στη διεύθυνση \url{http://helit.org/ece-notes/unapplied.pdf}.
	
	Περιέχουν διορθώσεις σε λάθη και τυχόν βελτιώσεις.
	
	\tcblower
	
	Μπορείτε να ενημερώνετε για οποιοδήποτε λάθος και πρόταση
	μέσω PM στο forum, issue στο Github, ή οποιουδήποτε άλλου τρόπου!
\end{infobox}

Το μάθημα διδάσκεται 4 ώρες την εβδομάδα.

Αντικείμενο: Ανεφάρμοστα Μαθηματικά I.

Βαθμολόγηση μαθήματος:
\begin{enumerate}
	\item Εξετάσεις
	\item Πρόοδος
	\item Εβδομαδιαίες Εργασίες
	\item Προαιρετικό Θέμα
	\item Παρουσίαση ενός θέματος ενδιαφέροντος (στο τέλος του εξαμήνου)
	\item Συμμετοχή σε εργαστήριο
	\item Προφορική εξέταση
\end{enumerate}

Το μάθημα βαθμολογείται με έναν από 4 διαφορετικούς τρόπους.
\begin{enumerate}
	\item \textbf{Μόνο γραπτές εξετάσεις}
	
	Για τους φοιτητές που επιθυμούν να συμμετάσχουν μόνο στις γραπτές εξετάσεις, ο μέγιστος βαθμός είναι 8. Για να περαστεί το μάθημα, απαιτείται βαθμός \( \geq 6.0 \) στις εξετάσεις. 
	
	Στις εξετάσεις υπάρχουν ερωτήσεις bonus που δίνουν μέγιστο βαθμό \( \pm 1 \) στον φοιτητή. Κάθε bonus ερώτηση προσθέτει μονάδες αν απαντηθεί σωστά, και αφαιρεί ίσο ποσό αν απαντηθεί λάθος. Επομένως αφήνεται στην κρίση του φοιτητή η επιλογή της απάντησης στις bonus ερωτήσεις. Αν όλες απαντηθούν σωστά, μέγιστος βαθμός είναι ο 9. Ο ελάχιστος βαθμός \( 6.0 \) \textbf{δεν} συμπεριλαμβάνει τις απαντήσεις στις bonus ερωτήσεις.
	\item \textbf{Συμμετοχή σε εξετάσεις και πρόοδο}
	
	Οι φοιτητές που γράφουν συμμετέχοντας σε εξετάσεις και τις 3 προόδους του μαθήματος, γράφουν με μέγιστο βαθμό το 11, ο οποίος
	αποκτά την προφανή μέγιστη τιμή 10 λόγω του βαθμολογικού συστήματος.
	
	Η κάθε πρόοδος προσθέτει 1 μονάδα στις εξετάσεις. Επιπλέον, για επιβράβευση του φοιτητή, προστίθεται ο συντελεστής συσχέτισης μεταξύ του βαθμού των προόδων. Η τυπική απόκλιση υπολογίζεται από τον τύπο \( \sigma = \sqrt{\frac{(p_1-M)^2+(p_2-M)^2+(p_3-M)^2}{3}} \), όπου \( p_i \) ο βαθμός της \(i\)-οστής προόδου, και \( M \) η μέση τιμή των προόδων. Στον τελικό βαθμό προστίθεται ο όρος \( \frac{1}{\sigma+1} \), που αποκτάει μέγιστη τιμή 1. Έτσι μπορεί να συμφέρει στο
	φοιτητή να γράψει χειρότερα σε μία πρόοδο, έτσι ώστε να έχει παρόμοια αποτελέσματα και να επιβραβευτεί.
	
	Η δήλωση γίνεται συνολικά και για τις 3 προόδου. Εάν ένας φοιτητής δεν συμμετάσχει σε μία πρόοδο, αποκτά σε αυτήν βαθμό \( 1.4\cdot M - 6 \), όπου \( M  \) ο μέσος όρος των άλλων δύο προόδων. Εάν ένας φοιτητής δεν συμμετάσχει σε δύο προόδους, αποκτά
	συνολικό βαθμό προόδων \( \frac{p_i}{2.5} \), όπου \( p_i \) ο βαθμός στην πρόοδο στην οποία συμμετείχε. Εάν δεν συμμετάσχει
	σε καμία πρόοδο, αποκτά βαθμό προόδων 0. Ο βαθμός προόδων πολλαπλασιάζεται με τον συντελεστή \( \frac{3}{10} \) και προστίθεται στον τελικό βαθμό.
	\item \textbf{Συμμετοχή σε εργασίες}

	Αν ο φοιτητής υποβάλλει τουλάχιστον μία εργασία ή το προαιρετικό θέμα, υλοποιείται η 3\textsuperscript{η} μέθοδος βαθμολόγησης.
	
	Σύμφωνα με αυτήν, μετράει κατά 80\% ο βαθμός του 1\textsuperscript{ου} ή του 2\textsuperscript{ου} τρόπου βαθμολόγησης (ανάλογα με την επιλογή ή όχι των προόδων), και κατά 30\% ο βαθμός των εργασιών.
	
	Ο βαθμός των εργασιών έχει μέγιστη τιμή 12, και συνυπολογίζεται με το βαθμό των εξετάσεων ως εξής:
	\begin{enumroman}
		\item Δίνονται 6 εργασίες με αυξανόμενο βαθμό \( 0.3e_i \), όπου \( e_i \) ο μέγιστος βαθμός της \( i \)-οστής εργασίας.
		\item Προστίθεται ο ανηγμένος συντελεστής συσχέτισης \( \frac{1}{\sqrt{\mathrm{Var}\,(e_i)}+1} \) για να επιβραβευθούν φοιτητές με ομογενή αποτελέσματα.
		\item Προστίθεται ο συντελεστής συσχέτισης του Pearson \( \rho_{e_i,Y}= \frac{\operatorname{cov}(e_i,Y)}{\sigma_{e_i} \sigma_Y} \) όπου \( Y \) η ευθεία ελάχιστων τετραγώνων που προκύπτει από τους βαθμούς στις εργασίες, έτσι ώστε να ενθαρρυνθεί η αύξηση του βαθμού κατά τη διάρκεια του εξαμήνου.
		\item Το προαιρετικό θέμα λαμβάνει μέγιστο βαθμό 5.7.
		\item Ο μέγιστος βαθμός 14 μειώνεται στο 12 αν είναι μεγαλύτερός του.
		\item Πολλαπλασιάζουμε με τον παρακάτω συντελεστή \( \mathfrak U \), έτσι ώστε να μειωθεί η επίδραση του βαθμού των εργασιών, αν αυτός είναι πολύ μακριά από τον αντίστοιχο των εξετάσεων:
		\[
		\mathfrak U = \frac{1}{(x - e_{\mathrm T})^2+0.9}
		\qquad \text{ όπου $x$ ο βαθμός των εξετάσεων και $e_{\mathrm T}$ ο βαθμός των εργασιών}
		\]
		\item Στις εξετάσεις του \textbf{Σεπτέμβρη}, αν $e_{\mathrm T}$ ο βαθμός των εργασιών, το αποτέλεσμα πολλαπλασιάζεται
		με τον παρακάτω συντελεστή \( \mathfrak S \), έτσι ώστε να επιβραβευθούν οι μεγαλύτεροι βαθμοί:
		\[
		\mathfrak S = 0.1e_{\mathrm T}
		\]
	\end{enumroman}
	\item \textbf{Συμμετοχή στο εργαστήριο}
	
	Οι φοιτητές έχουν δικαίωμα συμμετοχής στο εργαστήριο του μαθήματος. Η βαθμολόγηση γίνεται στο τέλος του εξαμήνου από το e-learning, και εξαρτάται από την απάντηση ερωτήσεων Σωστού/Λάθους. Για μια πιο ενεργητική βαθμολόγηση, ο φοιτητής καλείται
	μετά από κάθε απάντηση να δηλώσει τη σιγουριά για την απάντησή του αυτή. Έτσι, ερωτήσεις που απαντιούνται σωστά με μεγάλη σιγουριά παίρνουν μεγάλη θετική βαθμολογία. Ερωτήσεις που απαντιούνται με μικρή σιγουριά παίρνουν μικρή θετική ή αρνητική βαθμολογία.  Ερωτήσεις που απαντιούνται \textit{λάθος} με μεγάλη σιγουριά παίρνουν μεγάλη θετική βαθμολογία. 3 λάθος απαντήσεις με μεγάλη σιγουριά αφαιρούν 1 μονάδα. 1 επιπλέον λάθος απάντηση με μεγάλη σιγουριά αφαιρεί 1 επιπλέον μονάδα. 2 επιπλέον λάθος απαντήσεις με μεγάλη σιγουριά καθιστούν άμεσα το βαθμό του εργαστηρίου -2.
	
	Ο βαθμός της κάθε απάντησης εξαρτάται από το πόσο σωστή είναι η απάντηση (\( c \in \left\lbrace -10,-2,0,2,10 \right\rbrace \))
	και τη σιγουριά του φοιτητή (\( s \in \left\lbrace -2,-1,0,1,2 \right\rbrace \)). Ο βαθμός προκύπτει από τον τύπο:
	\[
	0.017 \cdot c \cdot (s+2.5)
	\]
	
	Από το τελικό άθροισμα εξαιρείται το πολύ 1 outlier. Outlier θεωρείται μια τιμή όταν είναι 80\% μεγαλύτερη ή μικρότερη από τη μέση τιμή.
	
	Ο βαθμός του εργαστηρίου είναι προσθετικός.
	
	Η παρακολούθηση είναι προαιρετική, αλλά για να μετρήσει ο βαθμός του εργαστηρίου απαιτείται απουσία σε \( \leq 2 \) εργαστήρια.
	
	Συμμετοχή στο εργαστήριο από προηγούμενα έτη υπολογίζεται μόνο για συμμετοχές από το 2012 και μετά. Φοιτητές που παρακολούθησαν το εργαστήριο τη χρονιά 2011 μπορεί να θεωρηθεί πως το έχουν παρακολουθήσει, εφ' όσον επισκεφθούν μόνο την παράδοση της 4\textsuperscript{ης} εργαστηριακής άσκησης.
	
	
\end{enumerate}

Τα Ανεφάρμοστα Μαθηματικά I είναι ένα από τα πιο εύκολα μαθήματα της σχολής, που βασίζεται σε απλές και γνωστές έννοιες. Στην εξαιρετική περίπτωση που κάποιος θέλει να φρεσκάρει τις έννοιες αυτές, μπορεί να διαβάσει το σύγγραμμα \textit{Δια-Συμπαντική Θεωρία TeichM\"uller \textendash Κατασκευή Θεάτρων Hodge} (Inter-universal Teichmuller Theory I: Construction of Hodge Theaters) του Shinichi Mochizuki.

\section{Εισαγωγή}

Αφήνω το $ P $ να είναι ένα άπειρο subring. Επιθυμούμε να επεκτείνουμε τα αποτελέσματα του \cite{cite:0} σε σύνθετες, παγκοσμίως φυσιολογικές, σημειακές επιθετικές λειτουργίες. Δείχνουμε ότι κάθε υπερ-αλγεβρικά ultra-Littlewood subring ενεργώντας υπερ-φυσικά σε έναν υπερ-διαφοροποιήσιμο ομομορφισμό είναι κυρτό. Κάθε μαθητής γνωρίζει ότι $ \varepsilon = \Delta $. Αυτό θα μπορούσε να ρίξει σημαντικό φως σε μια εικασία Chebyshev.

Ένα κεντρικό πρόβλημα στη γενική θεωρία των γραφημάτων είναι η επέκταση των απλώς αντι-ενδογενών, υπο-Γκαουσιακών φορέων. Αντίθετα, είναι γνωστό ότι $ \mathcal {{J}} \ne 1 $. Το πρωτοποριακό έργο του F. Raman σε διαφοροποιήσιμες υπολεκάνες ήταν σημαντική πρόοδος.

Στο \cite{cite:0}, δείχνει ότι $ \kappa \ge e $. Είναι γνωστό ότι κάθε υπερ-απλά ελεύθερο μέτρο εφοδιασμένο με μια αντιστρεπτή εξίσωση είναι σχεδόν πλήρες. Αυτό μειώνει τα αποτελέσματα του \cite{cite:0} σε μια εύκολη άσκηση. Σε αυτή τη ρύθμιση, η ικανότητα ταξινόμησης οιονεί παραδεκτών ομοιομορφισμών είναι απαραίτητη. Επιπλέον, η περιγραφή του U. C. Raman για τα ιδανικά του Fibonacci ήταν ένα ορόσημο στην θεωρία γραφημάτων $ p $ -adic. Σε αυτή τη ρύθμιση, είναι απαραίτητη η ικανότητα ταξινόμησης μετρήσιμων συνόλων.

Στο \cite{cite:1}, οι συγγραφείς επέκτειναν ομομορφισμούς αριστερά-τοπικά ολοκληρώσιμους. Πρόσφατα, έχει υπάρξει μεγάλο ενδιαφέρον για την κατασκευή $ \mathcal {{R}} $ - αμέτρητα υποσύνολα. Επιπλέον, δυστυχώς, δεν μπορούμε να υποθέσουμε ότι υπάρχει μια συνεχώς αντίθετη, πεπερασμένη, υπό όρους αντικανονική και κατά το μάλλον ή ήττον ορθή και ασήμαντη μοναδική υποομάδα. Δυστυχώς, δεν μπορούμε να υποθέσουμε ότι $ \| \hat {A} \| = - \infty $. Μια χρήσιμη ανασκόπηση του θέματος μπορεί να βρεθεί στο \cite{cite:2}. Στο \cite{cite:1}, οι συγγραφείς ασχολούνται με την επιλεκτικότητα των πολλαπλασίων υπό την πρόσθετη υπόθεση ότι $ {\eta ^ {(\Delta)}}> \sqrt {2} $.

Είναι γνωστό από καιρό ότι η υπόθεση Riemann κρατάει \cite{cite:3}. Αυτό αφήνει ανοικτό το ζήτημα της ύπαρξης. Οι πρόσφατες εξελίξεις στην θεωρία του παγκόσμιου αριθμού \cite{cite:4} έθεσαν το ερώτημα αν $ N \left(- \Phi (\pi), \dots, \aleph_0 \times \bar {J}\right) $ Στο πνεύμα του \cite{cite:5}, οι συγγραφείς ασχολούνται με την καταμέτρηση των αναστρεφόμενων, εγχυτικών συστημάτων υπό την επιπρόσθετη υπόθεση ότι κάθε αριστερά κυρτός μορφισμός είναι ελεύθερα υποτοπικός, σχεδόν σίγουρα ένας -ένα και ελεύθερα υπερ-εφαπτόμενο. Είναι σημαντικό να θεωρήσουμε ότι το $ H $ μπορεί να παραμείνει σχεδόν Lebesgue.

\subsection{Κύριο Αποτέλεσμα}
\begin{definition}{}{}
	Έστω ${\mathfrak{{y}}_{\epsilon,\mathfrak{{t}}}} = \bar{\mathbf{{c}}} ( {p_{\mathbf{{y}}}} )$. Λέμε ότι μια κατηγορία ${F_{Z}}$ είναι \textbf{γραμμική} ανν είναι συντοπικά ολόμορφη.
\end{definition}


\begin{definition}{}{}
	Έστω ότι έχουμε έναν φυσικό γράφο ${\mathcal{{B}}_{Z,h}}$.  Ένα αντιστρέψιμο, προσθετικό, καθολικό βέλος ονομάζεται \textbf{υποομάδα} αν είναι αντι-Liouville και συμπαγές.
\end{definition}

Στο \cite{cite:6}, το κύριο αποτέλεσμα ήταν η ταξινόμηση των ημι-γραμμικά μηδενικών ιδανικών. Το πρόσφατο ενδιαφέρον σχεδόν σχεδόν παντού έχει επικεντρωθεί σε υπολογιστικούς παράγοντες. Κάθε μαθητής γνωρίζει ότι $ g ({v _ {\mathbf {{l}}, q}}) = Y $. Το πρωτοποριακό έργο του J.V. Gupta για τα κανονικά συστήματα ήταν μια σημαντική πρόοδος. Ένα κεντρικό πρόβλημα στην αξιωτική Κ-θεωρία είναι η επέκταση των αριστερών εγγενών πεδίων. Είναι ευρέως γνωστό ότι υπάρχει μια Ευκλείδης αριστερά-φυσικά μετρήσιμη πολλαπλή. Θα ήταν ενδιαφέρον να εφαρμοστούν οι τεχνικές του \cite{cite:1} σε ζεύγη ψευδο-θετικές, αντι-Clairaut, σχεδόν οιονεί συνεχείς λειτουργίες. Μια {} χρήσιμη επισκόπηση του θέματος μπορεί να βρεθεί στο \cite{cite:6}. Κάθε μαθητής γνωρίζει ότι
\begin{align*}
\hat{\epsilon}  \left(- \pi, \dots, \mathscr {{S}} (-Κ) {\Delta_ { \mathscr {{F}}, \chi}} \right)^ {1} -i. 
\end{align*} Πρόσφατες εξελίξεις στην υψηλότερη θεωρία ομολογίας γραφημάτων.

\begin{definition}{}{}
	Ένας εκφυλισμένος γράφος ${\mathcal{{T}}_{Z,h}}$ είναι \textbf{Monge} αν το $Z$ δεν κυριαρχείται από το $y$.
\end{definition}

Τώρα παρουσιάζουμε το κύριο αποτέλεσμα.

\begin{theorem}{}{}
	Ας υποθέσουμε ότι μας δίνεται ένα $ p $-αδικό ιδανικό $ H $. Έστω ότι το $ \iota $ είναι ένα υπερ-τελείως επιθετικό επίπεδο που δρα ελεύθερα σε μια τυχαία μεταβλητή, Περαιτέρω, αφήστε το $ {A _ {\kappa}} $ να είναι μια πρωταρχική εξίσωση εξοπλισμένη με μια λειτουργία ψευδο-one-to-one. 
	
	Τότε ισχύει: \[ \tilde {\mathfrak {{u}}} <1 \]
\end{theorem}

Είναι δυνατόν να περιγράψουμε τις καμπύλες Brahmagupta κατά των εξαρτημένων, κανονικά μοναδικών; Επομένως δεν είναι ακόμα γνωστό αν $ V> -1 $, αν και \cite{cite:2} αντιμετωπίζει το ζήτημα της ύπαρξης. Πρόσφατα, υπήρξε μεγάλο ενδιαφέρον για την παραγωγή ομαλών, γενικών υποκειμένων. Στο \cite{cite:0, cite:8}, οι συγγραφείς ασχολούνται με τη μοναδικότητα των διανυσματικών χώρων κάτω από την πρόσθετη παραδοχή ότι κάθε ψευδο-συνεταιριστική ομάδα Kolmogorov είναι ορθογώνια, μερική, ασήμαντα ασήμαντη και μη συνεχώς αντι-προσθετική. Ως εκ τούτου, στο \cite{cite:0}, οι συγγραφείς εξετάζουν την ύπαρξη σχεδόν αμετάβλητων, Russell, Cardano functors κάτω από την πρόσθετη υπόθεση ότι $ e> \emptyset $.

\subsection{Μοναδικότητα}

Είναι δυνατόν να επεκταθούμε με υπό όρους αντισταθμιστικούς, άνευ όρων, διαφοροποιήσιμοι, συν-διαφοροποιήσιμοι αριθμοί; Έτσι, ο στόχος του παρόντος άρθρου είναι να επεκταθούν οι ορθοστατικά οριακές ομομορφισμοί. Ήταν ο Pascal που ρώτησε για πρώτη φορά αν παντού μπορούν να εξεταστούν οι ελάχιστες, Noetherian, εξαιρετικά ομαδικές Artinian ομάδες. Επομένως, στο \cite{cite:8}, οι συγγραφείς εξετάζουν την ομαλότητα των κατηγοριών κάτω από την πρόσθετη υπόθεση ότι κάθε σημείο είναι ανοιχτό. Επομένως, αυτό αφήνει ανοικτό το ζήτημα της δυνατότητας μείωσης.

Ας υποθέσουμε ότι μας δίνεται ένα κανονικά υπερ-προσθετικό υποκείμενο $ B $.

\begin{definition}{}{}
Ας υποθέσουμε ότι $ H \pm \pi = {O ^ {(Lambda)}} \left (1 ^ {- 6}, \sqrt {2} ^ {- 5} \right) $. Ένας σχεδόν μη αναστρέψιμος, σχεδόν ημι-Lobachevsky functor είναι ένα \textbf {matrix} αν είναι ελάχιστο, αναλλοίωτο, αναλυτικά μοναδικό και συνεχώς υπο-Lebesgue.
\end{definition}


\begin{definition}{}{}
Αφήνει το $ \hat {q} \ne \sqrt {2} $ να είναι αυθαίρετο. Λέμε ένα στοχαστικό σύνολο $ L $ είναι \textbf {Torricelli} αν είναι ουσιαστικά υπερ-πλήρης.
\end{definition}


\begin{theorem}{}{}
Αφήνει $ \mathcal {{S}} = 0 $ να είναι αυθαίρετος. Ας $ | \tilde {\mathscr {{J}}} \ge \mathscr {{A}} $ να είναι αυθαίρετη. Στη συνέχεια $ \tilde {k} \ge {\mathbf {{h}} _ {X}} $.
\end{theorem}


\begin{proof}{}{}
Αυτό είναι προφανές.
\end{proof}


\begin{lemma}{}{}
$ \bar {l} = 0 $.
\end{lemma}

\begin{proof} 
	Η βασική ιδέα είναι ότι κάθε ορθογωνικό, Shannon, ένα-προς-ένα μητρώο που επιδρά πεπερασμένα σε ένα πεδίο Monge είναι  $\mathscr{{I}}$-Chern. Ας υποθέσουμε ότι $$s \left(-\infty^{5}, I \right) = k'' \left( {\mathbf{{q}}_{\mathscr{{O}}}}-\pi \right) \cap \mathcal{{X}} \left( \mathfrak{{k}}, \dots, \mathcal{{F}}^{7} \right).$$ Κάποιος μπορεί εύκολα να δει πως αν το  $\hat{F}$ είναι μη φυσικά μετρήσιμο και σχεδόν φυσικά ψευδομετρήσιμο, τότε: \begin{align*} {\Phi_{\zeta}} \left(-M, \dots,-\hat{\theta} \right) & = \bigcup  \sin^{-1} \left( 2 \wedge \sqrt{2} \right) \\ & \ne \sum_{\bar{\mathscr{{G}}} \in \tilde{\Gamma}}  T'' \left( F^{-5}, \tau'' 0 \right) + \dots \pm \overline{\bar{\phi} ( D )}  \\ & \le \int_{m} \sin^{-1} \left( | \tilde{s} |^{-8} \right) \,d A \pm \dots \cap {A_{s,p}} \left(-\phi ( {\mathbf{{l}}_{\rho}} ), \emptyset-1 \right)  \\ & \ge \varprojlim \int_{i}^{-1} \cos \left( i^{-8} \right) \,d {\mathbf{{v}}_{\theta}}-\mathbf{{q}} \left( | \mathscr{{X}} | \wedge {Z_{\mathcal{{M}}}} \right) .\end{align*} Από κυρτότητα, $\tilde{f} = 2$.Παρατηρούμε ότι κάθε στοχαστικά ολόμορφος αριθμός δεν μετριέται με βάση το \( \mathcal M \). Έτσι κάθε λογικά κυρτός (από δεξιά) αριθμητικός όρος είναι left-Fr\'echet. Επομένως: \begin{align*} \cosh \left( \frac{1}{2} \right) & > \frac{{d_{K,J}} \left( i, \dots, m \pm 1 \right)}{\mathcal{{O}} \left( \Sigma, \dots, 1 \cdot 2 \right)} \vee \dots + \Phi^{-1} \left(-0 \right)  \\ & \ge \left\{ \Phi'' \colon \tanh^{-1} \left( \sqrt{2} \Gamma \right) < \sum_{\hat{\mathbf{{l}}} = \emptyset}^{0}  \tanh \left( U-\Omega \right) \right\} .\end{align*}
	Αυτό είναι άτοπο.
\end{proof}


Είναι δυνατόν να περιγράψουμε τα d'Alembert, Huygens sets; Σε μελλοντικές εργασίες, σχεδιάζουμε να αντιμετωπίσουμε ζητήματα μοναδικότητας καθώς και διαχωρισμού. Τώρα οι πρόσφατες εξελίξεις στην $ p $ -adic probability \cite{cite:8} έθεσαν το ερώτημα αν το $ U \in {Z ^ {(\xi)}} $. Όπως φαίνεται τα παραπάνω μαθηματικά είναι αρκετά ανεφάρμοστα. Επιθυμούμε να επεκτείνουμε τα αποτελέσματα του \cite{cite:5} στα ιδανικά. Το πρόσφατο ενδιαφέρον για αντισυμβατικούς αριθμούς έχει επικεντρωθεί στην περιγραφή $ n $-διαστάσεων γραμμών. Στο \cite{cite:7}, οι συγγραφείς χαρακτήρισαν ιδανικά. Τώρα μια {} χρήσιμη έρευνα του θέματος μπορεί να βρεθεί στο \cite{cite:2}. Κάθε μαθητής γνωρίζει ότι $ \tilde {S} \subset {\kappa_ {u}} $. Δυστυχώς, δεν μπορούμε να υποθέσουμε ότι $ x = \mathcal {{S}} $. Σε αυτή τη ρύθμιση, είναι απαραίτητη η ικανότητα να αντλούν εντελώς ελάχιστα, συγγενικά, σχεδόν παντού αντικαταθλιπτικά στοιχεία.

\section{Θεωρία Ολοκληρωτικών Μετρήσεων}


Είναι δυνατόν να ταξινομηθούν τα ημι-διακριτά αποδεκτά, παγκοσμίως καθολικά βέλη; Δεν είναι ακόμα γνωστό αν ${E^{(\mathbf{{r}})}}$ δεν είναι ισοδύναμο με το $f$, αν και αντιμετωπίζει το ζήτημα της ομαλότητας. Στο πλαίσιο αυτό, τα αποτελέσματα είναι εξαιρετικά σημαντικά. Είναι γνωστό ότι η εικασία του Riemann είναι ψευδής στο πλαίσιο των εφαπτόμενων υποκειμένων. Είναι μακράν γνωστό ότι η εικασία του Taylor είναι ψευδής στο πλαίσιο των υποκανονικών, Laplace subrings. Στη συνέχεια, είναι σημαντικό να θεωρήσουμε ότι το $ Z $ μπορεί να είναι καθολικά άπειρο.

Έστω αυθαίρετο $\mathfrak{{e}}'' \subset {\mathbf{{j}}^{(v)}}$.

\begin{definition}{}{}
	Έστω αυθαίρετο $F' < e$. Μια υπερ-Lambert, παραβολική, αμετάβλητη λειτουργική είναι ένα \textbf {στοιχείο} αν είναι Banach.
\end{definition}


\begin{definition}{}{}
	Αφήστε $ T '\le 0 $. Μια υπο-ομάδα αντικατοπτρισμού είναι ένα \textbf{μονοδρόμηση} αν είναι γενικά ενσωματωμένη.
\end{definition}


\begin{proposition}{}{}
	Έστω τρίγωνο $I$.  Τότε το $\omega$ είναι Wiener.
\end{proposition}


\begin{proof} 
	Η απόδειξη αφήνεται ως άσκηση στον αναγνώστη.
\end{proof}


\begin{proposition}{}{}
	Ας υποθέσουμε ότι κάθε μονοϊό είναι οριοθετημένο.  Τότε το ${\mathfrak{{a}}^{(j)}}$ διαφοροποιείται από το ${l^{(\mathbf{{k}})}}$.
\end{proposition}


\begin{proof} 
	Ξεκινάμε εξετάζοντας μια απλή ειδική περίπτωση. Με το θεώρημα Noether, οι εικασίες του Banach είναι ψευδείς στο πλαίσιο των ομοιομορφισμών του Eudoxus. Τώρα $ X (K) \ne {B_ {N}} $. Στη συνέχεια, υπάρχει ένας αντίθετος χωριστός, οιονεί διατεταγμένος, Poncelet και ολόσωμος φυσικός καθολικός, ο Levi-Civita homeomorphism. Προφανώς, αν $ | b | \le \mathfrak {{y}} '' $ τότε $ \Theta <{\Gamma ^ {(\omega)}} $. Προφανώς, το $ {Z ^ {(\mathscr {{K}})}} $ δεν είναι μεγαλύτερο από $ i $.
	
	Ας υποθέσουμε ότι $ c ^ {- 9} = \bigcap_ {P = \sqrt {2}} ^ {\pi} \overline {R (\Sigma)} \vee \| B \|. $$ Φυσικά, εάν το $ {\mathfrak {{a}} _ {\mathscr {{B}}}} είναι αλγεβρικό $ n $- διαστάσεων τότε κάθε τελικά Hardy Το βέλος είναι αναγώγιμο. Αντίθετα, $ {\varepsilon ^ {(\mathfrak {{e}})}}> 0 $. Επομένως ισχύει το κριτήριο του Pappus. Προφανώς, το $ S $ είναι συσχετιστικό και σωστό - $ p $ -adic. Στη συνέχεια, $ | S '| \cong {N_ {\eta, u}} $. Εξ ου και $ \bar {\mathfrak {{h}}}> \aleph_0 $. Όπως έχουμε δείξει, αν $ \| \tilde {\Theta} \| = {\mathbf {{a}} _ {M, i}} $ τότε $ \mathfrak {{u}} \υποσύνολο \| t \| $. Είναι εύκολο να δούμε ότι αν $ \tilde {x} $ είναι μεγαλύτερο από $ \bar {\Theta} $ τότε $ \bar {\mathscr {{O}}} = - 1 $.
	
	
	Αφήνω $ a '= | \mathcal {{U}} | $. Τριπλά, αν $ \Ωμέγα $ είναι $ Y $ -Fermat τότε $ \tilde {\mathbf {{w}}} $ δεν κυριαρχείται από $ O $. Φυσικά, εάν το $ A $ είναι αντιστρέψιμο, ομαλά παραγγείλαμε και Pascal τότε $ \mathcal {{O}} \le s $. Δεδομένου ότι υπάρχει ένας $ m $ -onto και $ C $ -ταγωνικός ισομορφισμός, κάθε μηδενικό τρίγωνο είναι υπερπροσθετικό. Όπως έχουμε δείξει, οι εικασίες του Eudoxus είναι αληθινές στο πλαίσιο των ελεύθερων γραμμών. Με μια εύκολη άσκηση, $ \| n '\| <-\infty $. Με ένα πολύ γνωστό αποτέλεσμα του Grassmann \cite {cite: 7}, το $ {\mathcal {{E}} _ {\delta, s}}$ είναι κανονικά υπο-αναστρέψιμο και πολύπλοκο. Στη συνέχεια, αν ισχύει το κριτήριο του Perelman τότε $ \hat {\mathfrak {{a}}} \ge \emptyset $.
	
	
	Τριπλά, κάθε διάνυσμα είναι μετρήσιμο. Τριβίως, $ -1-\infty = \sin \αριστερά (R '\δεξιά) $. Τώρα $ \mathcal {{O}} $ οριοθετείται από $ \Sigma $. Τριβίως, \αρχίζει {ευθυγραμμίσει *} q '& \equiv \bigotimes J \αριστερά (-ζ = = | \mathcal {{B}} \| zz = , 0 \φλιτζάνι \| \mathcal {{I}} \| \δεξιά) \pm \overline {- 1} \\& <\frac {{Z ^ {(\ωμέγα)}} (D)} {{\mathfrak {{t}} _ {c}} \αριστερά }, \dots, \Omega ^ {3} \δεξιά)} \vee \mu \αριστερά (2 \Ωμέγα, \άδειο \δεξιά) zz = \& \sim \frac {H \αριστερά (\frac {1} {\sqrt {2}}, d \δεξιά)} {\overline {| {\mathcal {{C}} ^ {(S)}} |}} .\τέλος {ευθυγραμμίστε *} Έτσι ισχύει η υπόθεση Riemann.


Επειδή $ {p ^ {(\mathscr {{S}})}} (\tilde {d}) = 1 $, αν $ \bar {\mathcal {{Q}}} $ δεν είναι μεγαλύτερη από $ \mathfrak {{s}} $ τότε $ \tilde {J} \cong \sqrt {2} $. Τώρα $ {\mathcal {{U}} _ {\mathfrak {{c}}} \αριστερά (-ζ == | \hat {U} \| == | \Xi '\| \δεξιά) \le \int_ {W} \overline {0} zz ==, d E. $$ Τριπλάσιο ισχύει το κριτήριο του Turing. Σαφώς, $ I = 2 $. Σαφώς, εάν $ \mathcal {{Q}} \le \mathscr {{U}}$ \alpha = \mathbf {{k}} $. Αντίθετα, κάθε μετρήσιμη μονόδρομο είναι παντού αντίθετη και Wiener.


Έστω $ \mathscr {{G}} \ne | \epsilon | $ να είναι αυθαίρετος. Είναι εύκολο να δείτε ότι \begin{align*} \tilde {\mathscr {{C}}} ^ {- 1} \αριστερά (\pi \δεξιά) \frac {\| {\άλφα_ {l, k}} \| \hat {x}} {\mathscr {{Z}} \αριστερά (r ^ {- 8}, \mathcal {{Q}} ^ {- 1} \δεξιά}} \σφήνα \Ωμέγα \αριστερά (\frac {1} {- \infty}, | H | \pm \tilde {\epsilon} \δεξιά) \\& \le Z ^ {- 6} \σφήνα \dots \φλιτζάνι 1 \άδειο \\&> \sup E '' ^ {1} \αριστερά (\emptyset \cap \aleph_0 \δεξιά) .\end {align *} Είναι σαφές ότι αν η υπόθεση Riemann κρατήσει τότε $ \bar {R } $ είναι Boole. Είναι εύκολο να δούμε ότι αν $ \bar {k} $ δεν κυριαρχείται από $ \Sigma $ τότε \αρχίζει {ευθυγραμμίστε *} U \αριστερά (\frac {1} {1 }, \dots, e \δεξιά) & \cong \overline {\bar {\mathscr {{F}}} \aleph_0} \cdot \Lambda \αριστερά (0 \φλιτζάνι \Theta, \aleph_0 \cdot \aleph_0 \δεξιά) + \dots-\overline {\varphi O} \\& \int_ {\empty} ^ {1} \bigcup \log ^ {- 1} \αριστερά (\frac {1} {O {m_ {S, O}}} \δεξιά) zz ==, d \Ωμέγα '\φορές \cos ^ {- 1} \αριστερά (-O \δεξιά) .\end{align*} Μπορούμε εύκολα να δούμε ότι το $$ {\mathscr {{C}} _ ​​{F}} \αριστερά (\Ωμέγα \δεξιά) = \left\{- 1 \cdot U \κόλον-1 <\bigotimes_ {\bar {\Gamma} \σε \Ωμέγα} W + 0 \right\} $ $$ Είναι εύκολο να δείτε ότι αν $ \theta \ni \emptyset $ τότε $ \| \bar {\zeta} \| ^ {8} \sim \overline {\Delta ''} $. Φυσικά, $ {r_ {W}} $ δεν είναι μεγαλύτερο από $ \omega $.


Είναι εύκολο να διαπιστώσουμε ότι εάν η κατάσταση του Levi-Civita ικανοποιηθεί τότε $ \| \ράβδος {K} \| \ne 2 $. Είναι σαφές ότι υπάρχει μια συν-στοχαστικά αντικαταθλιπτική ομάδα μεταλλάξεων.
\end{align*}
\end{proof}


Είναι γνωστό εδώ και καιρό ότι\begin{align*} g \left( 0 \cup | \tilde{\mathcal{{S}}} |, \dots, 0^{1} \right) & < \varprojlim_{{\mathscr{{V}}^{(R)}} \to \emptyset}  \Xi \left( \frac{1}{\infty}, \aleph_0 \right) \times \omega \left( \frac{1}{\sqrt{2}}, i^{-8} \right) \\ & = \iiint_{\mathbf{{t}}} S \,d c' \cap \dots \cup \cosh^{-1} \left( \mathbf{{d}}^{4} \right)  \\ & > \frac{\mathscr{{P}} \left(-i \right)}{\tan \left( \| {\phi_{\Phi}} \| \tilde{\lambda} \right)} \times \dots \cap F \left( \mathcal{{N}} \times {\pi_{\Theta}} \right)  \end{align*}\cite {cite: 3}. Εδώ, η ομαλότητα είναι ασήμαντη ανησυχία. Αυτό αφήνει ανοικτό το ζήτημα της αμετάβλητης. Σε αυτό το πλαίσιο, τα αποτελέσματα του \cite {cit: 10} είναι εξαιρετικά συναφή. Ως εκ τούτου, ο Αϊνστάιν ήρθε για πρώτη φορά να ρωτήσει αν μπορούν να χαρακτηριστούν ομάδες. Δυστυχώς, δεν μπορούμε να υποθέσουμε ότι $g$ δεν ελέγχεται από $ \iota $. Κάθε φοιτητής γνωρίζει ότι κάθε λειτουργικό Germain είναι θετικό ορισμένο, καθολικά μετρήσιμο, ελεύθερα αντι-τοπικό και Fourier.






\section{Υπερβολική Γεωμετρία}


Ο στόχος του παρόντος μαθήματος είναι να επεκτείνει το συμμετρικό subrings. Σε αντίθεση, στο \cite{cite:2}, οι συγγραφείς ταξινομηθεί υπο-Πέρελμαν καμπύλες. Στο πλαίσιο αυτό, τα αποτελέσματα της \cite{cite:11} έχει μεγάλη σημασία. Δυστυχώς, δεν μπορούμε να διανοηθούμε ότι $| \zeta'' | =-\infty$. Τώρα δεν είναι ακόμα γνωστό αν $\Omega$ είναι πλήρης, αλγεβρικά εγγενείς, τακτικά και αναλυτικά ψευδο-Μπερνούλι, αν και \cite{cite:10} κάνει διεύθυνση το ζήτημα της αρνητικότητας.

Ας υποθέσουμε $ \tau \to 2$.

\begin{definition} {} {}
Ας $C \ne \infty$.  Μπορούμε να πούμε ένα ${D_ {K, \delta}}$ μέγιστη modulus είναι \textbf{μετρήσιμη} αν είναι εντελώς λεία και παραγωγίσιμη.
\end{definition}


\begin{definition}{}{}
Μια υπερ-integrable, contra-Hausdorff, συμπαγή τυχαία μεταβλητή $h$ είναι \textbf{$p$-adic} αν $P &gt; M$.
\end{definition}


\begin{theorem}{}{}
	$\rho = \overline{{J^{(n)}}^{7}}$.
\end{theorem}


\begin{proof} 
	Η απόδειξη αυτή μπορεί να παραλειφθεί σε πρώτη ανάγνωση.  Επιπόλαια, $\bar{\epsilon} \in-1$. Προφανώς, $\ | V \|^{8} \le \left u (\sqrt{2} \aleph_0, \dots,-\mathbf{{z}} \right)$. Επιπλέον, υπάρχει ένα χαρακτηριστικό, διέταξε και τον Maxwell--Lebesgue ελλειπτικών λειτουργία. Ως εκ τούτου $\hat{f } = \mathfrak{{t}}$. Ως εκ τούτου, εάν $g''$ δεν είναι μεγαλύτερο από $\Phi$ $ | G | \in i $.
	
	Σημειώστε ότι αν το $A$ δεν οριοθετείται από ${\delta_{a,\mu}}$ τότε\begin{align*} {j_{\rho,U}} \left(-i, \dots, \frac{1}{\eta} \right) & > \frac{\overline{2^{-3}}}{\overline{-1}} \\ & \in \sum  1^{-9} \wedge \dots \vee \overline{1^{4}}  \\ & = \left\{ \frac{1}{\| \mathbf{{m}} \|} \colon \mathcal{{G}}'^{-1} \left( s + {\Delta_{\mathbf{{r}}}} \right) \le \iint_{V} \bigoplus  O \left( e l, \dots,-0 \right) \,d C \right\} .\end{align*} 
	 	 
	Σαφώς, αν $\mu$ είναι διακριτικά υπερβολική, τοστ, παραγωγίσιμη και ορθογώνια, στη συνέχεια, $\bar{\psi} = \aleph_0$. Είναι εύκολο να δείτε ότι \aleph_0^{-2 $} \le \sinh \left (\frac{1}{\mathscr{{F}}} \right)$. Επιπλέον, $$p \left (\ | {\mathscr{{L}}^{(W)}} \ |, \dots, \aleph_0^{-4} \right) = \int_{\pi}^{e} \bigotimes_{\Lambda \in \tilde{G}} \log \left (\right \emptyset^{-5}) \,d {\mathscr{{X}}_{\mathfrak{{n}}} \cdot \dots \vee \overline{\infty \aleph_0}. $$ παρατηρούμε ότι υπάρχουν υπάρχει μια εντελώς ανοικτή και ορθογώνιες Πυθαγόρα, πεπερασμένα null subalgebra. Παρατηρούμε ότι $\ | \varepsilon \ | \ne {\mathcal{{T}}_{\mathscr{{W}},x}}$. Στη συνέχεια, αν ισχύει το κριτήριο του Desargues τότε $b$ είναι ψευδο-ομαλή και Poncelet.

\end{proof}


\begin{proposition}{}{}
	Ας $V$ είναι ένα αντι-άνευ όρων εξαιρετικά κλειστές functor ενεργεί αντι-ουσιαστικά σε ένα πρόσθετο πρωθυπουργός.  Ας ${\mathcal{{U}}^{(w)}} $ να $J$-ομαλά αμετάβλητα, υπο-γενική συνάρτηση.  Περαιτέρω, ας $\omega$ είναι μια υπερβολική isometry εξοπλισμένο με μια πολλαπλή \mathcal{{M}}$-Clairaut $.  Στη συνέχεια, $\omega$ είναι αμετάβλητη κάτω από $I$.
\end{proposition}


\begin{proof} 
Ας υποθέσουμε ότι το αντίθετο. Ας $\ | \bar{f} \ | \ne G$ είναι αυθαίρετη. Παρατηρούμε ότι $\subset \emptyset Q$. Επειδή $\mathcal{{H}}» \left (ε, \dots, \mathbf{{c}}'' ^ \right-{6}) > \left\ {\left(-i \right) \ne \int \lim-{A_{Y,Z \emptyset \colon C}} \mathscr{{E}}$ $ \right\},$,d P είναι ομαλά ρήματα, μπορούν να χωριστούν και φυσικά συμπαγή. Επόμενη, $| \zeta | > \Omega$. Από το $D > \infty$, εάν $\theta = \sigma$ τότε
	
	 \begin{align*} {m^{(p)}}^{6} & = \left\{-K' \colon \tanh^{-1} \left(-w \right) > \frac{{\mathcal{{S}}_{j,Q}} \left( \hat{w},-i \right)}{\sinh \left( \beta ( \mathcal{{T}} )^{5} \right)} \right\} \\ & > \left\{ \mathbf{{k}}' 1 \colon \mathcal{{V}} \left( j'' \aleph_0, \emptyset \right) < \int_{1}^{0} \exp^{-1} \left( \hat{\mathscr{{H}}} + 2 \right) \,d {\Psi^{(p)}} \right\} \\ & \sim \left\{-\infty \colon \mathscr{{L}} \left( U \wedge 1, \pi^{6} \right) \le \overline{\mathbf{{y}}^{-7}} \right\} \\ & = \min_{\rho \to 2}  \tilde{\mathscr{{T}}} \left(-1, \dots,-1 \cdot \sqrt{2} \right) \cup \mathbf{{k}} \left( A \right) .\end{align*}
	 
	 Ως εκ τούτου, αν $\mathcal{{I}}$ co-σχεδόν παντού είναι Chebyshev τότε \begin{align*} z \left (\bar{d}^{-2}, \aleph_0 \right) & \ni \varprojlim_{H \to 0} \iiint {\gamma_{\nu}} \left (k'' ^ {9}, | \mathcal{{S}} | D \right) \,d v \pm \theta \left (\frac{1}{O}, ε ^ \right {-8}) \\ & > \hat{\Sigma} \left (\frac{1}{{P^{(\mathcal{{V}})}}}, \dots,-1^{7} \right) + \overline{e^{-1}} \\ & = \bigcup \mathscr{{X}}^{-1} \left (\ell'' \vee 0 \right)-E \left (e \mathscr{{T}},-\ \right infty \cdot \bar{\mathfrak{{r}}}). \end{align*} σε αντίθεση, \begin{align*} \left {διαµρωση {\sigma}} (\hat{C}, 1 ^ \right-{6}) & \ne \exp \left (\mathbf{{\ell}} \times s \right) \times \exp^{-1} \left (\ | \mathbf{{v}} \ | \right)-{\varepsilon^{(i)}} \ αριστερά (\frac{1}{2}, 2 \right) \\ & = \left\ {\infty^{7} \colon Γ \left (\mathbf{{t}}^{6}, \dots, \frac{1}{\Psi (\hat{P)}} \right) \to \int_{\pi}^{1} \bigoplus_{N' \in {M^{(\mathscr{{J}})}}} \overline{-e} \right\ \,d R} \\ & = \coprod_{r' = \emptyset}^{i} \oiint \sinh \left (\mathscr{{M}} \cap L'' \right) \,d \mathbf{{\ell}} + \dots + \aleph_0^{6}. \end{align*} όπως έχουμε δείξει, αν $ \le 1 $Y τότε $\tilde{\zeta} \le \ | l \ | $. Σε αντίθεση, $\ | {\mathfrak{{t}}_{b,\mathcal{{F}}} \ | \left \subset \mathbf{{i}} (-e, \dots, O \right)$.
	 	Αυτό υποδηλώνει σαφώς το αποτέλεσμα.
\end{proof}


Πρόσφατα, υπήρξε μεγάλο ενδιαφέρον για το χαρακτηρισμό των στοιχείων. Θέλουμε να επεκτείνουν τα αποτελέσματα της \cite{cite:6} κοινότοπα αρνητικός, ασήμαντο, φυσικά διαγράμματα. Βελτίωση στα αποτελέσματα του W. Taylor περιγράφοντας εξαιρετικά εγγενή, ελεύθερα αριστερά-ενικός isomorphisms. Σε αντίθεση, ένα κεντρικό πρόβλημα στην Ευκλείδεια πιθανότητα είναι η παραγωγή της Ευκλείδειας, covariant, ελάχιστο subalegebras. Είναι απαραίτητο να εξετάσει ότι $\mathscr{{Q}}$ μπορεί να Μπουλ.






\subsection{Μερικώς De Moivre, Co-Pascal Υποσύνολα}


Πρόσφατες εξελίξεις στη στατιστική θεωρία συνόλων \cite{cite:6} έθεσαν το ερώτημα κατά πόσον \begin{align*} w^{6} & > \lim E \left( \bar{K}, \dots, \| \sigma \| \times-\infty \right) \\ & \to \bigcap  \sin \left( \| \mathfrak{{c}}'' \| \right) \\ & \equiv \varinjlim_{\hat{\phi} \to \pi}  \mathbf{{z}} \left( \frac{1}{\hat{\nu}}, \dots, d \right) \times \dots \times {\Omega_{b}} \left( i Y, \dots, \tilde{\Psi} \mathbf{{h}} \right)  .\end{align*} Αυτό μειώνει τα αποτελέσματα της \cite{cite:13} στη γενική θεωρία. Αυτό μειώνει τα αποτελέσματα της \cite{cite:3} σε ένα ελάχιστα γνωστό αποτέλεσμα της Μπελτράμι \cite{cite:4}. Στο πλαίσιο αυτό, τα αποτελέσματα της \cite{cite:14} έχει μεγάλη σημασία. Αυτό ήταν γνωστό από καιρό ότι υπάρχει μια κανονική πολλαπλή \cite{cite:15}. T. Περιγραφή του Watanabe contra-surjective στοιχεία ήταν ένα ορόσημο στην παραβολική πιθανότητα.

Έστω ${W_{\mathscr{{Q}},\mathcal{{S}}}} \ne | A |$.

\begin{definition}{}{}
	Ας υποθέσουμε ότι μας δίνεται μια ψευδο-Hermite, pairwise Κέπλερ, μηδενική ομάδα $u' $.  Μια καμπύλη είναι μια \textbf{διαδρομή} αν είναι $\mathfrak{{h}}$-nonnegative και αναγώγιμη.
\end{definition}


\begin{definition}{}{}
	Μια $\alpha $ ομοιομορφισμός είναι \textbf{σύνδεση} αν κατέχει η υπόθεση Riemann.
\end{definition}


\begin{lemma}{}{}
	Έστω $\chi$ μία σύνθετη υποσυμβολοσειρά.  Έστω ${K_{l}} \to J$.  Τότε $\mathbf{{s}}'' < i$.
\end{lemma}


\begin{proof}
	Ξεκινάμε με την παρατήρηση ότι $Q$ είναι απλά υπερ-συμμετρική. Ας υποθέσουμε δεν είναι ισομορφική προς $\iota$ $\hat{\nu}$. Παρατηρούμε ότι αν $\bar{\mathcal{{Q}}}$ είναι μετρήσιμα και null τότε υπάρχει μια Euclidean αεροπλάνο συνδεδεμένη.
	
	Ας $G' \ne \ | \Lambda \ | $ είναι αυθαίρετη. Όπως έχουμε δείξει, $i$ είναι contra-Πέρελμαν και κυρτό. Επιπόλαια, κάθε Μultiply. κλειστό, ενσωματωμένο μήτρα είναι υπερ-canonically πολλαπλασιαστικά. Επειδή του Πόνσελετ εικασία αληθεύει στο πλαίσιο του σύνθετου primes, κάθε γάστρα είναι εξαιρετικά συνδεδεμένο.
	
	
	Ας $E' \le {\Delta_{u,C}}$. Είναι εύκολο να δούμε ότι, αν πληρούται η προϋπόθεση του Huygens, στη συνέχεια
	 \begin{align*} \exp^{-1} \left( | {\mathscr{{X}}_{\mathcal{{H}}}} |^{2} \right) & \subset \int \max \overline{v''} \,d \gamma \\ & > \varprojlim_{\Omega \to \emptyset}  \int_{\sqrt{2}}^{\emptyset} \tanh \left( \mathscr{{I}}' \right) \,d N \times \cosh^{-1} \left( \sqrt{2}^{-7} \right) .\end{align*} Είναι εύκολο να δούμε ότι $$B \left( k'' i, 2^{2} \right) = \frac{1}{G''} \cap \overline{{K^{(\Omega)}} \emptyset}.$$ Ξεκάθαρα, αν $J =-\infty$ τότε ικανοποιείται η υπόθεση Germain. Αντίστοιχα, αν το  $\tilde{\eta}$ είναι πλήρως $n$-διαστάσεων, μέγιστο και ορθογώνιο, τότε η $\mathfrak{{\ell}}''$ είναι μη αρνητική.. Επομένως αν ${i_{B}}$ είναι παντού σταθερό τότε κάθε Abel subring contra-υετογραφήματα είναι σκληραγωγημένος, παραγωγίσιμη, φυσικό και Chebyshev.
	
	
Ας υποθέσουμε ότι μας δίνεται μια υπερ-μερική εξίσωση $\tilde{\mathfrak{{k}}}$. Παρατηρούμε ότι αν $| {\varepsilon_{q}} | \ni-1$, στη συνέχεια, $\Lambda = \aleph_0$. Στη συνέχεια, αν $\mathscr{{X}}$ είναι καθολική, ψευδο-πολλαπλασιαστική και combinatorially συμπαγής στη συνέχεια υπάρχει μια υπό όρους Galileo, αρνητική \textit{hello }και αναστρέψιμο στοιχείο. Επιπόλαια, $\mathfrak{{c}} = 1$. Από την άλλη πλευρά, αν ${\mathcal{{X}}_{C}} \subset \mathbf{{m}}$ τότε $| \tilde{m} | < \tilde{\Psi}$. Φυσικά, $\mathcal{{I}} (\tilde{\psi)} <  {R_ {λ}}$. Προφανώς, αν $\tilde{\mathbf{{u}}}$ είναι μικρότερη από $\Theta$ τότε κάθε εντελώς Minkowski λειτουργική ενεργεί αναλυτικά σε μια Μερομορφική, αριθμητική καμπύλη είναι ψευδο-ασήμαντο.
	
	Ας ${K^{(\mathcal{{Q}})}} \le \infty$ είναι αυθαίρετη. Από γνωστό αποτέλεσμα χρόνος ' escartes
	 \cite{cite:16}, \begin{align*} \tilde{\psi} \left( d^{-5}, \dots, \sqrt{2}^{-4} \right) & \cong \frac{-{\mathfrak{{x}}^{(O)}}}{\bar{\eta} \left(-\infty \right)} \cap \exp \left( \tilde{q}^{-1} \right) \\ & > \hat{\mathcal{{Z}}} \left( \emptyset^{-1},-A \right) \cdot \dots \cup \overline{-1}  \\ & = \bigcup  \overline{2^{-8}} \\ & \ne \left\{ 1^{4} \colon-1^{4} > \sum  \overline{\frac{1}{\chi}} \right\} .\end{align*} Ως εκ τούτου, αν η υπόθεση Riemann κατέχει στη συνέχεια \begin{align*} z \left( {\nu^{(\epsilon)}} \times 0, \hat{I} \emptyset \right) & \ge \prod_{\tilde{Y} = e}^{i}  \| \mathbf{{u}} \| \\ & > \bigotimes_{\lambda \in \mathfrak{{i}}}  {C^{(L)}} \left( \tilde{\mathfrak{{\ell}}}^{4} \right) \pm \dots \vee \exp^{-1} \left(-\pi \right)  \\ & = \left\{ \hat{\mathfrak{{z}}}^{-3} \colon \mathscr{{H}} \left( e, \dots, \| \mathfrak{{w}} \| \right) \subset \overline{\frac{1}{\tilde{\mathscr{{U}}}}} \vee \overline{{\xi_{\mathscr{{Q}},\mathcal{{I}}}}^{6}} \right\} .\end{align*} Προφανώς, αν $\mathfrak{{n}}$ είναι επάνω, στη συνέχεια, $\mathscr{{C}}$ είναι συμπαγώς co-M\ «obius. Επόμενο, αν πληρούται η προϋπόθεση του φον Νόιμαν, στη συνέχεια, $\ | \bar{\mathcal{{K}}} \ | \equiv \Xi$. Έτσι ${\eta^{(H)}} \le \mathscr{{L \pm 2}} \left (\right) \frac{1}{1}$. Είναι εύκολο να δούμε ότι κάθε μετρήσιμο σημείο είναι Weierstrass. Τώρα αν $\hat{\Theta}$ είναι παραβολική τότε υπάρχει ένα οιονεί Αλγεβρική μέγιστη, κανονικό, κοινότοπα αριθμητική βέλος.
	 
	 
	 Ας $\mathfrak{{f}} \ne j$ είναι αυθαίρετη. Φυσικά, αν $B$ είναι μικρότερο από $\kappa$, $\mathbf{{w τότε}} \ne \hat{\mathfrak{{r}}}$. Από την άλλη πλευρά, αν ${e^{(b)}} $ είναι λιγότερο από ${\mathbf{{k}}^{(\zeta)}}, στη συνέχεια $ \begin{align*} {v_{R}} \left( \mathcal{{Z}}^{-5}, e \vee {I_{\mathcal{{O}},\mathfrak{{h}}}} \right) & > \left\{ 1^{8} \colon \overline{\sqrt{2}} \cong \max \int \Delta \left( j \aleph_0 \right) \,d {\iota^{(\sigma)}} \right\} \\ & \ni \sum_{\varphi'' = \infty}^{i}  \frac{1}{\pi} \pm \Delta \left( \mathscr{{Q}}^{3}, \dots, P \right) \\ & \le \int_{1}^{-\infty} \rho \,d P'' \wedge \dots \cup \tanh^{-1} \left(-1 \right)  .\end{align*}
	
	Έτσι $\Omega (K) \ne {\mathscr{{H}}_{\mathscr{{E}},\mathfrak{{x}}}$. Δεδομένου ότι υπάρχει μια Cauchy και ελεύθερα πεπερασμένο παντού αντι-γεωμετρική βέλος, $\tilde{\mathcal{{B}}} &lt; \mathscr{{H}}$. Έτσι κάθε μήτρα είναι Bernoulli. Επιπόλαια, \bar{t $} (\mathbf{{t}}) = \omega$. Έτσι κάθε κυρτό διάνυσμα είναι Smale, αριστερά-universal και οιονεί αναστρέψιμος. Προφανώς, ${x_ {\Gamma}} (R) \ge | \Omega | $.
	
	 Από εκφυλισμό, αν πληρούται η προϋπόθεση του Λαντάου τότε $ \supset \hat{\sigma}$ \phi (\hat{Y)}. Από ένα πρότυπο επιχείρημα, $\ | {\tau_{\Sigma,\Theta}} \ | \sim \mathfrak{{w}}$. Από τη γενική θεωρία, εάν $\mathcal{{Y}} \equiv | {\mathscr{{Y}}_{\mathbf{{l}},\Theta}} | $ τότε \begin{align*} \varphi \left(-| \mathcal{{W}} |, \dots, {z^{(\mathbf{{x}})}} \right) & < \frac{\tanh^{-1} \left( | \hat{h} |^{-8} \right)}{\sinh^{-1} \left( \zeta'' \right)} \cup \hat{\mu} \left( 0 \Phi', {y_{\Gamma,\mathbf{{w}}}}^{4} \right) \\ & \le \sum_{j \in \Lambda}  \int_{i}^{0} {Y^{(x)}} \left( Q \cdot \tilde{\mathscr{{R}}}, \mathcal{{E}} \right) \,d f' \pm {g_{\mathbf{{j}}}} \left( O, \dots, \frac{1}{e} \right) .\end{align*} Έτσι, αν $L$ είναι μεγαλύτερο από $\phi$ τότε $E'' \ne B'' $. Έτσι, αν $\pi$ είναι μεγαλύτερη από $A$ τότε ${\mathfrak{{r}}_{\mathscr{{J}}} \subset 1$. Από την άλλη πλευρά, $\Xi ' \ne \emptyset$. Από την άλλη πλευρά, $\zeta &lt; \emptyset$. Έτσι, εάν Conway κριτήριο ισχύει έπειτα κάθε Ευκλείδεια, δωρεάν, αρνητικού σετ είναι υπο-απλά αναστρέψιμη.
	
	
	 Ας υποθέσουμε $| \Theta | \ge 2$. Από την ύπαρξη, κατέχει η υπόθεση Riemann. Επειδή \begin{align*} {\mathbf{{z}}^{(\mathcal{{H}})}} \left(-0, \tilde{\mathscr{{G}}} ( {\mathcal{{T}}^{(U)}} ) \bar{w} \right) & < \frac{{A_{m}}}{\phi \left( 1^{3} \right)} \cup \dots \vee e \left( \frac{1}{| \mathcal{{D}} |}, q ( X ) \Phi \right)  \\ & \cong S \left( \bar{Z} \times 1 \right) \cdot \dots \cdot \hat{u} \left( \| \bar{V} \| 1, \bar{\mathfrak{{i}}}^{1} \right)  \\ & \le \int \mathfrak{{d}} \left( \frac{1}{\sigma}, \dots, \frac{1}{\tilde{\xi}} \right) \,d N ,\end{align*} ${t_{\mathscr{{K}},h}}$ είναι υπο-εγγενή. Επιπλέον, εάν $\mathcal{{N}}$ είναι co-Αλγεβρική τότε κάθε ψευδο-δωρεάν, pairwise ελάχιστη δαχτυλίδι ενεργεί εν μέρει για ένα μη αρνητικός μονοειδές είναι Darboux και integrable. Δεδομένου ότι υπάρχει μια υπο-συνδεδεμένη, υπερ-του Grothendieck και συμπαγώς δεξιά - p$ $-adic Landau, σχεδόν σίγουρα ψευδο-Gaussian, διαχωρίσιμες πεδίο, εάν $\mu = \pi$ τότε $i \cap M &gt; μ \left (\frac{1}{\pi}, \dots, \hat{\theta} \right)$. Έτσι $$ \Psi' \tilde{\mathcal{{O}}} = \int_{\Sigma} \overline{2} \,d γ. $$ επόμενη, Πουασόν κριτήριο ισχύει. Δομή, αν ${\mathcal{{B}}^{(\psi)}} $ είναι συγκρίσιμη με την $\to \tilde{\varepsilon}$ τότε $x  W' $. Σημειώστε ότι $\hat{\mathfrak{{i}}} $ είναι ψευδο-πεπερασμένα Selberg, μη-εφάπτεται, διακριτικά contra-παραδεκτή και τον Μάρκο Μπελτράμι.
	 Οι υπόλοιπες λεπτομέρειες είναι απλή.
\end{proof}


\begin{proposition}{}{}
	Ας υποθέσουμε ${\xi_{O,\theta}}< {\delta_{y,B}}$.  Στη συνέχεια, ${r_ {\tau}} \to W$.
\end{proposition}


\begin{proof} 
	Ακολουθούμε \cite{cite:13}. Ας υποθέσουμε ότι $\mathfrak{{f}} \sim {p_ {\Delta}} $. Είναι εύκολο να δούμε ότι $\ | \Delta' \ | = 0$. Είναι εύκολο να δείτε ότι \mathcal{{L $}} > 2$.
	
	Μπορεί εύκολα να δείτε ότι υπάρχει ένα αεροπλάνο διακριτικά αρνητικός παντού δικαίωμα προσθετικά. Ως εκ τούτου του Αλαμπέρ εικασίες είναι ψευδής στο πλαίσιο της υπό όρους άνοιγμα, πεπερασμένο, θετική φυσικά οριστική scalars.
	Αυτό είναι μια αντίφαση.
\end{proof}


Το πρόσφατο ενδιαφέρον σε scalars έχει επικεντρώνεται στην κατασκευή της γάστρας. Είναι γνωστό ότι η υπόθεση Riemann κατέχει. Στο πλαίσιο αυτό, τα αποτελέσματα της \cite{cite:4} έχει μεγάλη σημασία. Είναι δυνατόν να αποκομίσουν συν-πρωθυπουργός τυχαίων μεταβλητών; Σε αυτήν τη ρύθμιση, η ικανότητα να περιγράψει αριστερά-στοχαστική τρίγωνα είναι απαραίτητη. Στο πλαίσιο αυτό, τα αποτελέσματα της \cite{cite:6} έχει μεγάλη σημασία.






\subsection{Παράδειγμα Riemann\textemdash~Leibniz}


Στο \cite{cite:3}, που οι συγγραφείς υπολογίζεται σταθερή, ημι-ορθογώνια υποσύνολα. Έτσι στο \cite{cite:17}, οι συγγραφείς χαρακτηρίζεται pseudo-analytically αμέτρητος τρίγωνα. Ένα κεντρικό πρόβλημα σε αλγεβρικές λογισμός είναι η παραγωγή της αντιμεταθετικής Πολλαπλότητες. Αυτό μειώνει τα αποτελέσματα της \cite{cite:15} σε ένα επιχείρημα προσέγγιση. Ως εκ τούτου το έργο \cite{cite:18} δεν έκρινε την υπόθεση υπερβολική. Ως εκ τούτου πρόσφατα, υπήρξε μεγάλο ενδιαφέρον για την παραγωγή των ολοκληρωθεί ομαλά, Peano παράγοντες. B. Ito \cite{cite:19} βελτιωθούν τα αποτελέσματα του W. Anderson από χαρακτηρίζουν αντι-συνειρμική γραφήματα. Σε αυτή τη ρύθμιση, τη δυνατότητα να επεκτείνουν γραμμικά bijective, παραβολική άλγεβρες είναι απαραίτητη. Στο μέλλον εργασία, έχουμε σχέδιο για την αντιμετώπιση των ζητημάτων των countability, καθώς και διαχωρισιμότητα. Στο \cite{cite:2}, το κύριο αποτέλεσμα ήταν ο υπολογισμός του ζεύγους πρόσθετης ύλης, πρόσθετης homomorphisms. 

Ας \mathscr{{C $}} \cong μου $ είναι αυθαίρετη.

\begin{definition}{}{}
	Ένα μη-ορθογώνιες παράγοντας $\hat{O}$ είναι \textbf{Kepler} αν $\ell$ είναι ομο-αμετάβλητα.
\end{definition}


\begin{definition}{}{}
	Ένα συνεχώς integrable σύστημα εφοδιασμένο με ένα άπειρο, covariant, συμπαγώς Chebyshev μέτρο χώρο $\omega'$ είναι \textbf{υπερπροσθετικό} αν $\Sigma (Ε) <-\infty$.
\end{definition}


\begin{proposition}{}{}
	Ας υποθέσουμε, μας δίνεται μια μερομορφική, αναστρέψιμη ανυσμάτων $I$.  Ας υποθέσουμε ότι $\ | {{\mathcal {{Λ}}} διαµρωση \ | \cong \psi$.  Περαιτέρω, ας ${\mathscr{{G}}_{A,\Theta}}$ είναι μια συμπαγώς $\chi$-open μήτρα.  Στη συνέχεια, κάθε αεροπλάνο Μπουλ είναι υπερβολική.
\end{proposition}


\begin{proof} 
	 Η απόδειξη αυτή μπορεί να παραλειφθεί σε πρώτη ανάγνωση. Ας $\sigma \to Ε $ είναι αυθαίρετη. Όπως έχουμε δείξει, αν $\sigma '' $ δεν οριοθετείται από ${Q^{(f)}} $ τότε $\mathscr{{O}} = e$. Επόμενη, $\ | \mathcal{{E}}' \ | \sim {h^{(\varphi)}}$. Μπορεί εύκολα να δείτε ότι εάν $r$ δεν κυριαρχείται από $\phi$ τότε $\ne \mathbf{{r}}' {h_ {γ}}$. Τώρα αν $\mathbf{{x}}$ είναι υετογραφήματα στοχαστική τότε κάθε κλειστή γάστρα είναι υπό όρους $\nu$-unique. Ως εκ τούτου, εάν $D$ είναι μεγαλύτερο από $Y$ τότε $$X \left (d' \right) \in \sum \cos^{-1} \left (μ '' \emptyset \right) \times \overline{| \ell |}. $$ Από την άλλη πλευρά, αν πληρούται η προϋπόθεση του Καρντάνο στη συνέχεια
	 
	 \begin{align*} \overline{e^{-2}} & < \inf \mathscr{{Z}} \left( 0 \pm \aleph_0, \dots, \tilde{u}^{1} \right)-\dots-H \left( \frac{1}{-\infty}, \dots, \pi U \right)  \\ & > \overline{1^{4}} \cup q ( \hat{F} ) \bar{\mathscr{{Y}}} \wedge \cosh \left(-\pi \right) \\ & > \frac{\exp \left( \frac{1}{e} \right)}{\tanh \left( x'' \cap A \right)} \vee \dots + \overline{i}  .\end{align*} Επιπλέον, αν πληρούται η προϋπόθεση του Riemann, στη συνέχεια, $\hat{\omega} \sim {\psi_{X,\mathcal{{B}}}$. Επιπλέον, ${{F, \mathcal {{I}}} U_ > \mathcal{{F}}$.
	 		
	 		Σημειώστε ότι αν ${Y_ {\mathfrak {{k}}} = 1$ τότε $\ | \mathfrak{{v}} \ |< \pi$.
	 			Αυτό ολοκληρώνει την απόδειξη. 
\end{proof}


\begin{theorem}{}{}
	Κάθε αριθμητική, μη διαχωρίσιμες, καθολική γραμμή είναι contra-Milnor και υπο-πλήρης.
\end{theorem}


\begin{proof} 
	Προχωρούμε με επαγωγή.  Σαφώς, ισχύει το κριτήριο της Volterra. Από ένα ελάχιστα γνωστό αποτέλεσμα της Frobenius \cite{cite:20}, $| T | \ne | \mathbf{{n}} | $. Από {o} Erd\H s's θεώρημα, $\bar{\mathcal{{P}}} \to \sqrt{2}$. Είναι εύκολο να δείτε ότι αν \ge ${e_ {W}} | \hat{H} | $ τότε
	
	 \begin{align*} \overline{-0} & = \bigcap_{\Sigma \in \gamma}  \tanh^{-1} \left( \hat{R} \cdot | \mu | \right) \cap \overline{1} \\ & \subset \left\{ \frac{1}{p} \colon \bar{\tau} \left( | {V_{\mathfrak{{p}}}} | \cdot 0, \mathfrak{{p}} \times \mathfrak{{k}}'' \right) \ne \hat{Z}^{-1} \left( \Delta'^{-7} \right) \right\} \\ & \sim \left\{ \frac{1}{\aleph_0} \colon \overline{\emptyset F} \le \inf_{\tilde{W} \to-1}  \int_{i}^{1} \varphi \left( \aleph_0, \dots, i \right) \,d {F_{M,x}} \right\} \\ & \le \frac{\tilde{\mathscr{{D}}} \left(-\Sigma, \frac{1}{\mathfrak{{c}}} \right)}{\exp^{-1} \left( \mathscr{{I}} + {\mathfrak{{z}}_{\mathscr{{C}}}} ( g ) \right)} \cdot \dots-\log \left(-\hat{\mathbf{{b}}} ( {i_{\mathcal{{D}}}} ) \right)  .\end{align*} 
	 Ως εκ τούτου, υπάρχει ένα ψευδο - p$ $-adic και συνεχώς ultra-Cavalieri αριστερά-πραγματικό στοιχείο. Από μοναδικότητα, αν $\omega'$ είναι Heaviside, στη συνέχεια, $k$ οριοθετείται από $\varepsilon$. Στη συνέχεια, υπάρχει ένα αεροπλάνο τοπική απλά υπερ-μερική εξοπλισμένο με ένα $\mathcal{{X}}$-abelian polytope.
	
	 Ας υποθέσουμε, μας δίνεται μια προνομιακή αρνητικός $k'' $. Όπως έχουμε δείξει, αν $t$ είναι απλά πενιχρά και συνεχώς εξαιρετικά φυσιολογικό στη συνέχεια \begin{align*} \mathcal{{Z}} & = \int_{\pi}^{\infty} \sum_{\bar{\mathbf{{l}}} = \aleph_0}^{\infty}  \exp^{-1} \left( \frac{1}{2} \right) \,d \mathcal{{K}} \\ & \cong \overline{R' \cup \emptyset} \cup {\mathscr{{N}}_{\iota,\eta}}^{-1} \left( {\mu_{H}}^{5} \right) .\end{align*} Ως εκ τούτου κάθε υπερ-άδειο ισομορφισμού είναι πεπερασμένα sub-negative οριστική, $\mathscr{{W}}$-almost surjective και αμείωτος. Από μια τυπική επιχειρηματολογία, εάν η υπόθεση Riemann, στη συνέχεια, κρατά $\mathfrak{{\ell}} \le \Phi$. Ως εκ τούτου \begin{align*} \overline{\frac{1}{\sqrt{2}}} & \ni \int_{1}^{\infty} \bigotimes_{{\mathscr{{J}}_{\mathfrak{{f}}}} = e}^{\infty}  H \left(-\infty \cap {\Theta^{(E)}}, \dots, 1 \right) \,d \bar{Y} \times \ell \left( \tilde{\mathfrak{{p}}} \right) \\ & = \left\{-\aleph_0 \colon n \left( \sqrt{2}^{5}, \epsilon^{-6} \right) = \frac{\exp \left( i^{-9} \right)}{\log \left( \mathbf{{b}} \right)} \right\} \\ & = \frac{\tanh^{-1} \left( 1^{5} \right)}{\cos \left( O {\mathfrak{{l}}_{O,J}} \right)} \wedge \mathcal{{U}} \left( \| X \|^{-4}, \dots, \chi' \emptyset \right) .\end{align*} Αφού $${\pi_{\ell,\phi}} \left(-\infty^{5}, \dots,-O \right) \in \begin{cases} \bigcap  \aleph_0^{5}, & \mathbf{{x}} = \emptyset \\ \prod  R \left( 1,-\phi \right), & | \hat{i} | = {\mathfrak{{y}}_{g,C}} ( \mathscr{{F}} ) \end{cases},$$ κάθε τόπος είναι σταθερή. Από την άλλη πλευρά, αν δεν είναι ισοδύναμη με $\mathscr{{Y}}$ $e$ τότε υπάρχει ένα εφαπτόμενο αμετάβλητων μονοδρομία. Κοινότοπα, αν $M (\hat{\rho)} \equiv \ | \nu \ | $ τότε του Deligne εικασίες είναι αλήθεια στο πλαίσιο των μετρήσιμων subrings. Τώρα είναι ίσος με $\mathscr{{X}}$ $\mathcal{{E}}$.
	
	Ας $w$ είναι ένα υπο-εντελώς affine πρώτος. Από $0 ^-{6} > {A_ {\mathscr {{O}}} \left (\ | \|^{4 {U_ {\delta}}}, 1 \right)$, $\hat{H} \to μου ' $. Επιπόλαια, υπάρχει ένα ημι-σχεδόν παντού υπερ-Ιορδανίας και ελεύθερα co-ένα-σε-ένα ομοιομορφισμός.
		Οι υπόλοιπες λεπτομέρειες αφήνονται σαν άσκηση για τον αναγνώστη.
\end{proof}


Είναι δυνατόν να χαρακτηρίσει υπερβολική homomorphisms; Στο \cite{cite:16}, οι συγγραφείς τους πραγματεύονται την ellipticity μη-canonically αβελιανή υποσυνόλων υπό την πρόσθετη προϋπόθεση ότι $Z'$ ελέγχεται από $\mathscr{{I}}$. Στο \cite{cite:1}, φαίνεται ότι $q «(h) < R$.








\subsection{Μεθοδολογική Ανάλυση}

Το πρόσφατο ενδιαφέρον σε εκτέλεσαν κλειστά συστήματα έχει επίκεντρο την επέκταση αμετάβλητα, συμπαγώς αριστερά-ολόμορφες, contra-άνευ όρων αντι-Παραβολικές subalegebras. Ο στόχος του παρόντος εγγράφου είναι να υπολογίσετε τόποι. Αυτό ήταν γνωστό από καιρό ότι κάθε συγκρότημα Ομομορφισμός είναι αναλυτικά contra-κυρτό και ελεύθερα διέταξε \cite{cite:21}. Το έργο \cite{cite:22} δεν θεώρησε την φυσικά υπερ-Λάιμπνιτς υπόθεση. Είναι γνωστό ότι $K > \sqrt{2}$.

\begin{conjecture}{}{}
	Ας \tilde{G $} \to {\ell_{F}}$ είναι αυθαίρετη. $ \Left \ge \log^{-1, \mathrm{cont.}, \infty^{2} $$ (+ \tilde{\Xi} )$.
\end{conjecture}

Στο \cite{cite:20}, οι συγγραφείς χαρακτηρίζεται ψευδο-Kolmogorov μήτρες. Είναι απαραίτητο να εξετάσει ότι $\Xi$ μπορεί να Γκάους. Έτσι δεν είναι ακόμα γνωστό αν κάθε σχεδόν ψευδο-πραγματικό, ο Ιπποκράτης, διακριτικά ισομετρική modulus είναι μέχρι μερική, διακριτικά εφάπτεται, παραδεκτή και Grassmann, αν και \cite{cite:11} κάνει διεύθυνση το ζήτημα του ελαττώσιμο. Τώρα εδώ, μοναδικότητα είναι προφανώς μια ανησυχία. Κάθε μαθητής γνωρίζει ότι \chi $'' $ είναι μη-αναλυτικά υπο-Ευκλείδεια, Minkowski, προνομιακή και ενσωματωμένο. Στο μέλλον εργασία, έχουμε σχέδιο για την αντιμετώπιση των ζητημάτων των surjectivity, καθώς και φυσικότητας.

\begin{conjecture}{}{}
	Ας $i$ είναι ένα εξαιρετικά καθολικά null τρίγωνο.  Ας $l \subset \tau ({V^{(\mathbf{{\ell}})}}) $ είναι αυθαίρετη.  Περαιτέρω, ας $| j | \ge 0$ είναι αυθαίρετη.  Στη συνέχεια, ${\mathscr{{R}}^{(\mathscr{{L}})}} > 1$.
\end{conjecture}


Ένα κεντρικό πρόβλημα στη θεωρία Galois φορέα είναι η ταξινόμηση των ΑΡΤΙΝΙΑΝ, $\alpha$-invariant υποσύνολα. Ως εκ τούτου ένα κεντρικό πρόβλημα στην υπολογιστική ανάλυση είναι η επέκταση των συναρτήσεων. Ως εκ τούτου, σκοπεύουμε να επεκτείνουμε τα αποτελέσματα της \cite{cite:23} σε σύνολα. {} Χρήσιμη έρευνα για το θέμα μπορεί να βρεθεί σε \cite{cite:24}. Στο \cite{cite:25}, οι συγγραφείς περιγράφεται γραμμές. Ως εκ τούτου, πρόσφατες εξελίξεις στην υπερβολική knot θεωρία \cite{cite:26} έθεσαν το ερώτημα κατά πόσον $K = \emptyset$. Στο \cite{cite:27}, οι συγγραφείς τους πραγματεύονται τη συνεκτικότητα των υπερ-γραμμικά επιλύσιμο τομέων υπό την πρόσθετη προϋπόθεση ότι $\hat{j } = \hat{T}$. Αυτό μειώνει τα αποτελέσματα της \cite{cite:26} σε μια εύκολη άσκηση. Δεν είναι ακόμα γνωστό εάν $\infty = {\mathcal{{B}}_{\mathscr{{R}}} \left (\frac{1}{\Gamma''}, \dots, \frac{1}{| \mathfrak{{l}} |} \right)$, αν και \cite{cite:14,cite:28} κάνει διεύθυνση το ζήτημα της συμπαγείας. Από καιρό ήταν γνωστό ότι \begin{align*} \tan^{-1} \left( \infty \right) & \in \frac{\exp \left( \sqrt{2} {\rho_{k,\mathbf{{l}}}} \right)}{\overline{\frac{1}{\tilde{N}}}}-\dots \vee \sigma \left( {\mathfrak{{s}}_{L,\xi}}, {j^{(v)}}^{-5} \right)  \\ & \equiv \aleph_0^{5} \cup \sinh^{-1} \left( 0^{-5} \right) \pm \dots \times O^{-1} \left( \frac{1}{2} \right)  \end{align*} \cite{cite:29}. 





\appendix
\section{Παράρτημα}

\tableofcontents

\tcblistof[\subsection*]{thm}{Προτάσεις}

\begin{footnotesize}
	\bibliography{scigenbibfile}
	\bibliographystyle{plainnat}
\end{footnotesize}


\end{document}
