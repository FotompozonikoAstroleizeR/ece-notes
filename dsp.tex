% !TeX program = xelatex
\documentclass[11pt,a4paper,notitlepage,fleqn]{article}

\usepackage{amsmath}
\usepackage{amsfonts}
\usepackage{amssymb}
\usepackage{libs/commath2}
\usepackage[table]{xcolor}
\usepackage[hidelinks,draft=false]{hyperref}
\usepackage[skins,theorems]{tcolorbox}
\usepackage{titlesec}
\usepackage{tikz}
\usepackage{libs/circuitikz} % use our own recent version to make sure some bugs are fixed
\usepackage{pgfplots}
\usepackage{mathtools}
\usepackage[makeroom]{cancel}
\usepackage{mathrsfs}
\usepackage{wrapfig}
%\usepackage{subcaption}
%\usepackage{floatrow}
\usepackage{esint}
\usepackage{enumitem}
%\usepackage{bm}
\usepackage{relsize}
\usepackage{xfrac}
\usepackage{comment}
\usepackage{siunitx}
\usepackage{multicol}
%\usepackage{MnSymbol}
\usepackage[obeyDraft,disable]{todonotes}
%\usepackage{morefloats} % oh no!
%\usepackage[linesnumbered,lined]{algorithm2e}
\usepackage{glossaries}
\usepackage{xifthen}


\pgfplotsset{compat=1.13}
\usetikzlibrary{arrows.meta}
\usetikzlibrary{patterns}
\usetikzlibrary{decorations.pathmorphing}
\usetikzlibrary{decorations.markings}
\usetikzlibrary{backgrounds}
\usetikzlibrary{shapes.misc}
\usetikzlibrary{shapes.multipart}
\usetikzlibrary{shadows.blur}
\usetikzlibrary{fadings}
\usetikzlibrary{intersections}
\usetikzlibrary{arrows.meta}
\usetikzlibrary{calc}
\usetikzlibrary{matrix}
\usetikzlibrary{positioning}
\usetikzlibrary{shapes}
\usetikzlibrary{shadings}

\tcbuselibrary{breakable}
\tcbuselibrary{skins}
\tcbuselibrary{xparse}

\tikzset{cross/.style={cross out, draw,
        minimum size=2*(#1-\pgflinewidth),
        inner sep=0pt, outer sep=0pt}}
\tikzset{
    mark position/.style args={#1(#2)}{
        postaction={
            decorate,
            decoration={
            	post length=1mm, % ??? Magic to fix "Dimension
            	pre length=1mm, % ???  too large" errors.
                markings,
                mark=at position #1 with \coordinate (#2);
            }
        }
    }
}
\tikzset{
	arrow at/.style args={#1}{
		postaction={
			decorate,
			decoration={
				post length=1mm, % ??? Magic to fix "Dimension
				pre length=1mm, % ???  too large" errors.
				markings,
				mark=at position #1 with {\arrow{>}};
			}
		}
	}
}
\makeatletter
\tikzset{
  use path for main/.code={%
    \tikz@addmode{%
      \expandafter\pgfsyssoftpath@setcurrentpath\csname tikz@intersect@path@name@#1\endcsname
    }%
  },
  use path for actions/.code={%
    \expandafter\def\expandafter\tikz@preactions\expandafter{\tikz@preactions\expandafter\let\expandafter\tikz@actions@path\csname tikz@intersect@path@name@#1\endcsname}%
  },
  use path/.style={%
    use path for main=#1,
    use path for actions=#1,
  }
}
\makeatother

\pgfmathdeclarefunction{sinc}{1}{%
	\pgfmathparse{abs(#1)<0.01 ? int(1) : int(0)}%
	\ifnum\pgfmathresult>0 \pgfmathparse{1}\else\pgfmathparse{sin(#1 r)/#1}\fi%
}
\pgfmathdeclarefunction{gauss}{2}{%
	\pgfmathparse{1/(#2*sqrt(2*pi))*exp(-((x-#1)^2)/(2*#2^2))}%
}

\usepackage[left=2cm,right=2cm,top=2cm,bottom=2cm]{geometry}

%\usepackage[no-math]{fontspec}
%\usepackage{fontspec}
\usepackage{mathspec}
%\usepackage{newtxtext,newtxmath}
%\usepackage{unicode-math}
%\setmainfont{texgyretermes-regular.otf}
%\setsansfont{texgyreheros-regular.otf}
%\newfontfamily\greekfont[Script=Greek]{Linux Libertine O}
%\newfontfamily\greekfontsf[Script=Greek]{Linux Libertine O}
\usepackage{polyglossia}
%\newfontfamily\greekfont[Script=Greek]{texgyretermes-regular.otf}
\newfontfamily\greekfontsf[Script=Greek]{texgyreheros-regular.otf}
\newfontfamily\greekfonttt[Script=Greek]{Latin Modern Mono}
%\usepackage[greek]{babel}
\setdefaultlanguage{greek}
\setotherlanguage{english}

%\usepackage[utf8]{inputenc}
%\usepackage[greek]{babel}


%\usepackage{tkz-euclide} % loads  TikZ and tkz-base
%\usetkzobj{angles} % important you want to use angles

\newlist{enumparen}{enumerate}{1}
\setlist[enumparen]{label=(\arabic*)}
\newlist{enumpar}{enumerate}{1}
\setlist[enumpar]{label=\arabic*)}

\newlist{enumgreek}{enumerate}{1}
\setlist[enumgreek]{label=\alph*.}
\newlist{enumgreekparen}{enumerate}{1}
\setlist[enumgreekparen]{label=(\alph*)}
\newlist{enumgreekpar}{enumerate}{1}
\setlist[enumgreekpar]{label=\alph*)}


\newlist{enumroman}{enumerate}{1}
\setlist[enumroman]{label=(\roman*)}

\newlist{enumlatin}{enumerate}{1}
\setlist[enumlatin]{label=(\alph*)}

\newlist{invitemize}{itemize}{1}
\setlist[invitemize]{noitemsep,label=}

\input{libs/fiximplies}
\input{libs/sphere}

\makeatletter
\let\anw@true\anw@false

%\newcommand{\attnboxed}[1]{\textcolor{red}{\fbox{\normalcolor\m@th$\displaystyle#1$}}}
\makeatother
\tcbset{highlight math style={enhanced,colframe=red,colback=white,%
        arc=0pt,boxrule=1pt,shrink tight,boxsep=1.5mm,extrude by=0.5mm}}
\newcommand{\attnboxed}[1]{\tcbhighmath[colback=red!5!white,drop fuzzy shadow,arc=0mm]{#1}}
\newcommand{\infoboxed}[1]{%
	\tcbhighmath[colframe=blue!50!white,colback=blue!5!white,arc=0mm]{#1}}
\titleformat{\section}{\bf\Large}{Κεφάλαιο \thesection}{1em}{}
\newtcolorbox{attnbox}[1]{colback=red!5!white,%
    colframe=red!75!black,fonttitle=\bfseries,title=#1}
\newtcbox{quickattnbox}[1]{colback=red!5!white,%
	colframe=red!75!black,fonttitle=\bfseries,title=#1}
\newtcolorbox{infobox}[1]{colback=blue!5!white,%
    colframe=blue!75!black,fonttitle=\bfseries,title=#1}

\tcbset{frogbox/.style={enhanced jigsaw,%
		overlay first={\foreach \x in {0cm} {
				\begin{scope}[shift={([xshift=-0.2cm]title.west)}]
					\draw[very thick,green!65!black!50!white,latex-] (0,0) -- ++(-1.5,0);
\end{scope}}}}}
\tcbset{frogtitle/.style={
attach boxed title to top left=
{xshift=0mm,yshift=-0.50mm},
boxed title style={skin=enhancedfirst jigsaw,
	bottom=0mm,
	interior style={fill=none,
		left color=green!20!black,
		right color=gray}}
}}
\DeclareTColorBox{exercise}{ O{} }{
	enhanced jigsaw,
	breakable,parbox=false,
	%title style={left color=gray!50!white!50!green,opacity=.5,right color=white},
	subtitle style={%boxrule=1pt,
		colback=yellow!50!red!25!white,fontupper=\bfseries},
	coltitle=black,colbacktitle=green!90!black!25!white,colframe=black,
	frame hidden,
	boxrule=0mm,
	%boxrule=1mm,
	leftrule=0.8pt,toprule=0.8pt,rightrule=0pt, %reserve space
	borderline west={0.8pt}{0pt}{white!25!black},%---- draw line
	borderline north={0.8pt}{0pt}{white!25!black},%---- draw line
	interior hidden,
	%frame style={left color=black,right color=white},
	sharp corners=all,
	%frogbox, %TODO: frogbox
	before lower={\tcbsubtitle[before skip=\baselineskip]{Λύση}},lower separated=false,
	before title={\textbf{Άσκηση\ifthenelse{\isempty{#1}}{}{: }}},
	title={\ifthenelse{\isempty{#1}}{\hspace{0pt}}{#1}}%
}

\AtBeginDocument{%
\let\arg\relax
\let\Re\relax
\let\Im\relax
\DeclareMathOperator{\arg}{Arg}
\DeclareMathOperator{\Re}{Re}
\DeclareMathOperator{\Im}{Im}
}
\DeclareMathOperator{\sinc}{sinc}
\DeclareMathOperator{\sgn}{sgn}
\DeclareMathOperator{\erf}{erf}
\DeclareMathOperator{\cov}{cov}
\DeclareMathOperator{\atand}{atan2}

\newenvironment{absolutelynopagebreak}
{\par\nobreak\vfil\penalty0\vfilneg
	\vtop\bgroup}
{\par\xdef\tpd{\the\prevdepth}\egroup
	\prevdepth=\tpd}

\DeclareSIUnit \voltampere { VA } %apparent power 
\DeclareSIUnit \var { VAr } %volt-ampere reactive - idle power 
\DeclareSIUnit \decade { dec } %decade

% Global amount of samples
% Set to a higher value (e.g. 200) for nicer graphs
% Set to a low value (e.g. 10) for performance
% NOTE: Check the sample variables below for further measurements
\newcommand*{\gsamples}{200}

% Equals command as a workaround for CircuiTikZ bug
% not allowing the = sign in labels
\newcommand*{\equals}{=}

\newcommand{\nesearrow}{%
	\,%
	\smash{\raisebox{-1.1ex}
		{$%
			\stackrel{\displaystyle\nearrow}{\displaystyle\searrow}%
			$}}%
}
\newcommand{\degree}{^{\circ}} % not great
\newcommand\numberthis{\addtocounter{equation}{1}\tag{\theequation}} % add an equation number to a number-less math environment

% Provided commands
\providecommand\dif{d}
\providecommand\od[2]{\frac{#1}{#2}}

\newtcbtheorem[number within=section,list inside=thm]{theorem}{Θεώρημα}%
{colback=green!5,colframe=green!35!black,colbacktitle=green!35!black,fonttitle=\bfseries,enhanced,attach boxed title to top left={yshift=-2mm,xshift=-7mm},width=.9\textwidth,arc=.7mm}{th}
\newtcbtheorem[number within=section,list inside=defn]{defn}{Ορισμός}%
{colback=blue!5,colframe=cyan!35!black,colbacktitle=blue!35!black,fonttitle=\bfseries,enhanced,attach boxed title to top left={yshift=-2mm,xshift=-2mm}}{def}

% Locus plot utilities
\tikzset{locus/.style={orange!50!red!70!brown}}
\tikzset{locuspole/.style={draw=red!30!black,cross,inner sep=2.5pt,fill=white,fill opacity=.6,thick,label={[below]-90:#1}}}
\tikzset{locuszero/.style={draw=red!30!black,circle,inner sep=2pt,fill=white,fill opacity=.6,thick,label={[below]-90:#1}}}
\tikzset{locusbreak/.style={rounded corners=1.5pt,inner sep=2pt,draw,top color=brown,bottom color=black,fill opacity=.8,label={[below]-90:#1}}}

% New plotting utilities
\def\lowsamples{18}
\def\hisamples{40}
\def\timecolour{blue!50!cyan}

\tikzstyle{timecolour}=[\timecolour]



\title{ΨΕΣ
	\\
	{ 
		\normalsize Ψηφιακή Επεξεργασία Σήματος
		\\
		\normalsize Σημειώσεις από τις παραδόσεις\footnote{Όπως διδάσκονται στο τμήμα \textit{Ηλεκτρολόγων Μηχανικών και Μηχανικών Υπολογιστών} στο \textit{Αριστοτέλειο Πανεπιστήμιο Θεσσαλονίκης}.}
	}}
\date{Οκτώβριος 2018
	\\
	{ 
		\small Τελευταία ενημέρωση: \today
	}
}
\author{
	Για τον κώδικα σε \LaTeX, ενημερώσεις και προτάσεις:
	\\
	\url{https://github.com/kongr45gpen/ece-notes}}

\setallmainfonts(Digits,Latin,Greek){Asana Math}
\setmainfont{Noto Serif}
\setsansfont{Ubuntu}
\usepackage{polyglossia}
\newfontfamily\greekfont[Script=Greek,Scale=1.00]{Liberation Serif}

\hypersetup{pdftitle = {Ψηφιακή Επεξεργασία Σήματος}}

\let\mytodo\todo
\renewcommand{\todo}[1]{\par\mytodo[inline,noline]{#1}}


\begin{document}
\maketitle

\hrule
\vspace{50pt}

\begin{infobox}{Λάθη \& Διορθώσεις}
	Οι τελευταίες εκδόσεις των σημειώσεων βρίσκονται στο Github
	(\url{https://github.com/kongr45gpen/ece-notes/raw/master/dsp.pdf}) ή
	στη διεύθυνση \url{http://helit.org/ece-notes/dsp.pdf}.
	
	Περιέχουν διορθώσεις σε λάθη και τυχόν βελτιώσεις.
	
	\tcblower
	
	Μπορείτε να ενημερώνετε για οποιοδήποτε λάθος και πρόταση
	μέσω PM στο forum, issue στο Github, ή οποιουδήποτε άλλου τρόπου.
\end{infobox}

\section{Εισαγωγή}

\lecture{1}{1/10/2018}
	
Στο μάθημα της \textbf{Ψηφιακής Επεξεργασίας Σήματος} ασχολούμαστε
με προβλήματα όπως το εξής:

Παλιότερα, αν μας ζητούνταν να κατασκευάσουμε ένα φίλτρο (π.χ.
ζωνοπερατό από 20 Hz - 500 kHz), μπορούσαμε πολύ εύκολα να αγοράσουμε μια σακούλα με
όλα τα εξαρτήματα (πυκνωτές, αντιστάσεις, \textellipsis) από ένα
κατάστημα και να φτιάξουμε το κύκλωμα.

Αν αργότερα μας ζητούσαν να φτιάξουμε ένα φίλτρο 20 Hz - 300 kHz, θα
έπρεπε να ξαναπάμε στο κατάστημα και να αγοράσουμε ξανά νέα σακούλα και νέα εξαρτήματα
και να ξανακατασκευάσουμε το κύκλωμα.

Από τη στιγμή όμως που εφευρέθηκαν οι υπολογιστές, φανταστήκαμε να τους
χρησιμοποιήσουμε και για να πραγματοποιήσουμε την παραπάνω διαδικασία και
να αλλάζουμε προδιαγραφές όσο συχνά θέλουμε χωρίς να τρέχουμε στο κατάστημα.

Αυτή η διαδικασία απαιτεί τα εξής:
\begin{itemize}
	\item Πρέπει να μετατρέψουμε το \textit{πραγματικό} \textit{αναλογικό}
	σήμα σε μια μορφή που αναγνωρίζει αυτός ο υπολογιστής. Δηλαδή να μετατραπεί
	από αναλογικό σε ψηφιακό. Αυτό γίνεται με δύο βήματα:
	\begin{enumerate}
		\item \textbf{Δειγματοληψία}. Πρώτα πρέπει με έναν συγκεκριμένο τρόπο
		να λάβουμε \textit{δείγματα} σε \textit{διακριτές} στιγμές του αρχικού
		αναλογικού σήματος. Αφού ο υπολογιστής δεν μπορεί να αποθηκεύσει
		άπειρες τιμές, αναγκαστικά θα λάβουμε ένα πεπερασμένο εύρος τους.
		
		\begin{tikzpicture}
		\def\points{(0,0) (0.2,0.2) (0.4,0.5) (0.6,0.7)
			(0.8,0.9) (1,0.7) (1.2,0.6) (1.4,1) (1.6,0.8) (1.8,0.8) (2,1.1) (2.2,1.4)
			(2.4, 1.2) (2.6,1.1) (2.8,1) (3,0.95)
		}
		\def\pointsc{(0,0),(0.2,0.2),(0.4,0.5),(0.6,0.7),(0.8,0.9),(1,0.7),(1.2,0.6),(1.4,1),(1.6,0.8),(1.8,0.8),(2,1.1),(2.2,1.4),(2.4, 1.2),(2.6,1.1),(2.8,1),(3,0.95)
		}
		\def\pointsr{(0,0),(0.2,0.21),(0.4,0.42),(0.6,0.63),(0.8,0.84),(1,0.63),(1.2,0.63),(1.4,1.05),(1.6,0.84),(1.8,0.84),(2,1.05),(2.2,1.47),(2.4, 1.26),(2.6,1.05),(2.8,1.05),(3,0.84)
		}
		
		\draw[->] (0,0) -- (3,0) node[below] {$t$};
		\draw[->] (0,0) -- (0,2) node[left] (xt) {$x(t)$} node[right,align=left,scale=.7] {αναλογικό\\σήμα};
		\draw (xt.south) node[below,scale=.7,yshift=2mm] {$\in\mathbb R$};
		
		\draw[\timecolour,thick]
		plot [smooth] coordinates \points;
		
		\foreach \p in \pointsc
		\filldraw[red,top color=white,bottom color=blue,opacity=.4] \p circle(0.05);
		
		\draw[<->,thick,opacity=.9]
		(3.75,1) -- ++(1,0);
		
		\begin{scope}[xshift=5.5cm]
		\draw[->] (0,0) -- (3,0) node[below] {$n\in\mathbb N$};
		\draw[->] (0,0) -- (0,2) node[left] (xn) {$x(n)$} node[right,align=left,scale=.7] {διακριτό\\σήμα};
		
		\draw (xn.south) node[below,scale=.7,yshift=2mm] {$\in\mathbb R$};
		
		\foreach \p in \pointsc
		\filldraw[draw=black,top color=white,bottom color=blue,opacity=.4] \p circle(0.05);
		\end{scope}
		
		\end{tikzpicture}
		
		Αυτό το σήμα ονομάζεται \textbf{διακριτό} (\textbf{discrete}).
		\item \textbf{Κβάντωση}. Στον αναλογικό κόσμο, οι πεπερασμένες τιμές έχουν
		άπειρη ακρίβεια. Όμως στον υπολογιστή δεν μπορούμε να αποθηκεύσουμε άπειρα
		δεκαδικά ψηφία, αλλά πρέπει να τα αποθηκεύσουμε σε πεπερασμένες στάθμες
		που μας επιτρέπεται να χρησιμοποιηθούν.
		
		\begin{tikzpicture}
		\def\points{(0,0) (0.2,0.2) (0.4,0.5) (0.6,0.7)
			(0.8,0.9) (1,0.7) (1.2,0.6) (1.4,1) (1.6,0.8) (1.8,0.8) (2,1.1) (2.2,1.4)
			(2.4, 1.2) (2.6,1.1) (2.8,1) (3,0.95)
		}
		\def\pointsc{(0,0),(0.2,0.2),(0.4,0.5),(0.6,0.7),(0.8,0.9),(1,0.7),(1.2,0.6),(1.4,1),(1.6,0.8),(1.8,0.8),(2,1.1),(2.2,1.4),(2.4, 1.2),(2.6,1.1),(2.8,1),(3,0.95)
		}
		\def\pointsr{(0,0),(0.2,0.21),(0.4,0.42),(0.6,0.63),(0.8,0.84),(1,0.63),(1.2,0.63),(1.4,1.05),(1.6,0.84),(1.8,0.84),(2,1.05),(2.2,1.47),(2.4, 1.26),(2.6,1.05),(2.8,1.05),(3,0.84)
		}
		
		\draw[->] (0,0) -- (3,0) node[below] {$t$};
		\draw[->] (0,0) -- (0,2) node[left] (xt) {$x(t)$} node[right,align=left,scale=.7] {αναλογικό\\σήμα};
		\draw (xt.south) node[below,scale=.7,yshift=2mm] {$\in\mathbb R$};
		
		\draw[blue!50!cyan!80!brown,thick]
		plot [smooth] coordinates \points;
		
		\foreach \p in \pointsc
		\filldraw[red,top color=white,bottom color=blue,opacity=.4] \p circle(0.05);
		
		\draw[<->,thick,opacity=.9]
		(3.75,1) -- ++(1,0);
		
		\begin{scope}[xshift=5.5cm]
		\draw[->] (0,0) -- (3,0) node[below] {$n\in\mathbb N$};
		\draw[->] (0,0) -- (0,2) node[left] (xn) {$x(n)$} node[right,align=left,scale=.7] {διακριτό\\σήμα};
		
		\draw (xn.south) node[below,scale=.7,yshift=2mm] {$\in\mathbb R$};
		
		\foreach \y in {0,0.21,...,1.6}
		\draw[blue,opacity=.5,densely dashed] (0,\y) -- ++(3,0);
		
		\foreach \p in \pointsc
		\filldraw[draw=black,top color=white,bottom color=blue,opacity=.4] \p circle(0.05);
		\end{scope}
		
		\draw[->,thick,opacity=.9]
		(9.5,1) -- ++(1,0);
		
		\begin{scope}[xshift=11.25cm]
		\draw[->] (0,0) -- (3,0) node[below] {$n\in\mathbb N$};
		\draw[->] (0,0) -- (0,2) node[left] (xn) {$x(n)$} node[right,align=left,scale=.7] {ψηφιακό\\σήμα};
		
		\draw (xn.south) node[below,scale=.9,yshift=1mm] {$\in\mathbb Z$};
		
		\foreach \y in {0,0.21,...,1.6}
		\draw[blue,opacity=.5,densely dashed] (0,\y) -- ++(3,0);
		
		\foreach \p in \pointsr
		\filldraw[draw=black,top color=white!50!red,bottom color=blue,opacity=.4] \p circle(0.05);
		\end{scope}
		
		\end{tikzpicture}
		
		Αυτό το σήμα ονομάζεται \textbf{ψηφιακό} (\textbf{digital}).
	\end{enumerate}

	Σε αυτό το μάθημα, παρά τον τίτλο του, θα ασχοληθούμε με \textbf{διακριτά σήματα}, και
	όχι ψηφιακά.
	
	Για να επεξεργαστούμε τα σήματα, υποθέτουμε ότι θα υπάρχει μια συσκευή που
	μετατρέπει το \textit{Αναλογικό σήμα} σε \textit{Διακριτό σήμα} με τέτοιον τρόπο
	ώστε να μπορούμε να γυρίσουμε πίσω και να μπορούμε να το επεξεργαστούμε με έναν
	ασφαλή τρόπο που θα επιστρέψει σίγουρα σωστό αποτέλεσμα. Αυτό το εξασφαλίζει το
	\textbf{θεώρημα δειγματοληψίας} (Nyquist-Shannon), σύμφωνα με το οποίο για να
	αναπαραστήσουμε ένα ζωνοπερατό σήμα, αρκεί να το δειγματοληπτήσουμε με συχνότητα διπλάσια
	της μέγιστης συχνότητας του σήματος.
	
	Στην πραγματικότητα βέβαια, το θεώρημα δειγματοληψίας απαιτεί να δειγματοληπτούμε για άπειρο
	χρόνο, κάτι μη πραγματικά εφικτό. Μάλιστα, τα πραγματικά σήματα είναι \textit{χρονοπερατά}, άρα
	\textit{μη ζωνοπερατά}, επομένως εν γένει δεν εφαρμόζεται το θεώρημα δειγματοληψίας. Όμως η παραπάνω
	διαδικασία μπορούμε να πούμε ότι δίνει προσεγγιστικά ορθό αποτέλεσμα.
	
	Ακόμα, υπάρχουν σήματα που είναι από τη φύση τους ψηφιακά, όπως τιμές χρηματιστηρίου, ακολουθίες,
	δεδομένα, followers στο instagram, \textellipsis
	
	\item Παραμένει η ίδια η \textbf{επεξεργασία} του σήματος. Αυτή γίνεται με κώδικα ενός αλγορίθμου
	που εκτελεί πράξεις, δεδομένου ότι έχει επεξεργαστεί το σήμα σωστά.
\end{itemize}

Σχηματικά:

\begin{tikzpicture}

\pgfdeclarelayer{foreground}
\pgfsetlayers{main,foreground}

\draw[fill=yellow,draw opacity=.7,fill opacity=.15] (2.2,1.6) rectangle (5.2,-0.7) node[below,opacity=.7,circle,draw,inner sep=1pt,outer sep=2pt] {$A$};
\draw[fill=yellow,draw opacity=.7,fill opacity=.15] (-0.4,-2.3) rectangle (8.39,-3.7) node[below,opacity=.7,circle,draw,inner sep=1pt,outer sep=2pt] {$B$};

\draw (-2,0.5) node[scale=.9,align=center] {αναλογικός\\κόσμος};

\draw (0,0) --(1,0);
\draw (0,0) --(0,1);
\draw[\timecolour] plot[variable=\x,domain=0:1,samples=10,smooth]
(\x,{\x*((1+rand*0.3))});

\ctikzset{bipoles/length=.6cm}

\begin{pgfonlayer}{foreground}
\begin{scope}[xshift=3cm,yshift=1cm,local bounding box=scope1]
\draw (-0.1,0) node {};
\draw (0,0) to[L] ++(0.5,0);
\draw (0,-0.3) to[C] ++(0.5,0);
\draw (0,-0.7) to[R] ++(0.5,0);
\draw (0,-1.1) to[D] ++(0.5,0);

\draw (1.2,0) node {$H(\omega)$};
\draw (1.2,-0.4) node {$H(s)$};
\draw (1.2,-0.8) node {$h(t)$};
\draw (1.2,-1.2) node {???};
\end{scope}
\end{pgfonlayer}
\draw[fill=white] (scope1.north west) rectangle (scope1.south east);

\draw[<-] (scope1.west) -- ++(-1,0) node[left] {$x(t)$};
\draw[->] (scope1.east) -- ++(1,0) node[right] {$y(t)$};

\begin{scope}[xshift=6.5cm]
\draw (0,0) --(1.1,0);
\draw (0,0) --(0,1);
\draw[\timecolour] plot[variable=\x,domain=0:1,samples=10,smooth]
(\x,{(1-\x)*((0.9+rand*0.3))});
\end{scope}

\begin{scope}[yshift=-3cm]
\draw (-2,0) node[scale=.9,align=center] {ψηφιακός\\κόσμος};

\draw (1,0) node[rectangle,draw] (b1) {A$\to$D converter};
\draw (4,0) node[rectangle,scale=.8,draw,align=center] (b2)
{Algorithm\\Code\\ALU+Memory};
\draw (7,0) node[rectangle,draw] (b3) {D$\to$A converter};

\draw[<-] (b1.west) -- ++(-0.2,0) node[left,scale=.8] {$x(t)$};
\draw[->] (b1) -- (b2);
\draw[->] (b2) -- (b3);
\draw[->] (b3.east) -- ++(0.15,0) node[right,scale=.8] {$y(t)$};
\end{scope}
\end{tikzpicture}

Θέλουμε η διαδικασία \( A \) που είναι ένα \textbf{πραγματικό, φυσικό} φίλτρο ή κύκλωμα, να βγάζει
περίπου ίδιο αποτέλεσμα με μια διαδικασία \( B \) που υλοποιούμε ψηφιακά, δηλαδή:
\[
A \simeq B
\]
ή, ισοδύναμα για τα σήματα \( x(t) \) και \( y(t) \):
\begin{align*}
	A\left[x(t)\right] &\simeq B\left[x(t)\right] \\
	y_{\mathrm{analog}}(t) &\simeq y_{\mathrm{digital}}(t)
\end{align*}

\paragraph{}
Στο μάθημα θα ασχοληθούμε μόνο με \textbf{διακριτά}, όχι αυστηρά ψηφιακά σήματα. Παρακάτω στις σημειώσεις οι δύο όροι
συχνά θα χρησιμοποιούνται εναλλάξ, αλλά θα αναφέρονται πάντα στο απλώς \emph{διακριτό} σήμα.

\subsection{Ιδιότητες}
Όταν μιλάμε για διακριτό σήμα, μιλάμε ουσιαστικά για μία ακολουθία, όπως τη γνωρίζουμε από τα μαθηματικά.

\paragraph{Πράξεις}
Οι πράξεις σημάτων ορίζονται όπως και στις ακολουθίες. Ασχολούμαστε με τιμές ίδιων \textbf{δεικτών} (indices) \( n \):
\begin{enumerate}
	\item \textbf{Πρόσθεση}: \( x(n) + y(n) = z(n) \)
	\item \textbf{Πολλαπλασιασμός ακολουθιών}: \( x(n) \cdot y(n) = z(n) \)
	\item \textbf{Πολλαπλασιασμός αριθμού-ακολουθίας}: \( a\in\mathbb R,\ z(n) = a\cdot x(n) \)
\end{enumerate}

\paragraph{Άθροισμα γεωμετρικής προόδου}
Η γεωμετρική πρόοδος έχει σημαντική θέση στην ψηφιακή επεξεργασία σήματος (αφού στο αναλογικό σήμα
είχαμε το σημαντικό \( e^{j\omega t} \), εδώ είναι σημαντικό το \( e^{j\omega n} \) που εκφράζει συνεχή πολλαπλασιασμό με το \( e^{j\omega } \)):
\[
\sum_{n=0}^{\infty} a^n
= \begin{cases}
\displaystyle \frac{1}{1-a} &,\quad |a|<1\\
\text{αποκλίνει}&,\quad |a|>1
\end{cases}
\]
(για \( a=1 \) αποκλίνει, και για \( a=-1 \) ταλαντεύεται. Επίσης, η παραπάνω σχέση ισχύει και για
\( a\in\mathbb C \)).

\textbf{Προσοχή} ότι η παραπάνω σχέση ισχύει μόνο όταν το \( n \) ξεκινάει από το 0. Για παράδειγμα,
\( \sum_{n=1}^{\left(\frac{1}{2}\right)^n} = \frac{1}{1-\frac{1}{2}}-1 = 1 \).

Για \textbf{πεπερασμένο αριθμό όρων}:
\[
\sum_{n=0}^{N} a^n
= \frac{1-a^{n+1}}{1-a}
\]

\subsubsection{Χρήσιμοι τύποι ακολουθιών}
\paragraph{Περιοδική ακολουθία}
Η περιοδική ακολουθία περιέχει όρους που επαναλαμβάνονται, όπως και μία περιοδική συνάρτηση.
Μαθηματικά:
\[
\exists N \in \mathbb Z: \ \forall n \in \mathbb Z: \ x(n) = x(n+N)
\]
δηλαδή η περιοδική μας ακολουθία έχει μια \textbf{περίοδο \( \mathbb N \)} που είναι ακέραιος αριθμός.

\textbf{Προσοχή!} Δεδομένου ότι η \( \cos(\omega t) \) είναι \textit{περιοδική}, θα μπορούσε κάποιος
να φαντασεί ότι και η \( \cos(\omega n) \) είναι \textit{περιοδικό διακριτό σήμα}. Αν το δούμε μαθηματικά:
Έστω \( \exists N \in \mathbb Z:\ \cos(\omega n)=\cos\left( \omega (n+N) \right)
\implies \omega n = \omega n + \omega N + kπ \implies N = \frac{π}{\omega } \notin \mathbb Z \).

\begin{tikzpicture}[scale=.8]
\def\freq{0.5}

\draw (0,0) node[left,align=right,scale=.8] {περιοδικό\\ψηφιακό\\σήμα};

\draw (0,0) -- (10,0);
\draw (0,-1.5) -- (0,1.5);

\begin{scope}[green!50!black,opacity=.6]
\draw (0.5,1) -- (0.5,-1.5);
\draw (1,0) -- (1,-1.5);
\draw[<->] ([xshift=-1mm]1,-1.3) -- ([xshift=1mm]0.5,-1.3) node[below,yshift=-3mm,scale=.6,midway] {$f_s=2\;\mathrm{Hz}$};
\end{scope}

\draw[thick,timecolour]
plot[domain=0:9,samples=\lowsamples,smooth] (\x,{sin(\freq*2*pi*\x r)});

\foreach \x in {0.5,1,...,9} {
	\def\sample{sin(\freq*2*pi*\x r)}
	\draw[dashed,orange] (\x,0) -- (\x,{\sample});
	\filldraw[fill opacity=.8,fill=orange!50!white] (\x,{\sample}) circle (2.5pt);
}

\begin{scope}[yshift=-4cm]
\draw (0,0) node[left,align=right,scale=.8] {\emph{μη} περιοδικό\\ψηφιακό\\σήμα};

\draw (0,0) -- (10,0);
\draw (0,-1.5) -- (0,1.5);

\begin{scope}[green!50!black,opacity=.6]
\draw (0.55,{sin(\freq*2*pi*0.55 r)}) -- (0.55,-1.5);
\draw (1.1,{sin(\freq*2*pi*1.1 r)}) -- (1.1,-1.5);
\draw[<->] ([xshift=-1mm]1.1,-1.3) -- ([xshift=1mm]0.55,-1.3) node[below,yshift=-3mm,scale=.6,midway] {$f_s=1.28\sqrt{2}\;\mathrm{Hz}$};
\end{scope}

\draw[thick,timecolour]
plot[domain=0:9,samples=\lowsamples,smooth] (\x,{sin(\freq*2*pi*\x r)});

\foreach \x in {0.55,1.1,...,9} {
	\def\sample{sin(\freq*2*pi*\x r)}
	\draw[dashed,red] (\x,0) -- (\x,{\sample});
	\filldraw[fill opacity=.8,fill=red!50!white] (\x,{\sample}) circle (2.5pt);
}
\end{scope}
\end{tikzpicture}

Πρακτικά, δειγματοληπτούμε σε διαφορετικά σημεία, άσχετα ίσως από την περίοδο του σήματος. Γενικότερα,
όταν δειγματοληπτούμε περιοδικά αναλογικά σήματα, δεν θα παίρνουμε πάντα περιοδικά διακριτά πίσω.

Μάλιστα, κάτι άλλο περίεργο όταν δειγματοληπτούμε είναι πως για διαφορετικά αναλογικά σήματα, μπορεί
να πάρουμε το ίδιο ψηφιακό!

\begin{tikzpicture}
\draw (0,1.5) -- (0,-1.5);
\draw[->] (0,0) -- (5,0) node[below] {$t$};

\draw[\timecolour!40!black,very thick]
plot[samples=\lowsamples,smooth,domain=0:4,variable=\x]
(\x,{1.1*sin((\x*1.25*pi) r)});
\draw[\timecolour,very thick]
(0,0) -- (1/4*1.6,1) -- (3/4*1.6,-1) -- (5/4*1.6,1) -- (7/4*1.6,-1)
-- (9/4*1.6,1);
\end{tikzpicture}

Στο παραπάνω σχήμα, δειγματοληπτώντας στις κορυφές και στα μηδενικά, θα πάρουμε το ίδιο πράγμα από
τα δύο σήματα.

\paragraph{Άρτιες \& Περιττές ακολουθίες}
\begin{align*}
	\text{άρτια (even)} &\quad \forall n \in \mathbb Z \ x_e(n) = x_e(-n) \\
	\text{περιττή (odd)} &\quad \forall n \in \mathbb Z \ -x_o(n) = x_o(-n)
\end{align*}

Μπορούμε να μετατρέψουμε οποιαδήποτε ακολουθία σε ένα άρτιο και ένα περιττό μέρος:
\begin{align*}
	x_e(n) &= \frac{x(n) + x(-n)}{2}\\
	x_o(n) &= \frac{x(n) - x(-n)}{2}
\end{align*}

\subsubsection{Χαρακτηριστικά Μεγέθη}
\begin{enumerate}
	\item \textbf{Μέση τιμή}:
	\( \displaystyle
	\overline{x(n)} = \frac{\displaystyle \sum_{n=0}^{N} x(n)}{N+1}
	 \)
	\item \textbf{Ενεργός τιμή}:
	\( 
	\displaystyle
	\overline{\overline{x(n)}} = \left[
	\frac{\displaystyle \sum_{n=0}^{N} x^2(n)}{N+1}
	\right]^{\sfrac{1}{2} }
	 \)
	\item \textbf{Στιγμιαία ισχύς}:
	\( 
	\displaystyle
	p(n) = x^2(n)
	 \)
	\item \textbf{Μέση Ισχύς}:
	\( 
	\displaystyle
	p = \overline{p(n)} = \frac{\displaystyle \sum_{n=0}^{N} x^2(n)}{N+1}
	 \)
	\item \textbf{Ενέργεια}:
	\( 
	\displaystyle
	W = \sum_{n=0}^{N} x^2(n) = (N+1)p
	 \)
\end{enumerate}

\pagebreak[3]

\subsubsection{Χρήσιμες ακολουθίες}
\begin{enumpar}
	\item \textbf{Εκθετική ακολουθία}: \todo{Box this in a cool way}
	\begin{minipage}{.5\textwidth}
		\begin{knowledgebox}{Εκθετική ακολουθία}
		\[ \displaystyle
		x(n) = Ae^{sn} = Aa^{(\sigma + j \omega )n}
		\]
		\end{knowledgebox}

		για την οποία λαμβάνουμε τις εξής περιπτώσεις για τις σταθερές:
		\begin{itemize}
			\item \( a=e \) και \( s = \sigma < 0 \):
			\[
			x(n) = Ae^{-|σ|n}
			\]
			(γεωμετρική πρόοδος με λόγο \( e^{-|σ|} \))
			\item \( a=e \) και \( s = \pm j\omega  \):
			\[
			x(n) = A\left[
			cos(\omega n) \pm j \sin(\omega n)
			\right]
			\]
			
			Είναι περιοδική \textit{μόνο} εάν \( \frac{\pi}{\omega } \in \mathbb Q \)
		\end{itemize}
	\end{minipage}
	\begin{minipage}{.5\textwidth}
		\begin{tikzpicture}
		\tikzstyle{sample}=[bottom color=orange,top color=\timecolour,fill opacity=.7,rounded corners=2.5pt,thick]
		
		\draw[->] (-3,0) -- (3.5,0);
		\draw[->] (0,0) -- (0,2)  node[right] {$x(n)$};
		
		\def\sf{0.7}
		\def\sx{0.11}
		\def\sy{0.09}
		
		\foreach \x in {-4,-3,...,4} {
			\draw (\x*\sf-0.1,0.1) -- (\x*\sf+0.1,-0.1);
			\draw (\x*\sf,-0.1) node[below,scale=.9] {$\x$};
		}
		
		\foreach \x in {-4,-3,...,4} {
			\filldraw[sample] (\x*\sf + \sx,{exp(\x/3)/2 + \sy}) rectangle ++(-\sx*2,-\sy*2);
		}
		\end{tikzpicture}
	\end{minipage}
    \item \textbf{Δέλτα του Kronecker}:

    \begin{minipage}{.5\textwidth}
    	\begin{knowledgebox}{Δέλτα του Kronecker}
    		\[ \displaystyle
    		\delta(n) = \begin{cases}
    		1 &\quad n=0\\
    		0 &\quad n\neq 0
    		\end{cases}
    		\]
    	\end{knowledgebox}
    \end{minipage}
	\begin{minipage}{.5\textwidth}
		\begin{tikzpicture}
		\tikzstyle{sample}=[bottom color=orange,top color=\timecolour,fill opacity=.7,rounded corners=2.5pt,thick]
		
		\draw[->] (-3,0) -- (3.5,0);
		\draw[->] (0,0) -- (0,2)  node[right] {$\delta(n)$};
		
		\def\sf{0.7}
		\def\sx{0.11}
		\def\sy{0.09}
		
		\foreach \x in {-4,-3,...,4} {
			\draw (\x*\sf-0.1,0.1) -- (\x*\sf+0.1,-0.1);
			\draw (\x*\sf,-0.1) node[below,scale=.9] {$\x$};
		}
		
		\foreach \x in {1,2,...,4} {
			\filldraw[sample] (\x*\sf + \sx,0 + \sy) rectangle ++(-\sx*2,-\sy*2);
			\filldraw[sample] (-\x*\sf + \sx,0 + \sy) rectangle ++(-\sx*2,-\sy*2);
		}
		\filldraw[sample] ( + \sx,1 + \sy) rectangle ++(-\sx*2,-\sy*2);
		\draw (0+0.15,1) node[right] {$1$};
		
		\end{tikzpicture}
	\end{minipage}
    \item \textbf{Βηματική ακολουθία step}:
    
    \begin{minipage}{.5\textwidth}
    \begin{knowledgebox}{Βηματική ακολουθία}
    	\[
    	u(n) = \begin{cases}
    	1 &\quad n \geq 0\\
    	0 &\quad n < 0
    	\end{cases}
    	\]
    \end{knowledgebox}
    
	\end{minipage}
	\begin{minipage}{.5\textwidth}
		\begin{tikzpicture}
		\tikzstyle{sample}=[bottom color=orange,top color=\timecolour,fill opacity=.7,rounded corners=2.5pt,thick]
		
		\draw[->] (-3,0) -- (3.5,0);
		\draw[->] (0,0) -- (0,2)  node[right] {$\mathrm u(n)$};
		
		\def\sf{0.7}
		\def\sx{0.11}
		\def\sy{0.09}
		
		\foreach \x in {-4,-3,...,4} {
			\draw (\x*\sf-0.1,0.1) -- (\x*\sf+0.1,-0.1);
			\draw (\x*\sf,-0.1) node[below,scale=.9] {$\x$};
		}
		
		\foreach \x in {1,2,...,4} {
			\filldraw[sample] (\x*\sf + \sx,1 + \sy) rectangle ++(-\sx*2,-\sy*2);
			\filldraw[sample] (-\x*\sf + \sx,0 + \sy) rectangle ++(-\sx*2,-\sy*2);
		}
		\filldraw[sample] ( + \sx,1 + \sy) rectangle ++(-\sx*2,-\sy*2);
		\draw (0-\sx,1) node[left] {$1$};
		
		\end{tikzpicture}
	\end{minipage}
    
    Μάλιστα, ισχύει ότι:
    \[
    u(n) = \sum_{m=-\infty}^{n} δ(m)
    \]
    κάτι που μας θυμίζει αντίστοιχα από το αναλογικό σήμα ότι \( u(t) = \int_{-\infty}^{t} δ(τ)\dif τ \)
\end{enumpar}

\subsubsection{Συνέλιξη}
Μπορούμε να μετατρέψουμε τη \textbf{συνέλιξη} του αναλογικού σήματος στο ψηφιακό.

Στο αναλογικό, θυμόμαστε ότι, σύμφωνα με τον ορισμό της συνέλιξης:
\[
x(t) = x(t) * δ(t) = \int_{-\infty}^{\infty} x(τ)δ(t-τ) \dif τ
\]
και στο ψηφιακό, μπορούμε να έχουμε κάτι αντίστοιχο: 
\begin{equation}
\label{eq:deltaconv}
x(n) = \sum_{m=-\infty}^{\infty} x(m) δ(n-m)
\end{equation}

\begin{defn}{Συνέλιξη}{}
	Η \textbf{συνέλιξη} δύο διακριτών σημάτων ορίζεται ως εξής:
	\[
	z(n) = x(n) * y(n) = \sum_{k=-\infty}^\infty x(k)y(n-k)
	\]
\end{defn}

\paragraph{Απόδειξη σχέσης \eqref{eq:deltaconv}}
Έχουμε:
\begin{align*}
	\sum_{m=-\infty}^{\infty} x(m)\delta(n-m) &=
	\sum_{m=-\infty}^{n-1} x(m) \cdot 0 + x(n) \cdot \delta(n-n) + \sum_{m=n+1}^{-\infty} x(m) \cdot 0 = x(m)
\end{align*}

\subsection{Συστήματα}
Στον αναλογικό κόσμο, ένα σύστημα ήταν ένα κουτί που έπαιρνε σήματα εισόδου, τα επεξεργαζόταν, και έβγαζε σήματα εξόδου. Μαθηματικά, είναι μια απεικόνιση συναρτήσεων \( x(t) \) εισόδου σε συναρτήσεις εξόδους.

Επομένως, μπορούμε να ορίσουμε το \textbf{ψηφιακό σύστημα} ως ένα σύστημα που απεικονίζει ακολουθίες
σε ακολουθίες.

 \begin{tikzpicture}[scale=.8]
\draw (0,0) node[rectangle,inner sep=18pt,draw] (s) {Σύστημα};
\draw[->] (s.east) -- ++(1,0) node[right] {$y(n)$};
\draw[<-] (s.west) -- ++(-1,0) node[left] {$x(n)$};

\draw[->,gray!50!brown!50!black] (s) ++(0.2,-2) node[below] {Νόμος $T$} to[bend left] (s);
\end{tikzpicture}

Αντίστοιχα, ένα \textbf{υβριδικό σύστημα} απεικονίζει συναρτήσεις σε ακολουθίες. Δηλαδή έχει είσοδο αναλογικό σήμα, και έξοδο ψηφιακό.

\paragraph{Γραμμικό Σύστημα}
Μπορούμε σε αυτό το σημείο να δώσουμε τον ορισμό του \textbf{γραμμικού συστήματος} που συναντάμε συνέχεια,
για ένα ψηφιακό σύστημα \( T \). Έστω οι έξοδοι \( y_1(n) = T\left[x_1(n)\right] \),
\( y_2(n) = T\left[x_2(n)\right] \). Το σύστημα είναι γραμμικό ανν:
\begin{gather*}
\forall x_1,x_2\quad,\quad \forall a_1,a_2\in\mathbb C:\\
a_1T\left[x_1(n)\right]+a_2T\left[x_2(n)\right] = T\left[a_1x_1(n)+a_2x_2(n)\right]
\end{gather*}

Παραδείγματα:
\begin{itemize}
	\item Το \( y(n) = ax(n) + b \) \textit{δεν} είναι γραμμικό, λόγω του \( b \).
	\item Το \( y(n) = nx(n) \) \textit{είναι} γραμμικό.
\end{itemize}

\paragraph{Αμετάβλητο Κατά τη Μετατόπιση Σύστημα (ΑΚΜ)}
\begin{align*}
	y(n) &= T\left[x(n)\right] \\
	y(n-n_0) &= T\left[x(n-n_0)\right]
\end{align*}
δηλαδή, αν το ενοχλήσουμε τη στιγμή 2 ή τη στιγμή 50, θα δώσει την ίδια έξοδο, ξεκινώντας αντίστοιχα
από τη στιγμή 2 ή τη στιγμή 50.

Παραδείγματα:
\begin{itemize}
	\item Το \( y(n) = ax(n) + b \) \textit{είναι} είναι αμετάβλητο κατά τη μετατόπιση.
	\item Το \( y(n) = nx(n) \) \textit{δεν} είναι αμετάβλητο κατά τη μετατόπιση, λόγω του όρου \( n \).
\end{itemize}

Το σύστημα που προκύπτει από μία διαφορική εξίσωση είναι αμετάβλητο κατά τη μετατόπιση όταν οι συντελεστές των παραγώγων του δεν εξαρτώνται από το χρόνο.

\paragraph{Λίγη προσοχή}
Χρειάζεται κάποια προσοχή στο χειρισμό ΑΚΜ και γραμμικών συστημάτων.

Έστω τα συστήματα που εκφράζουν το νόμο του \emph{Ohm} (\( V=IR \)) σε μια \emph{αντίσταση}:
\begin{align*}
	y_1(t) &= R(t) \cdot x(t)\\
	y_2(t) &= R(x) \cdot x(t)
\end{align*}

Στο πρώτο σύστημα η αντίσταση εξαρτάται από το χρόνο (π.χ. διάβρωση), και στο δεύτερο εξαρτάται από την είσοδο (π.χ. αύξηση θερμοκρασίας \( \implies \) αλλαγή αντίστασης για μεγαλύτερα ρεύματα).

Το πρώτο σύστημα είναι γραμμικό αλλά \emph{όχι} ΑΚΜ, αφού η \( R(t) \) εξαρτάται από το χρόνο.
Το δεύτερο σύστημα είναι ΑΚΜ αλλά \emph{όχι} γραμμικό, αφού η \( R(x) \) εξαρτάται από την είσοδο.

Είναι \textbf{λάθος} να πούμε πως έστω \( R(t)=x(t) \implies y_1(t)=x^2(t)\ \text{(μη γραμμικό)} \), καθώς
η \( R(t) \) είναι μια παράμετρος του συστήματος που δεν μπορεί να είναι ίση με τις διαφορετικές πιθανές
εισόδους του. Παρομοίως, είναι \emph{λάθος} να θεωρήσουμε ότι \( R(x) = R(x(t)) = R(t) \implies y_2(t) = R(t)x(t) \ \text{(μη ΑΚΜ)} \) (δηλαδή ότι αφού η \( R \) εξαρτάται από το \( x \) και το \( x \) εξαρτάται από το χρόνο, άρα η \( R \) εξαρτάται μόνο από το χρόνο).

\lecture{2}{5/10/2018}

\paragraph{}
Πιο αυστηρά, ένα σύστημα \( y(n) = T\left[x(n)\right] \) είναι μία \textbf{απεικόνιση}
από το σύνολο όλων των ακολουθιών \( x(n) \in \mathbb R \) (ή \( \mathbb C \))
στο σύνολο όλων των ακολουθιών \( y(n) \in \mathbb R  \) (ή \( \mathbb{C} \))

\subsubsection{Συνέλιξη}
Θυμόμαστε ότι ένα \emph{ψηφιακό} σήμα είναι ίσο με την \emph{ψηφιακή} συνέλιξή του
με την \( δ(n) \) \eqref{eq:deltaconv}:
\begin{align*}
	x(n) &= \sum_{k=-\infty}^{\infty} x(k)δ(n-k)
	\intertext{Άρα, εφαρμόζοντας το σύστημα στη παραπάνω σχέση:}
	y(n) &= T\left[x(n)\right] = T\left[
	x(k)δ(n-k)
	\right]
	\intertext{Και, αν το $T$ είναι \textbf{γραμμικό}:}
	y(n) &= \sum_{k=-\infty}^{\infty} T\left[x(k)δ(n-k)\right]
	\\ &= \sum_{k=-\infty}^{\infty} x(k) \cdot T\left[δ(n-k)\right]
\end{align*}
Δηλαδή η \textbf{έξοδος} του συστήματος σε κάποια είσοδο \( x(n) \) είναι προκύπτει από τη συνέλιξη
της εισόδου με την \textbf{κρουστική απόκριση} του συστήματος (απόκριση στη \( δ(n) \)), την οποία ορίζουμε:
\begin{knowledgebox}{Κρουστική απόκριση}
	Έστω ότι δίνουμε σε ένα σύστημα είσοδο το δέλτα του Kronecker \( δ(n) \).
	
	Τότε η έξοδός του \( T\left[δ(n)\right] \) είναι η κρουστική απόκριση, την οποία ονομάζουμε
	\( h(n) \):
	\begin{align*}
		h(n) &= T\left[\delta(n)\right]
		\intertext{Μάλιστα, αν θεωρήσουμε ότι το σύστημα είναι ΑΚΜ, ισχύει ακόμα:}
		h(n-k) &= T\left[δ(n-k)\right]
	\end{align*}
\end{knowledgebox}
άρα το παραπάνω σύστημα γράφεται:
\[
y(n)= \sum_{k=-\infty}^{\infty} x(k) \cdot h(n-k)
\]

\begin{theorem}[title=Συμπέρασμα]{}{}
	Αν ένα σύστημα \( T:x(n) \to y(n) \) είναι γραμμικό, τότε ορίζω την έννοια της
	\textbf{κρουστικής απόκρισης} του \( t \) ως \( h(n) = T\left[δ(n)\right] \).
	Αν επιπροσθέτως το \( T \) είναι ΑΚΜ, τότε για οποιαδήποτε είσοδο \( x(n) \) μπορώ να γράψω ότι
	η έξοδος θα δίνεται ως:
	\[
	y(n) = \sum_{k=-\infty}^{\infty} x(k)h(n-k) = x(n) * h(n)
	\quad \text{(συνέλιξη των διακριτών ακολουθιών)}
	\]
\end{theorem}

\paragraph{Ιδιότητες}
Οι ιδιότητες της διακριτής συνέλιξης είναι ίδιες με αυτές του πολλαπλασιασμού:
\begin{enumgreekpar}
	\item \textbf{Αντιμεταθετική}: \( x(n) * y(n) = y(n) * x(n) \)
	\item \textbf{Προσεταιριστική}: \( x_1 * (x_2*x_3) = (x_1*x_2)*x_3 \)
	\item \textbf{Επιμεριστική} με πρόσθεση: \( x_1 * (x_2+x_3) = x_1*x_2 + x_1*x_3 \)
	\item \textbf{Ουδέτερο στοιχείο} η \( δ(n) \):\[
	x(n) * δ(n) = x(n)
	\]
\end{enumgreekpar}

\paragraph{Χρήσιμες πληροφορίες}
\begin{itemize}
	\item Μπορούμε να πούμε ότι ένα ψηφιακό σήμα έχει \textbf{διάρκεια}.
	
	Αν \emph{ξεκινάει} (πρώτο μη-0 στοιχείο) στο \( n=D_1 \) και \emph{τελειώνει} (τελευταίο μη-0 στοιχείο) στο \( n=U_1 \), τότε μπορούμε να πούμε ότι έχει διάρκεια:
	\[
	T = U_1-D_1 + 1
	\]
	(προσοχή στον όρο \( +1 \)!)
	
	\todo{graph 9 right}
	
	\item Έστω δύο σήματα:
	\begin{align*}
		x_1(n) \text{ έχει διάρκεια } & N_1\\
		x_2(n) \text{ έχει διάρκεια } & N_2
	\end{align*}
	τότε η συνέλιξή τους έχει διάρκεια:
	\[
	T_3 = T_1+T_2-1
	\]
	
	\begin{questionbox}{Άσκηση για το σπίτι}
 		Φανταστείτε ότι σας δίνονται αρχές \( U_1 \) και τέλη \( D_1 \). Για παράδειγμα, μια ακολουθία
		ξεκινά από το 1813 και τελειώνει στο 1980. Γνωρίζουμε ότι αν συνελίξουμε τέτοιες ακολουθίες
		μεταξύ τους, θα πάρουμε σήματα διάρκειας \( T_3 = T_1+T_2-1 \).
		
		Ζητείται να βρεθεί μια σχέση που, δεδομένων των \( U_1,U_2,D_1,D_2 \), να βρίσκει \textbf{από ποιό σημείο \( U_3 \)} ξεκινάει το αποτέλεσμα της συνέλιξης.
		
		\tcblower
		Η απάντηση θα είναι \( U_3 = U_1+U_2 \) και \( D_3 = D_1+D_2 \)
	\end{questionbox}

	%\item
	\todo{??? See what happens with x(n) * u(n-k)}
\end{itemize}

\paragraph{Παραδείγματα}
\hspace{0pt}

\begin{exercise}
	Να συνελιχθούν οι συναρτήσεις:
	\begin{align*}
		x_1(n) &= δ(n) + 2δ(n-1) + 3δ(n-2) + 4δ(n-3)\\
		x_2(n) &= δ(n+1) -2δ(n) + δ(n-1)
	\end{align*}
	\tcblower
	Περιμένουμε η συνέλιξη, με βάση αυτά που είδαμε παραπάνω, να έχει διάρκεια:
	\[
	4+3-1=6
	\]
	
	Πραγματοποιούμε τους υπολογισμούς:
	\begin{align*}
		x_1 * x_2 &=
		x_1(n) * \left[ δ(n+1)-2δ(n)+δ(n+1) \right]
		\\ &= x_1(n) * δ(n+1) - x_1(n)*\left( 2 δ(n) \right) + x_1(n)*δ(n-1)
		\\ &= x_1(n+1) - 2x_1(n) + x_1(n-1)
		\\ &= δ(n+1) + 2δ(n) + 3δ(n-1) + 4δ(n-2) - 2δ(n) - 4δ(n-1)
		\\ &\hphantom{=} - 6δ(n-2) - 8δ(n-3) + δ(n-1) +2δ(n-2) + 3δ(n-3)+4δ(n-4)
		\\ &= δ(n+1) -5δ(n-3) + 4δ(n-4)
	\end{align*}
	κάτι που όντως έχει διάρκεια 6.
\end{exercise}

\begin{exercise}
Έστω ότι έχουμε συναρτήσεις που ξεκινούν και τελειώνουν σε διάφορα σημεία:

\begin{minipage}{.5\textwidth}
	\[
	y(n) = \begin{cases}
	β^{n-n_0} &\quad n \geq n_0\\
	0 &\quad n < n_0
	\end{cases}
	\]
\end{minipage}
\begin{minipage}{.5\textwidth}
	\[
	x(n) = \begin{cases}
	a^{n} &\quad 0 \geq n \geq N\\
	0&\text{αλλού}
	\end{cases}
	\]
\end{minipage}
\paragraph{}

Να βρεθεί η συνέλιξή τους.

\tcblower

Μπορούμε να γράψουμε τις παραπάνω συναρτήσεις απλούστερα, εκμεταλλευόμενοι τη \textbf{βηματική step function}:
\begin{align*}
	y(n) &= β^{n-n_0} \cdot \mathrm u (n-n_0)\\
	x(n) &= a^n \cdot \left[
	\mathrm u(n) - \mathrm u(n-N-1)
	\right]
\end{align*}
(με \textit{προσοχή} στον όρο \( -1 \) μέσα στη \( \mathrm{u} \))

Άρα η συνέλιξη βρίσκεται:
\begin{align*}
	z(n) &= x(n) * y(n) \\
	&= \sum_{k=-\infty}^{\infty} β^{n-n_0} \mathrm u (n-n_0)a^n \left[
	\mathrm u(n) - \mathrm u(n-N-1)
	\right]
	\\ &= \sum_{k=-\infty}^{\infty} \left[
	β^{k-n_0}u(k-n_0)\cdot a^{n-k} \left[ u(n-k)-u(n-k-N-1) \right]
	\right]
	\\ &= \sum_{k=-\infty}^{\infty} β^{k-n_0}
	a^{n-k} u(k-n_0)u(n-k)
	- \sum_{k=-\infty}^{\infty} β^{k-n_0}u(k-n_0)a^{n-k} u (n-k-N-1)
	\\ &= β^{-n_0}a^n \sum_{k=n_0}^{n} β^k a^{-k} u (n-n_0)
	- β^{-n_0}a^n \sum_{k=n_0}^{n-N-1}β^ka^{-k}u(n-N-1-n_0)
	\intertext{Προσπαθώντας να φτάσουμε έναν όρο γεωμετρικής προόδου, θα κάνουμε το άθροισμα να ξεκινάει από το 0:}
	\\ &=
	a^n β^{-n_0}\left[
	\left(\frac{β}{a}\right)^{n_0}\sum_{k=0}^{n-n_0}\left(\frac{β}{a}\right)^k
	u(n-n_0) - \left(\frac{β}{a}\right)^{n_0}
	\sum_{k=0}^{n-n_0-N-1}\left(\frac{β}{a}\right)^k u(n-N-1-n_0)
	\right]
	\intertext{και ήρθε η ώρα να εκμεταλλευτούμε τους τύπους αθροίσματος γεωμετρικής προόδου:}
	\\ &= a^{n-n_0}\left[
	\frac{\left(\frac{β}{a}\right)^{n-n_0+1}-1}{\frac{β}{a}-1}u(n-n_0)
	- \frac{\left(\frac{β}{a}\right)^{n-N_0-1}}{\frac{β}{a}-1}u(n-N-1-n_0)
	\right]
\end{align*}

\end{exercise}

\begin{exercise}
	Έστω η ακολουθία:
	\[
	x(n) = a^n\ \forall n
	\]
	και οι \( y(n), \ z(n) \) αυθαίρετες.
	
	Να δειχθεί ότι:
	\[
	\left[
	x(n)y(n)
	\right]*\left[
	x(n)z(n)
	\right] = x(n)\left[
	y(n)*z(n)
	\right]
	\]
	
	\tcblower
	
	Το αριστερό μέλος της ζητούμενης σχέσης είναι:
	\begin{align*}
		\left[
		x(n)y(n)
		\right]*\left[
		x(n)z(n)
		\right] &=
		\sum_{k=-\infty}^{\infty}
		x(k)y(k)x(n-k)z(n-k)
		\\ &= \sum_{k=-\infty}^{\infty} a^k y(k)a^{n-k} z(n-k)
		\\ &= a^n \sum_{k=-\infty}^{\infty} y(k) z(n-k)
		\\ &= x(n) \cdot \left[y(n) * z(n)\right]
	\end{align*}
\end{exercise}

\begin{exercise}
	Να υπολογιστεί η ενέργεια και η ισχύς του σήματος:
	\[
	x_1(n) = \left( \frac{1}{2} \right)^n u(n)
	\]
	\tcblower
	Η ενέργεια δίνεται από τον τύπο:
	\begin{align*}
		E_{x_1} &=
		\sum_{n=-\infty}^{\infty}x_1^2(n) = \sum_{n=-\infty}^{\infty}\left(\frac{1}{2}\right)^{2n}
		u(n) = \sum_{n=0}^{\infty}\left( \frac{1}{4} \right)^n
		\\ &= \frac{1}{1-\frac{1}{4}} = \frac{4}{3}
	\end{align*}
	
	Έχουμε αποδείξει ότι το σήμα είναι σήμα ενέργειας, άρα σίγουρα δεν είναι σήμα ισχύος, δηλαδή η ισχύς του είναι:
	\[
	P=0
	\]
	
	Εναλλακτικά, η ισχύς είναι η ενέργεια διά τη διάρκεια του σήματος, δηλαδή:
	\[
	P = \frac{\sfrac{4}{3} }{\infty} = 0
	\]
\end{exercise}

\begin{exercise}
	Να υπολογιστεί η ενέργεια \& η ισχύς του σήματος:
	\[
	x_2(n) = e^{j\left( n\frac{π}{2} + \frac{π}{8} \right)}
	\]
	\tcblower
	Έχουμε:
	\[
	E_{x_2} = \sum_{n=-\infty}^{\infty} \left|x_2(n)\right|^2 = \sum_{n=-\infty}^{\infty} |1|^2 = \infty
	\]
	
	Η ισχύς του σήματος είναι:
	\[
	P_a = \lim_{N\to \infty}\frac{1}{2N+1}
	\sum_{n=-N}^{N} \cancelto{1}{\left|x_2(n)\right|^2}
	= \lim_{N\to \infty}\frac{2N+1}{2N+1} = 1
	\]
\end{exercise}

\begin{exercise}
	Δίνονται τα σήματα:
	
	\begin{align*}
	x(n) &= 2^n u(-n-1)\\
	y(n) &= 4^n u(-n-1)
	\end{align*}
	
	Να βρείτε τη συνέλιξή τους:
	\[
	x(n) * y(n) = z(n)
	\]
	
	\tcblower
	Έχουμε:
	\begin{align*}
		z(n) &= \sum_{k=-\infty}^{\infty} x(k)y(n-k)
		\\ &= \sum_{k=-\infty}^{\infty}
		2^k u(-k-1) 4^{n-k} u \left(-(n-k)-1\right)
		\\ &= 4^n\sum_{k=-\infty}^{\infty}\left(\frac{2}{4}\right)^k u(-k-1)u(-n-1+k)
		\\ &= 4^n\sum_{k=n+1}^{-1} \left(\frac{1}{2}\right)^k u(-1-n-1)
		\\ &= 4^n\left(\frac{1}{2}\right)^{n+1}\ \cdot \ \sum_{k=0}^{-n-2}\left(\frac{1}{2}\right)^k
		u(-1-n-1)
		\\ &= 4^n \left(\frac{1}{2}\right)^{n+1}
		\frac{\left(\frac{1}{2}\right)^{-n-1}-1}{\frac{1}{2}-1}u(-n-2)
		\\ &= \left(2^n-2^{2n+1}\right)u(-n-2)
	\end{align*}
\end{exercise}

\begin{questionbox}{Άσκηση για το σπίτι}
	Να γίνει η συνέλιξη των δύο ακολουθιών:
	\begin{align*}
		x(n) &= u(-n-1)\\
		y(n) &= \left(\frac{1}{2}\right)^n u(n)
	\end{align*}
\end{questionbox}

\lecture{3}{8/10/2018}

\subsubsection{Ευστάθεια}
Την ευστάθεια μπορούμε να την ορίσουμε με διαφορετικούς ασύμβατους τρόπους.

Εμείς θα χρησιμοποιούμε την \textbf{ευστάθεια κατά ΦΕΦΕ (Φραγμένη Είσοδος \textendash~Φραγμένη Έξοδος)} (BIBO \textemdash~ Bounded Input \textendash~Bounded Output).

\begin{defn}{Φραγμένη ακολουθία}{}
	Μια \textbf{φραγμένη} ακολουθία \( x(n) \)είναι αυτή που δεν πιάνει τιμές μέχρι το άπειρο:
	\[
	\left|x(n)\right| < M < \infty \quad \forall n \in \mathbb Z
	\]
\end{defn}

\begin{defn}{Ευστάθεια συστήματος κατά ΦΕΦΕ}{}
	Αν \( \forall x(n) \) φραγμένη είσοδο, η έξοδος \( y(n) \) ενός συστήματος 
	είναι \textbf{φραγμένη}, τότε το σύστημα είναι \textbf{ευσταθές κατά ΦΕΦΕ (BIBO)}.
\end{defn}

\begin{theorem}{Πόρισμα}{}
	Αν το σύστημα γνωρίζουμε ότι είναι \emph{γραμμικό} και \emph{ΑΚΜ} (άρα \( \exists\ h(n) \), δηλαδή υπάρχει η κρουστική του απόκριση), τότε το σύστημα είναι \textbf{ευσταθές} ανν
	\[
	\sum_{n=-\infty}^{\infty} \left|h(n)\right| < \infty
	\]
\end{theorem}
\subparagraph{Απόδειξη}
Για τις δύο κατευθύνσεις:
\begin{itemize}
	\item Εξετάζω ότι \( \sum_{n=-\infty}^{\infty} \left|h(n)\right| < \infty \implies \text{ευστάθεια κατά BIBO} \):
	
	Αφού το άθροισμα της κρουστικής απόκρισης είναι φραγμένο, θα ισχύει \( \displaystyle
	\sum_{n=-\infty}^{\infty} \left|h(n) \right| < L \) για κάποιο (μεγάλο) \( L \).
	
	Έστω \( x(n) \) φραγμένη είσοδος \( \iff \left|x(n)\right| < M \ \forall n \), η έξοδος θα είναι:
	\begin{align*}
		y(n) &= \sum_{k=-\infty}^{\infty} h(k)x(n-k)
		\\ \implies \left|y(n)\right| &= \left|\sum_{k=-\infty}^{\infty} h(k)x(n-k) \right|
		\leq \sum_{k=-\infty}^{\infty} \left|h(k)\right|\left|x(n-k)\right|
		\\ \implies \left|y(n)\right| &< M \sum_{k=-\infty}^{\infty} \left|h(n)\right|
		= N
	\end{align*}
	
	Άρα \( \left|y(n)\right| < N \), άρα η έξοδος είναι φραγμένη.
	
	\item Εξετάζω ότι \( \text{BIBO} \implies \sum_{n=-\infty}^{\infty} \left|h(n)\right| < \infty \).
	
	Εναλλακτικά θα εξετάσω το \textit{αντιθετοαντίστροφο}, δηλαδή ότι
	\( \displaystyle \sum_{n=-\infty}^{\infty} \left|h(n)\right| = \infty \implies
	\text{όχι BIBO} \)
	
	Ορίζω μία νέα ακολουθία \( x(n) \) ως εξής:
	\[
	x(n) = 
	\begin{cases}
	\frac{h^*(-n)}{\left| h(-n) \right|} 
 &\quad \text{αν } h(-n) \neq 0\\
	 0 &\quad \text{αλλού}
	 \end{cases}
	\]
	
	Η \( x(n) \) είναι φραγμένη, αφού έχει μοναδιαίο ή μηδενικό μέτρο παντού. Επομένως είναι \emph{φραγμένη}. Θα την πετάξουμε ως είσοδο στο σύστημα για να αποδείξουμε ότι αυτό
	δίνει \textbf{μη φραγμένη έξοδο, για φραγμένη είσοδο}, άρα είναι αυτό ασταθές κατά BIBO:
	
	\begin{align*}
		y(n) &= \sum_{k=-\infty}^{\infty} h(k)x(n-k) =  \sum_{k=-\infty}^{\infty} \frac{h(k)h^*(k-n)}{\left|h(k-n)\right|}\\
		\intertext{Θέτουμε συγκεκριμένα $y=0$:}
		y(0) & =  \sum_{k=-\infty}^{\infty} \frac{h(k)h^*(k)}{\left|h(k)\right|}
		= \sum_{k=-\infty}^{\infty} \left|h(k)\right|
		= \infty.
	\end{align*}
\end{itemize}

\subsubsection{Αιτιατότητα}
\begin{defn}{Αιτιατό σύστημα}{}
	Ένα σύστημα είναι \textbf{αιτιατό} αν η έξοδος εξαρτάται μόνο από \textbf{παρελθούσες τιμές} της εισόδου ή/και την τρέχουσα (παρούσα).
	
	Μαθηματικά:
	\[
	y(n) = f\left( x(k),x(λ),x(ξ), \dots, x(p) \right)
	\]
	όπου \( κ,λ,ξ,\dots,p \leq n \)
\end{defn}

\begin{theorem}{}{}
	Ένα γρ. ΑΚΜ σύστημα θα είναι \textbf{αιτιατό} αν:
	\[
	h(n) = 0 \ \forall n<0
	\]
	δηλαδή αν η κρουστική απόκριση είναι \textbf{αιτιατή ακολουθία}.
\end{theorem}
\subparagraph{Απόδειξη}
Η έξοδος του συστήματος θα είναι:
\begin{align*}
	y(n) &= \sum_{k=-\infty}^{\infty} h(k)x(n-k)
\end{align*}
Εμείς θέλουμε ο δείκτης του \( x \) να μην είναι ποτέ \emph{μεγαλύτερος} από το \( n \),
διότι σε αυτήν την περίπτωση η έξοδος θα εξαρτιόταν από το \emph{μέλλον} του \( x(n) \). Δηλαδή
το \( y(n) \) θα εξαρτάται από τα \( x(n+1), x(n+2), \dots \). Άρα θέλουμε \( k > 0 \).

Αυτό εξασφαλίζεται όταν \( h(k) = 0\ \forall k < 0 \). Τότε η έξοδος θα είναι:
\begin{align*}
	y(n) &= \sum_{k=0}^{\infty} h(k)x(n-k)
\end{align*}

Το παραπάνω βέβαια, δεδομένου του ορισμού της συνέλιξης, γράφεται και ως εξής:
\begin{align*}
	y(n) &= h(n) * x(n) = x(n) * h(n)
	\\ &= \sum_{k=-\infty}^{\infty}x(k)h(n-k)
	\\ &= \sum_{k=-\infty}^{n} x(k)h(n-k)
\end{align*}

\subsubsection{Μερικοί χρήσιμοι ορισμοί για ακολουθίες}
\begin{enumgreekpar}
	\item \textbf{Αιτιατή ακολουθία:} \( x(n) = 0 \quad \forall\ n<0 \)
	\item \textbf{Ακολουθία δεξιάς πλευράς:} \( \exists \ M \in \mathbb Z:\ x(n) = 0 \quad \forall \ n<M \)
	\item \textbf{Ακολουθία αντιαιτιατή:} \( x(n) = 0\quad \forall\ n \geq 0 \)
	\item \textbf{Ακολουθία αριστερής πλευράς:} \( \exists\ M \in \mathbb Z : \ x(n) = 0 \quad \forall \ n>M \)
	\item \textbf{Ακολουθία πεπερασμένης διάρκειας:} \( \exists \ M,N\ M<N:\ \forall n < M \text{ ή } n>N:\ x(n) = 0 \)
\end{enumgreekpar}

Κάθε αιτιατή ακολουθία είναι δεξιάς πλευράς.

\section{Συνεχής μετασχηματισμός Fourier διακριτού σήματος (Discrete-Time Fourier Transform)}

\subsection{Απόκριση γραμμικού ΑΚΜ συστήματος σε εκθετική είσοδο}
Μελετάμε την απόκριση συστήματος σε είσοδο:
\[
x(n) = a^n
\]
όπου \( a\in\mathbb{C} \).

Τότε η έξοδος είναι:
\begin{align*}
	y(n) = h(n) * x(n)  &= \sum_{k=-\infty}^{\infty} h(k) x(n-k)
	\\ &= \sum_{k=-\infty}^{\infty} h(k) a^{n-k}
	\\ &= a^n \sum_{k=-\infty}^{\infty} h(k)a^{-k}
	\\ &= x(n) \cdot \sum_{k=-\infty}^{\infty} \underbrace{h(k)a^{-k}}_{\mathclap{
			\raisebox{-1.5ex}{\footnotesize\text{αριθμός που δεν εξαρτάται από το $n$}}
		}}
	\\ &= x(n) \cdot H
\end{align*}

Το τελικό αποτέλεσμα δηλαδή θα είναι η είσοδος \textbf{πολλαπλασιασμένη με έναν αριθμό} που
εξαρτάται από την κρουστική απόκριση του συστήματος και τη βάση \( a \) της εισόδου.
\begin{itemize}
	\item
Έστω ότι:
\[
\infoboxed{a = e^{j\omega }} \quad \implies x(n) = e^{j\omega n}
\]
για \( \omega \in \mathbb R \) μία συνεχή μεταβλητή.

Άρα η έξοδος θα είναι:
\[
y(n) = x(n) \cdot \boxed{\sum_{k=-\infty}^{\infty} h(k) e^{-j\omega k}}
\]
το πλαίσιο είναι ουσιαστικά μία συνάρτηση που εξαρτάται από το \( e^{j\omega}  \), ή
πρακτικά μόνο από το \( \omega  \). Ονομάζουμε αυτήν τη συνάρτηση \( H(\omega ) \):
\[
H(\omega ) = \sum_{k=-\infty}^{\infty} h(k)e^{-j\omega k}
\]
αυτή ουσιαστικά είναι η \textbf{απόκριση συχνότητας του συστήματος}.

Υπενθυμίζουμε ότι το \( \omega  \) είναι \textbf{συνεχής μεταβλητή}. Μπορούμε να ορίσουμε
έναν συνεχή μετασχηματισμό Fourier ως εξής:
\[
x(n) \rightarrow X(\omega ) = \sum_{n=-\infty}^{\infty} x(n) e^{-j\omega n}
\]
\item
Έστω ότι:
\[
\infoboxed{a = z}
\]
δηλαδή το \( a \) ανήκει οπουδήποτε στο μιγαδικό επίπεδο
Τότε:
\[
y(n) = x(n) \cdot \boxed{\sum_{k=-\infty}^{\infty} h(k) z^{-k}}
\]
\end{itemize}

\begin{defn}{Απόκριση συχνότητας}{}
	Ως \textbf{απόκριση συχνότητας} του συστήματος ορίζουμε τη συνάρτηση:
	\begin{align*}
		H\left(e^{j\omega }\right) &= \sum_{k=-\infty}^{\infty}h(k)e^{-j\omega k}
		\\ &= H_R\left(e^{j\omega }\right) + j H_I\left(e^{j\omega }\right)
		\\ &=
		\underbrace{\left|H\left(e^{j\omega }\right)\right|}_{\mathclap{\text{απόκριση πλάτους}}}
		\exp\underbrace{\left[j \mathop{arg} \left(H\left(e^{j\omega t}\right)\right)\right]}_{\mathclap{\text{απόκριση φάσης}}}
	\end{align*}
\end{defn}

\subsubsection{Απόκριση συστήματος σε ημιτονοειδείς συναρτήσεις}
Έστω η είσοδος:
\[
x(n) = A\cos\left( \omega_0 n + \phi \right)
\]

Υπενθυμίζουμε ότι η είσοδος εν γένει \textbf{δεν είναι περιοδική}.
\todo{Add reference}

Αντικαθιστούμε το συνημίτονο:
\[
x(n) = A\frac{e^{j\phi}e^{j\omega_0 n} + e^{-j\phi}e^{-j\omega_0n}}{2}
\]
οπότε κατά τα παραπάνω, η έξοδος θα είναι:
\begin{align*}
	y(n) &= \frac{A}{2}e^{j\phi}e^{j\omega_0 n} H\left(e^{j\omega_0}\right)
	+ \frac{A}{2}e^{-j\phi}e^{-j\omega_0 n} H\left(e^{-j\omega_0}\right)
\end{align*}

Εφ' όσον θεωρηθεί ότι \( h(n) \in \mathbb R \), τότε θα ισχύει \( H\left( e^{j\omega_0} \right) = H^*\left(e^{-j\omega_0}\right) \). Άρα τελικά:
\begin{align*}
	y(n) &= \frac{A}{2} e^{j\phi} e^{j\omega_0 n} \left| H\left(e^{j\omega_0}\right) \right|
	\exp \left[j \mathop{arg}\left( H\left(e^{j\omega_0}\right) \right)\right]
	+ \frac{A}{2} e^{j\phi} e^{-j \omega_0 n}
	\left| H\left(e^{j\omega_0}\right) \right| \exp \left[-j\mathop{arg} \left(
	H\left(e^{j\omega_0}\right)
	\right)\right]
	\\ &= A\left| H\left( e^{j\omega_0} \right) \right|
	\cos\left( \omega_0 n + \phi + \mathop{arg}\left(H\left(e^{j\omega_0}\right)\right) \right)
\end{align*}

\begin{exercise}
	Να βρεθεί η απόκριση συχνότητας του συστήματος με μοναδιαία απόκριση \( h(n) = a^n \mathrm{u}(n) \) για \( |a| < 1 \).
	
	\tcblower
	Εύκολα φαίνεται ότι το σύστημα είναι αιτιατό και ευσταθές.
	
	Για την απόκρισή του, από τον ορισμό της έχουμε:
	\begin{align*}
		H\left(e^{j\omega }\right)
		&= \sum_{n=-\infty}^{\infty} h(n) e^{-j\omega n}
		\\ &= \sum_{n=0}^{\infty} a^n e^{-j\omega n}
		\\ &= \sum_{n=0}^{\infty}\left( ae^{-j\omega } \right)^n
		\intertext{Σε αυτό το σημείο εκμεταλλευόμαστε το άθροισμα άπειρων όρων γεωμετρικής προόδου:}
		&= \frac{1}{1-ae^{-j\omega }} \quad \text{(επειδή $\left|ae^{-j\omega }\right|=|a|<1$)}
		\intertext{Άρα η απόκριση πλάτους \& συχνότητας θα είναι:}
		\left| H\left(e^{j\omega }\right) \right|
		&= \frac{1}{\left(1+a^2-2a\cos \omega \right)^{\sfrac{1}{2} }}
		\\
		\mathop{arg}\left(H\left(e^{j\omega }\right)\right)
		&= -\arctan \left( \frac{a\sin\omega}{1-a\cos\omega} \right)
	\end{align*}
\end{exercise}

\begin{exercise}
	Δίνεται γρ. ΑΚΜ σύστημα με:\[
	h(n) = \frac{1}{2}δ(n) + δ(n-1) + \frac{1}{2}δ(n-2)
	\]
	
	Να χαρακτηριστεί και να βρεθεί η απόκριση συχνότητάς του.
	\tcblower
	Το σύστημα είναι αιτιατό \& ευσταθές.
	
	Σύμφωνα με τον ορισμό, έχουμε:
	\begin{align*}
		H\left(e^{j\omega }\right) &= \sum_{n=-\infty}^{\infty}
		h(n) e^{-j\omega n}
		\\ &= \sum_{n=-\infty}^{\infty}
		\left[
		\frac{1}{2}δ(n) + δ(n-1) + \frac{1}{2}δ(n-2)
		\right]e^{-j\omega n}
		\\ &= \frac{1}{2}e^{-j\omega 0}
		+ e^{-j\omega 1}
		+ \frac{1}{2}e^{-j\omega 2}
		\\ &= e^{-j\omega } \left[ \frac{1}{2}e^{j\omega } + 1 + \frac{1}{2}e^{-j\omega }  \right]
		\\ &= e^{-j\omega } (1 + \cos \omega)
		\intertext{Άρα σε πλάτος και φάση:}
		\left|
		H\left(e^{j\omega }\right)
		\right| &= \left|1+\cos \omega \right| = 1 + \cos \omega 
		\\
		\mathop{arg}\left[H\left(e^{j\omega }\right)\right]
		&= \mathop{arg} \left[
		e^{-j\omega(1+\cos \omega) }
		\right]
		= \mathop{arg}\left[e^{-j\omega}\right] +\underbrace{ \mathop{arg}\left[1+\cos\omega\right]}_{\mathclap{\text{επειδή $1+\cos\omega \geq 0$}}}
		= -\omega + 0 = -\omega
	\end{align*}
	
	\todo{Diagrams: Plot and phase of above}
\end{exercise}

\lecture{4}{12/10/2018}
\subsection{Μετασχηματισμός Fourier}
Παραπάνω ορίσαμε το συνεχή μετασχηματισμό Fourier στο διακριτό σήμα ως εξής:
\[
H(e^{j\omega }) = \sum_{n=-\infty}^{\infty} h(n)e^{-j\omega n}
\]
και, αν το σύστημα είναι ευσταθές κατά BIBO, έχουμε:
\begin{align*}
	h(n) \text{ ευσταθές}
	&\iff \sum_{n=-\infty}^{\infty} \left|h(n)\right| < \infty
	\\ &\iff \sum_{n=-\infty}^{\infty} \left|h(n)\right|\left|e^{-j\omega n}\right| < \infty
	\\ &\iff \left|h(n)e^{-j\omega n}\right| < \infty
	\implies \sum_{n=-\infty}^{\infty}h(n)e^{-j\omega n} < \infty
	\implies \exists \text{ } H\left(e^{j\omega }\right)
\end{align*}

Πρακτικά, διαπιστώνουμε ότι \textit{αν το σύστημα είναι ευσταθές}, τότε \textit{το σύστημα έχει απόκριση
συχνότητας}, δηλαδή ορίζεται ο μετασχηματισμός Fourier του.

Ένα ερώτημα που προκύπτει είναι αν ισχύει το αντίστροφο, δηλαδή αν η ύπαρξη του \( H\left(e^{j\omega }\right) \implies \) ότι το σύστημα είναι \emph{ευσταθές}. Για να το διαπιστώσουμε αυτό, θεωρούμε
ένα αντιπαράδειγμα (\emph{χαμηλοπερατού φίλτρου}):
\begin{minipage}{.5\textwidth}
	\[
	H\left(e^{j\omega }\right)
	\begin{cases}
		1 &\quad |\omega | < \omega_c\\
		0 &\quad \text{αλλού}
	\end{cases}
	\]
\end{minipage}
\begin{minipage}{.5\textwidth}
	\todo{Plot H(e^jw) continuously (LPF)}
\end{minipage}

Το παραπάνω σύστημα \textbf{δεν είναι ευσταθές}! Πράγματι, η αντίστοιχη κρουστική απόκρισή του είναι
\( h(n) = \frac{\sin\omega_c n}{\pi n} \). Το σύστημα δεν είναι αιτιατό (αφού \( h(n)\neq 0 \impliedby n <0 \)), και το άθροισμα \( \displaystyle S = \sum_{n=-\infty}^{\infty} \left| \frac{\sin\omega_c n}{\pi n} \right| \) τείνει στο \( \infty \).

Επομένως, η ύπαρξη του μετασχηματισμού Fourier \emph{δεν} συνεπάγεται την ευστάθεια του συστήματος.

\subsubsection{Ιδιότητες της απόκρισης συχνότητας}
Ο \emph{συνεχής} μετασχηματισμός Fourier στο διακριτό σήμα:
\begin{enumerate}
	\item Είναι συνάρτηση \textbf{συνεχούς} μεταβλητής \( \omega \in (-\infty,\infty) \)
	\item Η \( H\left( e^{j\omega } \right) \) είναι \textbf{περιοδική} με περίοδο \( 2π \).
\end{enumerate}

\todo{Check 2,3,4}

Επιπλέον, αν για το αρχικό σήμα στο χρόνο ισχύει \( h(n)\in\mathbb R \):
\begin{enumerate}
	\item[3] Η \( \left|H\left(e^{j\omega }\right)\right| \) είναι \textbf{άρτια} συνάρτηση του \( \omega  \) και συμμετρική ως προς τον άξονα \( \omega = π \).
	\todo{Graph 11 (optional)}
	\item[4] Το \( \arg H\left(e^{j\omega }\right) \) είναι \textbf{περιττή} συνάρτηση του \( \omega  \)
	και αντισυμμετρική ως προς τον άξονα \( \omega = \pi \).
	\todo{Graph 12 (optional)}
\end{enumerate}

\begin{questionbox}{Άσκηση για το σπίτι}
	Να αποδειχθούν οι ιδιότητες \textbf{3}, \textbf{4}.
\end{questionbox}

\begin{exercise}
	Να βρεθεί ο συνεχής Μ/Σ Fourier της:
	\[
	h(n) = \begin{cases}
	1 & \quad 0 \leq n < N\\
	0 & \quad \text{αλλού}
	\end{cases}
	\]
	\tcblower
	Η δοθείσα συνάρτηση γράφεται απλούστερα ως:
	\[
	h(n) = u(n) - u(n-N)
	\]
	
	Επομένως μπορούμε να υπολογίσουμε αναλυτικά το μετασχηματισμό της:
	\begin{align*}
		H\left(e^{j\omega }\right)
		&= \sum_{n=-\infty}^{\infty} h(n) e^{-j\omega n}
		\\ &= \sum_{n=0}^{N-1} 1 e^{-j\omega n}
		\\ &= \frac{1-e^{-j\omega n}}{1-e^{-j\omega }}
		\\ &= \frac
		{
			e^{-j\omega \sfrac{N}{2} } 
			\left(
				e^{j\omega \sfrac{N}{2} }-e^{-j\omega \sfrac{N}{2} }
			\right)
		}
		{ 
			e^{-j\sfrac{\omega }{2} }  
			\left( e^{j\sfrac{\omega }{2}  }\right)
			-e^{-j\sfrac{\omega }{2} 	}
		}
	\\ &= 	\frac{\sin\left(\frac{\omega N}{2}\right)}{\sin\left(\frac{\omega}{2}\right)}
	\end{align*}
	Άρα:
	\begin{align*}
		\left|H\left(e^{j\omega }\right)\right|
		 &= \left|
		\frac{\sin\left(\frac{\omega N}{2}\right)}{\sin\left(\frac{\omega}{2}\right)}
		\right|
		\\
		\arg\left(H\left(e\right)\right) &= \mathop{atan2}\left(0, \frac{\sin\left(\frac{\omega N}{2}\right)}{\sin\left(\frac{\omega}{2}\right)}\right)
	\end{align*}
\end{exercise}

\begin{enumerate}
	\item[5.]
	Μπορούμε να χωρίσουμε κάθε απόκριση συχνότητας σε ένα περιττό και ένα άρτιο κομμάτι:
	\[
	h(n) = h_o(n) + h_e(n)
	\]
	όπου \( h_o(n) = j\Im \left\lbrace H\left(e^{j\omega }\right)\right\rbrace \)
	και \( h_e = \Re\left\lbrace H\left(e^{j\omega }\right) \right\rbrace \)
\end{enumerate}

\paragraph{Μετασχηματισμός σήματος}
Ακριβώς όπως ορίσαμε το μετασχηματισμό της κρουστικής απόκρισης, μπορούμε να μετασχηματίσουμε
και την είσοδο \( x(n) \):
\[
X\left(e^{j\omega }\right) = \sum_{-\infty}^{\infty} x(n)e^{-j\omega n}
\]

Τότε, μπορούμε δοθέντος του συνεχούς Μ/Σ Fourier ενός σήματος, να γυρίσουμε πίσω στο ίδιο το σήμα:
\[
x(n) = \frac{1}{2π} \int_{-π}^{π} X\left(e^{-j\omega }\right)e^{j\omega n}\dif \omega 
\]

Εδώ δίνουμε προσοχή στο ότι \textbf{ολοκληρώνουμε} και δεν αθροίζουμε, αφού στο μετασχηματισμό αυτόν
η συχνότητα είναι \emph{συνεχής μεταβλητή}.

\subparagraph{Απόδειξη}
Έχουμε:
\begin{align*}
	X\left(e^{j\omega }\right) &= \sum_{k=-\infty}^{\infty} x(k)e^{-j\omega k}
	\implies  \\ X\left(e^{j\omega }\right)e^{j\omega n} &= \sum_{k=-\infty}^{\infty} x(k)e^{j\omega (n-k)}
	\\ \implies \int_{-π}^{π}X\left(e^{j\omega }\right) e^{j\omega n}\dif\omega
	&= \sum_{k=-\infty}^{\infty} x(k) \int_{-π}^{π} e^{j\omega (n-k)}\dif\omega
	= \sum_{k=-\infty}^{\infty} x(k) 2πδ(n-k) = 2\pi x(n).
\end{align*}

\paragraph{Θεώρημα Parseval}\hspace{0pt}

\begin{theorem}{Parseval's Theorem}{}
	Το θεώρημα του Parseval συνδέει την \textbf{ενέργεια} του σήματος με το Μ/Σ Fourier του:
	\[
	E = \sum_{n=-\infty}^{\infty} \left|x(n)\right|^2
	= \frac{1}{2\pi} \int_{-π}^{π}X\left(e^{j\omega }\right)X^*\left(e^{j\omega }\right)\dif \omega 
	\]
\end{theorem}
\begin{questionbox}{Άσκηση για το σπίτι}
	Να αποδειχθεί το θεώρημα του Parseval.
\end{questionbox}


\paragraph{Πολλαπλασιασμός \& Συνέλιξη}
\hspace{0pt}

\begin{theorem}{}{}
	Η \textbf{συνέλιξη} στο χρόνο είναι \textbf{πολλαπλασιασμός} στη συχνότητα του DTFT:
	\[
	x(k) * h(k) \leftrightarrow X\left(e^{j\omega }\right) H\left(e^{j\omega }\right)
	\]
\end{theorem}

\subparagraph{Απόδειξη}
Έχουμε:
\begin{align*}
	y(n) &= \sum_{k=-\infty}^{\infty}
	h(n-k)x(k)
	\implies \\
	\sum_{n=-\infty}^{\infty}y(n)e^{-j\omega n} &= \sum_{n=-\infty}^{\infty}
	e^{-j\omega n}\sum_{k=-\infty}^{\infty} h(n-k)x(k)
	\implies \\
	Y\left(e^{j\omega }\right) &= \sum_{k=-\infty}^{\infty} x(k) \sum_{n=-\infty}^{\infty}e^{-j\omega n}h(n-k)
    \\ &=
    \sum_{k=-\infty}^{\infty}x(k)e^{-j\omega k}\sum_{n=-\infty}^{\infty}e^{-j\omega (n-k)}h(n-k)
    \\ &= X\left(e^{j\omega }\right) H\left(e^{j\omega }\right)
\end{align*}

\section{Μετασχηματισμός Z}
Ο \textbf{μετασχηματισμός Z} είναι μια επέκταση του μετασχηματισμού Fourier στον διακριτό κόσμο, που
θα μπορούσε να μοιάσει με το μετασχηματισμό Laplace στο συνεχές σήμα.

Θυμόμαστε ότι η έξοδος ενός συστήματος είναι:
\[
y(n) = h(n) * x(n)
\]
και για μία είσοδο εκθετικής μορφής:
\[
x(n) =a^n
\]
η έξοδος θα είναι:
\[
y(n) = a^n \underbrace{\sum_{k=-\infty}^{\infty} h(k)a^{-k}}_{H(a)}
\]
ή, θεωρώντας ότι το \( a \) είναι κάποιο μιγαδικό \( z \):
\[
y(n) = z^n \sum_{k=-\infty}^{\infty} h(k)z^{-k}
\]

Τώρα μπορούμε να ορίσουμε το μετασχηματισμό \( Z \) ως εξής:
\begin{defn}{Μετασχηματισμός Z}{}
	Ο μετασχηματισμός Z μιας συνάρτησης \( x(n) \) ορίζεται ως εξής:
	\[
	X(z) = \sum_{n=-\infty}^{\infty} x(n) z^{-n}
	\]
\end{defn}

Για να έχει νόημα ο παραπάνω ορισμός, πρέπει να συγκλίνει το εξής άθροισμα:
\[
\sum_{n=-\infty}^{\infty} \left| x(n)z^{-n} \right| < \infty
\]
ή, ισοδύναμα:
\[
\sum_{n=-\infty}^{\infty} \left\lvert x(n) \right\rvert \left\lvert z\right\rvert^{-n} < \infty
\]

Αυτό που καθορίζει το αν θα συγκλίνουμε είναι το μέτρο \( |z| \). Οι περιοχές ίσου μέτρου ορίζουν
κύκλους ή donuts στο μιγαδικό επίπεδο του \( z \):
\todo{Graph 13}

Από τη μιγαδική ανάλυση αποδεικνύεται πως η περιοχή σύγκλισης δεν μπορεί να έχει κάποια άλλη περίεργη
μορφή:
\todo{Graph 14}

Επομένως, για το \( |z| \) θα ισχύει \( D < |z| < U \) για κάποια \( D, U \). Αυτά ορίζουν το \textbf{ROC} (Region Of Convergence) του μετασχηματισμού Z, χωρίς το οποίο ο μετασχηματισμός δεν έχει νόημα.

\begin{exercise}
	Να βρεθεί ο Z μετασχηματισμός του \( x(n) = a^n\mathrm u(n) \).\tcblower
	\begin{align*}
		X(z) &= \sum_{n=-\infty}^{\infty} x(n) z^{-n}
		= \sum_{n=0}^{\infty} a^n z^{-n} = \sum_{n=0}^{\infty} \left(\frac{a}{z}\right)^n
		\\ &= \frac{1}{1-\frac{a}{z}} = \frac{z}{z-a} \qquad \text{ αν $\left|\frac{a}{z}=1\right|$}
	\end{align*}
	
	Άρα γράφουμε συμβολικά:
	\[
	x(n) = a^n\mathrm{u}(n) \xrightarrow{ZT} X(z) = \frac{z}{z-a}\qquad \text{ROC}: |z|>|a|
	\]
\end{exercise}

\begin{exercise}
	Να βρεθεί ο μετασχηματισμός Z του \( x(n)= -a^n \mathrm{u}(-n-1) \).
	\tcblower
	Έχουμε:
	\begin{align*}
		X(z) &= \sum_{n=-\infty}^{\infty} x(n) z^{-n}
		= \sum_{n=-\infty}^{-1} -a^n z^{-n}
		= \sum_{n=-\infty}^{-1} -\left(\frac{a}{z}\right)^n
		\intertext{Προσπαθώ να ξεκινήσω την άθροιση από το 0 για να εφαρμόσω τύπο γεωμετρικής προόδου:}
		 &=
		\sum_{n=1}^{\infty} -\left(\frac{a}{z}\right)^{-n}
		= -\sum_{n=1}^{\infty}\left(\frac{z}{a}\right)^n
		= 1-\sum_{n=0}^{\infty}\left(\frac{z}{a}\right)^n
		\\ &= 1 - \frac{1}{1-\frac{z}{a}} & \left|\frac{z}{a}\right|<1
		\\ &= \frac{1-\frac{z}{a}-1}{1-\frac{z}{a}} = \frac{-\frac{z}{a}}{1-\frac{z}{a}}
		= \frac{-z}{a-z} = \frac{z}{z-a}
	\end{align*}
	
	Εδώ παρατηρούμε κάτι ενδιαφέρον: Ο τύπος του μετασχηματισμού Z είναι \textbf{ακριβώς ίδιος} με
	αυτόν της προηγούμεης άσκησης, παρ' όλο που προέρχονται από διαφορετικές συναρτήσεις. Η διαφορά
	εδώ βρίσκεται στην \textbf{περιοχή σύγκλισης} (ROC), που είναι διαφορετική στις δύο ασκήσεις.
	Εδώ φαίνεται και η σημασία του δεδομένου της περιοχής σύγκλισης.
\end{exercise}

\begin{exercise}
	Να βρεθεί ο μετασχηματισμός Z της συνάρτησης:\[
	x(n) = a^n \mathrm{u}(n) - b^n \mathrm{u}(-n-1)
	\]
	\tcblower
	Έχουμε:
	\begin{align*}
		X(z) &= \sum_{n=-\infty}^{\infty} x(n)z^{-n}
		\\ &= \sum_{n=-\infty}^{\infty} \left[
		a^n \mathrm{u}(n) - b^n \mathrm{u}(-n-1)
		\right]z^{-n}
		\\ &= \sum_{n=0}^{\infty} a^n z^{-n}
		- \sum_{n=-\infty}^{-1}b^n z^{-n}
		\\ &= \frac{z}{z-a} + \frac{z}{z-b} \qquad \text{για $|z|>|a|$ και $|z|<|b|$}
	\end{align*}
	
	Η \( X(z) \) υπάρχει αν υπάρχει η περιοχή σύγκλισης (ROC): \( |a|<|z|<|b| \). Για να συμβαίνει
	αυτό, πρέπει να ισχύει η συνθήκη \( |a|<|b| \). Διαφορετικά, δεν υπάρχει ο μετασχηματισμός.
\end{exercise}

\subsection{Περιοχή σύγκλισης}
\begin{itemize}
	\item Για \textbf{ακολουθίες δεξιάς πλευράς} \todo{link defn} \todo{slightly underline links} (\( x(n) = 0 \impliedby n<n_0 \)), αν \( n_0 \geq 0 \), η περιοχή σύγκλισης είναι:
	\[
	|z| > |z_1|
	\]
	\subparagraph{Απόδειξη}
	Ο μετασχηματισμός \( Z \) της ακολουθίας θα είναι:
	\[
	\sum_{n=n_0 \geq 0}^{\infty} x(n)z^{-n}
	\]
	
	Υποθέτουμε ότι συγκλίνει για κάποιο \( |z| = |z_1| \)
	Θέλουμε η σειρά \( \sum_{n=n_0\geq 0}^{\infty} \left| x(n)z_1^{-n} \right| < \infty \) να συγκλίνει.
	
	\[
	|z| > |z_1| \implies |z|^{-n} < |z_1|^{-n}
	\]
	αν \( n \geq 0 \) .
	
	\paragraph{}
	Αν αντίθετα, \( n_0 < 0 \):
	\[
	\sum_{n=n_0}^{\infty} \cdots = \sum_{n=n_0}^{-1}\cdots + \sum_{n=0}^{\infty}
	\]
	που οδηγεί σε ένα αντίστοιχο αποτέλεσμα, πλην του κύκλου στο άπειρο:
	\[
	|z_1| < |z| < \infty
	\]
	\item Για \textbf{ακολουθίες αριστερής πλευράς} (\( x(n) = 0 \impliedby n > n_0 \)), τότε (αντίστοιχα με παραπάνω):
	
	\[
	X(z) = \sum_{n=-\infty}^{n_0} x(n)z^{-n}
	\]
	
	Τότε θα ισχύει:
	\[
	\begin{cases}
	n_0 \leq 0 &\implies \text{ROC } =\ |z| < |z_1|\\
	n_0 > 0 &\implies \text{ROC } =\ 0 < |z| < |z_1|
	\end{cases}
	\]
	\item Για \textbf{ακολουθίες πεπερασμένου μήκους} (\( x(n) = 0\ \forall\ n<N,n>M,M>N \)), ο μετασχηματισμός Z θα είναι:
	\[
	X(z) = \sum_{n=N}^{M} x(n) z^{-n}
	\]
	
	Εδώ δεν έχουμε άπειρο άθροισμα. Κινδυνεύουμε με απειρισμούς μόνον όταν \( z = 0 \) ή \( z = \infty \).
	Πιο συγκεκριμένα:
	\[
	\begin{cases}
	\text{Αν $N < 0$ \& $M > 0$} &\implies \text{ROC: } 0<|z|<\infty\\
	\text{Αν $N > 0$ \& $M > 0$} &\implies \text{ROC: } 0<|z|\\
	\text{Αν $N < 0$ \& $M < 0$} &\implies \text{ROC: } |z|<\infty
	\end{cases}
	\]
\end{itemize}

\lecture{5}{15/10/2018}

\subsection{Αντίστροφος μετασχηματισμός Z}
Ο ευθύς μετασχηματισμός Z είναι:
\[
X(z) \overset{\triangle}{=} \sum_{n=-\infty}^{\infty} x(n)z^{-n}
\quad \mathcal{R}_{x(z)}
\]

Αποδεικνύεται ότι ο \textbf{αντίστροφος μετασχηματισμός Z} είναι:
\[
x(n) = \frac{1}{2 π j} \ointctrclockwise\limits_{c \in \mathcal{R}_{x(z)z^{n-1}}} x(z)z^{n-1} \dif z
\]

Η παραπάνω σχέση βασίζεται στο ολοκλήρωμα του Cauchy, \( \frac{1}{2πj} \ointctrclockwise z^{k-1}\dif z = \begin{cases}
1 &\quad k=0 \\ 0&\quad k\neq 0
\end{cases} \)

\subsection{Μέθοδοι υπολογισμού μετασχηματισμού Z}
\subsubsection{Ολοκληρωτικά υπόλοιπα}
Με βάση τη μιγαδική ανάλυση, γνωρίζουμε ότι:
\[
x(n) = \sum \left\lbrace 
\mathop{Res} \left[ x(z)z^{n-1} \right]
 \right\rbrace
\]
για τους πόλους \textbf{εντός} του \( C \in \mathcal{R}_{x(z)z^{n-1}} \).
\todo{Graph 15}
Η παραπάνω σχέση συμπεριλαμβάνει και πόλους που βρίσκονται \emph{εκτός των ορίων περιοχής σύγκλισης} αλλά
\emph{μέσα στα όρια του κύκλου που αυτές ορίζουν}.

\paragraph{Παράδειγμα}
Ένα σύστημα έχει 4 πόλους. Πόσες είναι οι μέγιστες δυνατές περιοχές σύγκλισης;
\subparagraph{Απάντηση}
Ένας πόλος, ως \emph{απομονωμένο} ανώμαλο σημείο, δεν μπορεί να βρίσκεται στην περιοχή σύγκλισης.

Τοποθετούμε τους πόλους στο πραγματικό επίπεδο για ευκολία, και θεωρούμε περιοχές σύγκλισης με ανοιχτά
όρια. Επειδή μια περιοχή σύγκλισης πρέπει να έχει πόλο στα όριά της, αλλά δεν μπορεί να περιέχει πόλο,
οι δυνατές επιλογές θα μοιάζουν ως εξής:
\todo{Graph 16 (5 graphs)}

Άρα έχουμε 5 δυνατότητες για περιοχή σύγκλισης.

\paragraph{Υπολογισμός ολοκληρωτικών υπολοίπων}
Σύμφωνα με τη μιγαδική ανάλυση, τα ολοκληρωτικά υπόλοιπα που ψάχνουμε μπορούν να βρεθούν ως εξής:
\[
\mathop{Res}\left[
X(z)z^{n-1} \ @\ z=z_0
\right] = \frac{1}{(s-1)!} \left.\od[s-1]{}{z} A(z)\right|_{z=z_0}
\]
όπου \( z_0 \) ο πόλος, \( s \) η πολλαπλότητά του, και:
\[
A(z) = X(z)z^{n-1}(z-z_0)^s
\]

\begin{exercise}
	Δίνεται ο μετασχηματισμός Z μιας ακολουθίας:\[
	X(z) = \frac{1}{1-az^{-1}} \qquad |z|>|a|
	\]
	
	Ποιά είναι η αρχική ακολουθία \( x(n) \);
	\tcblower
	Το \( X(z) \) γράφεται και ως:
	\[
	X(z) = \frac{z}{z-a}
	\]
	
	Εξετάζω πόσο κάνει το \( X(z)z^{n-1} \):
	\[
	X(z)z^{n-1} = \frac{z^n}{z-a}
	\]
	που έχει \textbf{πόλο} στο \( z_0 = a \) με \textbf{πολλαπλότητα} \( s_0 = 1 \). Στην περίπτωση που \( n = -k <0 \), προστίθεται και ένας παραπάνω πόλος \( z_1=0 \) πολλαπλότητας
	\( s_1 = |n|= k \).
	
	\begin{itemize}
		\item \textbf{Αν \( n\geq 0 \)}, τότε:
		έχουμε \textbf{1 πόλο} στο \( z_0=a \) πολλ/τας \( s_0=1 \). Άρα από τη θεωρία:
		\todo{add relation}\begin{align*}
			x(n) &= \res\left[
			\frac{z^n}{z-a} \ @ \ z=a
			\right] = \frac{1}{(s-1)^1} \left. \frac{\dif}{\dif \, z^{-1}} z^n \right|_{z=a}
			\\ &= \cdots
		\end{align*}
		\todo{complete}
		\item \textbf{Αν \( n < 0 \)}, τότε:
		\todo{συμπλήρωση}
		\begin{align*}
			x(n) &= \res\left[
			\frac{1}{(z-a)z^{-n}}\ @ \ z=a
			\right] = \left.\frac{1}{z^{-n}}\right|_{z=a} = a^n
			\\ &+ \res\left[
			\frac{1}{(z-a)z^{-n}}\ @ \ z=0
			\right] = \frac{1}{(k-1)!}\left.\od[k-1]{}{z} \frac{1}{z-a}\right|_{z=0}
			\\ &= a^n + \frac{1}{(k-1)!}(-1)^{k-1}\left.\frac{1}{(z-a)^k}\right|_{z=0}
			\\ &= a^n + (-1)^{k-1} \frac{1}{(-a)^k}
			= a^n + (-1)^{k-1}(-a)^n = a^n - a^n = 0
		\end{align*}
	\end{itemize}

    Τελικά, το αποτέλεσμα στο οποίο καταλήξαμε είναι το εξής:
    \[
    \left.\begin{aligned}
    n \geq 0 &\quad x(n)=a^n\\
    n < 0 &\quad x(n) = 0
    \end{aligned} \right\rbrace \implies x(n) = a^n\mathrm{u}(n)
    \]
    ένα αποτέλεσμα ίδιο με αυτό της "αντίστροφης" άσκησης που λύσαμε προηγουμένως.
    \todo{add ref}
\end{exercise}

\begin{exercise}
	Να βρεθεί το αρχικό σήμα του μετασχηματισμού:
	\[
	X(z) = \frac{z(z-b) + z(z-a)}{(z-a)(z-b)}
	\qquad \mathcal{R}_x : |a|<|z|<|b|
	\]
	\tcblower
	Αν και η άσκηση λύνεται άμεσα μετά την παρατήρηση ότι \( X(z) = \frac{z}{z-a}+\frac{z}{z-b} \), εδώ
	θα χρησιμοποιήσουμε τη μεθοδολογία των ολοκληρωτικών υπολοίπων.
	
	Έχουμε:
	\begin{align*}
		X(z)z^{n-1} &= \frac{(2z-a-b)}{(z-a)(z-b)}z^n
		\intertext{Άρα:}
		x(n) &= \frac{1}{2\pi j} \ointctrclockwise_C \frac{(2z-a-b)}{(z-a)(z-b)}z^n\dif z
		= \frac{1}{2πj}\ointctrclockwise_C I(z)\dif z
	\end{align*}
	όπου ορίσαμε για ευκολία \( I(z) = \frac{(2z-a-b)}{(z-a)(z-b)}z^n \).
	
	Το \( I(z) \) έχει πόλους \( z \) πολλαπλότητας \( s \):
	\[ I(z) \rightarrow
	\begin{cases}
	z_0 = a &\quad s_0=1\\
	z_1 = b &\quad s_1=1
	\end{cases}
	\]
	
	\begin{itemize}
		\item \textbf{Αν \( n\geq 0 \)}, τότε:
		\[
		x(n) = \res\left[
		I(z) \ @ \ z=a
		\right] =\left. \frac{z^n(2z-a-b)}{(z-b)}\right|_{z=a}
		= \frac{a^n(2a+a-b)}{(a-b)} = a^n
		\]
		
		Εδώ δεν συμπεριλάβαμε τον πόλο \( z_1=b \) επειδή είναι \emph{εκτός} του χωρίου που περικλείεται
		από την περιοχή σύγκλισης.
		
		Επομένως: \[
		x(u) = a^n \quad \text{όταν } n \geq 0
		\]
		\item \textbf{Αν \( n < 0 \)}, τότε οι πόλοι του \( I(z) \), λόγω του όρου \( z^n \), είναι:
		\[
		I(z) \rightarrow \begin{cases}
		z_0 = a &\quad s_0 = 1\\
		z_1 = b &\quad s_1 = 1\\
		z_2 = 0 &\quad s_2 = -n = k
		\end{cases}
		\]
		
		Άρα η ακολουθία είναι:
		\begin{align*}
			x(n) &= \res\left[
			I(z) \ @ \ z=a
			\right] + \res\left[
			I(z) \ @ \ z=0
			\right]
			\intertext{επειδή $a,0 \in \mathop{int}(C)$, ενώ $b \notin \mathop{int}(C)$}
			&= \left. \frac{2z-a-b}{z^{-n}(z-b)} \right|_{z=a}
			+ \frac{1}{(-n-1)!}\left. \od[-n-1]{}{z} \left( \frac{2z-a-b}{(z-a)(z-b)} \right)\right|_{z=0}
		\end{align*}
		
		Για να υπολογίσουμε τη "δύσκολη" παραπάνω παράγωγο, εφαρμόζουμε ένα τρικ. Θέτουμε \( z = \sfrac{1}{ρ}  \), οπότε \( \dif z = -\frac{1}{ρz}\dif ρ \), και η περιοχή σύγκλισης γίνεται
		\( |a|^{-1} > |ρ|> |b|^{-1} \). Και τότε:
		\begin{align*}
		I(z)\dif z &= -\frac{ρ^{-n}(2ρ^{-1}-a-b)}{\left( ρ^{-1}-a \right)\left( ρ^{-1}-b \right)}\frac{1}{ρ^2} \dif\rho
		\\ &= -\frac{ρ^{-n}(2ρ^{-1}-a-b)}{(1-aρ)(1-bρ)}\dif ρ
		\end{align*}
		
		Άρα τελικά:
		\begin{align*}
			x(n) &= \frac{1}{2πj} \ointclockwise_C \left[
			-\frac{ρ^{-n}(2ρ^{-1}-a-b)}{(1-aρ)(1-bρ)}
			\right]
			\dif ρ
		\end{align*}
		
		Στο νέο υπολογισμό με βάση το \( ρ \) και όχι το \( z \), έχει αλλάξει το επίπεδο όπου προβάλλουμε
		τους πόλους. Η περιοχή σύγκλισης μοιάζει πάλι με donut, αλλά η καμπύλη \( C \) διαγράφεται τώρα
		ωρολογιακά και όχι αντιωρολογιακά, λόγω της αλλαγής μεταβλητής \( \dif z = -\frac{1}{ρz}\dif ρ \).
		
		Άρα:
		\begin{align*}
			x(n) &= \frac{1}{2πj} \ointctrclockwise_C \left[
			\frac{ρ^{-n}(2ρ^{-1}-a-b)}{(1-aρ)(1-bρ)}
			\right]
			\dif ρ
			\\ &= \res\left[
			-\frac{ρ^{-n}(2ρ^{-1}-a-b)}{(1-aρ)(1-bρ)} \ @ \ \rho = \frac{1}{b}
			\right]
			\\ &= \frac{1}{ab}\left[
			   \frac{ρ^{-n-1}\left[ 2-(a+b)ρ \right]}{\left( ρ-\frac{1}{a} \right)\left(ρ-\frac{1}{b}\right)}
			\right]\ @ \ \rho = \frac{1}{b}
			\\ &= \left.\frac{1}{ab}\left[
			\frac{ρ^{-n-1}\left( 2-(a+b)ρ \right)}{\left( ρ-\frac{1}{a} \right)}
			\right]\right|_{\rho = \sfrac{1}{b} }
			\\ &= \cdots = -b^n
		\end{align*}
		
		Άρα τελικά:
		\[
		x(n) = -b^n \text{ για } n<0
		\]
	\end{itemize}

    Επομένως, η ζητούμενη ακολουθία είναι:
    \[
    x(n) = a^n\mathrm{u}(n) - b^n \mathrm{u}(-n-1)
    \]
\end{exercise}

\subsubsection{Μέθοδος συνεχούς διαίρεσης}
\begin{exercise}
	Δίνεται ο Μ/Σ Z:
	\[
	X(z) = \frac{2z^2 - (a+b)z}{z^2 - (a+b)z+ab}
	\qquad |a|<|b|<|z|
	\]

	Να βρεθεί η \( x(n) \).
	\tcblower
	\begin{comment}
	Έχουμε:
	\begin{align*}
		X(z) &= \frac{z}{z-a} + \frac{z}{z-b}
	\end{align*}
	Επομένως:
	\[
	x(n) = \left(a^n+b^n\right)\mathrm u(n)
	\]
	\end{comment}
	
	\todo{Graph 19 (optional)}
	Η μέθοδος αυτή απαιτεί ένα είδος διαίρεσης των πολυωνύμων \( 2z^2 - (a+b)z \)
	και \( z^2-(a+b)z + ab \).
	Το πηλίκο θα προκύψει \( 2 + (a+b)z^{-1} + (a^2+b^2)z^{-2} + (a^3+b^3)z^{-3} + \dots \), κάτι
	που μοιάζει με τον ορισμό του μετασχηματισμού Z: \( \sum x(n)z^{-n} \), επομένως η \( x(n) = \left(a^n+b^n\right)\mathrm{u}(n) \).
	
	Αντίστοιχα, για περιοχή σύγκλισης \( |z|<|a|<|b| \), θα βρίσκαμε \( x(n)=-\left(a^n+b^n\right)\mathrm{u}(-n-1) \). Όμως δεν μπορούμε να χρησιμοποιήσουμε τη μέθοδο αυτήν
	για περιοχές της μορφής \( |a|<|z|<|b| \).
\end{exercise}

\lecture{6}{19/10/2018}
\subsection{Ιδιότητες του μετασχηματισμού Z}
\begin{enumpar}
	\item \textbf{Γραμμικότητα}
	
	Έστω οι \( N \) σε πλήθος ακολουθίες \( x_k(n) \) και οι μετασχηματισμοί Z τους:
	\begin{align*}
		x_k(n) &\xrightarrow{\mathcal{Z}T} X_k(z) \qquad k=1,2,\dots,N
		\intertext{και έστω μία ακολουθία $y(n)$ που είναι γραμμικός συνδυασμός των παραπάνω:}
		y(n) &= \sum_{k=1}^{N} a_kx_k(n) \qquad a_k \in \mathbb{C} \text{ σταθερές}
		\intertext{με μετασχηματισμό Z:}
		Y(z) &= \sum_{k=1}^{N} a_k X_k(z)
		\intertext{ο οποίος έχει περιοχή σύγκλισης:}
		\mathcal{R}_{Y(z)} &\supseteq \bigcap_{k=1}^N \mathcal{R}_{X_k(z)}
	\end{align*}
	δηλαδή η \emph{περιοχή σύγκλισης}
	του γραμμικού συνδυασμού είναι \emph{τουλάχιστον} η τομή όλων των περιοχών σύγκλισης των
	επιμέρους ακολουθιών. Ο όρος \emph{τουλάχιστον} (ή αντίστοιχα το σύμβολο \( \supset \)) εκφράζουν
	ότι η περιοχή σύγκλισης της \( Y(z) \) μπορεί να είναι μεγαλύτερη των επιμέρους. Αυτό μπορεί για
	παράδειγμα να συμβαίνει όταν προσθέτουμε τις ακολουθίες με \( \mathcal{Z}\mathrm{T} \): \( \frac{1}{z-1} - \frac{1}{z-1} = 0 \), ή τις ακολουθίες \( \frac{z}{z-1}-\frac{1}{z-1}=\frac{z-1}{z-1}=1 \) όπου απαλείφεται ο πόλος.
	
	\item \textbf{Μετατόπιση}
	
	Έστω η ακολουθία \( x(n) \) και ο μετασχηματισμός της:
	\begin{align*}
		x(n) &\xrightarrow{\mathcal{Z} T} X(z) \qquad \mathcal{R}_X
		\intertext{και έστω ότι τη μετατοπίζουμε κατά \( n_0 \in \mathbb Z \) (είναι απαραίτητο ο \( n_0 \) να είναι \textbf{ακέραιος})}
		\infoboxed{y(n) = x(n-n_0)} &\xrightarrow{\mathcal{Z} T} \infoboxed{z^{-n_0} X(z)}
	\end{align*}
	
	\paragraph{Απόδειξη}
	Έχουμε: \todo{ROC?}
	\begin{align*}
		Y(z) &= \mathcal{Z}T \left\lbrace y(n) \right\rbrace
		= \sum_{n=-\infty}^{\infty} x(n-n_0)z^{-n} \underset{n-n_0 \leftarrow n}{=} \sum_{n=-\infty}^{\infty} x(n) z^{-(n+n_0)}
		\\ &= z^{-n_0}\sum_{n=-\infty}^{\infty} x(n)z^{-n}
		= z^{-n_0}X(z)
	\end{align*}
	
	Η περιοχή σύγκλισης εξαρτάται από το ποιούς πόλους εξαφανίζει και εμφανίσει το \( z^{-n_0} \)
	
	\item \textbf{Πολλαπλασιασμός με εκθετική ακολουθία (εκθετικό σήμα)}
	
	Δίνεται η ακολουθία:
	\begin{align*}
		x(n) &\xrightarrow{\mathcal{Z}T} X(z) \qquad \mathcal{R}_x : r_- < |z| < z_+
		\intertext{και την πολλαπλασιάζουμε με ένα εκθετικό $a^n$ όπου $a\in\mathbb{C}$:}
		\infoboxed{y(n) = a^n x(n)} &\xrightarrow{\mathcal{Z}T} \infoboxed{Y(z) = X\left(\frac{z}{a}\right)}
	\end{align*}
	
    \paragraph{Απόδειξη} Έχουμε:
    \begin{align*}
    	Y(z) &= \sum_{n=-\infty}^{\infty} a^n x(n) z^{-n} =
    	\sum_{n=-\infty}^{\infty} x(n) \left(\frac{z}{a}\right)^{-n}
    	= X\left(\frac{z}{a}\right)
    \end{align*}
    
    Για να εκτελέσουμε την τελευταία πράξη, πρέπει να ισχύει:
    \begin{gather*}
    	\left|\frac{z}{a} \right|  \in \mathcal{R}_x \implies
    	r_- < \left|\frac{z}{a}\right| < r_+ \implies
    	\boxed{|a|r_- < |z| < |a|r_+}
    \end{gather*}
    
    \item \textbf{Παραγώγιση του μετασχηματισμού}
    \[
    \infoboxed{nx(n) \xrightarrow{\mathcal{Z}T} -z\od{X(z)}{z}}
    \]
    
    \paragraph{Απόδειξη}
    \begin{align*}
    	\mathcal{Z}T\left\lbrace nx(n) \right\rbrace
    	&= \sum_{n=-\infty}^{\infty} nx(n)z^{-n}
    	= \sum_{n=-\infty}^{\infty} x(n) \left( -z \od{z^{-n}}{z} \right)
    	\\ &= -z\od{}{z}\left(\sum_{n=-\infty}^{\infty} x(n)z^{-n}\right)
    	= -z\od{}{z}X(z)
    \end{align*}
    
    \item \textbf{Χρονική αντιστροφή}
    
    Για την ακολουθία \( x(n) \):
    \begin{align*}
    	x(n) &\xrightarrow{\mathcal{Z}T} X(z) \qquad \mathcal{R}_x: r_- < |z| < r_+\\
    	\infoboxed{y(n) = x(-n)} &\xrightarrow{\mathcal{Z}T} \infoboxed{Y(z) = X(z^{-1})}
    \end{align*}
    
    \subparagraph{Απόδειξη}
    Έχουμε:
    \begin{align*}
    	Y(z) &= \sum_{n=-\infty}^{\infty} x(-n)z^{-n}
    	\underset{-n \leftarrow n}{=} \sum_{n=-\infty}^{\infty} x(n)z^{n}
    	= \sum_{n=-\infty}^{\infty} x(n) \left(z^{-1}\right)^n
    	\\ &= X\left( z^{-1} \right)
    \end{align*}
    
    Με την απαίτηση για το πεδίο σύγκλισης:
    \[
    r_- < |z^{-1}| < r_+ \implies \boxed{\frac{1}{r_+} < |z| < \frac{1}{r_-}}
    \]
    
    \item \textbf{Συνέλιξη στο χρόνο}
    
    Έχουμε τη συνέλιξη δύο ακολουθιών \( x \) και \( y \):
    \begin{alignat*}{2}
    	x(n) &\xrightarrow{\mathcal{Z}T} X(z), &&\qquad \mathcal{R}_x\\
    	y(n) &\xrightarrow{\mathcal{Z}T} Y(z), &&\qquad \mathcal{R}_y\\
    	\infoboxed{w(n) = x(n)*y(n)} &\xrightarrow{\mathcal{Z}T} \infoboxed{W(z) = X(z)Y(z)}, &&\qquad \boxed{\mathcal{R}_W \supseteq \mathcal{R}_x \cap \mathcal{R}_y}
    \end{alignat*}
    
    \item \textbf{Γινόμενο σημάτων}
    Έχουμε τον πολλαπλασιασμό δύο ακολουθιών \( x \) και \( y \):
    \begin{alignat*}{2}
    x(n) &\xrightarrow{\mathcal{Z}T} X(z), &&\qquad \mathcal{R}_x\\
    y(n) &\xrightarrow{\mathcal{Z}T} Y(z), &&\qquad \mathcal{R}_y
    \end{alignat*}
    
    Για τον υπολογισμό του στο πεδίο \( Z \), υπολογίζουμε, με βάση τον αντίστροφο μετασχηματισμό
    Z: \todo{add ref}
    \begin{align*}
    W(z) &= \sum_{n=-\infty}^{\infty} x(n)y(n)z^{-n} 
    = \sum_{n=-\infty}^{\infty} x(n) \frac{1}{2πj} \ointctrclockwise\limits_{C:υ\in\mathcal{R}_Y} Y(υ)υ^{n-1}\dif υ z^{-n}
    \\ &= \frac{1}{2πj} \ointctrclockwise_{C:υ\in\mathcal{R}_Y}
    Y(υ) υ^{-1} \sum_{n=-\infty}^{\infty} x(n) υ^n z^{-n} \dif υ\\
    &\underset{\frac{Z}{υ} \in \mathcal{R}_x}{=} \frac{1}{2πj} \ointctrclockwise_{C:υ\in\mathcal{R}_Y} Y(υ)υ^{-1} X\left(\frac{z}{υ}\right)\dif υ
    \\
    W(z) &= \frac{1}{2πj} \ointctrclockwise_{C:\stackrel{υ\in\mathcal{R}_Y}{\frac{z}{υ}\in\mathcal{R}_x}}
    Y(υ)υ^{-1} X\left( \frac{z}{υ} \right)\dif υ
    \end{align*}
    
    Δεδομένων των περιορισμών, πρέπει:
    \[
    \left.
    \begin{matrix}
    r_-^Y < |υ| < r_+^Y\\
    r_-^X < \left|\frac{z}{υ}\right| < r_+^X
    \end{matrix}
    \right\rbrace
    \mathcal{R}_W: r_-^Xr_-^Y < |z| < r_+^Yr_+^X
    \]
\end{enumpar}

\begin{exercise}
	Να υπολογιστεί ο μετασχηματισμός Z του:
	\[
	x(n) = δ(n)
	\]
	\tcblower
	Έχουμε:
	\[
	X(z) = \sum_{n=-\infty}^{\infty} δ(n)z^{-n} = 1 \quad \forall z\in \mathbb C 
	\]
\end{exercise}

\begin{exercise}
	Να υπολογιστεί ο μετασχηματισμός Z του:
	\[
	x(n) = δ(n-n_0)
	\]
	\tcblower
	Έχουμε:
	\[
	X(z) = \sum_{n=-\infty}^{\infty} δ(n-n_0)z^{-n} = z^{-n_0}
	\]
	
	Η περιοχή σύγκλισης εξαρτάται από το \( n_0 \) και είναι:
	\[
	\begin{cases}
	n_0 > 0 &\quad 0 < |z|\\
	n_0 < 0 &\quad |z| < \infty
	\end{cases}
	\]
\end{exercise}

\begin{exercise}
	Να βρεθεί ο Μ/Σ Z:
	\[
	x(n) = u(n)
	\]
	\tcblower
	Έχουμε:
	\begin{align*}
		X(z) &= \sum_{n=-\infty}^\infty u(n)z^{-n} = \sum_{n=0}^{\infty} z^{-n}
		= \frac{1}{1-z^{-1}}
	\end{align*}
	
	Πρέπει \( |z^{-1}| < 1 \), άρα η περιοχή σύγκλισης είναι ROC: \( |z|>1 \).
\end{exercise}

\begin{exercise}
	Βρείτε Μ/Σ Z:
	\[
	x(n) = u(n-n_0)
	\]
	\tcblower
	Εκμεταλλευόμενοι την ιδιότητα του μετασχηματισμού Z, έχουμε:
	\begin{align*}
		X(z) &=z^{-n_0} \cdot \mathcal{Z}T\left\lbrace u(n) \right\rbrace
		= z^{-n_0}\frac{z}{z-1}
		\\ &= \frac{z^{-n_0+1}}{z-1}
	\end{align*}
	
	Για την περιοχή σύγκλισης, έχουμε \( |z|>1 \) από το μετασχηματισμό της
	\( \mathrm{u}(t) \).
	\begin{itemize}
		\item Όταν ο εκθέτης είναι αρνητικός, δηλαδή \( -n_0+1 < 0 \iff n_0 > 1 \), έχουμε
		πόλους στο 0 και στο 1. Επειδή αποκλείεται (λόγω του \( |z|>1 \)) να
		βρισκόμαστε μέσα από τον μοναδιαίο κύκλο, αναγκαστικά το \( \mathcal{R}_X \) θα
		είναι \( |z|>1 \).
		\item Για \( -n_0 + 1 > 0 \iff n_0 < 1 \), έχουμε τον περιορισμό \( 1<|z|<\infty \) όταν
		\( n_0 < 0 \). \todo{Explain}
	\end{itemize}
\end{exercise}

\begin{exercise}
	Αν:
	\[
	X(z) = \frac{z}{3z^2-4z+1}
	\]
	να βρεθεί η \( x(n) \) για κάθε πιθανή ROC.
	\tcblower
	Πρώτα πρέπει να βρούμε τους \textbf{πόλους} της συνάρτησης, για να υπολογίσουμε τις πιθανές ROC.
	
	Έχουμε:
	\begin{align*}
	X(z) &= \frac{z}{3\left(z^2-\frac{4}{3}z+\frac{1}{3}\right)} = \frac{z}{3(z-1)\left(z-\frac{1}{3}\right)}\\
	&= \frac{1}{2}\frac{z}{z-1} - \frac{1}{2}\frac{z}{z-\sfrac{1}{3} }
		\end{align*}
	
	Επομένως έχουμε πόλους στα σημεία \( 1,\frac{1}{3} \).
	
	\todo{Graph 20}
	
	
	%Για να υπολογίσουμε τα παρακάτω, χρησιμοποιούμε τους μετασχηματισμούς:
	%\begin{align*}
%		content...
	%\end{align*}
	\todo{Add samples}
	
	\begin{itemize}
		\item \textbf{Αν ROC: \( |z|>1 \)}
		
		Τότε, δεδομένου του Μ/Σ Z της \( u(n) \) (\( U(z) = \frac{1}{1-\sfrac{1}{z} }, |z|>1 \)) έχουμε:
		\todo{fix}
		\[
		x(n) = \frac{1}{2} \mathrm{u}(n) - \frac{1}{2}
		-\frac{1}{2} \left(\frac{1}{3}\right)^n \mathrm{u}(n)
		\]
		\item \textbf{Αν ROC: \( |z| < \frac{1}{3} \)}:
		\[
		x(n) = -\frac{1}{2}\mathrm{u}(-n-1) + \frac{1}{2}\left(\frac{1}{3}\right)^n \mathrm{u}(-n-1)
		\]
		\item \textbf{Αν ROC: \( \frac{1}{3} < |z| < 1 \)}:
		\[
		x(n) = -\frac{1}{2} \mathrm{u}(-n-1) - \frac{1}{2}\left(\frac{1}{3}\right)^n \mathrm{u}(n)
		\]
	\end{itemize}
\end{exercise}

\begin{exercise}
	Να βρεθεί η \( x(n) \) του μετασχηματισμού Z:
	\[
	X(z) =\frac{z+1 }{3z^2-4z+1} \qquad |z|>1
	\]
	\tcblower
	Έχουμε:
	\begin{align*}
		\frac{X(z)}{z} &= \frac{z+1}{3z(z-1)\left(z-\frac{1}{3}\right)}
		= \frac{A}{z} + \frac{B}{z-1} + \frac{\Gamma}{z-\sfrac{1}{3} }
		\intertext{όπου προκύπτει \( A=1,\quad B=1,\quad \Gamma = -2 \), άρα:}
		X(z) &= 1 + \frac{z}{z-1} - 2\frac{z}{z-\sfrac{1}{3} } \implies\qquad |z| > 1 \\
		x(n) &= δ(n) + u(n) - 2\left(\frac{1}{3}\right)^n \mathrm{u}(n)
	\end{align*}
	
	\paragraph{}
	Εναλλακτικά, χωρίς να διαιρέσουμε αυθαίρετα με το \( z \):
	\begin{align*}
		X(z) &= \frac{z+1}{3z^2-4z+1} = \frac{z}{3z^2 - 4z+1} + \frac{}{3z^2-4z+1}\implies \qquad |z|>1
		\\ x(n) &= \frac{1}{2}\mathrm{u}(n) - \frac{1}{2}\left(\frac{1}{3}\right)^n \mathrm{u}(n)
		+ \frac{1}{2}\mathrm{u}(n-1) -\frac{1}{2}\left(\frac{1}{3}\right)^{n-1} \mathrm{u}(n-1)
	\end{align*}
	
	Εδώ, αν και φαινομενικά βγάλαμε διαφορετικά αποτελέσματα με τους δύο διαφορετικούς τρόπους προσέγγισης
	του προβλήματος, στην πραγματικότητα τα αποτελέσματα είναι \emph{ίδια}. Πράγματι, θέτοντας \( n=0,n=1,\dots \) θα πρέπει να πάρουμε το ίδιο αποτέλεσμα.
\end{exercise}

\begin{exercise}
	Να βρεθεί ο αντίστροφος μετασχηματισμός Laplace του:
	\[
	X(z) = \frac{z^4+z^2}{\left(z-\frac{1}{2}\right)\left(z-\frac{1}{4}\right)}
	\qquad \frac{1}{2}<|z|<\infty
	\]
	\tcblower
	Αυθαίρετα διαιρούμε με το \( z \), και έχουμε:
	\begin{align*}
		\frac{X(z)}{z} &= \frac{z^3+z}{\left(z-\frac{1}{2}\right)\left(z-\frac{1}{4}\right)}
		= z + \frac{3}{4} + \frac{\frac{23}{16}z - \frac{3}{32}}{z^2-\frac{3}{4}z+\frac{1}{8}}
		\\ &= z + \frac{3}{4} + \frac{\sfrac{5}{2} }{z-\sfrac{1}{2} } - \frac{\sfrac{17}{6} }{z-\sfrac{1}{4} }
		\\ &= z^2 + \frac{3}{4}z + \frac{5}{2}\frac{z}{z-\sfrac{1}{2} } - \frac{17}{16}\frac{z}{z-\sfrac{1}{4} } \implies 
		\hspace{60pt} \infty > |z| > \frac{1}{2}
		\\
		x(n) &= δ(n+2) + \frac{3}{4} δ(n+1) + \frac{5}{2} \left(\frac{1}{2}\right)^n \mathrm{u}(n)
		-\frac{17}{16} \left(\frac{1}{4}\right)^n \mathrm{u(n)}
	\end{align*}
\end{exercise}

\begin{exercise}
	Να βρείτε το μετασχηματισμό Z της ακολουθίας:
	\[
	x(n) = (n-2)a^{(n-2)}\cos\left( \omega_0 (n-2) \right)
	\mathrm{u}(n-2)
	\]
	\tcblower
	Έχουμε:
	\begin{align*}
		\mathcal{Z}\left\lbrace x(n) \right\rbrace&= z^{-2}
		\mathcal{Z} \left\lbrace n a^n \cos\left(\omega_0 n\right) \mathrm{u}(n)\right\rbrace
		\\ &= z^{-2} \left[
		-z \od{}{z} \left[ \mathcal{Z}
		  \left\lbrace a^n\cos(\omega_0 n)\mathrm{u}(n) \right\rbrace
		 \right]
		\right]
		\\ &= z^{-2}\left[
		  -z \od{}{z}\left.\mathcal{Z}\left\lbrace \cos(\omega_0 n)\mathrm{u}(n) \right\rbrace\right|_{z=\frac{z}{a}}
		\right]
	\end{align*}
	
	Θέτουμε \( \phi(n) = \cos(\omega_0 n)\mathrm{u}(n) \), και θα βρούμε την \( \Phi(z) \). Έχουμε:
	\begin{align*}
		\mathrm{u}(n) \cos(\omega_0 n) &= \frac{e^{j\omega_0 n} + e^{-j\omega_0 n}}{2} \mathrm u(n)
		\\
		\mathrm{u(n)} e^{j\omega_0 n }&\xrightarrow{\mathcal{Z} T} \frac{1}{1-\frac{e^{j\omega_0}}{z}}
		= \frac{z}{z-e^{j\omega_0}} \qquad \left| \frac{e^{j\omega_0}}{z} \right| < 1 \implies |z|>1
		\\
		\mathrm{u}(n) e^{-j\omega_0 n}&\xrightarrow{\mathcal{Z}} \frac{z}{z-e^{-j\omega_0}} \qquad |z|>1
		\\
		\mathrm{u}(n) \cos(\omega_0 n)&\xrightarrow{\mathcal{Z}} \frac{1}{2}\left[
		\frac{z}{z-e^{j\omega_0}} + \frac{z}{z-e^{-j\omega_0}}
		\right]
		= \frac{z^2-z\cos\omega_0}{z^2-2z\cos\omega_0 + 1}\qquad |z|>1
	\end{align*}
	
	Άρα τελικά:
	\begin{align*}
		X(z) &= -z\od{}{z} \left(
		\frac{z^2-z\cos\omega_0}{z^2-2z\cos\omega_0+1}\middle|_{z=\frac{z}{a}}
		\right)
		\qquad |z|>|a|
	\end{align*}
\end{exercise}

\end{document}
