% !TeX program = xelatex
\documentclass[11pt,a4paper,notitlepage,fleqn,final]{article}

\usepackage{amsmath}
\usepackage{amsfonts}
\usepackage{amssymb}
\usepackage{libs/commath2}
\usepackage[table]{xcolor}
\usepackage[hidelinks,draft=false]{hyperref}
\usepackage[skins,theorems]{tcolorbox}
\usepackage{titlesec}
\usepackage{tikz}
\usepackage{libs/circuitikz} % use our own recent version to make sure some bugs are fixed
\usepackage{pgfplots}
\usepackage{mathtools}
\usepackage[makeroom]{cancel}
\usepackage{mathrsfs}
\usepackage{wrapfig}
%\usepackage{subcaption}
%\usepackage{floatrow}
\usepackage{esint}
\usepackage{enumitem}
%\usepackage{bm}
\usepackage{relsize}
\usepackage{xfrac}
\usepackage{comment}
\usepackage{siunitx}
\usepackage{multicol}
%\usepackage{MnSymbol}
\usepackage[obeyDraft,disable]{todonotes}
%\usepackage{morefloats} % oh no!
%\usepackage[linesnumbered,lined]{algorithm2e}
\usepackage{glossaries}
\usepackage{xifthen}


\pgfplotsset{compat=1.13}
\usetikzlibrary{arrows.meta}
\usetikzlibrary{patterns}
\usetikzlibrary{decorations.pathmorphing}
\usetikzlibrary{decorations.markings}
\usetikzlibrary{backgrounds}
\usetikzlibrary{shapes.misc}
\usetikzlibrary{shapes.multipart}
\usetikzlibrary{shadows.blur}
\usetikzlibrary{fadings}
\usetikzlibrary{intersections}
\usetikzlibrary{arrows.meta}
\usetikzlibrary{calc}
\usetikzlibrary{matrix}
\usetikzlibrary{positioning}
\usetikzlibrary{shapes}
\usetikzlibrary{shadings}

\tcbuselibrary{breakable}
\tcbuselibrary{skins}
\tcbuselibrary{xparse}

\tikzset{cross/.style={cross out, draw,
        minimum size=2*(#1-\pgflinewidth),
        inner sep=0pt, outer sep=0pt}}
\tikzset{
    mark position/.style args={#1(#2)}{
        postaction={
            decorate,
            decoration={
            	post length=1mm, % ??? Magic to fix "Dimension
            	pre length=1mm, % ???  too large" errors.
                markings,
                mark=at position #1 with \coordinate (#2);
            }
        }
    }
}
\tikzset{
	arrow at/.style args={#1}{
		postaction={
			decorate,
			decoration={
				post length=1mm, % ??? Magic to fix "Dimension
				pre length=1mm, % ???  too large" errors.
				markings,
				mark=at position #1 with {\arrow{>}};
			}
		}
	}
}
\makeatletter
\tikzset{
  use path for main/.code={%
    \tikz@addmode{%
      \expandafter\pgfsyssoftpath@setcurrentpath\csname tikz@intersect@path@name@#1\endcsname
    }%
  },
  use path for actions/.code={%
    \expandafter\def\expandafter\tikz@preactions\expandafter{\tikz@preactions\expandafter\let\expandafter\tikz@actions@path\csname tikz@intersect@path@name@#1\endcsname}%
  },
  use path/.style={%
    use path for main=#1,
    use path for actions=#1,
  }
}
\makeatother

\pgfmathdeclarefunction{sinc}{1}{%
	\pgfmathparse{abs(#1)<0.01 ? int(1) : int(0)}%
	\ifnum\pgfmathresult>0 \pgfmathparse{1}\else\pgfmathparse{sin(#1 r)/#1}\fi%
}
\pgfmathdeclarefunction{gauss}{2}{%
	\pgfmathparse{1/(#2*sqrt(2*pi))*exp(-((x-#1)^2)/(2*#2^2))}%
}

\usepackage[left=2cm,right=2cm,top=2cm,bottom=2cm]{geometry}

%\usepackage[no-math]{fontspec}
%\usepackage{fontspec}
\usepackage{mathspec}
%\usepackage{newtxtext,newtxmath}
%\usepackage{unicode-math}
%\setmainfont{texgyretermes-regular.otf}
%\setsansfont{texgyreheros-regular.otf}
%\newfontfamily\greekfont[Script=Greek]{Linux Libertine O}
%\newfontfamily\greekfontsf[Script=Greek]{Linux Libertine O}
\usepackage{polyglossia}
%\newfontfamily\greekfont[Script=Greek]{texgyretermes-regular.otf}
\newfontfamily\greekfontsf[Script=Greek]{texgyreheros-regular.otf}
\newfontfamily\greekfonttt[Script=Greek]{Latin Modern Mono}
%\usepackage[greek]{babel}
\setdefaultlanguage{greek}
\setotherlanguage{english}

%\usepackage[utf8]{inputenc}
%\usepackage[greek]{babel}


%\usepackage{tkz-euclide} % loads  TikZ and tkz-base
%\usetkzobj{angles} % important you want to use angles

\newlist{enumparen}{enumerate}{1}
\setlist[enumparen]{label=(\arabic*)}
\newlist{enumpar}{enumerate}{1}
\setlist[enumpar]{label=\arabic*)}

\newlist{enumgreek}{enumerate}{1}
\setlist[enumgreek]{label=\alph*.}
\newlist{enumgreekparen}{enumerate}{1}
\setlist[enumgreekparen]{label=(\alph*)}
\newlist{enumgreekpar}{enumerate}{1}
\setlist[enumgreekpar]{label=\alph*)}


\newlist{enumroman}{enumerate}{1}
\setlist[enumroman]{label=(\roman*)}

\newlist{enumlatin}{enumerate}{1}
\setlist[enumlatin]{label=(\alph*)}

\newlist{invitemize}{itemize}{1}
\setlist[invitemize]{noitemsep,label=}

\input{libs/fiximplies}
\input{libs/sphere}

\makeatletter
\let\anw@true\anw@false

%\newcommand{\attnboxed}[1]{\textcolor{red}{\fbox{\normalcolor\m@th$\displaystyle#1$}}}
\makeatother
\tcbset{highlight math style={enhanced,colframe=red,colback=white,%
        arc=0pt,boxrule=1pt,shrink tight,boxsep=1.5mm,extrude by=0.5mm}}
\newcommand{\attnboxed}[1]{\tcbhighmath[colback=red!5!white,drop fuzzy shadow,arc=0mm]{#1}}
\newcommand{\infoboxed}[1]{%
	\tcbhighmath[colframe=blue!50!white,colback=blue!5!white,arc=0mm]{#1}}
\titleformat{\section}{\bf\Large}{Κεφάλαιο \thesection}{1em}{}
\newtcolorbox{attnbox}[1]{colback=red!5!white,%
    colframe=red!75!black,fonttitle=\bfseries,title=#1}
\newtcbox{quickattnbox}[1]{colback=red!5!white,%
	colframe=red!75!black,fonttitle=\bfseries,title=#1}
\newtcolorbox{infobox}[1]{colback=blue!5!white,%
    colframe=blue!75!black,fonttitle=\bfseries,title=#1}

\tcbset{frogbox/.style={enhanced jigsaw,%
		overlay first={\foreach \x in {0cm} {
				\begin{scope}[shift={([xshift=-0.2cm]title.west)}]
					\draw[very thick,green!65!black!50!white,latex-] (0,0) -- ++(-1.5,0);
\end{scope}}}}}
\tcbset{frogtitle/.style={
attach boxed title to top left=
{xshift=0mm,yshift=-0.50mm},
boxed title style={skin=enhancedfirst jigsaw,
	bottom=0mm,
	interior style={fill=none,
		left color=green!20!black,
		right color=gray}}
}}
\DeclareTColorBox{exercise}{ O{} }{
	enhanced jigsaw,
	breakable,parbox=false,
	%title style={left color=gray!50!white!50!green,opacity=.5,right color=white},
	subtitle style={%boxrule=1pt,
		colback=yellow!50!red!25!white,fontupper=\bfseries},
	coltitle=black,colbacktitle=green!90!black!25!white,colframe=black,
	frame hidden,
	boxrule=0mm,
	%boxrule=1mm,
	leftrule=0.8pt,toprule=0.8pt,rightrule=0pt, %reserve space
	borderline west={0.8pt}{0pt}{white!25!black},%---- draw line
	borderline north={0.8pt}{0pt}{white!25!black},%---- draw line
	interior hidden,
	%frame style={left color=black,right color=white},
	sharp corners=all,
	%frogbox, %TODO: frogbox
	before lower={\tcbsubtitle[before skip=\baselineskip]{Λύση}},lower separated=false,
	before title={\textbf{Άσκηση\ifthenelse{\isempty{#1}}{}{: }}},
	title={\ifthenelse{\isempty{#1}}{\hspace{0pt}}{#1}}%
}

\AtBeginDocument{%
\let\arg\relax
\let\Re\relax
\let\Im\relax
\DeclareMathOperator{\arg}{Arg}
\DeclareMathOperator{\Re}{Re}
\DeclareMathOperator{\Im}{Im}
}
\DeclareMathOperator{\sinc}{sinc}
\DeclareMathOperator{\sgn}{sgn}
\DeclareMathOperator{\erf}{erf}
\DeclareMathOperator{\cov}{cov}
\DeclareMathOperator{\atand}{atan2}

\newenvironment{absolutelynopagebreak}
{\par\nobreak\vfil\penalty0\vfilneg
	\vtop\bgroup}
{\par\xdef\tpd{\the\prevdepth}\egroup
	\prevdepth=\tpd}

\DeclareSIUnit \voltampere { VA } %apparent power 
\DeclareSIUnit \var { VAr } %volt-ampere reactive - idle power 
\DeclareSIUnit \decade { dec } %decade

% Global amount of samples
% Set to a higher value (e.g. 200) for nicer graphs
% Set to a low value (e.g. 10) for performance
% NOTE: Check the sample variables below for further measurements
\newcommand*{\gsamples}{200}

% Equals command as a workaround for CircuiTikZ bug
% not allowing the = sign in labels
\newcommand*{\equals}{=}

\newcommand{\nesearrow}{%
	\,%
	\smash{\raisebox{-1.1ex}
		{$%
			\stackrel{\displaystyle\nearrow}{\displaystyle\searrow}%
			$}}%
}
\newcommand{\degree}{^{\circ}} % not great
\newcommand\numberthis{\addtocounter{equation}{1}\tag{\theequation}} % add an equation number to a number-less math environment

% Provided commands
\providecommand\dif{d}
\providecommand\od[2]{\frac{#1}{#2}}

\newtcbtheorem[number within=section,list inside=thm]{theorem}{Θεώρημα}%
{colback=green!5,colframe=green!35!black,colbacktitle=green!35!black,fonttitle=\bfseries,enhanced,attach boxed title to top left={yshift=-2mm,xshift=-7mm},width=.9\textwidth,arc=.7mm}{th}
\newtcbtheorem[number within=section,list inside=defn]{defn}{Ορισμός}%
{colback=blue!5,colframe=cyan!35!black,colbacktitle=blue!35!black,fonttitle=\bfseries,enhanced,attach boxed title to top left={yshift=-2mm,xshift=-2mm}}{def}

% Locus plot utilities
\tikzset{locus/.style={orange!50!red!70!brown}}
\tikzset{locuspole/.style={draw=red!30!black,cross,inner sep=2.5pt,fill=white,fill opacity=.6,thick,label={[below]-90:#1}}}
\tikzset{locuszero/.style={draw=red!30!black,circle,inner sep=2pt,fill=white,fill opacity=.6,thick,label={[below]-90:#1}}}
\tikzset{locusbreak/.style={rounded corners=1.5pt,inner sep=2pt,draw,top color=brown,bottom color=black,fill opacity=.8,label={[below]-90:#1}}}

% New plotting utilities
\def\lowsamples{18}
\def\hisamples{40}
\def\timecolour{blue!50!cyan}

\tikzstyle{timecolour}=[\timecolour]



\title{ΣΑΕ 2
	\\
	{ 
		\normalsize Συστήματα Αυτομάτου Ελέγχου II
		\\
		\normalsize Σημειώσεις από τις παραδόσεις
	}}
\date{Φεβρουάριος 2018
	\\
	{ 
		\small Τελευταία ενημέρωση: \today
	}
}
\author{
	Για τον κώδικα σε \LaTeX, ενημερώσεις και προτάσεις:
	\\
	\url{https://github.com/kongr45gpen/ece-notes}}

\setallmainfonts(Digits,Latin,Greek){Asana Math}
\setmainfont{Noto Serif}
\setsansfont{Ubuntu}
\usepackage{polyglossia}
\newfontfamily\greekfont[Script=Greek,Scale=1.00]{Liberation Serif}

\hypersetup{pdftitle = {ΣΑΕ 2}}


\begin{document}
\maketitle

\hrule
\vspace{50pt}

\begin{infobox}{Λάθη \& Διορθώσεις}
	Οι τελευταίες εκδόσεις των σημειώσεων βρίσκονται στο Github
	(\url{https://github.com/kongr45gpen/ece-notes/raw/master/sae1=2.pdf}) ή
	στη διεύθυνση \url{http://helit.org/ece-notes/sae2.pdf}.
	
	Περιέχουν διορθώσεις σε λάθη και τυχόν βελτιώσεις.
	
	\tcblower
	
	Μπορείτε να ενημερώνετε για οποιοδήποτε λάθος και πρόταση
	μέσω PM στο forum, issue στο Github, ή οποιουδήποτε άλλου τρόπου!
\end{infobox}
	
Το μάθημα περιλαμβάνει ένα μικρό προαιρετικό εργαστήριο. Η επιλογή γίνεται με βάση
προαιρετικής προόδου (μπαίνουν οι πρώτοι 18, εφ' όσον έχουν γράψει βαθμό \( \geq 6 \)).

Εφ' όσον δοθεί, η πρόοδος συμμετέχει κατά 25\% στον τελικό βαθμό
(και το υπόλοιπο 75\% στις εξετάσεις), για αυτούς που συμμετάσχουν
σε αυτήν. Διαφορετικά, μετράν οι εξετάσεις κατά 100\%. Το εργαστήριο μετράει προσθετικά
με μέγιστο βαθμό \( +3 \), εφ' όσον ο βαθμός της εξέτασης είναι από 4 και πάνω.

Αν δηλωθεί η πρόοδος, ο βαθμός της μετράει όπως παραπάνω, και δεν υπάρχει δυνατότητα να
ακυρωθεί, ακόμα και αν ο φοιτητής δεν την παραδώσει.

\newpage

\tableofcontents

\newpage

\subsection{Εισαγωγή}
Στα Συστήματα Αυτόματου Ελέγχου 2 οι αναλύσεις γίνονται στο πεδίο του \textbf{χρόνου}
και όχι της \textbf{συχνότητας}. Αυτό επιτρέπει να μελετηθούν συστήματα μη γραμμικά,
καθώς και συστήματα με περισσότερες από μία εισόδους και εξόδους.

Γενικότερα, τα ΣΑΕ έχουν εφαρμογές σε πολυάριθμους τομείς, όπως η αυτοκίνηση (φρένα
ABS, σύστημα πρόσφυσης, διατήρηση ευστάθειας σε πλαγιολίσθηση, \textellipsis), έλεγχος
κινητήρων, έλεγχος υπερμεγέθων τηλεσκοπίων, κατανομή πρόσβασης σε δίκτυα internet και
τηλεφωνικά, διαχείριση συστημάτων ενέργειας (για διανομή, ασφάλεια, αξιοπιστία, π.χ.
απόσβεση διαταραχών μετά από κεραυνό)\textellipsis

Θυμόμαστε ότι \textbf{\textit{σύστημα}} είναι οποιαδήποτε λειτουργική μονάδα που διεγείρεται
από κάποιες εισόδους, και επιστρέφει κάποιες εξόδους.

\begin{tikzpicture}
\draw (0,0) node[rectangle,draw,inner sep=15pt,minimum width=20pt] (r) {$S$};
\draw[<-] (r.west) -- ++(-2,0) node[above right] {είσοδος};
\draw[->] (r.east) -- ++(2,0) node[above left] {έξοδος};
\end{tikzpicture}

Ένα \textbf{ελεγχόμενο σύστημα} είναι τέτοιο ώστε να φροντίζουμε η έξοδος \( y \) να έχει μια
επιθυμητή τιμή \( r \), και το οποίο συχνά περιλαμβάνει και μια είσοδο διαταραχής
\( d \), που δεν μπορούμε να ελέγξουμε.

\begin{tikzpicture}
\draw (0,0) node[rectangle,draw,inner sep=15pt,minimum width=20pt] (r) {$S$};
\draw[<-] (r.west) ++ (0,-0.3) -- ++(-2,0) node[above right] {$u$};
\draw[<-] (r.west) ++ (0,0.3) -- ++(-2,0) node[above right] {$r$};
\draw[<-] (r.north) -- ++(0,1) node[midway,right] {$d(t)$};
\draw[->] (r.east) -- ++(2,0) node[above left] {$y$};
\end{tikzpicture}

Μια ακόμα χρήσιμη έννοια που μάθαμε είναι αυτή της ανάδρασης, στην οποία η έξοδος του συστήματος ανατροφοδοτείται στο σύστημα ως είσοδος, ίσως αφού περαστεί από έναν
ελεγκτή \( C \). Σε αυτά μπορούμε να προσθέσουμε μια μετρητική διάταξη \( M \) και έναν
ενεργοποιητή (actuator) που να μετατρέπει την έξοδο σε μορφή αποδεκτή από το σύστημα. Αυτά
είναι και τα \textbf{συστήματα κλειστού βρόχου}.

\begin{tikzpicture}
\draw (0,0) node[rectangle,draw,inner sep=15pt,minimum width=20pt] (r) {$S$};
\draw (-3,0) node[rectangle,draw,inner sep=10pt,minimum width=10pt] (c) {$C$};
\draw (-1.5,0) node[rectangle,draw,minimum height=30pt,minimum width=7pt] (e) {$E$};

\draw (-0.5,-1.5) node[rectangle,draw,minimum width=30pt,inner sep=8pt] (d) {$M$};

\draw[->] (e) -- (r) node[midway, above] {$u$};
\draw[->] (c) -- (e);
\draw[<-] (c.west) ++ (0,0.3) -- ++(-2,0) node[above right] {$r$};
\draw[<-] (r.north) -- ++(0,1) node[midway,right] {$d(t)$};
\draw[->] (r.east) -- ++(2,0) node[above left] {$y$} node[midway] (m) {};

\draw[->] (m.center) node[circ] {} |- (d) -- ++(-3.5,0) |- (c);
\end{tikzpicture}

Στόχος των μαθημάτων είναι ο σχεδιασμός του ελεγκτή \( C \) ώστε να ικανοποιούνται
συγκεκριμένες προδιαγραφές. Χρειάζεται βέβαια και μια διαισθητική κατανόηση των εννοιών.
Για παράδειγμα, αν έχουμε προδιαγραφή το σύστημα να έχει έξοδο \( 1 \) στη μόνιμη κατάσταση,
είναι προτιμότερο να φτάσει σε αυτήν με υπεραποσβεννύμενη απόκριση, παρά με ταλαντώσεις.
\todo{Graph 1}

Για τη \textbf{μοντελοποίηση} των συστημάτων μπορούμε είτε να υπολογίσουμε και να
αναλύσουμε τη φυσική λειτουργία του συστήματος, είτε να μελετήσουμε σύνολα εισόδων και
εξόδων ώστε να προβλέψουμε τη συμπεριφορά τους.

Στην πραγματικότητα βέβαια δεν θα μας δίνονται οι μαθηματικές προδιαγραφές, αλλά οι
φυσικές προδιαγραφές του συστήματος.

Υπάρχει μάλιστα η περίπτωση τα λειτουργικά κομμάτια των παραπάνω διατάξεων να μην
είναι συνδεδεμένα φυσικά μεταξύ τους, αλλά να βρίσκονται σε απόσταση, εισάγοντας
χρονικές καθυστερήσεις στη μεταφορά των σημάτων (π.χ. drones). Άλλα προβλήματα μπορεί
να είναι ο κβαντισμός των σημάτων (π.χ. για ασύρματη μεταφορά δεδομένων), περιορισμοί του
hardware ή του software.

\paragraph{Παράδειγμα}
Έστω ένα αυτοκίνητο μάζας \( m \) που κινείται σε δρόμο κλίσης \( \phi \) με ταχύτητα
\( y(t) \). Στο αυτοκίνητο ασκείται δύναμη οδήγησης \( u(t) \) και δύναμη του αέρα ανάλογη
με την ταχύτητα, με συντελεστή \( a \) (αεροδυναμικός συντελεστής).
Οι τριβές μεταξύ αυτοκινήτου και οδοστρώματος θεωρούνται αμελητέες.

Επιθυμούμε να σχεδιάσουμε έναν ελεγκτή \( u(t) \) που να ελέγχει τη δύναμη οδήγησης, ώστε
το αυτοκίνητο να κινείται με σταθερή ταχύτητα.

\subparagraph{Λύση}
\todo{Graph 2}

Από το νόμο του Νεύτωνα \( \sum F = m\ddot y \)έχουμε:
\[
m\dot y = u - ay -mg\sin\phi
\]
δηλαδή φτάσαμε σε μία διαφορική εξίσωση που μοντελοποιεί το πρόβλημα.

Παρατηρούμε μάλιστα τον όρο \( \left[-mg\sin\phi\right] \), ο οποίος δεν εξαρτάται από μεταβλητές που
μπορούμε να επηρεάσουμε, αλλά μόνο από την κλίση του δρόμου, που πιθανώς αλλάζει. Δηλαδή
αποτελεί την \textbf{είσοδο διαταραχών}.

Η επιθυμητή έξοδος του συστήματος που δίνεται ως είσοδο είναι \( y(t) = r \), και εδώ
θεωρούμε για παράδειγμα ότι \( r = \SI{25}{\meter/\second} \).

\begin{itemize}
	\item Ισχυρίζομαι ότι μπορώ να λύσω το πρόβλημα \textbf{χωρίς κλειστό βρόχο}, δηλαδή
	χωρίς να φτάνει στον ελεγκτή \( C \) η έξοδος \( y \):
	\[
	u(t) = kr(t)
	\]
	
	Και για απλότητα στους υπολογισμούς θεωρώ \( \phi = 0 \).
	
	Τότε, από το παραπάνω μοντέλο του συστήματος, προκύπτει η διαφορική εξίσωση:
	\[
	m\dot y = -ay + k\cdot25
	\]
	με λύση:
	\[
	y(t) = 25\frac{k}{a}\left(
	1-e^{-\frac{a}{m} t}
	\right),\quad t\geq 0
	\]
	ή, αν επιλέξουμε το κέρδος του ελεγκτή να είναι \( k = a \), η ταχύτητα θα γίνει
	όντως \( \SI{25}{\meter/\second} \). Όμως ο χρόνος αποκατάστασης εξαρτάται από τα
	\( a \) και \( m \), που είναι παράμετροι του ελεγχόμενου συστήματος, και δεν μπορούν
	να επηρεαστούν. Δηλαδή ο ρυθμός με τον οποίο συγκλίνουμε στο 25 εξαρτάται μόνο από τις
	παραμέτρους του ελεγχόμενου συστήματος. Συνήθως έχουμε μεγάλη μάζα \( m \) για το
	αυτοκίνητο και μικρό αεροδυναμικό συντελεστή. Επομένως το πηλίκο \( \frac{a}{m} \) είναι
	μικρό και η απόκριση λογικά αργή.
	
	Ακόμα δεν είναι γνωστή τις περισσότερες φορές η τιμή του συντελεστή \( a \). Αν για
	παράδειγμα ανοίξουμε ένα παράθυρο, μεταβάλλεται ανάλογα με το πόσο το έχουμε ανοίξει.
	Γενικότερα στη διαδικασία της σχεδίασης \textbf{δεν μπορούμε να χρησιμοποιήσουμε
		μεγέθη που δεν είναι γνωστά}.
	
	Επίσης, στην περίπτωση που θεωρήσουμε ότι \( \phi \neq 0 \), η λύση θα προκύψει μετά
	από πράξεις:
	\[
	y(t) = \underbrace{\left(\frac{25k}{a} - \frac{mg\sin\phi}{a}\right)}_{\text{θέλουμε } = 25}
	\left(  1-e^{-\frac{a}{m} t} \right)
	\]
	επομένως:
	\[
	k = \frac{25a + mg\sin\phi}{25}
	\]
	το οποίο πάλι δεν μπορεί να υπολογιστεί εύκολα, αφού η κλίση του δρόμου \( \phi \) κάθε
	φορά δεν είναι γνωστή.
	\item Χρησιμοποιούμε \textbf{αναλογικό ελεγκτή κλειστού βρόχου}:
	\[
	u(t) = k \cdot \Big( r(t) - y(t) \Big), \qquad k > 0 
	\]
	
	Τότε η λύση της διαφορικής εξίσωσης θα προκύψει, μετά από πράξεις:
	\[
	y(t) = \left(
	\frac{25k - mg\sin\phi}{a+k}
	\right)\left( 1-e^{-\left(\frac{a+k}{m}\right)t} \right)
	\]
	
	Όσο μεγαλώνουμε το \( k \), ο παράγοντας στον εκθέτη αυξάνεται, άρα μεγαλώνει η ταχύτητα
	σύγκλισης στη μόνιμη τιμή. Όσον αφορά τον πρώτο όρο, για \( k \) που φτάνει στο
	\( \infty \), έχουμε:
	\[
	\lim_{k \to \infty} \left( \frac{mg\sin\phi}{a+k} \right) = 0
	\]
	και ο πρώτος όρος γίνεται 25.
	
	Η μικρή διαφορά που υπάρχει για πεπερασμένες τιμές του \( k \) είναι ουσιαστικά το
	σφάλμα θέσης, που μπορούμε να εξαλείψουμε με έναν ολοκληρωτή.
\end{itemize}

\todo{Lesson 28/2/18, 2}

\subsection{Μοντελοποίηση Συστημάτων}
Η μοντελοποίηση συστημάτων, όπως αναφέραμε σε μια παράγραφο παραπάνω, γίνεται είτε με
φυσική μελέτη του συστήματος, είτε μελετώντας μερικές σχέσεις εισόδου-εξόδου, και εξάγοντας
συμπεράσματα από αυτές. Το αντικείμενο της μοντελοποίησης μελετάται εκτενώς σε μάθημα
επόμενου εξαμήνου, και εδώ θα κάνουμε μια εισαγωγή.

Τα πραγματικά συστήματα είναι από τη φύση τους πολύπλοκα, και επομένως ένα μοντέλο θα εισάγει
σχεδόν πάντα ένα \textbf{σφάλμα}. Το \textbf{σφάλμα μοντελοποίησης} \( e \) εκφράζει τη
διαφορά της τιμής \( \hat y \) που εξάγει το μοντέλο, από την τιμή \( y \) που εξάγει το
πραγματικό σύστημα.
\todo{Graph 3}
Στόχος της μοντελοποίησης είναι η ελαχιστοποίηση αυτού του σφάλματος, το οποίο εκφράζεται
μέσα από την προδιαγραφή της ακρίβειας.

Για παράδειγμα, μπορεί να θέλουμε να προσεγγίσουμε την έξοδο ενός πραγματικού συστήματος
σε μια "επικίνδυνη" είσοδο, την οποία είναι δύσκολο να εφαρμόσουμε στην πραγματικότητα,
αλλά εύκολο να θεωρήσουμε ως είσοδο στο μοντέλο.

Για τη μελέτη των συστημάτων χρησιμοποιούμε τις \textbf{μεταβλητές κατάστασης}, δηλαδή
τις μεταβλητές που είναι απαραίτητο να γνωρίζουμε για να περιγράψουμε πλήρως τη λειτουργία
του συστήματος. Ο ορισμός αυτός προέκυψε από τη μελέτη της κίνησης των πλανητών, η οποία
μπορεί να περιγραφεί από τη θέση και την ταχύτητα του καθενός.

Το σύνολο των τιμών των μεταβλητών κατάστασης ονομάζεται \textbf{σύνολο καταστάσεων}, και
κάθε τιμή (ή διάνυσμα καταστάσεων) ονομάζεται \textbf{κατάσταση}.

\paragraph{Παράδειγμα} Θεωρούμε ένα φυσικό σύστημα με ένα ελατήριο \( k \) και έναν αποσβεστήρα \( c \). Ασκούμε και
μία δύναμη \( u \). Σε αυτό θεωρούμε ότι η μεταβλητή \( q \) εκφράζει τη θέση του
σώματος. Τότε η χρονική παράγωγός της, \( \dot q \), εκφράζει την ταχύτητα του σώματος.
\todo{Graph 4}

Από το θεμελιώδη νόμο της μηχανικής (2\textsuperscript{ος} Νόμος Νεύτωνας):
\[
m\ddot q = -c(\dot q) - kq + u
\]
δηλαδή το \textbf{μοντέλο} του συστήματος είναι το:
\[
m\ddot q + c(\dot q) + kq = u
\]
Το σύστημα αυτό είναι αυτόνομο. Αν είχαμε κάποια εξωτερική δύναμη, θα ήταν μη αυτόνομο.
\begin{defn}{Αυτόνομα συστήματα}{}
	Συστήματα στα οποία \textbf{δεν} ενεργούν \textbf{εξωτερικές δυνάμεις} ονομάζονται
	\textbf{αυτόνομα}.
	
	Συστήματα στα οποία ενεργούν \textbf{εξωτερικές δυνάμεις}, οι οποίες λειτουργούν ως
	είσοδοι και μπορούμε να τις μεταβάλλουμε ώστε να αλλάξει η δυναμική συμπεριφορά του
	συστήματος, ονομάζονται \textbf{μη αυτόνομα}.
\end{defn}

\begin{defn}{Εξισώσεις κατάστασης}{}
	Γενικότερα, για τα συστήματά μας θα προκύπτει μία εξίσωση:
	\[
	F\left(
	q^{(n)},\ q^{(n-1)},\ \dots,\ \dot q,\ q,\ u
	\right) = 0
	\]
	
	Αυτή είναι διαφορική εξίσωση \textbf{τάξης \( n \)}, και μπορεί να αναλυθεί σε
	επιμέρους απλούστερες διαφορικές εξισώσεις της μορφής:
	\[
	\boxed{\begin{aligned}
		\dot x &= f(x,u) \\
		y &= h(x,u)
		\end{aligned}}
	\text{ όπου }\begin{aligned}
		x \in \mathbb R^n\ &\text{το \textbf{διάνυσμα μεταβλητών κατάστασης}},\\
		u \quad&\text{η \textbf{είσοδος του συστήματος}},\\
		y \quad&\text{η \textbf{έξοδος του συστήματος}},\\
		q \quad&\text{μία \textbf{μεταβλητή κατάστασης} ή άλλη παράμετρος του συστήματος}
		\end{aligned}
	\]
	
	Οι παραπάνω εξισώσεις εκφράζουν για κάθε στιγμή τη σχέση των μεταβλητών κατάστασης,
	και ονομάζονται \textbf{εξισώσεις κατάστασης}.
\end{defn}

Η λύση του προβλήματος μπορεί να προκύψει από οποιαδήποτε επιλογή μεταβλητών κατάστασης. Οι
δυνατές επιλογές όμως είναι άπειρες. Στα παρακάτω παραδείγματα προτείνουμε επιλογές που
θα επιστρέφουν σίγουρα αποτέλεσμα, αν και ίσως θα οδηγούμαστε εκεί με πιο αργό τρόπο.


\paragraph{Παράδειγμα}
\label{sec:nontd_system}
Έστω ένα σύστημα του οποίου η περιγραφή εκφράζεται από τον τύπο:
\[
y^{(n)} + a_1y^{(n-1)} + a_2y^{(n-2)} + \dots + a_{n-1}\dot y + a_n y = u
\label{eq:nontd_system}
\]
όπου \( u\in\mathbb R  \) η είσοδος και \( y\in\mathbb R  \) η έξοδος του συστήματος.

Εδώ παρατηρούμε ότι οι σταθερές \( a_i \) \textbf{δεν εξαρτώνται} από το \textbf{χρόνο}.

\textbf{Επιλέγουμε} να ορίσουμε τις μεταβλητές κατάστασης
\( (x_1,\ x_2,\dots,\ x_n) \) ως τις παραγώγους της εξόδου.
Η συγκεκριμένη επιλογή δουλεύει καλά:
\begin{align*}
	x_1 &= y \\
	x_2 &= \dot y \\
	&\vdots \\
	x_n &= y^{(n-1)}
\end{align*}

Η δυσκολία αλλά και ο στόχος που θέλουμε να πετύχουμε είναι η \textbf{εύρεση των \( n \) εξισώσεων κατάστασης} με βάση τις παραπάνω μεταβλητές. Εδώ παρατηρούμε ότι:
\begin{align*}
	\dot x_1 &= \dot y = x_2\\
	\dot x_2 &= \ddot y = x_3 \\
	&\vdots \\
	\dot x_{n-1} &= x_n
\end{align*}
και από την αρχική Διαφορική Εξίσωση \eqref{eq:nontd_system} έχουμε:
\[
\dot x_n = -a_1x_n -a_2x_{n-1} - \dots - a_{n-1}x_2 -a_nx_1 + u
\]
δηλαδή το σύνολο των εξισώσεων κατάστασης είναι:
\begin{align*}
\dot x_1 &= x_2\\
\dot x_2 &= x_3 \\
&\vdots \\
\dot x_{n-1} &= x_n\\
\dot x_n &= -a_1x_n -a_2x_{n-1} - \dots - a_{n-1}x_2 -a_nx_1 + u
\end{align*}

Σε \textbf{μορφή πίνακα}, οι \textbf{μεταβλητές κατάστασης} είναι:
\[
\dot X = \left[\begin{matrix}
\dot x_1 \\ \dot x_2 \\ \vdots \\ \dot x_n
\end{matrix}\right] = \left[
\begin{matrix}
0 & 1 & 0 & \cdots & 0 \\
0 & 0 & 1 & \cdots & 0 \\
\vdots & \vdots & \vdots & \ddots & 0\\
0 & 0 & 0 & \cdots & 1 \\
-a_n & - a_{n-1} & - a_{n-2} & \cdots & -a_1
\end{matrix}
\right] \left[\begin{matrix}
x_1 \\  x_2 \\ \vdots \\ x_{n-1} \\ x_n
\end{matrix}\right]
+ \left[\begin{matrix}
0\\0\\\vdots\\0\\1
\end{matrix}\right]u
\]
και η \textbf{έξοδος} του συστήματος προκύπτει από:
\[
y= \left[\begin{matrix}
1 & 0 & 0 & \cdots & 0
\end{matrix}\right]\left[\begin{matrix}
x_1\\x_2\\\vdots\\x_n
\end{matrix}\right]
\]

\paragraph{Παράδειγμα με μεγαλύτερη τάξη εξόδου}
\label{sec:nontd_highorder_system}
Έστω το σύστημα:
\[
y^{(n)} + a_1y^{(n-1)} + a_2y^{(n-2)} + \dots + a_{n-1}\dot y + a_n y =
b_0 u^{(n)} + b_1u^{(n-1)} + \dots + b_{n-1}\dot u + b_n u
\label{eq:nontd_highorder_system}
\]

Υπενθυμίζουμε ότι σκοπός είναι να βρούμε ένα σύνολο μεταβλητών κατάστασης που μπορούν
να παράγουν διαφορικές εξισώσεις που μπορούν να λυθούν, ώστε να περιγραφεί η λειτουργία του
συστήματος.

\begin{itemize}
	\item Έστω ότι ακολουθούμε την προσέγγιση του προηγούμενου παραδείγματος, δηλαδή οι
	μεταβλητές κατάστασης είναι:
	\begin{align*}
	x_1 &= y \\
	x_2 &= \dot y \\
	&\vdots \\
	x_n &= y^{(n-1)}
	\end{align*}
	
	Με λύση:
	\begin{align*}
	\dot x_1 &= \dot y = x_2\\
	\dot x_2 &= \ddot y = x_3 \\
	&\vdots \\
	\dot x_{n-1} &= x_n\\
	\dot x_n &= -a_1x_n -a_2x_{n-1} - \dots - a_{n-1}x_2 -a_nx_1 + b_0 u^{(n)} + b_1u^{(n-1)} + \dots + b_{n-1}\dot u + b_n u
	\end{align*}

     Η παραπάνω λύση είναι σωστή, όμως απαιτεί τη γνώση \textbf{παραγώγων υψηλής τάξης} της
     εισόδου \( u \), κάτι το οποίο ιδιαίτερα στα πραγματικά ΣΑΕ δεν μπορεί να υπολογιστεί,
     επειδή απαιτούνται οι μελλοντικές τιμές του συστήματος.
     \item Μία άλλη λύση που προτείνουμε είναι να \textbf{επιλέξουμε}:
     \begin{alignat*}{2}
     	x_1 &= y - \beta_0 u && \\
     	x_2 &= \dot x_1 - \beta_1 u &&= \dot y - \beta_0\dot u - \beta_1u \\
     	&\vdots && \\
     	x_n &= \dot{x_{n-1}} - \beta_{n-1}u
     	&&= y^{(n-1)} - \beta_0u^{(n-1)} - \dots - \beta_{n-2}\dot u - \beta_{n-1}u
     \end{alignat*}
     
     Οι όροι \( \beta_i \) δεν είναι ίδιοι με τους \( b_i \), αλλά ορίζονται ως εξής:
     \begin{align*}
     	\beta_0 &= b_0 \\
     	\beta_1 &= b_1 - a_1\beta_0\\
     	\beta_2 &= b_2 - a_1\beta_1 - a_2\beta_0\\
     	&\vdots\\
     	\beta_n &= b_n - a_1\beta_{n-1}-\dots - a_{n-1}\beta_1 - a_n\beta_0
     \end{align*}
     
     Το παραπάνω σύστημα μπορεί να λυθεί, και αποδεικνύεται ότι η κάθε εξίσωση περιέχει
     την είσοδο \( u \) χωρίς να είναι σε παράγωγο.
\end{itemize}

\begin{exercise}[Παράδειγμα]
	Στο παράδειγμα της προηγούμενης παραγράφου:
	\[
	m\ddot q + c\dot q + kq = u
	\]
	\tcblower
	Η παραπάνω εξίσωση είναι η \textbf{διαφορική εξίσωση του συστήματος}.
	
	Δεν εμφανίζονται ανώτερες παράγωγοι της εισόδου άρα χρησιμοποιούμε την απλή έκφραση
	από επάνω (\autoref{sec:nontd_system}).
	
	Η τάξη της εξίσωσης είναι \textbf{2}, άρα έχουμε \textbf{μόνο 2} μεταβλητές κατάστασης.
	Σύμφωνα με την παράγραφο \autoref{sec:nontd_system}, επιλέγουμε:
	\begin{alignat*}{4}
		x_1 &=q \qquad && \dot x_1&&=x_2 \\
		x_2&=\dot q \qquad && \dot x_2 &&= -\frac{c}{m}x_2 - \frac{k}{m}x_1
		+ \frac{1}{m}u
	\end{alignat*}
	ή, σε μορφή πίνακα:
	\begin{align*}
		\dot x &= \left[\begin{matrix}
		0 & 1 \\ -\frac{k}{m} & -\frac{c}{m}
		\end{matrix}\right]\left[\begin{matrix}
		x_1\\x_2
		\end{matrix}
		\right]+\left[\begin{matrix}
		0\\ \frac{1}{m}
		\end{matrix}\right]u\\
		y &= \left[\begin{matrix}
		1 & 0
		\end{matrix}\right]\left[\begin{matrix}
		x_1 \\ x_2
		\end{matrix}\right]
	\end{align*}
\end{exercise}
\begin{exercise}[Παράδειγμα]
	Να γραφούν οι εξισώσεις κατάστασης του κυκλώματος:
	
	\begin{circuitikz}[american]
		\draw (0,0) to[V=$v_s(t)$] (0,2)
		to[R=$R$,i>^=$i$] (2,2)
		to[cute inductor=$L$] (4,2)
		to[C=$C$] (4,0)
		-- (0,0)
		;
	\end{circuitikz}
	
	\tcblower
	Λύνουμε το σύστημα με βάση τη φυσική του (νόμοι Kirchoff)
	\begin{align}
		u_s(t) &=
		iR + L\od{i}{t} + \frac{1}{C} \int_{0}^{t} i \dif\tau
		\label{eq:ex0ceq}
		\\
		L\od[2]{i}{t} + R\od{i}{t} + \frac{1}{C} i &= \od{u_s(t)}{t}
		\todo{Hide this eq nlabel}
	\end{align}
	
	Η είσοδος του συστήματος είναι η ανεξάρτητη τροφοδοσία \( u_s(t) \), και η έξοδος
	το ρεύμα \( i \), δηλαδή:
	\begin{align*}
		u &= u_s(t) \\
		y &= i
	\end{align*}
	άρα η τελευταία διαφορική εξίσωση γράφεται απλούστερα:
	\[
	\ddot y + \frac{R}{L}\dot y + \frac{1}{CL} y = \frac{1}{L} \dot u
	= 0\ddot u + \frac{1}{L}\dot u + 0u
	\]
	
	Αφού η είσοδος \( u \) είναι ανώτερης τάξης στη διαφορική αυτή, επιλέγουμε τις μεταβλητές
	κατάστασης με βάση την παράγραφο \autoref{sec:nontd_highorder_system}:
	\begin{align*}
		x_1 &= y-\beta_0u,\qquad \beta_0 = 0\\
		x_2 &= \dot y-\beta_0 \dot u -\beta_1 u,\qquad \beta_1 = \frac{1}{L}
	\end{align*}
	και εκτελώντας πράξεις:
	\begin{align*}
		x_1 = y &\implies \boxed{\dot x_1 = x_2 + \frac{1}{L}u} \\
		x_2 = \dot x_1 - \beta_1 u \implies \dot x_2 = \ddot y - \frac{1}{L}\dot u
		\implies \dot x_2 = -\frac{R}{L}\dot y-\frac{1}{LC}y + \frac{1}{L}\dot u
			- \frac{1}{L} \dot u
		&\implies\boxed{\dot x_2 = -\frac{R}{L}x_2 - \frac{R}{L^2}u-\frac{1}{LC}x}\\
		\boxed{y=x_1}&
	\end{align*}
	
	\paragraph{Με διαφορετική επιλογή μεταβλητών κατάστασης}
	Στην παραπάνω λύση διαλέξαμε τις μεταβλητές κατάστασης όπως γνωρίζαμε από τη θεωρία
	των ΣΑΕ, οι οποίες οδήγησαν σε ένα αποτέλεσμα, αλλά με αρκετές πράξεις.
	
	Εναλλακτικά, μπορούμε να διαλέξουμε διαφορετικές μεταβλητές κατάστασης. Στο συγκεκριμένο
	πρόβλημα, επειδή ασχολούμαστε με ένα ηλεκτρικό κύκλωμα, επιλέγουμε τις \textbf{τυπικές
		μεταβλητές} που αντιστοιχούν στην \textbf{τάση του πυκνωτή} \( x_1 \) και στο
	\textbf{ρεύμα του πηνίου} \( x_2 \):
	\begin{align*}
		x_1 &= \frac{1}{C} \int_{0}^{t} i\dif t\\
		x_2 &= i \implies \dot x_2 = \od{i}{t}
	\end{align*}
	και από αυτά προκύπτει ότι:
	\begin{align*}
		\dot x_1 &= \frac{1}{C} x_2\\
		\dot x_2 &= \frac{x_1-x_2 R}{L} \qquad \text{λόγω της \eqref{eq:ex0ceq}}
	\end{align*}
\end{exercise}

\begin{exercise}
	Έστω το μηχανικό σύστημα:
	\todo{Graph 6}
	
	Να βρεθούν οι εξισώσεις κατάστασης, θεωρώντας ότι η έξοδος είναι η μετατόπιση του δεξιού
	σώματος, δηλαδή:
	\[
	y= q_1
	\]
	
	\tcblower
	
	Εφαρμόζοντας το νόμο του Νεύτωνα για κάθε σώμα, και αθροίζοντας τις δυνάμεις που
	ασκούνται στο καθένα, έχουμε:
	\begin{align}
		M_1\ddot q_1 &= u - k_1(q_1-q_2) - c_1(\dot q_1 - \dot q_2) 
		\label{eq:ex0eq1}
		\\
		M_2\ddot q_2 &= -k_1(q_2-q_1) - c_1(\dot q_2 - \dot q_1)-k_2q_2-c_2\dot q_2
		\label{eq:ex0eq2}
	\end{align}
	
	Το σύστημα αυτό είναι \textbf{4\textsuperscript{ης}} τάξης και 1\textsuperscript{ου}
	βαθμού, αφού έχουμε 2 εξισώσεις 2\textsuperscript{ης} τάξης. Επομένως πρέπει να βρούμε
	4 μεταβλητές και εξισώσεις κατάστασης.
	
	Λαμβάνουμε τις μεταβλητές κατάστασης με βάση τη θεωρία:
	\[
	x_1 = q_1, \hfill x_2=\dot q_1, \hfill x_3=q_2,\hfill x_4=\dot q_2
	\]
	οπότε, πιο καθαρά:
	\begin{align*}
		\dot x_1 &= x_2 \\
		\dot x_2 &= \frac{1}{M_1}u - \frac{k_1}{M_1}x_3
		+ \frac{k_1}{M_1}x_2 + \frac{C_1}{M_1}x_4
		\quad \text{(όπως προκύπτει από την \eqref{eq:ex0eq1})} \\
		\dot x_3 &= x_4\\
		\dot x_4 &= -\frac{k_1}{M_2}x_3 + \frac{k_1}{M_2}x_1 - \frac{c_1}{M_2}x_4
	    + \frac{c_1}{M_2}x_2 - \frac{k_2}{M_2}x_3 - \frac{c_2}{M_2}x_4
	    \quad \text{(όπως προκύπτει από την \eqref{eq:ex0eq2})}
	\end{align*}
	
	Σε μορφή πίνακα:
	\[
	\dot x = \left[\begin{matrix}
	0 & 1 & 0 & 0\\
	-\frac{k_1}{M_1} & -\frac{c_1}{M_1} & \frac{k_1}{M_1} & \frac{c_1}{M_1}\\
	0 & 0 & 0 & 1\\
	\frac{k_1}{M_2} & \frac{c_1}{M_2} & -\left(\frac{k_1+k_2}{M_2}\right)
	& -\left(\frac{c_1+c_2}{M_2}\right)
	\end{matrix}\right]\left[\begin{matrix}
	x_1 \\ x_2 \\ x_3 \\ x_4
	\end{matrix}\right] + \left[
	\begin{matrix}
	0 \\ \sfrac{1}{M_1} \\ 0 \\ 0
	\end{matrix}
	\right]u
	\]
\end{exercise}

\subsubsection{Μετασχηματισμός Laplace}
Θυμόμαστε μια υτυπική έκφραση ενός συστήματος συστήματος:
\[
y^{(n)} + a_1y^{(n-1)} + a_2y^{(n-2)} + \dots + a_{n-1}\dot y + a_n y =
b_0 u^{(n)} + b_1u^{(n-1)} + \dots + b_{n-1}\dot u + b_n u
\]

Αυτή μπορεί να μετασχηματιστεί κατά Laplace, όπως γνωρίζουμε:
\[
\left(
s^n + a_1s^{n-1} + \dots + a_{n-1}s + a_n
\right)Y(s) = \left(
b_0s^m = b_1s^{m-1} + \dots + b_{m-1}s+b_m
\right)U(s)
\]
και η \textbf{συνάρτηση μεταφοράς} προκύπτει κατά τα γνωστά:
\[
\frac{Y(s)}{U(s)} = \frac{
	b_0s^m = b_1s^{m-1} + \dots + b_{m-1}s+b_m
	}{
	s^n + a_1s^{n-1} + \dots + a_{n-1}s + a_n
	}
\] όπου συνήθως \( n \geq m \).

Προκύπτει το ερώτημα του πώς μπορεί να προκύψει η συνάρτηση μεταφοράς από τις εξισώσεις
κατάστασης, δηλαδή από την περιγραφή (σε μορφή πινάκων):
\begin{align*}
\dot x &= Ax	+ By\\
y &= Cx + Du
\end{align*}
όπου οι συντελεστές \( A,B,C,D \) είναι γνωστοί.

Μετασχηματίζοντας τις παραπάνω σχέσεις κατά Laplace, έχουμε:
\begin{align*}
	sX(s) &= AX(s) + BU(s) \\
	Y(s) &= CX(s) + DU(s)
\end{align*}

Εκτελούμε πράξεις:
\begin{align*}
	(sI-A)X(s) &= BU(s) \implies \\
	X(s) &= (sI-A)^{-1} B U(s) \implies \\
	Y(s) &= \left[ C(sI-A)^{-1}B+D \right]U(s)
\end{align*}
δηλαδή η συνάρτηση μεταφοράς προκύπτει από τον τύπο:
\[
\boxed{
	G(s) = \frac{Y(s)}{U(s)} = C(sI-A)^{-1}B+D
	}
\]

Ο υπολογισμός της συνάρτησης μεταφοράς μπορεί να φανεί χρήσιμος ανάλογα με τον τρόπο με
τον οποίο θέλουμε να λύσουμε ένα πρόβλημα. Για παράδειγμα, μπορεί να εφαρμοστεί αν θέλουμε
να εκμεταλλευτούμε τεχνικές του προηγούμενου εξαμήνου (π.χ. γεωμετρικός τόπος ριζών).

Υπενθυμίζουμε ότι σε ένα σύστημα αντιστοιχεί μία μοναδική διαφορική εξίσωση και μία μοναδική
συνάρτηση μεταφοράς, αλλά άπειρες διαφορετικές επιλογές μεταβλητών και εξισώσεων κατάστασης.

\subsubsection{Αριθμητική επίλυση διαφορικών εξισώσεων}
Για την αριθμητική επίλυση των διαφορικών εξισώσεων μπορούμε να χρησιμοποιήσουμε μια μέθοδο
όπως η \textbf{μέθοδος Euler}, προσομοιώνοντας ουσιαστικά το μοντέλο του συστήματος.
Εκμεταλλευόμαστε τον ορισμό της παραγώγου:
\[
\dot x = \frac{x(t+\Delta t) - x(t)}{\Delta(t)}
\]

Ορίζουμε μια συνάρτηση που εκφράζει την παράγωγο:
\[
\dot x = f(x)
\]
επομένως:
\[
x(t+\Delta t) = f\left( x(t) \right) \cdot \Delta t + x(t)
\]
όπως προκύπτει από τον ορισμό της παραγώγου.

Για μικρό \( \Delta t \) η μέθοδος αυτή οδηγεί στο επιθυμητό αποτέλεσμα.

\subsubsection{Μελέτη Ευστάθειας Συστήματος}
Θα μελετήσουμε \textbf{τρόπους εύρεσης της ευστάθειας} ενός συστήματος, οι οποίες όμως δεν απαιτούν
προσομοίωση ή αναλυτική επίλυση.

Ο λόγος που δεν μπορούμε να εφαρμόσουμε αναλυτική επίλυση, είναι ότι πολλές φορές είναι
δύσκολη η επίλυση των διαφορικών εξισώσεων που περιγράφουν τα συστήματα, ιδιαίτερα αν είναι
μη γραμμικές.

Από την άλλη μεριά, η προσομοίωση λειτουργεί μόνο για μία επιλογή αρχικών τιμών. Μπορεί να
δώσει ενδείξεις ευστάθειας, αλλά όχι να την αποδείξει για όλο το εύρος των άπειρων αρχικών
συνθηκών.

Πρέπει επομένως να βρούμε τρόπους να \textit{αποδείξουμε} την ευστάθεια ή μη ενός συστήματος,
η οποία να μην βασίζεται απλώς στην εμπειρία μας.

\paragraph{Παράδειγμα}
Έστω το σύστημα:
\[
\dot x = \left[\begin{matrix}
\dot x_1\\ \dot x_2
\end{matrix}\right] = \left[\begin{matrix}
x_2 \\ -\frac{c}{m}x_2 - \frac{k}{m}x_1
\end{matrix}\right] = f(x)
\]
το οποίο αναφέρεται σε ένα σώμα μάζας \( m \) που κινείται με την επίδραση ενός ελατηρίου
\( k \) και ενός αποσβεστήρα \( c \).

Η \textbf{ενέργεια} του συστήματος, όπως προκύπτει από τις γνώσεις μας στη φυσική, είναι:
\[
V(x_1,x_2) =
\underbrace{\frac{1}{2}kx_1^2}_{\mathclap{\text{Δυναμική Εν.}}}
+
\underbrace{\frac{1}{2}mx_2^2}_{\mathclap{\text{Κινητική Εν.}}}
\]

Αν την παραγωγίσουμε, έχουμε:
\begin{align*}
	\dot V &=
	kx_1\dot x_1 + mx_2\dot x_2 \\
	&= kx_1x_2 + mx_2 \left(
	-\frac{c}{m}x_2 - \frac{k}{m}x_1
	\right)
	\\ &= kx_1x_2 - cx_2^2 - kx_1x_2
	\\ &= -cx_2^2 \quad \leq 0
\end{align*}
δηλαδή παρατηρούμε ότι η \textbf{παράγωγος} της ενέργειας είναι \textbf{αρνητική}, άρα η ενέργεια του
συστήματος σχεδόν κάθε στιγμή μειώνεται (φθίνουσα)! Αυτό μπορούμε να το αντιληφθούμε αφού στο σύστημα δεν
ασκούνται εξωτερικές δυνάμεις, και εκτελεί κάποια ταλάντωση με μια απόσβαση που συνεχώς
αφαιρεί ενέργεια.

Δηλαδή κάθε στιγμή η ενέργεια είναι μικρότερη από την αρχική
\( V\left( x_1(0),\ x_2(0) \right) \):
\[
	V\left( x_1(0),\ x_2(0) \right) \geq \frac{1}{2}k x_1^2
	+ \frac{1}{2}mx_2^2
\]

Αυθαίρετα κρατάμε μόνο τον έναν όρο της παραπάνω σχέσης:
\begin{align*}
	V\left( x_1(0),\ x_2(0) \right) &\geq \frac{1}{2}k x_1^2 \implies \\
	\frac{1}{2}kx_1^2(0) + \frac{1}{2}mx_2^2(0) &\geq \frac{1}{2}kx_1^2
	\implies \boxed{
		x_1(t) \leq \sqrt{x_1^2(0) + \frac{m}{k}x_2^2(0)}
		}\quad \forall t \geq 0
\end{align*}
ή, αν κρατήσουμε τον άλλον όρο:
\begin{align*}
V\left( x_1(0),\ x_2(0) \right) &\geq \frac{1}{2}m x_x^2 \implies \\
\frac{1}{2}kx_1^2(0) + \frac{1}{2}mx_2^2(0) &\geq \frac{1}{2}mx_2^2
\implies \boxed{
	x_2(t) \leq \sqrt{\frac{k}{m}x_1^2(0) + x_2^2(0)}
}\quad \forall t \geq 0
\end{align*}

Παρατηρούμε δηλαδή ότι οι μεταβλητές κατάστασης είναι \textbf{φραγμένες}, δηλαδή δεν ξεπερνούν κάποια τιμή, το οποίο δηλώνει ευστάθεια του συστήματος, αφού δεν φτάνουν μέχρι
το \( \infty \).

\paragraph{}
Παραπάνω είδαμε ένα πρόβλημα στο οποίο \( \dot V(x_1,x_2) \leq 0 \). Η δυνατότητα ισότητας
(\( \leq \)) με το 0 δηλώνει ότι η συνάρτηση είναι απλά \textit{φθίνουσα}, και όχι
\textit{γνησίως φθίνουσα}, που σημαίνει ότι υπάρχουν διαστήματα στα οποία η παράγωγος
της ενέργειας είναι 0, και η ενέργεια παραμένει σταθερή χωρίς να μειώνεται. Έστω ένα τέτοιο
διάστημα στο παραπάνω σύστημα:
\[
\dot V(x_1,x_2) = 0
\xRightarrow{\dot V = -cx_2^2 \leq 0}
x_2 = 0
\xRightarrow[\dot x_2 = -\frac{c}{m} x_2 - \frac{k}{m} x_1]{\text{εξ. κατάστασης}}
x_1 = 0
\]

\begin{exercise}
	Έστω ένα ποδήλατο σε κάτοψη:
	\todo{Graph 7}
	
	Το ποδήλατο έχει κέντρο μάζας και περιστρέφεται γύρω από ένα σημείο \( O \).
	\tcblower
	
	Κυνηγούμε γωνίες και τις προσθέτουμε στο σχήμα:
	\todo{Graph 8}
	
	Στα ορθογώνια τρίγωνα που προκύπτουν μεταξύ γωνιών και πλευρών, έχουμε:
	\[
	\left.
	\begin{aligned}
	\tan\phi &= \frac{s}{r}\\
	\tan\delta &= \frac{L}{r}
	\end{aligned}\right\rbrace
	\implies \tan\phi = \frac{S}{L}\tan\delta
	\]
	
	Θεωρώντας ότι ο κάτω τροχός κινείται με ταχύτητα \( u_0 \), έχουμε:
	\[
	u\cos\phi = u_0
	\]
	
	Τώρα πρέπει να βρούμε μια διαφορική εξίσωση που να περιγράαφει την κίνηση του
	κέντρου βάρους ποδηλάτου. Έστω \( (x,y) \) οι συντεταγμένες του. Τότε:
	\begin{align*}
		\Aboxed{\od{x}{t} &=
		u \cos (\phi+\theta) = \frac{u_0\cos(\phi+\theta)}{\cos\phi}} \\
		\Aboxed{\od{y}{t} &=
		u \sin(\phi+\theta) = \frac{u_0\sin(\phi+\theta)}{\cos\phi}}
	\end{align*}
	
	Επίσης πρέπει να βρούμε εξισώσεις κατάστασης για τον προσανατολισμό του ποδηλάτου.
	
	Η γωνιακή ταχύτητα δίνεται από τη σχέση:
	\[
	\omega = \frac{u_0}{r} \qquad \text{ή} \qquad \omega = \od{\phi}{t}
	\]
	και, επειδή \( \tan\delta = \frac{L}{r} \):
	\[
	\omega = \frac{u_0}{L}\tan\delta
	\]
	
	Για μία μικρή μεταβολή \( \Delta \phi \) του προσανατολισμού του ποδηλάτου:
	\todo{Graph 9}
	
	Οι γωνίες \( \Delta \phi \) και \( \theta \) είναι μεταξύ τους κάθετες, άρα ίσες,
	επομένως:
	\[
	\boxed{\od{\theta}{t} = \frac{u_0}{L} \tan\delta}
	\]
\end{exercise}

\end{document}
