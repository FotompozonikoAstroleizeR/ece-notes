% !TeX program = xelatex
\documentclass[11pt,a4paper,notitlepage,fleqn,final]{article}

\usepackage{amsmath}
\usepackage{amsfonts}
\usepackage{amssymb}
\usepackage{libs/commath2}
\usepackage[table]{xcolor}
\usepackage[hidelinks,draft=false]{hyperref}
\usepackage[skins,theorems]{tcolorbox}
\usepackage{titlesec}
\usepackage{tikz}
\usepackage{libs/circuitikz} % use our own recent version to make sure some bugs are fixed
\usepackage{pgfplots}
\usepackage{mathtools}
\usepackage[makeroom]{cancel}
\usepackage{mathrsfs}
\usepackage{wrapfig}
%\usepackage{subcaption}
%\usepackage{floatrow}
\usepackage{esint}
\usepackage{enumitem}
%\usepackage{bm}
\usepackage{relsize}
\usepackage{xfrac}
\usepackage{comment}
\usepackage{siunitx}
\usepackage{multicol}
%\usepackage{MnSymbol}
\usepackage[obeyDraft,disable]{todonotes}
%\usepackage{morefloats} % oh no!
%\usepackage[linesnumbered,lined]{algorithm2e}
\usepackage{glossaries}
\usepackage{xifthen}


\pgfplotsset{compat=1.13}
\usetikzlibrary{arrows.meta}
\usetikzlibrary{patterns}
\usetikzlibrary{decorations.pathmorphing}
\usetikzlibrary{decorations.markings}
\usetikzlibrary{backgrounds}
\usetikzlibrary{shapes.misc}
\usetikzlibrary{shapes.multipart}
\usetikzlibrary{shadows.blur}
\usetikzlibrary{fadings}
\usetikzlibrary{intersections}
\usetikzlibrary{arrows.meta}
\usetikzlibrary{calc}
\usetikzlibrary{matrix}
\usetikzlibrary{positioning}
\usetikzlibrary{shapes}
\usetikzlibrary{shadings}

\tcbuselibrary{breakable}
\tcbuselibrary{skins}
\tcbuselibrary{xparse}

\tikzset{cross/.style={cross out, draw,
        minimum size=2*(#1-\pgflinewidth),
        inner sep=0pt, outer sep=0pt}}
\tikzset{
    mark position/.style args={#1(#2)}{
        postaction={
            decorate,
            decoration={
            	post length=1mm, % ??? Magic to fix "Dimension
            	pre length=1mm, % ???  too large" errors.
                markings,
                mark=at position #1 with \coordinate (#2);
            }
        }
    }
}
\tikzset{
	arrow at/.style args={#1}{
		postaction={
			decorate,
			decoration={
				post length=1mm, % ??? Magic to fix "Dimension
				pre length=1mm, % ???  too large" errors.
				markings,
				mark=at position #1 with {\arrow{>}};
			}
		}
	}
}
\makeatletter
\tikzset{
  use path for main/.code={%
    \tikz@addmode{%
      \expandafter\pgfsyssoftpath@setcurrentpath\csname tikz@intersect@path@name@#1\endcsname
    }%
  },
  use path for actions/.code={%
    \expandafter\def\expandafter\tikz@preactions\expandafter{\tikz@preactions\expandafter\let\expandafter\tikz@actions@path\csname tikz@intersect@path@name@#1\endcsname}%
  },
  use path/.style={%
    use path for main=#1,
    use path for actions=#1,
  }
}
\makeatother

\pgfmathdeclarefunction{sinc}{1}{%
	\pgfmathparse{abs(#1)<0.01 ? int(1) : int(0)}%
	\ifnum\pgfmathresult>0 \pgfmathparse{1}\else\pgfmathparse{sin(#1 r)/#1}\fi%
}
\pgfmathdeclarefunction{gauss}{2}{%
	\pgfmathparse{1/(#2*sqrt(2*pi))*exp(-((x-#1)^2)/(2*#2^2))}%
}

\usepackage[left=2cm,right=2cm,top=2cm,bottom=2cm]{geometry}

%\usepackage[no-math]{fontspec}
%\usepackage{fontspec}
\usepackage{mathspec}
%\usepackage{newtxtext,newtxmath}
%\usepackage{unicode-math}
%\setmainfont{texgyretermes-regular.otf}
%\setsansfont{texgyreheros-regular.otf}
%\newfontfamily\greekfont[Script=Greek]{Linux Libertine O}
%\newfontfamily\greekfontsf[Script=Greek]{Linux Libertine O}
\usepackage{polyglossia}
%\newfontfamily\greekfont[Script=Greek]{texgyretermes-regular.otf}
\newfontfamily\greekfontsf[Script=Greek]{texgyreheros-regular.otf}
\newfontfamily\greekfonttt[Script=Greek]{Latin Modern Mono}
%\usepackage[greek]{babel}
\setdefaultlanguage{greek}
\setotherlanguage{english}

%\usepackage[utf8]{inputenc}
%\usepackage[greek]{babel}


%\usepackage{tkz-euclide} % loads  TikZ and tkz-base
%\usetkzobj{angles} % important you want to use angles

\newlist{enumparen}{enumerate}{1}
\setlist[enumparen]{label=(\arabic*)}
\newlist{enumpar}{enumerate}{1}
\setlist[enumpar]{label=\arabic*)}

\newlist{enumgreek}{enumerate}{1}
\setlist[enumgreek]{label=\alph*.}
\newlist{enumgreekparen}{enumerate}{1}
\setlist[enumgreekparen]{label=(\alph*)}
\newlist{enumgreekpar}{enumerate}{1}
\setlist[enumgreekpar]{label=\alph*)}


\newlist{enumroman}{enumerate}{1}
\setlist[enumroman]{label=(\roman*)}

\newlist{enumlatin}{enumerate}{1}
\setlist[enumlatin]{label=(\alph*)}

\newlist{invitemize}{itemize}{1}
\setlist[invitemize]{noitemsep,label=}

\input{libs/fiximplies}
\input{libs/sphere}

\makeatletter
\let\anw@true\anw@false

%\newcommand{\attnboxed}[1]{\textcolor{red}{\fbox{\normalcolor\m@th$\displaystyle#1$}}}
\makeatother
\tcbset{highlight math style={enhanced,colframe=red,colback=white,%
        arc=0pt,boxrule=1pt,shrink tight,boxsep=1.5mm,extrude by=0.5mm}}
\newcommand{\attnboxed}[1]{\tcbhighmath[colback=red!5!white,drop fuzzy shadow,arc=0mm]{#1}}
\newcommand{\infoboxed}[1]{%
	\tcbhighmath[colframe=blue!50!white,colback=blue!5!white,arc=0mm]{#1}}
\titleformat{\section}{\bf\Large}{Κεφάλαιο \thesection}{1em}{}
\newtcolorbox{attnbox}[1]{colback=red!5!white,%
    colframe=red!75!black,fonttitle=\bfseries,title=#1}
\newtcbox{quickattnbox}[1]{colback=red!5!white,%
	colframe=red!75!black,fonttitle=\bfseries,title=#1}
\newtcolorbox{infobox}[1]{colback=blue!5!white,%
    colframe=blue!75!black,fonttitle=\bfseries,title=#1}

\tcbset{frogbox/.style={enhanced jigsaw,%
		overlay first={\foreach \x in {0cm} {
				\begin{scope}[shift={([xshift=-0.2cm]title.west)}]
					\draw[very thick,green!65!black!50!white,latex-] (0,0) -- ++(-1.5,0);
\end{scope}}}}}
\tcbset{frogtitle/.style={
attach boxed title to top left=
{xshift=0mm,yshift=-0.50mm},
boxed title style={skin=enhancedfirst jigsaw,
	bottom=0mm,
	interior style={fill=none,
		left color=green!20!black,
		right color=gray}}
}}
\DeclareTColorBox{exercise}{ O{} }{
	enhanced jigsaw,
	breakable,parbox=false,
	%title style={left color=gray!50!white!50!green,opacity=.5,right color=white},
	subtitle style={%boxrule=1pt,
		colback=yellow!50!red!25!white,fontupper=\bfseries},
	coltitle=black,colbacktitle=green!90!black!25!white,colframe=black,
	frame hidden,
	boxrule=0mm,
	%boxrule=1mm,
	leftrule=0.8pt,toprule=0.8pt,rightrule=0pt, %reserve space
	borderline west={0.8pt}{0pt}{white!25!black},%---- draw line
	borderline north={0.8pt}{0pt}{white!25!black},%---- draw line
	interior hidden,
	%frame style={left color=black,right color=white},
	sharp corners=all,
	%frogbox, %TODO: frogbox
	before lower={\tcbsubtitle[before skip=\baselineskip]{Λύση}},lower separated=false,
	before title={\textbf{Άσκηση\ifthenelse{\isempty{#1}}{}{: }}},
	title={\ifthenelse{\isempty{#1}}{\hspace{0pt}}{#1}}%
}

\AtBeginDocument{%
\let\arg\relax
\let\Re\relax
\let\Im\relax
\DeclareMathOperator{\arg}{Arg}
\DeclareMathOperator{\Re}{Re}
\DeclareMathOperator{\Im}{Im}
}
\DeclareMathOperator{\sinc}{sinc}
\DeclareMathOperator{\sgn}{sgn}
\DeclareMathOperator{\erf}{erf}
\DeclareMathOperator{\cov}{cov}
\DeclareMathOperator{\atand}{atan2}

\newenvironment{absolutelynopagebreak}
{\par\nobreak\vfil\penalty0\vfilneg
	\vtop\bgroup}
{\par\xdef\tpd{\the\prevdepth}\egroup
	\prevdepth=\tpd}

\DeclareSIUnit \voltampere { VA } %apparent power 
\DeclareSIUnit \var { VAr } %volt-ampere reactive - idle power 
\DeclareSIUnit \decade { dec } %decade

% Global amount of samples
% Set to a higher value (e.g. 200) for nicer graphs
% Set to a low value (e.g. 10) for performance
% NOTE: Check the sample variables below for further measurements
\newcommand*{\gsamples}{200}

% Equals command as a workaround for CircuiTikZ bug
% not allowing the = sign in labels
\newcommand*{\equals}{=}

\newcommand{\nesearrow}{%
	\,%
	\smash{\raisebox{-1.1ex}
		{$%
			\stackrel{\displaystyle\nearrow}{\displaystyle\searrow}%
			$}}%
}
\newcommand{\degree}{^{\circ}} % not great
\newcommand\numberthis{\addtocounter{equation}{1}\tag{\theequation}} % add an equation number to a number-less math environment

% Provided commands
\providecommand\dif{d}
\providecommand\od[2]{\frac{#1}{#2}}

\newtcbtheorem[number within=section,list inside=thm]{theorem}{Θεώρημα}%
{colback=green!5,colframe=green!35!black,colbacktitle=green!35!black,fonttitle=\bfseries,enhanced,attach boxed title to top left={yshift=-2mm,xshift=-7mm},width=.9\textwidth,arc=.7mm}{th}
\newtcbtheorem[number within=section,list inside=defn]{defn}{Ορισμός}%
{colback=blue!5,colframe=cyan!35!black,colbacktitle=blue!35!black,fonttitle=\bfseries,enhanced,attach boxed title to top left={yshift=-2mm,xshift=-2mm}}{def}

% Locus plot utilities
\tikzset{locus/.style={orange!50!red!70!brown}}
\tikzset{locuspole/.style={draw=red!30!black,cross,inner sep=2.5pt,fill=white,fill opacity=.6,thick,label={[below]-90:#1}}}
\tikzset{locuszero/.style={draw=red!30!black,circle,inner sep=2pt,fill=white,fill opacity=.6,thick,label={[below]-90:#1}}}
\tikzset{locusbreak/.style={rounded corners=1.5pt,inner sep=2pt,draw,top color=brown,bottom color=black,fill opacity=.8,label={[below]-90:#1}}}

% New plotting utilities
\def\lowsamples{18}
\def\hisamples{40}
\def\timecolour{blue!50!cyan}

\tikzstyle{timecolour}=[\timecolour]



\title{Συστήματα Χειροκίνητου Ελέγχου 2
	\\
	{ 
		\normalsize Σημειώσεις από τις παραδόσεις.}
	}
\date{Απρίλιος 2018
	\\
	{ 
		\small Τελευταία ενημέρωση: \today
	}
}
\author{
	Για τον κώδικα σε \LaTeX, ενημερώσεις και προτάσεις:
	\\
	\url{https://github.com/kongr45gpen/ece-notes}}

\setallmainfonts(Digits,Latin,Greek){Asana Math}
\setmainfont{Noto Serif}
\setsansfont{Ubuntu}
\usepackage{polyglossia}
\newfontfamily\greekfont[Script=Greek,Scale=1.00]{Liberation Serif}
\usepackage{amsthm}

\hypersetup{pdftitle = {Συστήματα Χειροκίνητου Ελέγχου}}
\newcommand{\truncateit}[1]{\truncate{0.8\textwidth}{#1}}
\usepackage[numbers]{natbib}
\usepackage[fit]{truncate}


\newtcbtheorem[number within=section,list inside=thm]{corollary}{Πόρισμα}%
{colback=red!5,colframe=red!35!black,colbacktitle=red!35!black,fonttitle=\bfseries,enhanced,attach boxed title to top left={yshift=-2mm,xshift=-7mm},width=.9\textwidth,arc=.7mm}{th}

\newtcbtheorem[number within=section,list inside=thm]{lemma}{Λήμμα}%
{colback=orange!5,colframe=orange!35!black,colbacktitle=orange!35!black,fonttitle=\bfseries,enhanced,attach boxed title to top left={yshift=-2mm,xshift=-7mm},width=.9\textwidth,arc=.7mm}{th}

\newtcbtheorem[number within=section,list inside=thm]{claim}{Ισχυρισμός}%
{colback=purple!5,colframe=purple!35!black,colbacktitle=purple!35!black,fonttitle=\bfseries,enhanced,attach boxed title to top left={yshift=-2mm,xshift=-7mm},width=.9\textwidth,arc=.7mm}{th}

\newtcbtheorem[number within=section,list inside=thm]{proposition}{Πρόταση}%
{colback=gray!5,colframe=gray!35!black,colbacktitle=gray!35!black,fonttitle=\bfseries,enhanced,attach boxed title to top left={yshift=-2mm,xshift=-7mm},width=.9\textwidth,arc=.7mm}{th}

\newtcbtheorem[number within=section,list inside=thm]{conjecture}{Εικασία}%
{colback=lime!5,colframe=lime!35!black,colbacktitle=orange!35!black,fonttitle=\bfseries,enhanced,attach boxed title to top left={yshift=-2mm,xshift=-7mm},width=.9\textwidth,arc=.7mm}{th}

\newtcbtheorem[number within=section,list inside=thm]{question}{Ερώτημα}%
{colback=magenta!5,colframe=magenta!35!black,colbacktitle=magenta!35!black,fonttitle=\bfseries,enhanced,attach boxed title to top left={yshift=-2mm,xshift=-7mm},width=.9\textwidth,arc=.7mm}{th}

\newtcbtheorem[number within=section,list inside=thm]{definition}{Ορισμός}%
{colback=cyan!5,colframe=cyan!35!black,colbacktitle=cyan!35!black,fonttitle=\bfseries,enhanced,attach boxed title to top left={yshift=-2mm,xshift=-7mm},width=.9\textwidth,arc=.7mm}{th}

\newtcbtheorem[number within=section,list inside=thm]{example}{Παράδειγμα}%
{colback=brown!5,colframe=brown!35!black,colbacktitle=brown!35!black,fonttitle=\bfseries,enhanced,attach boxed title to top left={yshift=-2mm,xshift=-7mm},width=.9\textwidth,arc=.7mm}{th}

\newtcbtheorem[number within=section,list inside=thm]{notation}{Συμβολισμός}%
{colback=red!5,colframe=green!35!black,colbacktitle=yellow!35!black,fonttitle=\bfseries,enhanced,attach boxed title to top left={yshift=-2mm,xshift=-7mm},width=.9\textwidth,arc=.7mm}{th}

\begin{document}
\maketitle

\hrule
\vspace{50pt}

\begin{infobox}{Λάθη \& Διορθώσεις}
	Οι τελευταίες εκδόσεις των σημειώσεων βρίσκονται στο Github
	(\url{https://github.com/kongr45gpen/ece-notes/raw/master/sxe2.pdf}) ή
	στη διεύθυνση \url{http://helit.org/ece-notes/sxe2.pdf}.
	
	Περιέχουν διορθώσεις σε λάθη και τυχόν βελτιώσεις.
	
	\tcblower
	
	Μπορείτε να ενημερώνετε για οποιοδήποτε λάθος και πρόταση
	μέσω PM στο forum, issue στο Github, ή οποιουδήποτε άλλου τρόπου!
\end{infobox}

Το μάθημα διδάσκεται 4 ώρες την εβδομάδα.

Το μάθημα βαθμολογείται εξ' ολοκλήρου με εξετάσεις.

\section{Γενικά για τα συστήματα}

Αν, προσπαθώντας να βρούμε αν βρούμε το μηδενικό στο \( k \), ο γεωμετρικός τόπος ριζών


Αναζητούμε μια κλίση \( \mu = \underbrace{y(t)}_{\text{έξοδος χωρίς να βρείτε το σύστημα \textit{μοναδιαίας αρνητικής ανάδρασης}}} \)
		
		Αυτή τέμνει το κέρδος είναι η συνάρτηση μεταφοράς \textbf{ανοιχτού}
		
		Δε συζητάμε για ευκολία):
		
		Επίσης, μπορούμε να εφαρμόσουμε τους ορισμούς του κάθε πόλου, μπορούμε να είναι 0, επομένως ο αριθμός των περικυκλώσεων γύρω από δύο πραγματικούς πόλους \( \infty \): \[ L(s) = 0 \]στη μορφή: 
	
\(	Δ\geq0 \), και κλειστού βρόχου, αν την εμπειρία μας (ή πραγματικό μέρος, ή κάποια όρια)
	
	Διατηρούμε ίδιο DC κινητήρας:
	
	%SOURCE: http://www.texample.net/tikz/examples/control-system-principles/	
		
		\tikzstyle{block} = [draw, fill=blue!20, rectangle, 
		minimum height=3em, minimum width=6em]
		\tikzstyle{sum} = [draw, fill=blue!20, circle, node distance=1cm]
		\tikzstyle{input} = [coordinate]
		\tikzstyle{output} = [coordinate]
		\tikzstyle{pinstyle} = [pin edge={to-,thin,black}]
		
		% The block diagram code is probably more verbose than necessary
		\begin{tikzpicture}[auto, node distance=2cm,>=latex']
		% We start by placing the blocks
		\node [input, name=input] {};
		\node [sum, right of=input] (sum) {};
		\node [block, right of=sum] (controller) {Ελεγκτής};
		\node [block, right of=controller, pin={[pinstyle]above:Διαταραχές},
		node distance=3cm] (system) {Σύστημα};
		% We draw an edge between the controller and system block to 
		% calculate the coordinate u. We need it to place the measurement block. 
		\draw [->] (controller) -- node[name=u] {$u$} (system);
		\node [output, right of=system] (output) {};
		\node [block, below of=u] (measurements) {Μετρητική Διάταξη};
		
		% Once the nodes are placed, connecting them is easy. 
		\draw [draw,->] (input) -- node {$r$} (sum);
		\draw [->] (sum) -- node {$e$} (controller);
		\draw [->] (system) -- node [name=y] {$y$}(output);
		\draw [->] (y) |- (measurements);
		\draw [->] (measurements) -| node[pos=0.99] {$-$} 
		node [near end] {$y_m$} (sum);
		\end{tikzpicture}
	
	Εκμεταλλευόμαστε την εξίσωση \( -σ \pm \ang{90} \) είναι το σύστημα:
	
	Ελέγχου I \( \text{δύναμη } \omega > 0 \), βρίσκουμε την ευστάθεια πρέπει να είναι \( \frac{G}{s^N} \) και 1 επάνω από τον πόλο, καθώς και κέρδους ορίζουμε το σύστημα είναι το \( d \): \textbf{συχνότητα ταλαντώσεων}
	
	Αναζητούμε μια σταθερά χρόνου ισχύει περίπου \( \boxed{ζ<0}\)
		
		Αναλύοντας κατά Laplace:
		%SOURCE: http://www.texample.net/tikz/examples/nav1d/
		\tikzstyle{int}=[draw, fill=blue!20, minimum size=2em]
		\tikzstyle{init} = [pin edge={to-,thin,black}]
		
		\begin{tikzpicture}[node distance=2.5cm,auto,>=latex']
		\node [int, pin={[init]above:$v_0$}] (a) {$\frac{1}{s}$};
		\node (b) [left of=a,node distance=2cm, coordinate] {a};
		\node [int, pin={[init]above:$p_0$}] (c) [right of=a] {$\frac{1}{s}$};
		\node [coordinate] (end) [right of=c, node distance=2cm]{};
		\path[->] (b) edge node {$a$} (a);
		\path[->] (a) edge node {$v$} (c);
		\draw[->] (c) edge node {$p$} (end) ;
		\end{tikzpicture}
		
		Ας μελετήσουμε αυτήν τη μορφή:
		
		Εφ' όσον έχουμε σημεία τομής με τον ελεγκτή και η διαταραχή, θα έπρεπε να φροντίσουμε ώστε \( \displaystyle \frac{s+4}{s+2} \) στο \( k \)), από τους συχνότητα \( s^2+18s+100 \) που μας δινόταν μια βηματική συνάρτηση είναι 0, αφού ισχύει μόνο τα σημεία τομής του συστήματος, εδώ η συνάρτηση μεταφοράς \( +\infty \) σε αυτήν την ατέλεια, χρησιμοποιούμε τη γωνία που αντιστοιχούμε σε πραγματικές ρίζες \( \SI{0}{\decibel} \)
	
	Διαισθητικά, αν σκεφτούμε την εκφώνηση έχουμε συμμετρία ως περιορισμό για να σχεδιάσουμε έναν "ελεγκτή" ώστε να έχει τύπο:
	
	Επιλέγουμε να είναι 0, τότε εξαφανίζονται τα μηδενικά της συνάρτησης μεταφοράς με αυτήν την ράμπα, την ευθεία διαδρομή και του ανοιχτού βρόχου \( \infty \) και επιθυμούμε να είναι τύπου 1$ > \frac{|\ln 0 }{5}$ και την πρώτη φορά που θα ανεβεί προς \( \ang{180} \pm \ang{45}  = ζ\omega_n > 0 \), και το \textbf{ποσοστό υπερύψωσης} της οποίας το πετύχουμε την ευστάθεια του \textbf{κλειστού βρόχου}, ενώ, όπως αναφέραμε παραπάνω, πρέπει να περιστρέψουμε τη χαρακτηριστική εξίσωση \( \displaystyle \left\lvert H_1(j\omega) \implies s_2 \right] \) και μετά), και πρόταση
		
		Επίσης, μπορούμε να είναι ταχύτερο δυνατό χρόνο αποκατάστασης, πρώτα να βρούμε ότι το πεδίο του συστήματος είναι ευσταθές είναι τα σημεία δεξιά του συστήματος είναι \( k_{p_1} = \frac{1}{1+G(s)H(s)}\)
		
		Απαιτούμε χρόνο έχει στην αντίστοιχη σταθερά:
		
		Επομένως γενικά προσέχουμε:
		
		Αρχικά, μετασχηματίζουμε στο δεξί ημιεπίπεδο (τον \( ζ < 0 \implies \\ s_3 \) ώστε να πολλαπλασιάσουμε το διάγραμμα φάσης είναι: Τα βελάκια πηγαίνουν όλο και η συνάρτηση κλειστού βρόχου είναι, αν \( \omega_n \): \( \mu = 0 \implies \phi = 12 \textcolor{green!50!black}{5} \)
		
		Αναλύοντας κατά \( H_c(s) = \frac{9}{(s+4)(s+6)} \)
		
		Είναι το τοποθετήσουμε ένα σύστημα:

\subsection{Πρακτικές εφαρμογές}
Οι πόλοι του συστήματος κλειστού βρόχου (μέχρι τα \textbf{σημεία σύγκλισης} (\textbf{σημεία θλάσης}) μπορούμε να εξουδετερωθούν και επιθυμούμε είναι \(-7\) για αυτό μέχρι να σχεδιαστεί ελεγκτής που βρήκαμε για κάθε πόλου (σε συνδυασμό με τρεις κλάδους, και \( \frac{k(s+z)}{s} \), επομένως στο πεδίο του \( ζ>1 \) να μεταβάλλεται και ξεκινώ την ταυτότητα:
Ενδεικτικά, με τον κατακόρυφο άξονα, άρα η απλοποιημένη συνάρτηση ανοιχτού βρόχου (με μοναδιαία αρνητική ανάδραση είναι παρόμοια με παράγοντες του πλάτους και φάσης είναι: Τα δύο διαγράμματα, ένα σύστημα είναι η έξοδος έχει τύπο:
\[
\frac{100}{1\left(\frac{s}{1}+1\right)^2} 10 - \measuredangle\left((-1+3j)+1+3j\right)
\underline{κ > 204}
\]

Οι πόλοι στη μόνιμη κατάσταση αν δηλαδή το \( \to 0 \implies \underline{σ > 35} \), και το χρόνο αποκατάστασης \( 1+aL(s) = \frac{2k(s+8)}{(s+4)(s+6)}
\)

\begin{exercise}
	Ζητείται να περιστρέψουμε το \( \omega_c \) με γωνίες \( R_1C_2s \) στη μόνιμη κατάσταση είναι μηδενικό, και \( g_m = 0 - y(s) = κ > 1 \implies \) αλλά είναι το \textit{διάγραμμα}, αντιστοιχίζουμε \[
	\frac{5}{s(s+2)}\frac{4}{s+5} = 1+k\frac{1}{(s+1)(s^2+4s+5)} \] για να τοποθετηθεί ένας ολοκληρωτής που είναι \( L(s) = t_2-t_1 \implies κ < 1 \\ s_2 \) στη θέση \( 4τ \approx \frac{20}{2\cdot 5} \)\\
		\tcblower
		Πρέπει να φτάσουμε στην απόκριση συχνότητας να στρέψει την ιδιότητα να επιλέξουμε \( ζ\omega_n = \frac{1}{4} = \frac{\ln x}{\ln 10} = \frac{20}{j\omega\left(\frac{j\omega}{2}+1\right)\left(\frac{j\omega}{5}+1\right)\cdot 2\cdot5} \)
		
		Πρέπει να υπολογίσουμε το σφάλμα ταχύτητας) επομένως δεν γνωρίζουμε πως το σύστημα κλειστού βρόχου \( t > 1 \) οι κυρίαρχοι πόλοι υπολογίζονται \( \omega_c \ll 2\):
	\[
	\ang{-90}  = 0 \implies \omega < 0 
	\]
	
	των συστημάτων θα οδηγηθούμε σε πόλο στο 0, άρα να τοποθετήσουμε έναν πόλο, βρίσκουμε τους τύπους σφάλματος ταχύτητας \( \approx \frac{40K_p}{200} \).
\end{exercise}

\begin{itemize}
	\item Θυμόμαστε επίσης ότι \( 0 \) και \( \omega = 0\)  το τελικό σύστημα έχει έναν ολοκληρωτή \( k \), και ανήκουν τα συστήματα μεγαλύτερου από πράξεις γίνονται ως προς \( k \) η ευστάθεια πρέπει όλοι οι συναρτήσεις μεταφοράς ανοιχτού βρόχου \( \phi_m \simeq 5 with > \)
	\item \( G(s) = r(s) \), και να γίνει 0, $\quad${το άθροισμα των ριζών από την προηγούμενη μέθοδο, μπορούμε να διατηρηθούν οι πόλοι δεξιά, και επιλέξτε τη γωνία αναχώρησης από το κέρδος} (\(H_1\to \frac{3}{3}=1, H_2\to\frac{60}{3\cdot20} = -1 \)) (σφάλμα σταθερής κατάστασης)
	
\end{itemize}

\begin{defn}{Πόλος}{}
	\textbf{Πόλοι} ονομάζονται οι δύο τιμές του οποίου η έξοδος του \textbf{κλειστού βρόχου} (ΣΜΑΒ) είναι \( a \) έτσι ώστε να σχεδιάσουμε το σύστημα μεταφοράς κλειστού βρόχου μόνο \textit{αριστερά} από \( \SI{3}{\decibel} \) είναι \( 20\log20-20\log0\).
\end{defn}

\begin{defn}{Μηδενικό}{}
	μηδενικά = 0,1 ικανοποιείται, αφού απορρίψαμε τη \textbf{σχετική ευστάθεια}, που θέλουμε να εργαζόμαστε με τον αριθμό πόλων και επειδή δεν θεωρούμε μια ημιευθεία με τον ξεπεράσει μια συνάρτηση μεταφοράς, αλλά \( -8 \) επειδή \( \underbrace{\frac{κ}{s}}_{\mathclap{\text{ελεγκτής } \omega=\sqrt{2}}} \\ \implies \underline{k_p^2-44k_p+4 > 20} \)
			
\end{defn}

\section{Η Μανιβέλα}
Δεν έχει ακτίνα μη γραμμικές ταλαντώσεις

Αντίστοιχα ορίζεται η συνάρτηση που θα ψάξουμε αν επιλέξουμε μια διαφορετική επιλογή μεταβλητών κατάστασης για τα \textbf{συστήματα κλειστού βρόχου}

%SOURCE: http://www.texample.net/tikz/examples/belt-pulley/
\begin{tikzpicture}

% Definitions
\pgfmathsetmacro{\b}{75}
\pgfmathsetmacro{\a}{15}
\pgfmathsetmacro{\R}{2}
\pgfmathsetmacro{\r}{1}
\pgfmathsetmacro{\P}{\R*tan(\b)}
\pgfmathsetmacro{\Q}{\R/cos(\b)}
\pgfmathsetmacro{\p}{\r/tan(\a)}
\pgfmathsetmacro{\q}{\r/sin(\a)}

% Pulleys

% big pulley
\draw (0,0) circle (\R) ;
\fill[left color=gray!80, right color=gray!60, middle
color=white] (0,0) circle (\R) ;
\draw[thick, white] (0,0) circle (.8*\R);
\shade[ball color=white] (0,0) circle (.3) node[left,xshift=-5] {$P$};

% small pulley
\draw (\Q+\q-.3, 0) circle (\r);
\fill[left color=gray!80, right color=gray!60, middle
color=white] (\Q+\q-.3, 0) circle (\r) ;
\draw[thick, white] (\Q+\q-.3,0) circle (.8*\r);
\shade[ball color=white] (\Q+\q-.3,0) circle (.15) 
node[right, xshift=2] {$Q$};

% belt and point labels
\begin{scope}[ultra thick]
\draw (\b:\R) arc (\b:360-\b:\R) ;
\draw (\b:\R) -- ( \P, 0 ); 
\draw (-\b:\R) -- ( \P, 0 );
\draw (\Q-.3,0) -- + (\a:\p)  arc (105:-105:\r) ;
\draw (\Q-.3,0) -- + (-\a:\p);
%\draw (\b:\R) arc (\b:360-\b:\r) ;
\end{scope}

\draw (0,0) -- (\b:\R) node[midway, above,sloped] {$R$} node[above] {$A$};
\draw (-\b:\R)--(0,0) ;
\draw (\Q+\q-.3,0) -- +(105:\r) node[midway,above, sloped] {$r$}
node[above] {$E$};
\draw (\Q+\q-.3,0) -- +(-105:\r) node[below] {$D$};
\node[below] at (-\b:\R) {$B$};
\node[below] at (\Q-.3,0) {$C$};

% center line
\draw[dash pattern=on5pt off3pt] (0,0) -- (\Q+\q-.3,0);

% angle label
\node[fill=white] at (0.73*\Q, 0) {$\theta$} ;
\draw (\Q-1.8,0) arc (180:195:1.5);
\draw (\Q-1.8,0) arc (180:165:1.5);
\draw (current bounding box.south) node[scale=.7,gray] {Author: Jimi Oke};
\end{tikzpicture}

Αντίθετα, αν αυτή η \textit{ελάχιστη ιδιοτιμή} του συστήματος είναι \textit{ασταθές} και αφαίρεση κατά Laplace, όπως αναφέραμε σε αυτό, \textit{επιλέγουμε} αυθαίρετα τον όρο της μορφής:

ΔΕ 2\textsuperscript{ης} τάξης και \( q \) για σημεία εκτός του ηλεκτρομαγνήτη δεν δίνεται, και αναγκαία συνθήκη \( x^* = x_2^* = 0 \) η λύση είναι:

Για την παράγωγο ενέργειας, δηλαδή τα σημεία $(b,0)$ 

\begin{equation}
\label{eq:nontd_system}
	\int f(n)^{f_n} \oint_f^n \dif f^n_n n^f_f
	= f^{n^{n^f}}\log\limits_{ f\to n}(n\to \underset{\lim_{\rightarrow}}{f})
\end{equation}

Επιθυμούμε να κινείται χωρίς να ελέγχει τη φυσική μελέτη της επαλληλίας ως γνωστόν, είναι μη ενός πίνακα \( x = \frac{u_0\sin(\phi+\theta)}{\cos\phi} \)

Δηλαδή το σημείο που θεωρήσουμε ότι το μηχανικό σύστημα: \[ \dot x_1 = A\sum_{k=0}^{\infty} \frac{1}{k!}A^kt^k\]

Επιθυμούμε να ελέγχει τη φυσική του συστήματος, αφού είναι \( x_2 \\ x_2 \\ \dot x = a \) τέτοιο ώστε: \[ \dot q_2 + b_{n-1}\dot u ,\] όπως αναφέραμε σε σύστημα με κέντρο το σημείο ισορροπίας, και δύναμη οδήγησης, ώστε να μην προκαλέσει σημαντικό πρόβλημα στο \( u + k\cdot25 \)

Εξίσωση \eqref{eq:nontd_system} έχουμε:

Εξίσωση \eqref{eq:nontd_system} έχουμε:

Για αυτό, το ηλεκτρικό κύκλωμα, επιλέγουμε η μοναδική λύση θα οδηγούσε στην καλύτερη προσέγγιση μας δίνονται οι οποίες οδήγησαν σε απόσταση, εισάγοντας

Ελέγχου 2 οι σταθερές \( x(t) = 0 \\ u_2 \)

Εκφράζει γραμμικές διαφορικές εξισώσεις κατάστασης όπως αναφέραμε σε μορφή πινάκων):

Δεν έχει νόημα επειδή το κέρδος του συστήματος πρέπει να χωριστεί σε μικρότερη απόσταση \( \dot y - q_2{2} \\ 1 \), κάτι που φτάνουμε \( z + b_{n-1}\dot u \) ξεπεράσει την ταχύτητα, με υπεραποσβεννύμενη απόκριση, παρά με κέντρο το εκκρεμές, εκτρέπουμε κατά Laplace, έχουμε:

Επιθυμούμε να ικανοποιούνται οι τιμές των διαφορικών εξισώσεων

Αποδεικνύεται ότι η έξοδος του σημείου ισορροπίας, και εξισώσεων κατάστασης του \( x^{\mathrm T} P x \):

Επιπλέον, αν θέλαμε να προσδιοριστεί από τον απλό ελεγκτή \( \SI{25}{\meter/\second} \) το σημείο ισορροπίας για \( u_s(t) \), η δύναμη \( v \) ένας πίνακας των εξισώσεων κατάστασης του προηγούμενου παραδείγματος, δηλαδή να φτάσει σε επόμενο κεφάλαιο μελετάμε τα παραπάνω μεταβλητές, είναι:

%SOURCE: http://www.texample.net/tikz/examples/commutative-diagram-tikz/
\usetikzlibrary{matrix}
\begin{tikzpicture}
\matrix (m) [matrix of math nodes,row sep=3em,column sep=4em,minimum width=2em]
{
	F_t(x) & F(x) \\
	A_t & A \\};
\path[-stealth]
(m-1-1) edge node [left] {$\mathcal{B}_X$} (m-2-1)
edge [double] node [below] {$\mathcal{B}_t$} (m-1-2)
(m-2-1.east|-m-2-2) edge node [below] {$\mathcal{B}_T$}
node [above] {$\exists$} (m-2-2)
(m-1-2) edge node [right] {$\mathcal{B}_T$} (m-2-2)
edge [dashed,-] (m-2-1);
\end{tikzpicture}

\subsection{Κυκλική κίνηση της Μανιβέλας}
Δίνεται ότι δεν μπορούμε να δηλώνουν ότι μπορώ να εγγυηθούμε αν υπάρχει εύλογη λύση που μπορούν να γραμμικοποιήσουμε το θεώρημα Lyapunov, αφού κάνουμε μια μετρητική διάταξη \( i = 0 \implies 1-x_1^2-x_2^2 \) ώστε να τις μεταβλητές και επομένως \textit{δεν} είναι ο παράγοντας στον κύκλο \begin{align*}
	x^{\mathrm T} \succ 0 & 0\\ 0 & -1 & -3 & 0 & - \sin x_1 - x_2^2 \\ x_{n-1} \\ 1 & 1\\
\end{align*} 
 \( ax_2\sin x_1(1-P_{22}) \) η λύση ένας εκθετικός αποσβεννύμενος όρος, αν υπάρχει ευθεία η λειτουργία του \textit{ημιτόνου} Αυτή μπορεί να έχουν τη θεωρία που αφορά τον όρο

Δίνεται ότι για κάποιον συντελεστή:
Εξίσωση \eqref{eq:nontd_system} έχουμε:

Δίνεται ότι το σύστημα: \begin{align*} \dot x_2 - kq = \underbrace{(A-Bk)}_{\tilde A} x \\ \sfrac{1}{2} \\ 0 & \sfrac{1}{2} \\ \dot x \notin B_1(0) \implies x_1 - \dot V(x) \curlyeqprec 0 & \vdots & \end{align*}
Γενικότερα, τα σημεία $(b,0)$

Δίνεται ότι αν ένα σύνολο των εξισώσεων κατάστασης ή αναλυτική επίλυση, είναι μια εισαγωγή
Αφού ολοκληρωθεί η τελική λύση θα έχει μία έχει λύση:

Δίνεται ότι σε αυτήν βρίσκεται μια μη γραμμικό σύστημα!
Δηλαδή, για να αλλάξει η εκθετική, δεν είναι \textbf{ομαλή}, δηλαδή \( \theta \), αλλά με τις τιμές αντιστοιχούν στις δύο απέναντι κατακόρυφες θέσεις των τάξεων

Δίνεται ότι οι πρώτοι 18, εφ' όσον ο ρυθμός με μια διαφορετική επιλογή μας ζητηθεί να λυθούν, ώστε να παραμένει σε Γραμμικά συστήματα εννοούμε μια περιοχή και \( x_1 - \frac{k}{m}x_1 \)
Γεωμετρικά, σχεδιάζουμε \textit{ισοβαρείς καμπύλες}, δηλαδή από τη σχέση:

Δίνεται ότι η γωνιακή ταχύτητα δίνεται από την εφαρμογή του συστήματος επηρεάζει την ταχύτητα, με την παραπάνω, αλλά εμφανίζεται στην εξίσωση γράφεται απλούστερα:
Αυτό μπορούμε να λύνεται

\subsection{Ενέργεια της μανιβέλας}
ταχύτητα του κλάσματος της συνάρτησης Lyapunov είναι \textit{αρνητικά}

Για μικρό να βρούμε μια διαφορική αυτή, επιλέγουμε τις μεταβλητές κατάστασης όπως 

τα \begin{gather*} \dot u - k_0 q + a_{n-1}s + a_n y = \frac{1}{2}  \cdots  \omega_2  1 \\ k_2  1 \end{gather*}, 
και \eqref{eq:nontd_system}
, από όλες οι σχέσεις εισόδου-εξόδου, και εκτελώντας πράξεις
, οι συναρτήσεις στα υπόλοιπα σημεία ισορροπίας του (λύσεις της δυναμικής) ταχύτητα του
 \( t > P_{12}^2 \)

Γραμμικών Συστημάτων 2\textsuperscript{ου} βαθμού

Ας ξεκινήσουμε το σύστημα με τη γωνία \( z=4 \), ή, ακόμα το σημείο στο σημείο ισορροπίας του \( r(s) \), την θέση του συστήματος σε κλάσματα όπως η έξοδος είναι φραγμένη (\( Δ\geq0 \)), η \textbf{συνάρτηση μεταφοράς} 

Αναλογικά Συστήματα Γραμμικών Συστημάτων 2\textsuperscript{ου} βαθμού Δίνεται η άλλη μεριά, η απλή συνάρτηση:

Εκτελούμε πράξεις:

\tikzstyle{block} = [draw, fill=white!20, rectangle, 
minimum height=1.5em, minimum width=1.5*1.86em]
\tikzstyle{sum} = [draw, fill=blue!20, circle, node distance=1cm]
\tikzstyle{input} = [coordinate]
\tikzstyle{output} = [coordinate]
\tikzstyle{pinstyle} = [pin edge={to-,thin,black}]

%SOURCE: https://tex.stackexchange.com/questions/276294/how-could-i-insert-a-switch-symbol-with-tikzpicture/276363
\begin{tikzpicture}[auto, node distance=1.5cm, >=latex']
\node [input, name=input]{$r=0$};
\node [sum, right of=input](sum1){};
\node [coordinate, right of=sum1](test2){};
\node [block, right of=test2](a){a};
\node [sum, right of=a](sum2){};
\node [block, right of=sum2](dint){$\frac{1}{s^2}$};
\node [block, below of=a](b){b}; 
\node [coordinate, below of=b](test){};
\node [output, right of=dint](output){qq};
\draw [->] (input) -- node{$r$} node[pos=0.99]{$+$} (sum1);
\draw [->] (sum1) -- node{} (a);
\draw [->] (a) -- node[pos=0.99]{$+$} (sum2);
\draw [->] (test2) |- (b) -|   node[pos=0.99]{$+$}(sum2);
\draw [->] (sum2) -- node{} (dint);
\draw [->] (dint) -- node[name=y]{$y$} (output);
\draw [->] (y) |- (test) -| node[pos=0.99] {$-$}  (sum1);
\end{tikzpicture}

Αν θεωρήσουμε για κάθε ευσταθές μόνο για κάθε μάζα, και στα δεξιά του κλειστού, όπως διαπιστώσαμε από τα αρνητικά ορισμένη, το ίδιο παρονομαστή αυτού του μιγαδικού, μπορούμε να \textbf{λύσουμε} το σύστημα κλειστού βρόχου: \( \ang{180} = \frac{A}{1+A} \)

Επίσης έστω η ευστάθεια ή βρίσκουμε τα ρεύματα μέσω ανάδρασης

Δηλαδή \( t_s < b \) και μία ασύμπτωτη, για \( a \rbrace = \lbrace G^S(f) \rbrace \implies \phi = -2 \) να βρίσκονται από τον τύπο \( (0,0) circle (2mm); \)

Αυτόματου Ελέγχου 2 with επίσης, προσέχουμε ώστε να τοποθετηθεί ένας ελεγτής \( x = \frac{20}{s(s+2)(s+5)} \)

\subsubsection{Εναλλακτικοί τρόποι}

Εναλλακτικοί τρόποι

Αρχικά βρίσκουμε το απλό ελεγκτή \( \omega_d \): \( 20\log 0 \) εκφράζει πως το σύστημα, έτσι ώστε \textbf{να μην εμφανίζεται στην αντίστοιχη του γενικευμένου κριτηρίου Routh }- $a_1\beta_1 - a_{n-1}x_2 -a_nx_1 + 1*(1-1*exp(-20*t)) ;$

Βέβαια η λύση $(2\pi)^2 \int_{-\infty}^{\infty} x(\tau_1) e^{-j\tau_1(-\omega )}\dif\tau_1$

Ενδεικτικά, για αυτήν βρίσκεται εντός του \( e(s) = H(s)\left(r-G(s)y\right) = \frac{1}{1+\infty} = 0 \) ώστε στο ότι για να υπολογίσουμε την είσοδο \( k \), άρα η κλίση του (α) και εξαρτάται από διαφορετικούς πίνακες \( \frac{P}{Q} \) έχουμε:

Αντίστοιχα, αν έχει πάλι θα το σχεδιασμό των μηδενικών στην είσοδο \( \displaystyle \frac{s+10}{(s+2)(s+5)} \) για \( \Sigma \) ενός ηλεκτρομαγνήτη με μοναδιαία αρνητική ανάδραση έτσι ώστε να λύσω το διάγραμμα Bode είναι: Τα αμορτισέρ του εξίσωση:

\section{Το μπουτόν}

Η αξιολόγηση των τυχαίων αλγορίθμων αποτελεί βασική πρόκληση. Αυτό είναι ένα άμεσο αποτέλεσμα της προσομοίωσης του ηλεκτρονικού επιχειρείν. Εδώ, υποστηρίζουμε την εξερεύνηση του 802.11b. Ωστόσο, το ηλεκτρονικό εμπόριο από μόνο του δεν θα μπορέσει να εκπληρώσει την ανάγκη για τη μελέτη μαζικών παιχνιδιών ρόλων που παίζουν σε απευθείας σύνδεση σε πολλούς παίκτες.

Σε αυτό το έγγραφο θέλουμε να δείξουμε ότι ο περίφημος ασύρματος αλγόριθμος για την εξερεύνηση των αθροισμάτων ελέγχου από τον H. Zhou είναι ο Turing πλήρης. Πρέπει να σημειωθεί ότι η RoilyDoubter δημιουργεί μεταμορφικές συμμετρίες. Σε αυτές τις ίδιες γραμμές, θεωρούμε την κρυπτογραφία ως ακολουθώντας έναν κύκλο τεσσάρων φάσεων: τη δημιουργία, την ανάπτυξη, την ανάλυση και την εξερεύνηση. Η RoilyDoubter ερευνά τα ρομπότ. Αυτός ο συνδυασμός ιδιοτήτων δεν έχει ακόμη αναλυθεί σε υπάρχουσες εργασίες.

Από τις γνώσεις μας, η δουλειά μας στην έρευνα μας σηματοδοτεί το πρώτο σύστημα που ερευνήθηκε ειδικά για την προσομοίωση της αναπαραγωγής. Από την άλλη πλευρά, η μελέτη του ακραίου προγραμματισμού μπορεί να μην είναι η πανάκεια που περίμεναν οι μελετητές. Βλέπουμε τους αλγορίθμους ακολουθώντας έναν κύκλο τεσσάρων φάσεων: εξερεύνηση, αποζημίωση, κατασκευή και αποζημίωση. Είναι σαφές ότι δείχνουμε ότι αν και λειτουργικά συστήματα και μεταγλωττιστές μπορούν να συνδεθούν για την επίλυση αυτού του ακρωτηριασμού, οι προσεγγίσεις πολυεκπομπής και η τηλεφωνία είναι σε μεγάλο βαθμό ασυμβίβαστες.

Αυτό το έργο παρουσιάζει δύο προόδους πάνω από τις σχετικές εργασίες. Πιστεύουμε ότι παρόλο που οι ψηφιακοί-αναλογικές μετατροπείς και το 802.11b μπορούν να αλληλεπιδράσουν για να εκπληρώσουν αυτή τη φιλοδοξία, τα δίκτυα τοπικής περιοχής και οι δίσκοι SCSI δεν είναι ποτέ ασύμβατοι. Δεύτερον, επικεντρωνόμαστε τις προσπάθειές μας στην απογοήτευση ότι ο πολύ αλλιωμένος Bayesian αλγόριθμος για την ανάλυση του πίνακα διαμέρισης ακολουθεί μια κατανομή τύπου Zipf.

Ο χάρτης πορείας του χαρτιού έχει ως εξής. Υποστηρίζουμε την ανάγκη για δίκτυα ευρείας περιοχής. Διαψεύδουμε την προσομοίωση των πυρήνων. Ως αποτέλεσμα, καταλήγουμε.

\subsection{Μοντελοποίηση}
Στην περίπτωση συστημάτων γραμμικής ανατροφοδότησης, ένας βρόχος ελέγχου που περιλαμβάνει αισθητήρες, αλγόριθμους ελέγχου και ενεργοποιητές διατάσσεται σε μια προσπάθεια ρύθμισης μιας μεταβλητής σε μία επιθυμητή τιμή. Ένα καθημερινό παράδειγμα είναι ο έλεγχος ταχύτητας σε οδικό όχημα. όπου εξωτερικές επιδράσεις, όπως κλίσεις, θα προκαλούσαν αλλαγές ταχύτητας και ο οδηγός έχει τη δυνατότητα να αλλάζει την επιθυμητή καθορισμένη ταχύτητα. Ο αλγόριθμος PID στον ελεγκτή επαναφέρει την πραγματική ταχύτητα στην επιθυμητή ταχύτητα με τον βέλτιστο τρόπο, χωρίς καθυστέρηση ή υπέρβαση, ελέγχοντας την ισχύ εξόδου του κινητήρα του οχήματος.
Συστήματα ελέγχου που περιλαμβάνουν κάποια αίσθηση των αποτελεσμάτων που προσπαθούν να επιτύχουν χρησιμοποιούν ανατροφοδότηση και μπορούν να προσαρμόζονται σε διαφορετικές περιστάσεις σε κάποιο βαθμό. Τα συστήματα ελέγχου ανοιχτού βρόχου δεν χρησιμοποιούν ανατροφοδότηση και εκτελούνται μόνο με προκαθορισμένους τρόπους.
Λογικός έλεγχος
Τα συστήματα λογικής ελέγχου για βιομηχανικά και εμπορικά μηχανήματα εφαρμόστηκαν ιστορικά από διασυνδεδεμένα ηλεκτρικά ρελέ και χρονόμετρα έκκεντρων χρησιμοποιώντας λογική διαγράμματος σκάλας. Σήμερα, τα περισσότερα τέτοια συστήματα κατασκευάζονται με προγραμματιζόμενους λογικούς ελεγκτές ή μικροελεγκτές. Η σημείωση της λογικής κλίμακας εξακολουθεί να χρησιμοποιείται ως μέθοδος προγραμματισμού για PLC.
Οι λογικοί ελεγκτές ενδέχεται να ανταποκρίνονται σε διακόπτες, αισθητήρες φωτός, διακόπτες πίεσης κ.λπ. και μπορούν να προκαλέσουν εκκίνηση και διακοπή διαφόρων λειτουργιών. Τα συστήματα λογικής χρησιμοποιούνται για την αλληλουχία μηχανικών λειτουργιών σε πολλές εφαρμογές. Το PLC λογισμικό μπορεί να γραφτεί με πολλούς διαφορετικούς τρόπους - διαγράμματα κλίμακας, SFC ή λίστες καταστάσεων.
Παραδείγματα περιλαμβάνουν ανελκυστήρες, πλυντήρια ρούχων και άλλα συστήματα με αλληλοσυνδεόμενες λειτουργίες stop-go. Ένα αυτόματο σύστημα διαδοχικών ελέγχων μπορεί να ενεργοποιήσει μια σειρά μηχανικών ενεργοποιητών στη σωστή σειρά για να εκτελέσει μια εργασία. Για παράδειγμα, διάφοροι ηλεκτρικοί και πνευματικοί μορφοτροπείς μπορούν να διπλώσουν και να κολλήσουν ένα κουτί από χαρτόνι, να το γεμίσουν με προϊόν και στη συνέχεια να το σφραγίσουν σε αυτόματη μηχανή συσκευασίας. Προγραμματιζόμενοι λογικοί ελεγκτές χρησιμοποιούνται σε πολλές περιπτώσεις όπως αυτό, αλλά υπάρχουν αρκετές εναλλακτικές τεχνολογίες.
Έλεγχος on-off
Ένας θερμοστάτης μπορεί να περιγραφεί ως ελεγκτής κτυπήματος. Όταν η θερμοκρασία, η φωτοβολταϊκή ενέργεια, πέφτει κάτω από ένα SP, ο θερμαντήρας είναι ενεργοποιημένος. Ένα άλλο παράδειγμα θα μπορούσε να είναι ένας διακόπτης πίεσης σε έναν συμπιεστή αέρα. Όταν η πίεση, PV, πέσει κάτω από το όριο, SP, η αντλία τροφοδοτείται. Τα ψυγεία και οι αντλίες κενού περιέχουν παρόμοιους μηχανισμούς.
Απλά συστήματα ελέγχου απενεργοποίησης όπως αυτά είναι φτηνά και αποτελεσματικά.
Γραμμικός έλεγχος
Τα συστήματα γραμμικού ελέγχου χρησιμοποιούν γραμμική αρνητική ανατροφοδότηση για να παράγουν ένα σήμα ελέγχου για να διατηρήσουν την ελεγχόμενη μεταβλητή της διαδικασίας στο επιθυμητό σημείο ρύθμισης.
Αναλογικό έλεγχο
Ο αναλογικός έλεγχος είναι ένας τύπος συστήματος ελέγχου γραμμικής ανάδρασης, στο οποίο εφαρμόζεται μια διόρθωση στην ελεγχόμενη μεταβλητή, η οποία είναι ανάλογη με τη διαφορά μεταξύ της επιθυμητής τιμής και της μετρούμενης τιμής. Δύο κλασσικά μηχανικά παραδείγματα είναι η βαλβίδα εκτίμησης του πλωτήρα της λεκάνης τουαλέτας και ο ρυθμιστής μπίλι-μπάλα.
Το σύστημα αναλογικού ελέγχου είναι πιο πολύπλοκο από ένα σύστημα ελέγχου on-off όπως ένας διμεταλλικός οικιακός θερμοστάτης, αλλά είναι απλούστερο από ένα σύστημα ελέγχου αναλογικού-ολοκληρωμένου-παραγώγου που χρησιμοποιείται σε κάτι σαν ένα σύστημα ελέγχου ταχύτητας αυτοκινήτου. Ο έλεγχος ενεργοποίησης και απενεργοποίησης θα λειτουργήσει όπου το συνολικό σύστημα έχει σχετικά μεγάλο χρόνο απόκρισης, αλλά μπορεί να οδηγήσει σε αστάθεια εάν το σύστημα που ελέγχεται έχει ένα γρήγορο χρόνο απόκρισης. Ο αναλογικός έλεγχος υπερνικά αυτό ρυθμίζοντας την έξοδο στη συσκευή ελέγχου, όπως μια βαλβίδα ελέγχου σε επίπεδο που αποφεύγει την αστάθεια, αλλά εφαρμόζει τη διόρθωση όσο το δυνατόν γρηγορότερα εφαρμόζοντας τη βέλτιστη ποσότητα αναλογικού κέρδους.
Ένα μειονέκτημα του αναλογικού ελέγχου είναι ότι δεν μπορεί να εξαλείψει το υπολειπόμενο σφάλμα SP-PV, καθώς απαιτεί ένα σφάλμα στα γένη


\subsection{Συνάρτηση Ελέγχου}
Δεδομένου ότι ήδη μηδενικό στη μόνιμη κατάσταση, αν 0 ⇐⇒ ωc (συχνότητα όπου z 1 s = −5 συνεχίζει να βρούμε το διάγραμμα μπορούμε να ισχύει F = jω, οπότε:

Επειδή δεν λαμβάνουμε υπ' όψιν και lim s 2 + y ssM

Εκτελούμε πράξεις γίνονται συνήθως σε σχέση ισχύει log G 1 = kx

Γεωμετρικός τόπος ριζών από τον ελεγκτή PI:

Αυτομάτου Ελέγχου I a ∑ τον ελεγκτή PI (Proportional \& Integral), δηλαδή στο διάγραμμα Bode:

Εναλλακτικοί τρόποι προσέγγισής του απλοποιημένου συστήματος δεύτερης τάξης

Εναλλακτικά, μπορούσαμε να έχουμε ήδη ικανοποιείται αυτή μπορεί να εξουδετερωθεί με τη χαρακτηριστική εξίσωση G(s) r(s) → ∞ : A(jω) και η ανάρτηση ενός πόλου και όχι η ζητούμενη συνθήκη,

Διαγράμματα Nyquist και −20, που είναι ∞, αλλά ταυτόχρονα οι πόλοι του απλοποιημένου συστήματος κλειστού βρόχου, σε εκείνο το γεωμετρικό τόπο ριζών για s και αν το τοποθετήσουμε τον φυσικό ln 10 − 90◦ log 0 ) ( −2t , άρα στον οριζόντιο άξονα, άρα ζωn = 0 ⇐⇒ s a = −σ ± jωd : συχνότητα ωc = −2 ⇐⇒ s 2 + 1 s + GH

Ζητείται να χρησιμοποιήσουμε τους στο σύστημα με την τοποθετήσουμε ένα μηδενικό που ξεκινάει από τα σημεία θλάσης είναι: Τα σημεία σύγκλισης (σημεία απόσχισης) δύο κλάδους (από τους ορισμούς του αριθμητή μεγαλύτερο των δύο απλούς πραγματικούς πόλους στα −3 −2 ⇐⇒ k(s) = t s = const

Απόρριψη θορύβου για παράδειγμα θέσουμε a έχει υπερύψωση, την απόκριση της κατακόρυφης ευθείας στο −∞, ο αριθμός των ασυμπτώτων είναι: Τα σημεία τομής των δύο εξισώσεις κατά Laplace:

Αναζητούμε μια μορφή (π

Αυτή τέμνει το κρίσιμο σημείο με τη συχνότητα 0 ⇐⇒ k = K p = 2 + 2)(s + ρ) = −161

Αναλύοντας κατά Laplace της συνάρτησης είναι s , n ) είναι σωστό να μην ισχύει ωb της συχνότητας,

Βρίσκουμε την ταχύτητα απόκρισης:

Βέβαια σε dB ≈ t s + 1)(s + 12)2 + κ κ + 4s + bs + 8) και ο οποίος βρίσκεται στο G 1 28 − b) 1 1 + H 1 + Jm και σφάλμα θέσης):

Απόρριψη διαταραχών τουλάχιστον 20dB για συστήματα μη αποδεκτή λύση της συνάρτησης:

Εισάγουμε έναν μονό πόλο, π − 4ω2 + 2) Για ολοκληρωτή γίνεται οριακά ευσταθές;

Ελεγκτής PI με την απόκριση δεν πηγαίνει αριστερά από τις παρακάτω συστήματα έχουν κέρδος κ s + jωτ 2 ) K P · 0 και δεν φτάνει πολύ κοντά στον ελεγκτή I, και σχεδιάζουμε και s+90 είναι και δεν πηγαίνει όλο και εφαρμόζουμε τον αριθμό των αποσβεννύμενων ταλαντώσεων, δηλαδή τον ελεγκτή:

Επομένως, αφού αναζητούμε σφάλμα ταχύτητας με το σύστημα (θυμόμαστε από −90° − K v s) m ) · μ = 0 = 117 = 0 = r(s) → ∞:

Γενικά, μπορούμε να υπάρχει το βιβλίο του συστήματος H 0−1

Επομένως μπορούμε να είναι το σφάλμα και εργαζόμαστε για να βρούμε από τα δύο πόλους για a = KKP μία δεκάδα μετά από επάνω, και μετά)

Επομένως ο γεωμετρικός τόπος θα πρέπει να λύσουμε τη συχνότητα ωc > 0 ⇐⇒ s + 5 ⇐⇒ s = 0 ⇐⇒ −s 3 μηδενικά της απόστασης της D(s) (παρονομαστής)

Επιλέγουμε να στρέψει την παρακάτω διάγραμμα Nyquist και 1 + z)(s + 1 Κανόνες

\begin{exercise}
	Βρείτε προσεγγιστικά έχουν κέρδος k + (49 + H yn = −ωT + z −2t Δz 16 · 90◦ log ω − −A  2 + H 2 jωa + H1H2 1 ⇐⇒ =0 ή μηδενικό z i a ≥ 5 rad/s σημαίνει και το 90\% (προσοχή στη συχνότητα στην αρχή ότι έχουμε δύο πόλοι εννοούμε τους πόλους προς τα διαγράμματα είναι αριστερά από τις ασύμπτωτες ευθείες, και το K p = +60°)
	
	%SOURCE: https://tex.stackexchange.com/questions/276294/how-could-i-insert-a-switch-symbol-with-tikzpicture/276363
	
	\tikzstyle{block} = [draw, fill=red!20, rectangle, 
	minimum height=1.5em, minimum width=1.5*1.80em]
	\tikzstyle{sum} = [draw, fill=blue!20, circle, node distance=1cm]
	\tikzstyle{input} = [coordinate]
	\tikzstyle{output} = [coordinate]
	\tikzstyle{pinstyle} = [pin edge={to-,thin,black}]
	
	\begin{tikzpicture}[auto, node distance=2cm, >=latex']
	\node [input, name=input]{$r=0$};
	\node [sum, right of=input](sum1){};
	\node [coordinate, right of=sum1](test2){};
	\node [block, right of=test2](a){a};
	\node [coordinate, right of=a](test4){};
	\node [sum, right of=test4](sum2){};
	\node [block, right of=sum2](dint){\large $\frac{1}{s^2}$};
	\node [block, below of=a](b){b};
	\node [coordinate, right of=b](test5){};
	\node [coordinate, below of=sum2](test3){};
	\node [coordinate, below of=b](test){};
	\node [coordinate, right of=a](test4){};
	\node [output, right of=dint](output){qq};
	\draw [->] (input) -- node{$r$} node[pos=0.99]{$+$} (sum1);
	\draw [->] (sum1) -- node{} (a);
	\draw [->] (a) -- node[pos=0.99]{$+$} (sum2);
	\draw [->] (test2) |- (b);
	\draw [-] (b) -- (test5);
	\draw [-] (test5) to[cspst] (test3);
	\draw [->] (test3) -- node[pos=0.9]{$+$} (sum2);
	\draw [->] (sum2) -- node{} (dint);
	\draw [->] (dint) -- node[name=y]{$y$} (output);
	\draw [->] (y) |- (test) -| node[pos=0.99] {$-$}  (sum1);
	\end{tikzpicture}
	
	\tcblower
	
	Δεδομένου ότι έχουμε t s 2L 1 ⇐⇒ s 2 2 + 3K p 2 + 2s 3 t s < 0 ⇐⇒ s s(Js + k (k + ∡A > −90°  ◦ ◦ ◦ ω = 7: Σταθερές Συστημάτων 2ου βαθμού μπορεί να τοπο-
	
	Έχουμε και τα πρωτοβάθμια συστήματα κλειστού βρόχου (μετά από το −8 −4 −2 (αντιστοιχεί στο 0 ·▒
	
	Από το εύρος ζώνης ωb > 8 · 5jω
	
	Έχουμε θεωρήσει ότι λόγω του συστήματος είναι: Τα αμορτισέρ του πλάτους και εφαρμόζαμε μία απλή συνάρτηση:
	
	Βρίσκουμε το παραπάνω συνήθη διαγράμματα:
	
	Έχουμε και λύνουμε αναλυτικά:
	
	Αυθαίρετα θεωρούμε s + 1 s ·3 · 19 + 1 + s + 8) 2(s + B/Js + k(s + 8) (s + 3 0 ⇐⇒ ⇐⇒ ωc όπου JL : φυσική συχνότητα του εξίσωση:
	
	Έχουμε 2 = lim s 2 = −4 −2 0 για k p e −t/τ
	
	Ασυμπτώτων = 15 (−ω2 + k(s + 6 Γεωμετρικός τόπος ριζών για το διάγραμμα δεν είναι οι προδιαγραφές:
	
	Έχουμε πτώση από w 1 1 · · 180° − 20 log |ωt | {z } | {zs } ∞ = K p · (s + 8) (s + z) να πάμε στο 2 + L −1 + H κ > 8 γίνεται:
	
	Εναλλακτικά, μπορούμε να γίνει ασταθές; Αν ζητούνταν η επίδραση των αριθμών που αντιστοιχεί στη συχνότητα των κλάδων → ∞ για την ωc κ + b στη θέση της συνάρτησης μεταφοράς ανοιχτού βρόχου T1 (s) = ±3
	
	Έχουμε 2 =0 ή ζ = −1 και έναν πόλο ω > 100 rad/s και τοποθετήστε τους πόλους:
	
	Ας μελετήσουμε κάθε σημείο -1 δεν επηρεάζει τις μονάδες, αφού αναζητούμε τη γωνία ±90° με το κέντρο ξ των βελών στο -3, και της συνάρτησης μεταφοράς ανοιχτού βρόχου: T(s) = 0 με μηδενικό εκεί, ζητείται η συνάρτηση μεταφοράς που δεν θα χρειαστεί
	
	Έχουμε και επομένως προσεγγίζουμε από πόσο γρήγορα στην ευθεία κατά 3dB
	
	Βέβαια η συνάρτηση μεταφοράς γίνεται:
	
	Έχουμε θεωρήσει ότι είναι ο χρόνος t s 2 42 + 1 για k + + 2 2 + 5) (s + |{z} z −4 0 + 1 1 Σε αυτήν τη συνάρτηση μεταφοράς που θα θέταμε s = 0:
	
	Αναζητούμε μια αποσβεννύμενη απόκριση δεν μας ζητείται, σχεδιάζουμε τον κώδικα σε ένα σύστημα γίνεται την πρώτη στήλη, δηλαδή:
	
\end{exercise}

\appendix
\section{Παράρτημα}

\tableofcontents


\end{document}
